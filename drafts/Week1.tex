\documentclass[12pt, oneside]{article}

\usepackage[letterpaper, scale=0.89, centering]{geometry}
\usepackage{fancyhdr}
\setlength{\parindent}{0em}
\setlength{\parskip}{1em}

\pagestyle{fancy}
\fancyhf{}
\rfoot{\href{https://creativecommons.org/licenses/by-nc-sa/2.0/}{CC BY-NC-SA 2.0} Version \today~(\thepage.)}

\input{../../resources/CSE20packages}

\begin{document}
\begin{flushright}
    \StrBefore{\currfilename}{.}
\end{flushright}

\section*{Before we start}

We are committed to fostering a learning environment for this course that supports a diversity of thoughts, 
perspectives and experiences, and respects your identities (including race, ethnicity, heritage, gender, sex, 
class, sexuality, religion, ability, age, educational background, etc.).  
Our goal is to create a diverse and inclusive learning environment where all students feel comfortable and can thrive. 

If you or someone you know is suffering from food and/or housing insecurities 
there are UCSD resources here to help:

Basic Needs Office: \href{https://basicneeds.ucsd.edu/}{https://basicneeds.ucsd.edu/}

Triton Food Pantry (in the old Student Center)
is free and anonymous, and includes produce: 

\href{https://www.facebook.com/tritonfoodpantry/}{https://www.facebook.com/tritonfoodpantry/}

Mutual Aid UCSD: \href{https://mutualaiducsd.wordpress.com/}{https://mutualaiducsd.wordpress.com/}

Financial aid resources, the possibility of emergency grant funding, and off-campus housing referral 
resources are available. See CAPS and your college dean.

If you find yourself in an uncomfortable situation, ask for help. 
We are committed to upholding University policies regarding nondiscrimination, sexual violence and sexual harassment.

Counseling and Psychological Services (CAPS) at 858 5343755 or \href{http://caps.ucsd.edu}{http://caps.ucsd.edu}


OPHD at (858) 534-8298, ophd@ucsd.edu , \href{http://ophd.ucsd.edu}{http://ophd.ucsd.edu}. 
CARE at Sexual Assault Resource Center at 858 5345793 sarc@ucsd.edu \href{http://care.ucsd.edu}{http://care.ucsd.edu}

\subsection*{Spring quarter philosophy}
Spring 2022 is still a transition quarter so please be patient with us as we do our best 
to deliver a great learning opportunity that meets the needs of all students and adheres to the university guidelines. 

Please do not come to class if you are sick or even think you might be sick.

Please reach out (minnes@eng.ucsd.edu) if you need support with extenuating circumstances.

Based on current UCSD guidelines (as of March 21), masks are required in class. 
All students who attend class must also be fully vaccinated against COVID-19
unless they have a university-approved exemption.
We will continue to follow the campus guidelines as updated on https://returntolearn.ucsd.edu/ .


\newpage

\section*{Introductions}
Class website: \href{https://cseweb.ucsd.edu/classes/sp22/cse105-a/}{https://cseweb.ucsd.edu/classes/sp22/cse105-a/}

{\bf Pro-tip}: the URL structure is your map to finding your course website for other CSE classes.

Instructor: Prof. Mia Minnes {\tiny{"Minnes" rhymes with Guinness}}, minnes@eng.ucsd.edu, 
\href{http://cseweb.ucsd.edu/~minnes}{http://cseweb.ucsd.edu/~minnes}

Our team: Six TAs and five tutors + all of you

Fill in contact info for students around you, if you'd like:

\vfill


On a typical week: {\bf MWF} Lectures + review quizzes, {\bf M} Discussion, {\bf Th} Homework / project due.
Office hours and Q+A on Piazza available throughout the week.
All dates are on \href{https://canvas.ucsd.edu/}{Canvas (click for link)} and details are on
 \href{https://theory-cs.github.io/website/overview_calendar.html}{course calendar (click for link)}.

\newpage Welcome to CSE 105: Introduction to Theory of Computation in Spring 2022!

\section*{CSE 105's Big Questions}
\begin{itemize}
   \item What problems are computers capable of solving?
   \item What resources are needed to solve a problem?
   \item Are some problems harder than others?
\end{itemize}

In this context, a {\bf problem} is defined as: ``Making a decision or computing a value based on some input"

Consider the following problems: 
\begin{itemize}
   \item Find a file on your computer
   \item Determine if your code will compile
   \item Find a run-time error in your code
   \item Certify that your system is un-hackable
\end{itemize}

Which of these is hardest?

\vfill

In Computer Science, we operationalize ``hardest'' as ``requires most resources'', where
resources might be memory, time, parallelism, randomness, power, etc.

To be able to compare ``hardness'' of problems, we use a consistent description of problems

{\bf Input}: String

{\bf Output}: Yes/ No, where Yes means that the input string matches the pattern or property described by the problem.


\newpage
\section*{Monday March 28}

%! app: Regular Languages
%! outcome: Regular expressions

Our motivation in studying sets of strings is that they can be used to encode problems.
To calibrate how difficult a problem is to solve, we describe how complicated the set of strings that encodes it is. 
How do we define sets of strings?


\vfill

How would you describe the language that has no elements at all?

\vfill

How would you describe the language that has all strings over $\{0,1\}$ as its elements?

\vfill

\newpage

**This definition was in the pre-class reading**
{\bf Definition 1.52}: A {\bf regular expression} over alphabet $\Sigma$
is a syntactic expression that can describe a language over $\Sigma$. The collection of all regular
expressions over $\Sigma$ is defined recursively:
\begin{itemize}
\item[] {\it Basis steps of recursive definition}
\begin{quote}    
    $a$ is a regular expression, for $a \in \Sigma$

    $\varepsilon$ is a regular expression

    $\emptyset$ is a regular expression
\end{quote}

\item[] {\it Recursive steps of recursive definition}
\begin{quote}
    $(R_1 \cup R_2)$ is a regular expression when $R_1$, $R_2$ are regular expressions 

    $(R_1 \circ R_2)$ is a regular expression when $R_1$, $R_2$ are regular expressions

    $(R_1^*)$ is a regular expression when $R_1$ is a regular expression 
\end{quote}
\end{itemize}
 

The {\it semantics} (or meaning) of the syntactic regular expression is the {\bf language
described by the regular expression}. The function that assigns a language to a regular expression
over $\Sigma$ is defined recursively, using familiar set operations:


\begin{itemize}
    \item[] {\it Basis steps of recursive definition}
    \begin{quote}    
        The language described by $a$, for $a \in \Sigma$, is $\{a\}$ and we write 
        $L(a) = \{a\}$
    
        The language described by $\varepsilon$ is $\{\varepsilon\}$ and we write 
        $L(\varepsilon) = \{ \varepsilon\}$
    
        The language described by $\emptyset$ is $\{\}$ and we write
        $L(\emptyset) = \emptyset$.
    \end{quote}
    
    \item[] {\it Recursive steps of recursive definition}
    \begin{quote}
        When $R_1$, $R_2$ are regular expressions, the language described by the regular
        expression $(R_1 \cup R_2)$ is the union of the languages described by $R_1$ and $R_2$, 
        and we write 
        $$L(~(R_1 \cup R_2)~) = L(R_1) \cup L(R_2) = \{ w \mid w \in L(R_1) \lor w \in L(R_2)\}$$
    
        When $R_1$, $R_2$ are regular expressions, the language described by the regular
        expression $(R_1 \circ R_2)$ is the concatenation of the languages described by $R_1$ and $R_2$, 
        and we write 
        $$L(~(R_1 \circ R_2)~) = L(R_1) \circ L(R_2) = \{ uv \mid u \in L(R_1) \land v \in L(R_2)\}$$
    
        When $R_1$ is a regular expression, the language described by the regular 
        expression $(R_1^*)$ is the {\bf Kleene star} of the language described by $R_1$ and we write
        $$L(~(R_1^*)~) = (~L(R_1)~)^* = \{ w_1 \cdots w_k \mid k \geq 0 \textrm{ and each } w_i \in L(R_1)\}$$
    \end{quote}
\end{itemize}
  
\newpage
For the following examples assume the alphabet is $\Sigma_1 =  \{0,1\}$:
    
The language described by the regular expression $0$ is $L(0) = \{ 0 \}$

The language described by the regular expression $1$ is $L(1)  = \{ 1 \}$

The language described by the regular expression $\varepsilon$ is $L(\varepsilon) = \{ \varepsilon  \}$

The language described by the regular expression $\emptyset$ is $L(\emptyset) = \emptyset$

The language described by the regular expression $(\Sigma_1 \Sigma_1 \Sigma_1)^*$ 
is $L(~(\Sigma_1 \Sigma_1 \Sigma_1)^*~) = $

\vfill

The language described by the regular expression $1^* \circ 1$ is $L(1^* \circ 1) = $

\vfill

    
\newpage
\subsection*{Review: Week 1 Monday}
\begin{enumerate}
\item Please complete the beginning of the quarter survey \href{https://forms.gle/9AaEcwwN5EvcJ4qp9}{https://forms.gle/9AaEcwwN5EvcJ4qp9}
\item We want you to be familiar with class policies and procedures so you are ready to have a successful quarter. 
Please take a look at the class website https://cseweb.ucsd.edu/classes/sp22/cse105-a/
and answer the questions about it on Gradescope.
\end{enumerate}

{\bf Pre class reading for next time}: Figure 1.4, Definition 1.5

\newpage
\subsection*{Week 1 Wednesday}

\begin{center}
    \begin{tabular}{|ll|}
    \hline
    Alphabet e.g. $\Sigma$, $\Gamma$ & 	non-empty finite set	 \\
    Symbol over $\Sigma$  & element of alphabet $\Sigma$\\
    String over $\Sigma$  &	finite list of symbols from $\Sigma$\\
    Language over $\Sigma$& set of strings over $\Sigma$ \\
    Empty set $\emptyset$ & the empty language\\
    Regular expression over $\Sigma$ e.g. $R$& syntactic expression built up recursively \\
    Language described by $R$,  $L(R)$ & set of strings matching pattern given by 
    regular expression\\
    \hline
    {\it Pages 3, 4, 13, 14, 64, 65}& \\
    \hline
    \end{tabular}
    \end{center}
    
    
    For the following True/False questions assume the alphabet is $\Sigma =  \{a,b,c\}$:
    
    \begin{center}
    \begin{tabular}{lcc}
    $a  \in L(a \cup b \cup c)$ & True & False \\
    $ab  \in L(~ (a \cup b)^*  ~)$ & True & False \\
    $ba \in L( ~ a^* b^* ~)$ & True & False \\
    $\varepsilon  \in L(a \cup b \cup c)$ & True & False \\
    $\varepsilon  \in L(~ (a \cup b)^*  ~)$ & True & False \\
    $\varepsilon \in L( ~ a^* b^* ~)$ & True & False \\
    \end{tabular}
    \end{center}
    
    
    
    \begin{center}
    \begin{tabular}{|ll|}
    \hline
    & \\
    Deterministic finite automaton & $M = (Q, \Sigma, \delta, q_0, F)$ \\
    Finite set of states $Q$  & Can  be labelled by any collection  of distinct names. Default: $q0, q1, \ldots$  \\
    Alphabet $\Sigma$ &   Each input to the automaton is a string over  $\Sigma$. \\
    Transition function $\delta$ &  Gives the next state based on current state of machine next input symbol\\
    Start state $q_0$ & Element of $Q$.  Each computation of the machine starts at the  start  state.\\
    Accept (final) states $F$ & $F \subseteq  Q$. Used to flag if the machine accepts or rejects
    an input string.\\
    Computation & The computation of a machine on an input string is a sequence of states \\
    &  in the machine,  starting with the initial state, determined by transitions \\
    & of the machine as it reads successive input symbols.
    \\
    $M$ accepts the input string & The computation of $M$ on the input string ends in an
    accept state.\\
    $M$ rejects the input string & The computation of $M$ on the input string ends in a
    nonaccept state.\\
    Language of $M$, $L(M)$ & The set of  all strings that  are each accepted by the machine $M$.\\
    aka language recognized by $M$ & \\
    & \\
    \hline
    {\it Pages 34-36}& \\
    \hline
    \end{tabular}
    \end{center}
    
    
    What is {\bf finite} about a deterministic finite automaton? (Select all that apply)
    \begin{itemize}
    \item The size of the machine (number of states, number of arrows)
    \item The number of strings that are accepted by the machine
    \item The length of the computations of the machine
    \end{itemize}
    
    
    
    
    \begin{figure}[h]
       \centering
       \includegraphics[width=3in]{../../resources/machines/Lect2DFA1.png} 
    \end{figure}
    
    The formal definition of this DFA is
    
    \vspace{100pt}
    
    
    \begin{center}
    \begin{tabular}{lcc}
    {\bf Input string} & {\bf Result}: this string is \ldots    & \\
    \hline
    $a$ & accepted by the DFA & rejected by the DFA \\
    $aa$ & accepted by the DFA & rejected by the DFA \\
    $ab$ & accepted by the DFA & rejected by the DFA \\
    $ba$ & accepted by the DFA & rejected by the DFA \\
    $bb$ & accepted by the DFA & rejected by the DFA \\
    $\varepsilon$ & accepted by the DFA & rejected by the DFA \\
    \end{tabular}
    \end{center}
    
    The language recognized by this DFA is
    
    
    \begin{figure}[h]
       \centering
       \includegraphics[width=3in]{../../resources/machines/Lect2DFA2.png} 
    \end{figure}
    
    
    
    The language recognized by this DFA is
    
    
    
    \begin{figure}[h]
       \centering
       \includegraphics[width=3in]{../../resources/machines/Lect2DFA3.png} 
    \end{figure}
    
    
    The language recognized by this DFA is
    
    

\end{document}