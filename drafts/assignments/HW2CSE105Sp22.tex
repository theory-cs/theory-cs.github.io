\documentclass[12pt, oneside]{article}

\usepackage[letterpaper, scale=0.8, centering]{geometry}
\usepackage{fancyhdr}
\setlength{\parindent}{0em}
\setlength{\parskip}{1em}

\pagestyle{fancy}
\fancyhf{}
\renewcommand{\headrulewidth}{0pt}
\rfoot{{\footnotesize Copyright Mia Minnes, 2021, Version \today~(\thepage)}}

\author{CSE20F21}

\newcommand{\instructions}{{\bf For all HW assignments:}

Weekly homework may be done individually or in groups of up to 3 students. 
You may switch HW partners for different HW assignments. 
The lowest HW score will not be included in your overall HW average. 
Please ensure your name(s) and PID(s) are clearly visible on the first page of your homework submission.

All submitted homework for this class must be typed. 
Diagrams may be hand-drawn and scanned and included in the typed document. 
You can use a word processing editor if you like (Microsoft Word, Open Office, Notepad, Vim, Google Docs, etc.) 
but you might find it useful to take this opportunity to learn LaTeX. 
LaTeX is a markup language used widely in computer science and mathematics. 
The homework assignments are typed using LaTeX and you can use the source files 
as templates for typesetting your solutions\footnote{To use this template, copy the source file (extension \texttt{.tex}) 
to your working directory or upload to Overleaf.}.


{\bf Integrity reminders}
\begin{itemize}
\item Problems should be solved together, not divided up between the partners. The homework is
designed to give you practice with the main concepts and techniques of the course, 
while getting to know and learn from your classmates.
\item You may not collaborate on homework with anyone other than your group members.
You may ask questions about the homework in office hours (of the instructor, TAs, and/or tutors) and 
on Piazza (as private notes viewable only to the Instructors).  
You \emph{cannot} use any online resources about the course content other than the class material 
from this quarter -- this is primarily to ensure that we all use consistent notation and
definitions we will use this quarter.
\item Do not share written solutions or partial solutions for homework with 
other students in the class who are not in your group. Doing so would dilute their learning 
experience and detract from their success in the class.
\end{itemize}

}
\usepackage{amssymb,amsmath,pifont,amsfonts,comment,enumerate,enumitem}
\usepackage{currfile,xstring,hyperref,tabularx,graphicx,wasysym}
\usepackage[labelformat=empty]{caption}
\usepackage{xcolor}
\usepackage{multicol,multirow,array,listings,tabularx,lastpage,textcomp,booktabs}

% NOTE(joe): This environment is credit @pnpo (https://tex.stackexchange.com/a/218450)
\lstnewenvironment{algorithm}[1][] %defines the algorithm listing environment
{   
    \lstset{ %this is the stype
        mathescape=true,
        frame=tB,
        numbers=left, 
        numberstyle=\tiny,
        basicstyle=\rmfamily\scriptsize, 
        keywordstyle=\color{black}\bfseries,
        keywords={,procedure, div, for, to, input, output, return, datatype, function, in, if, else, foreach, while, begin, end, }
        numbers=left,
        xleftmargin=.04\textwidth,
        #1
    }
}
{}

\newcommand\abs[1]{\lvert~#1~\rvert}
\newcommand{\st}{\mid}

\newcommand{\cmark}{\ding{51}}
\newcommand{\xmark}{\ding{55}}




\title{HW2 : Regular Languages and Automata Constructions}
\date{Due: on Gradescope}

\begin{document}
\maketitle
\thispagestyle{fancy}


{\bf In this assignment,}

You will practice reading and
applying definitions to get comfortable working with mathematical language. As
a result, you can expect to spend more time reading the questions and looking
up notation than doing calculations.

\instructions

You will submit this assignment via Gradescope
(\href{https://www.gradescope.com}{https://www.gradescope.com}) 
in the assignment called ``hw1-definitions-and-notation''.


{\bf Resources}: To review the topics you are working with 
for this assignment, see the class material from  Week 0 and 1.
We will post frequently asked questions and our answers to them in a 
pinned Piazza post.

{\bf Assigned questions}



{\bf Reading and extra practice problems}: Sipser Section 1.1, 1.2, 1.3 ( Section 0 for the prerequisites).
Chapter 1 exercises 1.1, 1.2, 1.3, 1.4, 1.5, 1.6, 1.7, 1.8, 1.9, 1.10, 1.11, 1.12, 1.14, 1.15, 1.16, 1.17, 1.18, 1.19, 1.20, 1.21, 1.22, 1.23.

{\bf Key Concepts}: Strings, languages, Kleene star, regular expressions, language described by a regular expression, deterministic finite automata (DFAs), regular languages, closure of the class of regular languages under certain operations, nondeterministic finite automata (NFA), equivalence of NFA and DFA and regular expressions.


\begin{enumerate}

\item  In video \texttt{019.CSE105Sp20.FiniteLanguagesAreRegular.mp4}, a general 
argument is sketched to build a DFA for an arbitrary finite language. In this question, 
you'll explore representations of finite languages further.
\begin{enumerate}
\item (3 points) Pick an alphabet (a nonempty finite set), and specify it, e.g.\ by filling in the blank 
$\Sigma =  \underline{\text{fill in your alphabet here}}$.

Then, pick a language of cardinality (size) $5$ over this alphabet, and specify it, e.g.\ by filling in the blank
\[
L =  \underline{\text{fill in your language here}}
\]
{\it Note: we encourage you to pay attention to syntax here.  There are many correct answers; please be 
precise in how you present the sets you choose.}

\item (10 points) Give a regular expression that describes the language $L$ you defined in part (a).  Briefly justify why your regular expression
works.
\item (10 points) Give a DFA that recognizes your language $L$ you defined in part (a).  Specify your DFA {\bf both} using a formal definition
{\bf and} a state diagram. Briefly justify why your DFA works.
\item (10 points) ({\it Graded for fair effort completeness}) Describe how you would generalize your DFA construction
for languages over $\Sigma$ with other {\bf finite} cardinalities. 
Include a description of how the number of states
in the machine might depend on the cardinality of $L$. Does this depend on the size of the alphabet you 
chose?
\end{enumerate}

\item To safeguard the privacy or security of a network, some software filters
the IP addresses that are allowed to send content to computers on the network. Each IP address
can be broken into parts that represent the source host of incoming traffic, including geographic data.
As a result, software needs to be designed to recognize whether certain substrings (representing
permitted hosts) are present (if the hosts are permitted to send data) and whether others
are absent (if those hosts are blocked from sending data).

In this question, you'll design ways to detect these patterns in strings and analyze their costs.

\begin{enumerate}
\item (10 points) Over the alphabet $\{0,1,2,3,4,5,6,7,8,9\}$ design a NFA that accepts each and only strings
 that have $427$, $953$, or $259$ as a substring. Your NFA should have {\bf no more than $8$ states}.
 Include the state diagram of your NFA. Briefly justify 
 your construction by explaining the role each state plays in the machine. {\it Note: you may 
 include the formal definition of your NFA, but this is not required.} In the context of network
security, this NFA would only allow traffic from IPs that have prior approval.

\item (10 points) Give a regular expression that describes the set of strings from part (a). Briefly justify
why your regular expression works.

\item (10 points)\footnote{This part of the question was revised (to be simpler) on April 11, 2020.} Next, suppose the network administrators want to block traffic from IP addresses
that have been associated with spammers. Over the alphabet~$\{0,1,2,3,4,5,6,7,8,9\}$, 
design a NFA that accepts each and only strings
 that do not have the substring $427$ and do not have the substring $953$.  Your NFA should have 
 {\bf no more than $7$ states}.  Briefly justify 
 your construction by explaining the role each state plays in the machine or by explaining the process of 
 obtaining this machine using the general constructions discussed in class and in the textbook.

{\it Bonus - not for credit; do not hand in}: extend your design so that you have a (new) NFA that recognizes
the complement of the language you considered in part (a).

\item (10 points) Give a regular expression that describes the set of strings over the alphabet $\{0,1,2,3,4,5,6,7,8,9\}$ 
that do not have the substring $427$. Briefly justify why your regular expression works.

{\it Bonus - not for credit; do not hand in}: extend your design so that you have a (new) regular expression
that describes
the complement of the language you considered in part (a).


\item (10 points) ({\it Graded for fair effort completeness}) 
One way to implement the designs from the previous parts is to convert each NFA to 
an equivalent DFA and then store the DFA's transition function as a look-up table and the 
set of accept states in an appropriate data structure.  To process an incoming 
IP address, the look up table would give the sequence of states visited by the computation 
of the DFA on the input IP address, and last state in the computation would be tested for membership
in the set of accept states. We would like to measure the
computational costs of this IP filtering operation for the two type of filters we discussed: allowing
all IP addresses from certain hosts (with specified substrings) and blocking all IP addresses from certain
hosts (avoiding specified substrings). We will measure computational costs
in terms of computation time and memory storage.  
\begin{enumerate}
\item Do you expect the computation time of the DFAs that are obtained 
from the NFAs for  part (a) and part (c) to be the same or 
different for an arbitrary input string?  Briefly explain your reasoning.
\item How do the memory storage needs of the DFAs obtained in part (a) and part (c) 
compare? Briefly explain your reasoning.
\end{enumerate}
\end{enumerate}


\item In this question, you'll practice working with formal general constructions
for DFAs and translating between state diagrams and formal definitions.
Consider the following
construction in the 
textbook for Chapter 1 Problem 34, which we include here for reference: 
``Let $B$ and $C$ be languages over $\Sigma = \{ 0,1\}$. Define
\[
B \overset{1}{\leftarrow} C= \{ w \in B ~\mid~\textrm{ for some $y \in C$, strings $w$ and $y$ contain equal 
numbers of $1$s }\}
\]
The class of regular languages is shown to be closed under the $\overset{1}{\leftarrow}$ operation
using the construction: Let $M_B = (Q_B, \Sigma, \delta_B, q_B, F_B)$ and $M_C = ( Q_C, \Sigma, \delta_C, q_C, F_C)$ be DFAs recognizing the languages $B$ and $C$, respectively.  We will now construct NFA
$M = (Q, \Sigma, \delta, q_0, F)$ that recognizes $B \overset{1}{\leftarrow} C$ as follows.  To decide
whether its input $w$ is in $B \overset{1}{\leftarrow} C$, the machine $M$ checks that $w \in B$, and 
in parallel nondeterministically guesses a string $y$ that contains the same number of $1$s as 
contained in $w$ and checks that $y \in C$.
\begin{itemize}
\item[{\bf 1.}] $Q = Q_B \times Q_C$
\item[{\bf 2.}] For $(q,r) \in Q$ and $a \in \Sigma_\varepsilon$, define
\[
\delta( ~((q,r), a)~) = \begin{cases}
\{ (\delta_B(q,0) , r ) \}  \qquad&\textrm{if } a = 0 \\
\{ (\delta_B( q,1) ,  \delta_C( r,1) ) \}  \qquad&\textrm{if } a = 1 \\
\{ (q, \delta_C( r,0 ))\}  \qquad&\textrm{if } a = \varepsilon\\
\end{cases}
\]
\item[{\bf 3.}] $q_0 = (q_B, q_C)$
\item[{\bf 4.}] $F = F_B \times F_C$~~~."
\end{itemize}

\begin{enumerate}
\item (10 points) Illustrate this construction by defining specific example DFAs $M_B$ and $M_C$ and including 
their state diagrams in your submission.   Choose $M_B$ to have two states and $M_C$
to have three states, and make sure that every state in each state diagram is reachable from the start state
of that machine.
Apply the construction above to create the NFA $M$  and include its state diagram in your submission.
{\it Note: you may 
 include the formal definition of your DFAs and NFA, but this is not required.} 

{\it Hint: Confirm that you have specified every required piece of the state diagram for $M$. E.g., 
label the states consistently with the construction, indicate the start arrow, specify each
accepting state, and include all required transitions.}

\item (10 points) ({\it Graded for fair effort completeness}) Describe the sets recognized by each of the machines you used in part (a): $M_B, M_C, M$.
If possible, give an example of a string that is in $B$ and in $B \overset{1}{\leftarrow} C$
and an example of a string that is in $B$ and not in $B \overset{1}{\leftarrow} C$. If any of these examples
do not exist, explain why not.
\end{enumerate}

\end{enumerate}
\end{document}