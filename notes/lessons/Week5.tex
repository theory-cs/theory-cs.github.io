\input{../../resources/lesson-head.tex}

\subsection*{Week 5 at a glance}

\subsubsection*{Textbook reading: Section 2.2, 2.1.}

\vspace{-20pt}

Before Monday, read Theorem 2.20.

Before Wednesday, read Example 2.18 (page 114).

Before Friday, read Figure 3.1.

For Week 6 Monday: Page 165-166 Introduction to Section 3.1.

\vspace{-20pt}

\subsubsection*{We will be learning and practicing to:}
\vspace{-20pt}

\begin{itemize}
    \item Clearly and unambiguously communicate computational ideas using appropriate formalism. Translate across levels of abstraction.
    \begin{itemize}
        \item Describe and use models of computation that don't involve state machines.
        \begin{itemize}
            \item {\bf Use context-free grammars and relate them to languages and pushdown automata.}
        \end{itemize}
        \item Use precise notation to formally define the state diagram of a Turing machine
        \item Use clear English to describe computations of Turing machines informally.
        \begin{itemize}
            \item {\bf Design a PDA that recognizes a given language.}
         \end{itemize}
       \item Give examples of sets that are context-free (and prove that they are).
       \begin{itemize}
          \item {\bf State the definition of the class of context-free languages}
          \item {\bf Explain the limits of the class of context-free languages}
          \item {\bf Identify some context-free sets and some non-context-free sets}
       \end{itemize}
    \end{itemize}
    \item Know, select and apply appropriate computing knowledge and problem-solving techniques. 
    Reason about computation and systems.
    \begin{itemize}
        \item Describe and closure properties of classes of languages under certain operations.
        \begin{itemize}
            \item {\bf Apply a general construction to create a new PDA or CFG from an example one.}
            \item {\bf Formalize a general construction from an informal description of it.}
            \item {\bf Use general constructions to prove closure properties of the class of context-free langugages.}
            \item {\bf Use counterexamples to prove non-closure properties of the class of context-free langugages.}
        \end{itemize}
    \end{itemize}
\end{itemize}

\vspace{-20pt}

\subsubsection*{TODO:}
\begin{list}{\itemsep-10pt}
   \item Schedule your Test 1 Attempt 1, Test 2 Attempt 1, Test 1 Attempt 2, and Test 2 Attempt 2 times 
   at PrairieTest (http://us.prairietest.com)
   \item Review Quiz 5 on PrairieLearn (http://us.prairielearn.com), complete by Sunday 11/4/2024
\end{list}

\newpage

\section*{Monday: Context-free languages}

%! app: context-free languages
%! outcome: Classify language, Find example languages, context-free grammars

These definitions are on pages 101-102.

\vspace{-20pt}

\begin{center}
    \begin{tabular}{|p{2.4in}cp{3.6in}|}
    \hline 
    {\bf Term} & {\bf Typical symbol} & {\bf Meaning} \\
     & or {\bf Notation} & \\
    \hline
    \hline
    {\bf Context-free grammar} (CFG) & $G$ & $G = (V, \Sigma, R, S)$ \\
    The set of {\bf variables}& $V$ & Finite  set of symbols that represent phases in production pattern\\
    The set of {\bf terminals} & $\Sigma$ & Alphabet of symbols of strings generated  by CFG \\
    & & $V \cap \Sigma = \emptyset$ \\
    The set of {\bf rules}& $R$ & Each rule is  $A \to u$ with $A \in V$ and $u  \in (V  \cup \Sigma)^*$\\
    The {\bf start} variable&  $S$  & Usually  on left-hand-side of first/ topmost rule \\
    & &\\
    {\bf Derivation} & $S \Rightarrow \cdots \Rightarrow w$& 
    Sequence  of substitutions in a  CFG (also written $S \Rightarrow^* w$). At each step, we can apply one rule 
    to one occurrence of a variable in the current string by substituting that occurrence of the variable with the 
    left-hand-side of the rule. The derivation must end when the current string has only terminals (no variables)
    because then there are no instances of variables to apply a rule to.\\
    Language {\bf generated} by the context-free grammar $G$ & $L(G)$ &The set of strings for which there is a derivation in $G$. 
    Symbolically: $\{  w \in \Sigma^* \mid S \Rightarrow^* w \}$ i.e. $$\{  w \in \Sigma^* \mid \text{there is  derivation in $G$ that ends
    in $w$} \}$$\\
    {\bf Context-free language} & & A language that is the language generated by some context-free grammar\\
    \hline
    \end{tabular}
\end{center}

\vfill

  
{\bf Examples of context-free grammars, derivations in those grammars, and the languages generated by those grammars}
  
$G_1 =  (\{S\}, \{0\}, R, S)$ with rules
  \begin{align*}
    &S \to 0S\\
    &S \to 0\\
  \end{align*}
  In  $L(G_1)$ \ldots 
  
\vfill

  Not in $L(G_1)$ \ldots 


  \vfill

\newpage
  $G_2 =  (\{S\}, \{0,1\}, R, S)$
  \[
  S \to 0S \mid 1S \mid \varepsilon
  \]
  In  $L(G_2)$ \ldots 
  
  \vspace{110pt}
  
  Not in $L(G_2)$ \ldots 

  \vspace{110pt}

  $(\{S, T\}, \{0, 1\}, R, S)$ with  rules
  \begin{align*}
  &S \to T1T1T1T \\
  &T \to  0T \mid 1T \mid \varepsilon
  \end{align*}

  In  $L(G_3)$ \ldots 
  
  \vspace{110pt}
  
  Not in $L(G_3)$ \ldots 

  \vspace{110pt}

\newpage
  $G_4 =  (\{A, B\}, \{0, 1\}, R, A)$ with rules
  \[
    A \to 0A0 \mid  0A1 \mid 1A0  \mid 1A1 \mid  1
  \]
  In  $L(G_4)$ \ldots 
  
  \vspace{110pt}
  
  Not in $L(G_4)$ \ldots 

  \vspace{110pt}


  \begin{comment}
    I moved the following to the review quiz.

    {\it Extra practice}: Is there a CFG $G$ with $L(G) = \emptyset$?


  Three different CFGs that each generate the  language $\{abba\}$
  
  \begin{align*}
  & ( \{ S, T, V, W\}, \{a,b\}, \{ S \to aT, T \to bV, V \to bW, W \to a\}, S)\\
  & \\ 
  & \\ 
  & \\ 
  & ( \{ Q \}, \{a,b\}, \{Q \to abba\}, Q) \\
  & \\ 
  & \\ 
  & \\
  & ( \{ X,Y \}, \{a,b\}, \{X \to aYa, Y \to bb\}, X) 
  & \\ 
  & \\ 
  \end{align*} 
\end{comment}
  \newpage
  Design a CFG to generate the  language $\{a^n b^n \mid  n  \geq  0\}$
  
  \vspace{100pt}
  
  {\it Sample derivation:} 
  
  \vspace{100pt}
  
  
  \vfill 


\newpage

    
\newpage
\subsection*{Wednesday: Context-free and non-context-free languages}

%! app: Regular Languages, Context-free Languages
%! outcome: Formal definition of automata, Informal definition of automata, Classify language, Find example languages




\newpage
\subsection*{Friday: Turing machines}

%! app: Decidable Languages, Undecidable Languages
%! outcome: Formal definition of automata, Informal definition of automata, Classify language, Find example languages

We are ready to introduce a formal model that will capture a notion of general purpose computation.
\begin{itemize}
\item {\it Similar to DFA, NFA, PDA}: input will be an arbitrary string over a fixed alphabet.
\item {\it Different from NFA, PDA}: machine is deterministic.
\item {\it Different from DFA, NFA, PDA}: read-write head can move both to the left and to the right,
and can extend to the right past the original input.
\item {\it Similar to DFA, NFA, PDA}: transition function drives computation one step at a time 
by moving within a finite set of states, always starting at designated start state.
\item {\it Different from DFA, NFA, PDA}: the special states for rejecting and accepting take effect immediately.
\end{itemize}

\vspace{-10pt}

(See more details: Sipser p. 166)

\vfill

Formally: a  Turing machine is $M= (Q, \Sigma, \Gamma, \delta, q_0, q_{accept}, q_{reject})$ 
where $\delta$ is the {\bf transition function} 
\[
  \delta: Q\times \Gamma \to Q \times \Gamma \times \{L, R\}
\]
The {\bf computation} of $M$ on a string $w$ over $\Sigma$  is:

\vspace{-10pt}

\begin{itemize}
\setlength{\itemsep}{0pt}
\item Read/write head starts at leftmost position on tape. 
\item Input string is written on $|w|$-many leftmost cells of tape, 
rest of  the tape cells have  the blank symbol. {\bf Tape alphabet} 
is $\Gamma$ with $\textvisiblespace\in \Gamma$ and $\Sigma \subseteq \Gamma$.
The blank symbol $\textvisiblespace \notin \Sigma$.
\item Given current state of machine and current symbol being read at the tape head, 
the machine transitions to next state, writes a symbol to the current position  of the 
tape  head (overwriting existing symbol), and moves the tape head L or R (if possible). 
\item Computation ends {\bf if and when} machine enters either the accept or the reject state.
This is called {\bf halting}.
Note: $q_{accept} \neq q_{reject}$.
\end{itemize}

The {\bf language recognized by the  Turing machine} $M$,  is  $L(M) = \{ w \in \Sigma^* \mid w \textrm{ is accepted by } M\}$,
which is defined as
\[
  \{ w \in \Sigma^* \mid \textrm{computation of $M$ on $w$ halts after entering the accept state}\}
\]




\newpage
\begin{multicols}{2}
\begin{tikzpicture}[->,>=stealth',shorten >=1pt, auto, node distance=2cm, semithick]
  \tikzstyle{every state}=[text=black, fill=none]
  
  \node[initial,state] (q0)          {$q0$};
  \node[state]         (q1) [right of=q0, xshift=40pt] {$q1$};
  \node[state,accepting]         (qacc) [above right of=q0, yshift=20pt] {$q_{acc}$};
  \node[state]         (qrej) [above right of=q1,yshift=20pt] {$q_{rej}$};
  
  \path (q0) edge [bend left=0] node {$\square; \square, R$} (qacc)
      (q0) edge  [bend left=20] node {$0; \square, R$} (q1)
      (q1) edge [bend left=20] node {$0; \square, R$} (q0)
      (q1) edge [bend left=0] node {$\square; \square, R$} (qrej)
      (qacc) edge  [loop above] node {\parbox{1cm}{$0; \square, R$\newline $\square; \square, R$}} (qacc)
      (qrej) edge  [loop above] node {\parbox{1cm}{$0; \square, R$\newline $\square; \square, R$}}  (qrej)
  ;
\end{tikzpicture}
\columnbreak
Formal definition:

\vspace{10pt}

Sample computation: 

\begin{tabular}{|c|c|c|c|c|c|c|}
\hline
\multicolumn{1}{|c}{$q0\downarrow$} &  \multicolumn{6}{c|}{\phantom{A}}\\
\hline
$0$ & $0$  & $0$ & $\textvisiblespace $& $\textvisiblespace $& $\textvisiblespace $&  $\textvisiblespace $\\
\hline
\multicolumn{7}{|c|}{\phantom{A}}\\
\hline
\phantom{AA} & \phantom{AA}& \phantom{AA}& \phantom{AA}& \phantom{AA}& \phantom{AA}& \phantom{AA} \\
\hline
\multicolumn{7}{|c|}{\phantom{A}}\\
\hline
\phantom{AA} & \phantom{AA}& \phantom{AA}& \phantom{AA}& \phantom{AA}& \phantom{AA}& \phantom{AA} \\
\hline
\multicolumn{7}{|c|}{\phantom{A}}\\
\hline
\phantom{AA} & \phantom{AA}& \phantom{AA}& \phantom{AA}& \phantom{AA}& \phantom{AA}& \phantom{AA} \\
\hline
\multicolumn{7}{|c|}{\phantom{A}}\\
\hline
\phantom{AA} & \phantom{AA}& \phantom{AA}& \phantom{AA}& \phantom{AA}& \phantom{AA}& \phantom{AA} \\
\hline
\end{tabular}
\end{multicols}
\vfill

The language recognized by this machine is \ldots

\vfill
 

{\bf Describing  Turing machines} (Sipser p. 185) To define a Turing machine, we could give a 
\begin{itemize}
\item {\bf Formal definition}: the $7$-tuple of parameters including set of states, 
input alphabet, tape alphabet, transition function, start state, accept state, and reject state; or,
\item {\bf Implementation-level definition}: English prose that describes the Turing machine head 
movements relative to contents of tape, and conditions for accepting / rejecting based on those contents.
\item {\bf High-level description}: description of algorithm (precise sequence of instructions), 
without implementation details of machine. As part of this description, can ``call" and run 
another TM as a subroutine.
\end{itemize}
  
\newpage
Fix $\Sigma = \{0,1\}$, $\Gamma = \{ 0, 1, \textvisiblespace\}$ for the Turing machines with  the following state diagrams:
  
\begin{center}
  \begin{tikzpicture}[->,>=stealth',shorten >=1pt, auto, node distance=2cm, semithick]
    \tikzstyle{every state}=[text=black, fill=none]
    
    \node[initial,state] (q0)          {$q0$};
    \node[state,accepting]         (qacc) [right of=q0, xshift=20pt] {$q_{acc}$};
    
    \path (q0) edge  [loop above] node {\parbox{1cm}{$\square; \square, R$}} (q0)
    ;
  \end{tikzpicture}
\end{center}

Example of string accepted: \\
Example of string rejected: \\


Implementation-level description

\vfill

High-level description

\vfill

\begin{center}
  \begin{tikzpicture}[->,>=stealth',shorten >=1pt, auto, node distance=2cm, semithick]
    \tikzstyle{every state}=[text=black, fill=none]
    
    \node[initial,state] (qrej)          {$q_{rej}$};
    \node[state,accepting]         (qacc) [right of=q0, xshift=20pt] {$q_{acc}$};
  \end{tikzpicture}
\end{center}

Example of string accepted: \\
Example of string rejected: \\


Implementation-level description

\vfill

High-level description

\vfill

\newpage
\begin{center}
  \begin{tikzpicture}[->,>=stealth',shorten >=1pt, auto, node distance=2cm, semithick]
    \tikzstyle{every state}=[text=black, fill=none]
    
    \node[initial,state] (q0)          {$q0$};
    \node[state,accepting]         (qacc) [right of=q0, xshift=20pt] {$q_{acc}$};
    
    \path (q0) edge  [bend left=0] node {\parbox{1cm}{$\square; \square, R$}} (qacc)
    ;
  \end{tikzpicture}
\end{center}

Example of string accepted: \\
Example of string rejected: \\


Implementation-level description

\vfill

High-level description

\vfill

\begin{center}
  \begin{tikzpicture}[->,>=stealth',shorten >=1pt, auto, node distance=2cm, semithick]
    \tikzstyle{every state}=[text=black, fill=none]
    
    \node[initial,state] (q0)          {$q0$};
    \node[state,accepting]         (qacc) [right of=q0, xshift=20pt] {$q_{acc}$};
    
    \path (q0) edge  [loop above] node {\parbox{1cm}{$1; \square, R$\\$0; \square, R$\\$\square; \square, R$}} (q0)
    ;
  \end{tikzpicture}
\end{center}

Example of string accepted: \\
Example of string rejected: \\


Implementation-level description

\vfill

High-level description

\vfill

\newpage



\end{document}
