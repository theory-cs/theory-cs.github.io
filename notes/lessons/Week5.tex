\documentclass[12pt, oneside]{article}

\usepackage[letterpaper, scale=0.89, centering]{geometry}
\usepackage{fancyhdr}
\setlength{\parindent}{0em}
\setlength{\parskip}{1em}

\pagestyle{fancy}
\fancyhf{}
\rfoot{\href{https://creativecommons.org/licenses/by-nc-sa/2.0/}{CC BY-NC-SA 2.0} Version \today~(\thepage.)}

\input{../../resources/CSE20packages}

\begin{document}
\begin{flushright}
    \StrBefore{\currfilename}{.}
\end{flushright}

\subsection*{Week 5 at a glance}

\subsubsection*{Textbook reading: Chapter 2}

\vspace{-20pt}

Before Monday, read Introduction to Section 2.1 (pages 101-102).

Before Wednesday, read Section 2.1

Before Friday, read Theorem 2.20.

For Week 6 Monday: Page 165-166 Introduction to Section 3.1.

\vspace{-20pt}

\subsubsection*{We will be learning and practicing to:}
\vspace{-20pt}

\begin{itemize}
    \item Clearly and unambiguously communicate computational ideas using appropriate formalism. Translate across levels of abstraction.
    \begin{itemize}
        \item Describe and use models of computation that don't involve state machines.
        \begin{itemize}
            \item {\bf Identify the components of a formal definition of a context-free grammar (CFG)}
            \item {\bf Derive strings in the language of a given CFG}
            \item {\bf Determine the language of a given CFG}
            \item {\bf Design a CFG generating a given language}
            \item {\bf Use context-free grammars and relate them to languages and pushdown automata.}
        \end{itemize}
        \item Use precise notation to formally define the state diagram of a Turing machine
        \item Use clear English to describe computations of Turing machines informally.
        \begin{itemize}
            \item {\bf Design a PDA that recognizes a given language.}
         \end{itemize}
       \item Give examples of sets that are context-free (and prove that they are).
       \begin{itemize}
          \item {\bf State the definition of the class of context-free languages}
          \item {\bf Explain the limits of the class of context-free languages}
          \item {\bf Identify some context-free sets and some non-context-free sets}
       \end{itemize}
    \end{itemize}
    \item Know, select and apply appropriate computing knowledge and problem-solving techniques. 
    Reason about computation and systems.
    \begin{itemize}
        \item Describe and prove closure properties of classes of languages under certain operations.
        \begin{itemize}
            \item {\bf Apply a general construction to create a new PDA or CFG from an example one.}
            \item {\bf Formalize a general construction from an informal description of it.}
            \item {\bf Use general constructions to prove closure properties of the class of context-free langugages.}
            \item {\bf Use counterexamples to prove non-closure properties of the class of context-free langugages.}
        \end{itemize}
    \end{itemize}
\end{itemize}

\vspace{-20pt}

\subsubsection*{TODO:}
\begin{list}{\itemsep-10pt}
   \item Schedule your Test 1 Attempt 1, Test 2 Attempt 1, Test 1 Attempt 2, and Test 2 Attempt 2 times
   at PrairieTest (http://us.prairietest.com) . The first Test 1 sessions are next week!
   \item Review Quiz 4 on PrairieLearn (http://us.prairielearn.com), due 2/5/2025
   \item Homework 3 submitted via Gradescope (https://www.gradescope.com/), due 2/6/2025
   \item Review Quiz 5 on PrairieLearn (http://us.prairielearn.com), due 2/12/2025
\end{list}

\newpage



\newpage
\subsection*{Monday: More Pushdown Automata}

%! app: Regular Languages, Context-free Languages
%! outcome: Formal definition of automata, Informal definition of automata, Nondeterminism, Classify language, Find example languages
    

{\bf Definition} A {\bf pushdown automaton} (PDA) is  specified by a  $6$-tuple $(Q, \Sigma, \Gamma, \delta, q_0, F)$
where $Q$ is the finite set of states, $\Sigma$ is the input alphabet,  $\Gamma$ is the stack alphabet,
\[
    \delta: Q \times \Sigma_\varepsilon  \times  \Gamma_\varepsilon \to \mathcal{P}( Q \times \Gamma_\varepsilon)
\]
is the transition function,  $q_0 \in Q$ is the start state, $F \subseteq  Q$ is the set of accept states.


%Draw the state diagram and give the formal definition of a PDA with $\Sigma = \Gamma$.

%\vfill

%Draw the state diagram and give the formal definition of a PDA with $\Sigma \cap \Gamma = \emptyset$.
    
%\vfill

%\newpage
For the PDA state diagrams below, $\Sigma = \{0,1\}$.

\vspace{-15pt}

\begin{center}
\begin{tabular}{c c}
Mathematical description of language & State diagram of PDA recognizing language\\
\hline
& $\Gamma = \{ \$, \#\}$ \hspace{2.3in} \\
& \\
& 
\begin{tikzpicture}[->,>=stealth',shorten >=1pt, auto, node distance=2cm, semithick]
    \tikzstyle{every state}=[text=black, fill=none]
    
    \node[initial,state] (q0)          {$q0$};
    \node[state]         (q1) [right of=q0, xshift=20pt] {$q1$};
    \node[state]         (q2) [below right of=q1, xshift=20pt] {$q2$};
    \node[state]         (q3) [right of=q2, xshift=20pt] {$q3$};
    \node[state,accepting]         (q4) [left of=q2, xshift=-20pt] {$q4$};
    
    \path (q0) edge [bend left=0] node {$\varepsilon, \varepsilon; \$$} (q1)
        (q1) edge  [loop above] node {$0, \varepsilon; \#$} (q1)
        (q1) edge [bend left=0] node {$\varepsilon, \varepsilon; \varepsilon$} (q2)
        (q2) edge  [bend left=20] node [midway, above] {$1, \#; \varepsilon$} (q3)
        (q3) edge  [bend left=20] node [midway, below] {$1, \varepsilon; \varepsilon$} (q2)
        (q2) edge  [bend left=0] node {$\varepsilon, \$; \varepsilon$} (q4)
    ;
\end{tikzpicture}
\\
& \\
& \\
\hline
& $\Gamma = \{ \sun, 1\}$ \hspace{2.3in} \\
& \\
& 
\begin{tikzpicture}[->,>=stealth',shorten >=1pt, auto, node distance=2cm, semithick]
    \tikzstyle{every state}=[text=black, fill=none]
    
    \node[initial,state] (q0)          {$q0$};
    \node[state]         (q1) [right of=q0, xshift=20pt] {$q1$};
    \node[state]         (q2) [below right of=q1, xshift=20pt] {$q2$};
    \node[state]         (q3) [right of=q2, xshift=20pt] {$q3$};
    \node[state,accepting]  (q4) [right of=q3, xshift=20pt] {$q4$};
    \node[state]         (q5) [above right of=q1, xshift=20pt] {$q5$};
    \node[state,accepting]  (q6) [right of=q5, xshift=20pt] {$q6$};
    
    \path (q0) edge [bend left=0] node {$\varepsilon, \varepsilon; \sun$} (q1)
        (q1) edge  [loop above] node {$1, \varepsilon; 1$} (q1)
        (q1) edge [bend left=0] node [below] {$\varepsilon, \varepsilon; \varepsilon$} (q5)
        (q1) edge [bend left=0] node [above]{$\varepsilon, \varepsilon; \varepsilon$} (q2)
        (q5) edge [loop above] node {$0, 1, ; \varepsilon$} (q5)
        (q5) edge [bend left=0] node {$\varepsilon, \sun; \varepsilon$} (q6)
        (q6) edge [loop above] node {$1, \varepsilon; \varepsilon$} (q6)
        (q2) edge [loop below] node {$0, \varepsilon; \varepsilon$} (q2)
        (q2) edge  [bend left=0] node {$\varepsilon, \varepsilon; \varepsilon$} (q3)
        (q3) edge  [loop below] node {$1, 1; \varepsilon$} (q3)
        (q3) edge  [bend left=0] node {$\varepsilon, \sun; \varepsilon$} (q4)
    ;
\end{tikzpicture}
\\
& \\
& \\
\hline
& \\
& \\
& \\
$\{ 0^i 1^j 0^k \mid i,j,k \geq 0 \}$ & \\
& \\
& \\

\end{tabular}
\end{center}
{\it Note: alternate notation is to replace $;$ with $\to$ on arrow labels.}

\newpage
Corollary: for each language $L$ over $\Sigma$, if there is an NFA $N$ with $L(N)=L$
then there is a PDA $M$ with $L(M) = L$

Proof idea: Declare stack alphabet to be $\Gamma = \Sigma$ and then 
don't use stack at all. 


 \vfill

\begin{comment}
{\it Extra practice}: Consider the state diagram of a PDA with input alphabet 
$\Sigma$ and stack alphabet $\Gamma$.

\begin{center}
\begin{tabular}{|c|c|}
\hline
Label & means \\
\hline
$a, b ; c$ when $a \in \Sigma$, $b\in \Gamma$, $c \in \Gamma$ 
& \hspace{3in} \\
& \\
& \\
& \\
& \\
&\\
\hline
$a, \varepsilon ; c$ when $a \in \Sigma$, $c \in \Gamma$ 
& \hspace{3in} \\
& \\
& \\
& \\
& \\
&\\
\hline
$a, b ; \varepsilon$ when $a \in \Sigma$, $b\in \Gamma$
& \hspace{3in} \\
& \\
& \\
& \\
& \\
&\\
\hline
$a, \varepsilon ; \varepsilon$ when $a \in \Sigma$
& \hspace{3in} \\
& \\
& \\
& \\
& \\
&\\
\hline
\end{tabular}
\end{center}


How does the meaning change if $a$ is replaced by $\varepsilon$?
\end{comment}

{\it Big picture}: PDAs are motivated by wanting to add some memory of unbounded size to NFA. How 
do we accomplish a similar enhancement of regular expressions to get a syntactic model that is 
more expressive?

DFA, NFA, PDA: Machines process one input string at a time; the computation of a machine on its input string 
reads the input from left to right.

Regular expressions: Syntactic descriptions of all strings that match a particular pattern; the language 
described by a regular expression is built up recursively according to the expression's syntax

{\bf Context-free grammars}: Rules to produce one string at a time, adding characters from the middle, beginning, 
or end of the final string as the derivation proceeds.\\

\vfill


\newpage
\section*{Wednesday: Context-free Grammars and Languages}

\input{../activity-snippets/day13.tex}
    
\newpage
\subsection*{Friday: Context-free and non-context-free languages}

\input{../activity-snippets/day14.tex}



\end{document}
