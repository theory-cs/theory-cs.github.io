\documentclass[12pt, oneside]{article}

\usepackage[letterpaper, scale=0.89, centering]{geometry}
\usepackage{fancyhdr}
\setlength{\parindent}{0em}
\setlength{\parskip}{1em}

\pagestyle{fancy}
\fancyhf{}
\rfoot{\href{https://creativecommons.org/licenses/by-nc-sa/2.0/}{CC BY-NC-SA 2.0} Version \today~(\thepage.)}

\input{../../resources/CSE20packages}

\begin{document}
\begin{flushright}
    \StrBefore{\currfilename}{.}
\end{flushright}

\subsection*{Week 5 at a glance}

\subsubsection*{Textbook reading: Section 2.2, 2.1.}

\vspace{-20pt}

Before Monday, read Theorem 2.20.

Before Wednesday, read Example 2.18 (page 114).

Before Friday, read Figure 3.1.

For Week 6 Monday: Page 165-166 Introduction to Section 3.1.

\vspace{-20pt}

\subsubsection*{We will be learning and practicing to:}
\vspace{-20pt}

\begin{itemize}
    \item Clearly and unambiguously communicate computational ideas using appropriate formalism. Translate across levels of abstraction.
    \begin{itemize}
        \item Describe and use models of computation that don't involve state machines.
        \begin{itemize}
            \item {\bf Use context-free grammars and relate them to languages and pushdown automata.}
        \end{itemize}
        \item Use precise notation to formally define the state diagram of a Turing machine
        \item Use clear English to describe computations of Turing machines informally.
        \begin{itemize}
            \item {\bf Design a PDA that recognizes a given language.}
         \end{itemize}
       \item Give examples of sets that are context-free (and prove that they are).
       \begin{itemize}
          \item {\bf State the definition of the class of context-free languages}
          \item {\bf Explain the limits of the class of context-free languages}
          \item {\bf Identify some context-free sets and some non-context-free sets}
       \end{itemize}
    \end{itemize}
    \item Know, select and apply appropriate computing knowledge and problem-solving techniques. 
    Reason about computation and systems.
    \begin{itemize}
        \item Describe and prove closure properties of classes of languages under certain operations.
        \begin{itemize}
            \item {\bf Apply a general construction to create a new PDA or CFG from an example one.}
            \item {\bf Formalize a general construction from an informal description of it.}
            \item {\bf Use general constructions to prove closure properties of the class of context-free langugages.}
            \item {\bf Use counterexamples to prove non-closure properties of the class of context-free langugages.}
        \end{itemize}
    \end{itemize}
\end{itemize}

\vspace{-20pt}

\subsubsection*{TODO:}
\begin{list}{\itemsep-10pt}
   \item Schedule your Test 1 Attempt 1, Test 2 Attempt 1, Test 1 Attempt 2, and Test 2 Attempt 2 times 
   at PrairieTest (http://us.prairietest.com)
   \item Review Quiz 5 on PrairieLearn (http://us.prairielearn.com), complete by Sunday 11/4/2024
\end{list}

\newpage

\section*{Monday: Context-free languages}

%! app: context-free languages
%! outcome: Classify language, Find example languages, context-free grammars, Formal definition of automata, Informal definition of automata

Warmup:   Design a CFG to generate the  language $\{a^i b^j \mid j \geq i  \geq 0\}$
  
\vfill
{\it Sample derivation:} 

\vspace{100pt}



Design a PDA to recognize the  language $\{a^i b^j \mid j \geq i  \geq 0\}$
  
\vspace{100pt}


\vfill
\newpage

{\bf Theorem  2.20}: A language is  generated by some context-free  grammar
if  and only if it is recognized by some push-down automaton.

Definition: a language is called {\bf context-free} if it is the language generated by a context-free grammar.
The class of all context-free language over a given alphabet $\Sigma$ is called {\bf CFL}.

Consequences:
\begin{itemize}
    \item Quick proof that every regular language is context free 
    \item To prove closure of the class of context-free languages under a given operation, we can choose 
    either of two modes 
    of proof (via CFGs or PDAs) depending on which is easier
    \item To fully specify a PDA we could give its $6$-tuple formal definition or we could give its input 
alphabet, stack alphabet, and state diagram.
An informal description of a PDA is a step-by-step description of how its computations 
would process input strings; the reader should be able to reconstruct the state diagram or formal 
definition precisely from such a descripton. The informal description of a PDA can refer to some 
common modules or subroutines that are computable by PDAs:
\begin{itemize}
  \item PDAs can ``test for emptiness of stack'' without providing details. 
  {\it How?} We can always push a special end-of-stack symbol, $\$$, at the start, before processing
  any input, and then use this symbol as a flag.
  \item PDAs can ``test for end of input'' without providing details.
  {\it How?} We can transform a PDA to one where accepting states are only those reachable 
  when there are no more input symbols.
\end{itemize}

\end{itemize}




\vfill

\newpage
Suppose $L_1$ and $L_2$ are context-free languages over $\Sigma$.  {\bf Goal}:  $L_1 \cup L_2$  is  also context-free.

{\it Approach 1: with  PDAs}

Let $M_1 = ( Q_1, \Sigma, \Gamma_1, \delta_1, q_1, F_1)$ and
$M_2 = ( Q_2, \Sigma, \Gamma_2, \delta_2, q_2, F_2)$ be PDAs with 
$L(M_1) =  L_1$  and  $L(M_2) = L_2$.

Define $M = $

\vfill

{\it Approach  2: with CFGs}

Let $G_1 = (V_1, \Sigma, R_1, S_1)$  and   $G_2 = (V_2, \Sigma, R_2, S_2)$  be CFGs  with
$L(G_1) =  L_1$  and  $L(G_2) = L_2$.

Define $G = $

\vfill

\newpage
Suppose $L_1$ and $L_2$ are context-free languages over $\Sigma$.  {\bf Goal}:  $L_1 \circ L_2$  is  also context-free.


{\it Approach 1: with  PDAs}

Let $M_1 = ( Q_1, \Sigma, \Gamma_1, \delta_1, q_1, F_1)$ and
$M_2 = ( Q_2, \Sigma, \Gamma_2, \delta_2, q_2, F_2)$ be PDAs with 
$L(M_1) =  L_1$  and  $L(M_2) = L_2$.

Define $M = $

\vfill

{\it Approach  2: with CFGs}

Let $G_1 = (V_1, \Sigma, R_1, S_1)$  and   $G_2 = (V_2, \Sigma, R_2, S_2)$  be CFGs  with
$L(G_1) =  L_1$  and  $L(G_2) = L_2$.

Define $G = $

\vfill


    
\newpage
\subsection*{Wednesday: Context-free and non-context-free languages}

%! app: context-free languages
%! outcome: Classify language, Find example languages, context-free grammars, Formal definition of automata, Informal definition of automata

{\bf Theorem  2.20}: A language is  generated by some context-free  grammar
if  and only if it is recognized by some push-down automaton.

Definition: a language is called {\bf context-free} if it is the language generated by a context-free grammar.
The class of all context-free language over a given alphabet $\Sigma$ is called {\bf CFL}.

Consequences:
\begin{itemize}
    \item Quick proof that every regular language is context free 
    \item To prove closure of the class of context-free languages under a given operation, we can choose 
    either of two modes 
    of proof (via CFGs or PDAs) depending on which is easier
\end{itemize}


Over $\Sigma = \{a,b\}$, let $L = \{ a^n b^m \mid n  \neq m \}$. {\bf Goal}: Prove $L$ is context-free.


\vfill

\newpage
Suppose $L_1$ and $L_2$ are context-free languages over $\Sigma$.  {\bf Goal}:  $L_1 \cup L_2$  is  also context-free.

{\it Approach 1: with  PDAs}

Let $M_1 = ( Q_1, \Sigma, \Gamma_1, \delta_1, q_1, F_1)$ and
$M_2 = ( Q_2, \Sigma, \Gamma_2, \delta_2, q_2, F_2)$ be PDAs with 
$L(M_1) =  L_1$  and  $L(M_2) = L_2$.

Define $M = $

\vfill

{\it Approach  2: with CFGs}

Let $G_1 = (V_1, \Sigma, R_1, S_1)$  and   $G_2 = (V_2, \Sigma, R_2, S_2)$  be CFGs  with
$L(G_1) =  L_1$  and  $L(G_2) = L_2$.

Define $G = $

\vfill

\newpage
Suppose $L_1$ and $L_2$ are context-free languages over $\Sigma$.  {\bf Goal}:  $L_1 \circ L_2$  is  also context-free.


{\it Approach 1: with  PDAs}

Let $M_1 = ( Q_1, \Sigma, \Gamma_1, \delta_1, q_1, F_1)$ and
$M_2 = ( Q_2, \Sigma, \Gamma_2, \delta_2, q_2, F_2)$ be PDAs with 
$L(M_1) =  L_1$  and  $L(M_2) = L_2$.

Define $M = $

\vfill

{\it Approach  2: with CFGs}

Let $G_1 = (V_1, \Sigma, R_1, S_1)$  and   $G_2 = (V_2, \Sigma, R_2, S_2)$  be CFGs  with
$L(G_1) =  L_1$  and  $L(G_2) = L_2$.

Define $G = $

\vfill
\newpage
{\it Summary}

Over a fixed alphabet $\Sigma$, a language $L$ is {\bf regular}

\vspace{-20pt}
\begin{center}
    iff it is described by some regular expression \\
    iff it is recognized by some DFA\\
    iff it is recognized by some NFA
\end{center}

Over a fixed alphabet $\Sigma$, a language $L$ is {\bf context-free}

\vspace{-20pt}
\begin{center}
    iff it is generated by some CFG\\
    iff it is recognized by some PDA
\end{center}

{\bf Fact}: Every regular language is a context-free language.

{\bf Fact}: There are context-free languages that are not nonregular.

{\bf Fact}: There are countably many regular languages.

{\bf Fact}: There are countably inifnitely many context-free languages.

{\it Consequence}: Most languages are {\bf not} context-free!

{\bf Examples  of non-context-free languages}

\begin{align*}
    &\{ a^n b^n c^n \mid 0 \leq n , n \in \mathbb{Z}\}\\
    &\{ a^i b^j c^k \mid 0 \leq i \leq j \leq k , i \in \mathbb{Z}, j \in \mathbb{Z}, k \in \mathbb{Z}\}\\
    &\{ ww \mid w \in \{0,1\}^* \}
\end{align*}
(Sipser Ex 2.36, Ex 2.37, 2.38)

There is a Pumping Lemma for CFL that can be used to prove a specific language is non-context-free: 
If $A$ is a context-free language, there there
is a number $p$ where, if $s$ is any string in $A$ of length at least $p$, then $s$ may be divided 
into five pieces $s = uvxyz$ where (1) for each $i \geq 0$, $uv^ixy^iz \in A$, (2) $|uv|>0$, (3) $|vxy| \leq p$.
{\it We will not go into the details of the proof or application of Pumping Lemma for CFLs this quarter.}


\newpage
\subsection*{Friday: Turing machines}

%! app: Regular Languages, Context-free Languages
%! outcome: Formal definition of automata, Informal definition of automata, Classify language, Find example languages

\includegraphics[width=2.5in]{../../resources/machines/Lect13TM1.png}

Formal definition: 


Sample computation: 

\begin{tabular}{|c|c|c|c|c|c|c|}
\hline
\multicolumn{1}{|c}{$q0\downarrow$} &  \multicolumn{6}{c|}{\phantom{A}}\\
\hline
$0$ & $0$  & $0$ & $\textvisiblespace $& $\textvisiblespace $& $\textvisiblespace $&  $\textvisiblespace $\\
\hline
\multicolumn{7}{|c|}{\phantom{A}}\\
\hline
\phantom{AA} & \phantom{AA}& \phantom{AA}& \phantom{AA}& \phantom{AA}& \phantom{AA}& \phantom{AA} \\
\hline
\multicolumn{7}{|c|}{\phantom{A}}\\
\hline
\phantom{AA} & \phantom{AA}& \phantom{AA}& \phantom{AA}& \phantom{AA}& \phantom{AA}& \phantom{AA} \\
\hline
\multicolumn{7}{|c|}{\phantom{A}}\\
\hline
\phantom{AA} & \phantom{AA}& \phantom{AA}& \phantom{AA}& \phantom{AA}& \phantom{AA}& \phantom{AA} \\
\hline
\multicolumn{7}{|c|}{\phantom{A}}\\
\hline
\phantom{AA} & \phantom{AA}& \phantom{AA}& \phantom{AA}& \phantom{AA}& \phantom{AA}& \phantom{AA} \\
\hline
\end{tabular}

\vfill

The language recognized by this machine is \ldots

{\it Extra practice:}

 \includegraphics[width=2in]{../../resources/machines/Lect13TM2.png} 


 Formal definition: 


 Sample computation: 
 

\newpage
{\it Sipser Figure  3.10}

{\bf Conventions in state diagram of TM}: $b \to R$ label means $b \to b, R$ and
all arrows missing from diagram represent transitions with output $(q_{reject}, \textvisiblespace , R)$

\begin{multicols}{2}
\vspace{-20pt}
\begin{center}
\includegraphics[width=4in]{../../resources/machines/Lect13TM3.png}
\end{center}

Implementation level description of this machine:
\begin{quote}
Zig-zag across tape to corresponding positions on either side of $\#$ to check whether the 
characters in these positions agree. If they do not, or if there is no $\#$, reject. If they 
do, cross them off.

Once all symbols to the left of the $\#$ are crossed off, check for any un-crossed-off symbols 
to the right of $\#$; if there are any, reject; if there aren't, accept.
\end{quote}

\columnbreak

Computation on  input  string  $01\#01$

\begin{tabular}{|c|c|c|c|c|c|c|}
\hline
\multicolumn{1}{|c}{$q_1 \downarrow$} &  \multicolumn{6}{c|}{\phantom{A}}\\
\hline
$0$ & $1$  & $\#$  & $0$ & $1$ & $\textvisiblespace $& $\textvisiblespace $\\
\hline
  \multicolumn{7}{|c|}{\phantom{A}}\\
  \hline
  \phantom{AA} & \phantom{AA}& \phantom{AA}& \phantom{AA}& \phantom{AA}& \phantom{AA}& \phantom{AA} \\
  \hline
  \multicolumn{7}{|c|}{\phantom{A}}\\
  \hline
  \phantom{AA} & \phantom{AA}& \phantom{AA}& \phantom{AA}& \phantom{AA}& \phantom{AA}& \phantom{AA} \\
  \hline
  \multicolumn{7}{|c|}{\phantom{A}}\\
  \hline
  \phantom{AA} & \phantom{AA}& \phantom{AA}& \phantom{AA}& \phantom{AA}& \phantom{AA}& \phantom{AA} \\
  \hline
  \multicolumn{7}{|c|}{\phantom{A}}\\
  \hline
  \phantom{AA} & \phantom{AA}& \phantom{AA}& \phantom{AA}& \phantom{AA}& \phantom{AA}& \phantom{AA} \\
  \hline
  \multicolumn{7}{|c|}{\phantom{A}}\\
  \hline
  \phantom{AA} & \phantom{AA}& \phantom{AA}& \phantom{AA}& \phantom{AA}& \phantom{AA}& \phantom{AA} \\
  \hline
  \multicolumn{7}{|c|}{\phantom{A}}\\
  \hline
  \phantom{AA} & \phantom{AA}& \phantom{AA}& \phantom{AA}& \phantom{AA}& \phantom{AA}& \phantom{AA} \\
  \hline
  \multicolumn{7}{|c|}{\phantom{A}}\\
  \hline
  \phantom{AA} & \phantom{AA}& \phantom{AA}& \phantom{AA}& \phantom{AA}& \phantom{AA}& \phantom{AA} \\
  \hline
  \multicolumn{7}{|c|}{\phantom{A}}\\
  \hline
  \phantom{AA} & \phantom{AA}& \phantom{AA}& \phantom{AA}& \phantom{AA}& \phantom{AA}& \phantom{AA} \\
  \hline
  \multicolumn{7}{|c|}{\phantom{A}}\\
  \hline
  \phantom{AA} & \phantom{AA}& \phantom{AA}& \phantom{AA}& \phantom{AA}& \phantom{AA}& \phantom{AA} \\
  \hline
  \multicolumn{7}{|c|}{\phantom{A}}\\
  \hline
  \phantom{AA} & \phantom{AA}& \phantom{AA}& \phantom{AA}& \phantom{AA}& \phantom{AA}& \phantom{AA} \\
  \hline
  \multicolumn{7}{|c|}{\phantom{A}}\\
  \hline
  \phantom{AA} & \phantom{AA}& \phantom{AA}& \phantom{AA}& \phantom{AA}& \phantom{AA}& \phantom{AA} \\
  \hline
  \multicolumn{7}{|c|}{\phantom{A}}\\
  \hline
  \phantom{AA} & \phantom{AA}& \phantom{AA}& \phantom{AA}& \phantom{AA}& \phantom{AA}& \phantom{AA} \\
  \hline
  \multicolumn{7}{|c|}{\phantom{A}}\\
  \hline
  \phantom{AA} & \phantom{AA}& \phantom{AA}& \phantom{AA}& \phantom{AA}& \phantom{AA}& \phantom{AA} \\
  \hline
  \multicolumn{7}{|c|}{\phantom{A}}\\
  \hline
  \phantom{AA} & \phantom{AA}& \phantom{AA}& \phantom{AA}& \phantom{AA}& \phantom{AA}& \phantom{AA} \\
  \hline
  \multicolumn{7}{|c|}{\phantom{A}}\\
  \hline
  \phantom{AA} & \phantom{AA}& \phantom{AA}& \phantom{AA}& \phantom{AA}& \phantom{AA}& \phantom{AA} \\
  \hline
  \multicolumn{7}{|c|}{\phantom{A}}\\
  \hline
  \phantom{AA} & \phantom{AA}& \phantom{AA}& \phantom{AA}& \phantom{AA}& \phantom{AA}& \phantom{AA} \\
  \hline
  \multicolumn{7}{|c|}{\phantom{A}}\\
  \hline
  \phantom{AA} & \phantom{AA}& \phantom{AA}& \phantom{AA}& \phantom{AA}& \phantom{AA}& \phantom{AA} \\
  \hline
  \multicolumn{7}{|c|}{\phantom{A}}\\
  \hline
  \phantom{AA} & \phantom{AA}& \phantom{AA}& \phantom{AA}& \phantom{AA}& \phantom{AA}& \phantom{AA} \\
  \hline
  \end{tabular}
\end{multicols}


The language recognized by this machine is
\[
  \{ w \# w \mid w \in \{0,1\}^* \}
\]

\newpage
{\it Extra practice}

Computation on  input  string  $01\#1$

\begin{tabular}{|c|c|c|c|c|c|c|}
\hline
\multicolumn{1}{|c}{$q_1\downarrow$} &  \multicolumn{6}{c|}{\phantom{A}}\\
\hline
$0$ & $1$  & $\#$  & $1$ & $\textvisiblespace $& $\textvisiblespace $&  $\textvisiblespace $\\
\hline
\multicolumn{7}{|c|}{\phantom{A}}\\
\hline
\phantom{AA} & \phantom{AA}& \phantom{AA}& \phantom{AA}& \phantom{AA}& \phantom{AA}& \phantom{AA} \\
\hline
\multicolumn{7}{|c|}{\phantom{A}}\\
\hline
\phantom{AA} & \phantom{AA}& \phantom{AA}& \phantom{AA}& \phantom{AA}& \phantom{AA}& \phantom{AA} \\
\hline
\multicolumn{7}{|c|}{\phantom{A}}\\
\hline
\phantom{AA} & \phantom{AA}& \phantom{AA}& \phantom{AA}& \phantom{AA}& \phantom{AA}& \phantom{AA} \\
\hline
\multicolumn{7}{|c|}{\phantom{A}}\\
\hline
\phantom{AA} & \phantom{AA}& \phantom{AA}& \phantom{AA}& \phantom{AA}& \phantom{AA}& \phantom{AA} \\
\hline
\multicolumn{7}{|c|}{\phantom{A}}\\
\hline
\phantom{AA} & \phantom{AA}& \phantom{AA}& \phantom{AA}& \phantom{AA}& \phantom{AA}& \phantom{AA} \\
\hline
\multicolumn{7}{|c|}{\phantom{A}}\\
\hline
\phantom{AA} & \phantom{AA}& \phantom{AA}& \phantom{AA}& \phantom{AA}& \phantom{AA}& \phantom{AA} \\
\hline
\multicolumn{7}{|c|}{\phantom{A}}\\
\hline
\phantom{AA} & \phantom{AA}& \phantom{AA}& \phantom{AA}& \phantom{AA}& \phantom{AA}& \phantom{AA} \\
\hline
\multicolumn{7}{|c|}{\phantom{A}}\\
\hline
\phantom{AA} & \phantom{AA}& \phantom{AA}& \phantom{AA}& \phantom{AA}& \phantom{AA}& \phantom{AA} \\
\hline
\multicolumn{7}{|c|}{\phantom{A}}\\
\hline
\phantom{AA} & \phantom{AA}& \phantom{AA}& \phantom{AA}& \phantom{AA}& \phantom{AA}& \phantom{AA} \\
\hline
\multicolumn{7}{|c|}{\phantom{A}}\\
\hline
\phantom{AA} & \phantom{AA}& \phantom{AA}& \phantom{AA}& \phantom{AA}& \phantom{AA}& \phantom{AA} \\
\hline
\multicolumn{7}{|c|}{\phantom{A}}\\
\hline
\phantom{AA} & \phantom{AA}& \phantom{AA}& \phantom{AA}& \phantom{AA}& \phantom{AA}& \phantom{AA} \\
\hline
\multicolumn{7}{|c|}{\phantom{A}}\\
\hline
\phantom{AA} & \phantom{AA}& \phantom{AA}& \phantom{AA}& \phantom{AA}& \phantom{AA}& \phantom{AA} \\
\hline
\multicolumn{7}{|c|}{\phantom{A}}\\
\hline
\phantom{AA} & \phantom{AA}& \phantom{AA}& \phantom{AA}& \phantom{AA}& \phantom{AA}& \phantom{AA} \\
\hline
\multicolumn{7}{|c|}{\phantom{A}}\\
\hline
\phantom{AA} & \phantom{AA}& \phantom{AA}& \phantom{AA}& \phantom{AA}& \phantom{AA}& \phantom{AA} \\
\hline
\multicolumn{7}{|c|}{\phantom{A}}\\
\hline
\phantom{AA} & \phantom{AA}& \phantom{AA}& \phantom{AA}& \phantom{AA}& \phantom{AA}& \phantom{AA} \\
\hline
\multicolumn{7}{|c|}{\phantom{A}}\\
\hline
\phantom{AA} & \phantom{AA}& \phantom{AA}& \phantom{AA}& \phantom{AA}& \phantom{AA}& \phantom{AA} \\
\hline
\multicolumn{7}{|c|}{\phantom{A}}\\
\hline
\phantom{AA} & \phantom{AA}& \phantom{AA}& \phantom{AA}& \phantom{AA}& \phantom{AA}& \phantom{AA} \\
\hline
\multicolumn{7}{|c|}{\phantom{A}}\\
\hline
\phantom{AA} & \phantom{AA}& \phantom{AA}& \phantom{AA}& \phantom{AA}& \phantom{AA}& \phantom{AA} \\
\hline
\multicolumn{7}{|c|}{\phantom{A}}\\
\hline
\phantom{AA} & \phantom{AA}& \phantom{AA}& \phantom{AA}& \phantom{AA}& \phantom{AA}& \phantom{AA} \\
\hline
\multicolumn{7}{|c|}{\phantom{A}}\\
\hline
\phantom{AA} & \phantom{AA}& \phantom{AA}& \phantom{AA}& \phantom{AA}& \phantom{AA}& \phantom{AA} \\
\hline
\end{tabular}



\end{document}
