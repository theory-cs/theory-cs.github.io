\input{../../resources/lesson-head.tex}

\section*{Let's get started}

We want you to be successful. 

We will work together to build an 
environment in CSE 105 that supports your learning
in a way that respects your
perspectives, experiences, and identities (including race, ethnicity, heritage, gender, sex, 
class, sexuality, religion, ability, age, educational background, etc.).  
Our goal is for you to  engage
with interesting and challenging concepts and 
feel comfortable exploring, asking questions, and thriving.\\

If you are skipping and stretching meals, or having difficulties affording or accessing food, you may be eligible for CalFresh, California's Supplemental Nutrition Assistance Program, that can provide up to \$292 a month in free money on a debit card to buy food. Students can apply at \url{benefitscal.com/r/ucsandiegocalfresh}.
The Hub Basic Needs Center empowers all students by connecting them to resources for food, stable housing and financial literacy. Visit their site at \url{basicneeds.ucsd.edu}.

Financial aid resources, the possibility of emergency grant funding, and off-campus housing referral 
resources are available: see your College Dean of Student Affairs.

If you find yourself in an uncomfortable situation, ask for help. 
We are committed to upholding University policies regarding nondiscrimination, sexual violence and sexual harassment.
Here are some campus contacts that could provide this help: 
Counseling and Psychological Services (CAPS) at 858 534-3755 or \href{http://caps.ucsd.edu}{http://caps.ucsd.edu}; 
OPHD at 858 534-8298 or ophd@ucsd.edu , \href{http://ophd.ucsd.edu}{http://ophd.ucsd.edu};
CARE at Sexual Assault Resource Center at 858 534-5793 or sarc@ucsd.edu , \href{http://care.ucsd.edu}{http://care.ucsd.edu}.


Please reach out (minnes@ucsd.edu) if you need support with extenuating circumstances affecting CSE 105.

\vfill

\section*{Introductions}
Class website on Canvas \href{https://canvas.ucsd.edu/}{https://canvas.ucsd.edu}


Instructor: Prof. Mia Minnes {\tiny{"Minnes" rhymes with Guinness}}, minnes@ucsd.edu, 
\href{http://cseweb.ucsd.edu/~minnes}{http://cseweb.ucsd.edu/~minnes}


Our team: One instructor + two TAs and five tutors + all of you

Fill in contact info for students around you, if you'd like:


\vfill

\newpage 
Welcome to CSE 105: Introduction to Theory of Computation in Winter 2025!

\section*{CSE 105's Big Questions}
\begin{itemize}
   \item What problems are computers capable of solving?
   \item What resources are needed to solve a problem?
   \item Are some problems harder than others?
\end{itemize}

In this context, a {\bf problem} is defined as: ``Making a decision or computing a value based on some input"

Consider the following problems: 
\begin{itemize}
   \item Find a file on your computer
   \item Determine if your code will compile
   \item Find a run-time error in your code
   \item Certify that your system is un-hackable
\end{itemize}

Which of these is hardest?

\vfill

In Computer Science, we operationalize ``hardest'' as ``requires most resources'', where
resources might be memory, time, parallelism, randomness, power, etc.

To be able to compare ``hardness'' of problems, we use a consistent description of problems

{\bf Input}: String

{\bf Output}: Yes/ No, where Yes means that the input string matches the pattern or property described by the problem.

\newpage

\subsection*{Weeks 0 and 1 at a glance}

\subsubsection*{Textbook reading: Chapter 0, Sections 1.3, 1.1}

\vspace{-15pt}

Before Monday, review class syllabus on Canvas (https://canvas.ucsd.edu/).

Before Wednesday, read Example 1.51.

{\it Notice}: we are jumping to Section 1.3 and then will come back 
to Section 1.1 next week.

Before Friday, read Definition 1.52 (definition of regular expressions) on page 64.

For Week 2 Monday: Figure 1.4 and Definition 1.5 (definition of finite automata) on pages 34-35.

{\it Textbook references: Within a chapter, each item is numbered consecutively. Example 1.51 is the fifty-first numbered item in chapter one.}


\subsubsection*{We will be learning and practicing to:}
%Define decision problem, Regular expressions
\begin{itemize}
\item Clearly and unambiguously communicate computational ideas using appropriate formalism. Translate across levels of abstraction.
\begin{itemize}
   \item Translate a decision problem to a set of strings coding the problem.
   \begin{itemize}
      \item {\bf Distinguish between alphabet, language, sets, and strings}
   \end{itemize}
   \item Use regular expressions and relate them to languages and automata.
   \begin{itemize}
      \item {\bf Write and debug regular expressions using correct syntax}
      \item {\bf Determine if a given string is in the language described by a regular expression}
   \end{itemize}
\end{itemize}
\end{itemize}

\subsubsection*{TODO:}
\begin{list}{\itemsep-10pt}
   \item \#FinAid Assignment on Canvas (complete as soon as possible) and read syllabus on Canvas
   \item Schedule your Test 1 Attempt 1, Test 2 Attempt 1, Test 1 Attempt 2, and Test 2 Attempt 2 times 
   at PrairieTest (http://us.prairietest.com)
   \item Review Quiz 1 on PrairieLearn (http://us.prairielearn.com), complete by 1/15/25
   \item Create a homework group, possibly by using the Piazza (https://piazza.com/) find-a-teammate tool
   \item Homework 1 submitted via Gradescope (https://www.gradescope.com/), due 1/16/25
\end{list}

\newpage

\section*{Week 1 Monday: Terminology and Notation}

%! app: Regular Languages
%! outcome: Define decision problem

The CSE 105 vocabulary and notation build on discrete
math and introduction to proofs classes.  Some of the conventions may 
be a bit different from what you saw before so we'll draw your attention to them.

For consistency, we will use the notation from this class' textbook\footnote{Page references are to 
the 3rd edition of Sipser's Introduction to the Theory of Computation,
available through various sources for approximately \$30. You may be able to 
opt in to purchase a digital copy through Canvas. Copies of the book are also available 
for those who can't access the book
to borrow from the course instructor, while supplies last (minnes@ucsd.edu)}.

These definitions are on pages 3, 4, 6, 13, 14, 53.

\begin{center}
    \begin{tabular}{|p{2.6in}cp{3.5in}|}
    \hline 
    {\bf Term} & {\bf Typical symbol} & {\bf Meaning} \\
     & or {\bf Notation} & \\
    \hline
    && \\
    Alphabet & $\Sigma$, $\Gamma$ & A non-empty finite set	 \\
    Symbol over $\Sigma$  & $\sigma$, $b$, $x$ & An element of the alphabet $\Sigma$\\ 
    String over $\Sigma$  &	$u$, $v$, $w$ & A finite list of symbols from $\Sigma$\\
    (The) empty string &$\varepsilon$ & The (only) string of length $0$\\
    The set of all strings over $\Sigma$ & $\Sigma^*$ & The collection of all possible strings formed from symbols from $\Sigma$ \\ 
    (Some) language over $\Sigma$& $L$ & (Some) set of strings over $\Sigma$ \\ 
    (The) empty language &$\emptyset$ & The empty set, i.e. the set that has no strings (and no other elements either)\\
    && \\
    \hline
    && \\
    The power set of a set $X$ &$\mathcal{P}(X)$ & The set of all subsets of $X$ \\
    (The set of) natural numbers &$\mathcal{N}$ & The set of positive integers \\ 
    (Some) finite set & & The empty set or a set whose distinct elements can be counted by a natural number\\
    (Some) infinite set & & A set that is not finite.\\ 
    &&\\
    \hline
    && \\
    Reverse of a string $w$ & $w^\mathcal{R}$  & write $w$  in  the opposite order, if $w = w_1 \cdots  w_n$ then $w^\mathcal{R} = w_n \cdots  w_1$. Note: $\varepsilon^\mathcal{R} = \varepsilon$\\
    Concatenating strings $x$ and $y$ & $xy$ &  take $x = x_1 \cdots x_m$, $y=y_1 \cdots y_n$ and form $xy = x_1 \cdots x_m y_1 \cdots y_n$\\
    String $z$ is a substring of string $w$ & & there are strings $u,v$ such that $w = uzv$\\
    String $x$ is a prefix of string $y$ & & there is a string $z$ such that $y = xz$ \\
    String $x$ is a proper prefix of string $y$ & & $x$ is a prefix of $y$ and $x \neq y$\\
    && \\
    \hline
    &&\\
    Shortlex order, also known as string order over alphabet $\Sigma$ & & Order strings over  $\Sigma$ first 
    by length and then according to the dictionary order, assuming symbols in $\Sigma$  have an ordering\\ \hline
    \end{tabular}
\end{center}

\vfill
    
\newpage
Write out in words the meaning of the symbols below: 

\[
    \{ a, b, c\}
\]

\phantom{The set whose elements are $a$, $b$, and $c$}

\[
    | \{a, b, a \} | = 2
\]

\phantom{The number of elements in the set $\{a,b,a\}$ is $2$.}

\[
    | aba | = 3
\]

\phantom{The length of the string $aba$ is $3$.}

\begin{comment}

\[
    (a, 3, 2, b, b)
\]

\phantom{The $5$-tuple whose first components is $a$, second component 
is $3$, third component is $2$, fourth component is $b$, and fifth component is $b$.}
\end{comment}

{\it Circle the correct choice}:

A {\bf string} over an alphabet $\Sigma$ is \underline{~~an element of $\Sigma^*$ ~~ OR ~~ a subset of $\Sigma^*$}.
    
A {\bf language} over an alphabet $\Sigma$ is \underline{~~an element of $\Sigma^*$ ~~ OR ~~ a subset of $\Sigma^*$}.


With $\Sigma_1 = \{0,1\}$ and 
$\Sigma_2 = \{a,b,c,d,e,f,g,h,i,j,k,l,m,n,o,p,q,r,s,t,u,v,w,x,y,z\}$
and $\Gamma = \{0,1,x,y,z\}$

{\bf True} or {\bf False}: $\varepsilon \in \Sigma_1$

{\bf True} or {\bf False}: $\varepsilon$ is  a string over $\Sigma_1$

{\bf True} or {\bf False}: $\varepsilon$ is a language over $\Sigma_1$

{\bf True} or {\bf False}: $\varepsilon$ is a prefix of some string over  $\Sigma_1$

{\bf True} or {\bf False}: There is a string over $\Sigma_1$ that is a proper prefix of $\varepsilon$
    

The first five strings over $\Sigma_1$ in string order, using the ordering $0 <  1$: \vfill
    
The first five strings over $\Sigma_2$ in string order, using the usual alphabetical ordering for single letters: \vfill




    
\newpage

\section*{Week 1 Wednesday: Regular expressions}

%! app: Regular Languages
%! outcome: Regular expressions

Our motivation in studying sets of strings is that they can be used to encode problems.
To calibrate how difficult a problem is to solve, we describe how complicated the set of strings that encodes it is. 
How do we define sets of strings?


\vfill

How would you describe the language that has no elements at all?

\vfill

How would you describe the language that has all strings over $\{0,1\}$ as its elements?

\vfill

\newpage

**This definition was in the pre-class reading**
{\bf Definition 1.52}: A {\bf regular expression} over alphabet $\Sigma$
is a syntactic expression that can describe a language over $\Sigma$. The collection of all regular
expressions over $\Sigma$ is defined recursively:
\begin{itemize}
\item[] {\it Basis steps of recursive definition}
\begin{quote}    
    $a$ is a regular expression, for $a \in \Sigma$

    $\varepsilon$ is a regular expression

    $\emptyset$ is a regular expression
\end{quote}

\item[] {\it Recursive steps of recursive definition}
\begin{quote}
    $(R_1 \cup R_2)$ is a regular expression when $R_1$, $R_2$ are regular expressions 

    $(R_1 \circ R_2)$ is a regular expression when $R_1$, $R_2$ are regular expressions

    $(R_1^*)$ is a regular expression when $R_1$ is a regular expression 
\end{quote}
\end{itemize}
 

The {\it semantics} (or meaning) of the syntactic regular expression is the {\bf language
described by the regular expression}. The function that assigns a language to a regular expression
over $\Sigma$ is defined recursively, using familiar set operations:


\begin{itemize}
    \item[] {\it Basis steps of recursive definition}
    \begin{quote}    
        The language described by $a$, for $a \in \Sigma$, is $\{a\}$ and we write 
        $L(a) = \{a\}$
    
        The language described by $\varepsilon$ is $\{\varepsilon\}$ and we write 
        $L(\varepsilon) = \{ \varepsilon\}$
    
        The language described by $\emptyset$ is $\{\}$ and we write
        $L(\emptyset) = \emptyset$.
    \end{quote}
    
    \item[] {\it Recursive steps of recursive definition}
    \begin{quote}
        When $R_1$, $R_2$ are regular expressions, the language described by the regular
        expression $(R_1 \cup R_2)$ is the union of the languages described by $R_1$ and $R_2$, 
        and we write 
        $$L(~(R_1 \cup R_2)~) = L(R_1) \cup L(R_2) = \{ w \mid w \in L(R_1) \lor w \in L(R_2)\}$$
    
        When $R_1$, $R_2$ are regular expressions, the language described by the regular
        expression $(R_1 \circ R_2)$ is the concatenation of the languages described by $R_1$ and $R_2$, 
        and we write 
        $$L(~(R_1 \circ R_2)~) = L(R_1) \circ L(R_2) = \{ uv \mid u \in L(R_1) \land v \in L(R_2)\}$$
    
        When $R_1$ is a regular expression, the language described by the regular 
        expression $(R_1^*)$ is the {\bf Kleene star} of the language described by $R_1$ and we write
        $$L(~(R_1^*)~) = (~L(R_1)~)^* = \{ w_1 \cdots w_k \mid k \geq 0 \textrm{ and each } w_i \in L(R_1)\}$$
    \end{quote}
\end{itemize}
  
\newpage
For the following examples assume the alphabet is $\Sigma_1 =  \{0,1\}$:
    
The language described by the regular expression $0$ is $L(0) = \{ 0 \}$

The language described by the regular expression $1$ is $L(1)  = \{ 1 \}$

The language described by the regular expression $\varepsilon$ is $L(\varepsilon) = \{ \varepsilon  \}$

The language described by the regular expression $\emptyset$ is $L(\emptyset) = \emptyset$


The language described by the regular expression $1^* \circ 1$ is $L(1^* \circ 1) = $

\vfill

The language described by the regular expression $(~(0 \cup 1) \circ (0 \cup 1) \circ (0 \cup 1) ~)^*$ 
is $L(~(~(0 \cup 1) \circ (0 \cup 1) \circ (0 \cup 1) ~)^*~) = $

\vfill

    
\newpage

\subsection*{Week 1 Friday: Regular expressions conventions}

%! app: Regular Languages
%! outcome: Regular expressions


{\bf Review}: Determine whether each statement below about regular expressions
over the alphabet $\{a,b,c\}$ is true or false:

\begin{comment}
True or False: \qquad 
   $a  \in L(~(a \cup b )~\cup c)$
\end{comment}

True or False: \qquad 
   $ab  \in L(~ (a \cup b)^*  ~)$
   
True or False: \qquad    
   $ba \in L( ~ a^* b^* ~)$
   
True or False: \qquad 
   $\varepsilon  \in L(a \cup b \cup c)$
   
True or False: \qquad 
   $\varepsilon  \in L(~ (a \cup b)^*  ~)$

True or False: \qquad 
   $\varepsilon \in L( ~ aa^* \cup bb^* ~)$

\vfill

{\it Shorthand and conventions} (Sipser pages 63-65)

\vspace{-20pt}

\begin{center}
    \begin{tabular}{|ll|}
    \hline
    & \\
    \multicolumn{2}{|l|}{Assuming $\Sigma$ is the alphabet, we use the following conventions}\\
    & \\
    $\Sigma$   & regular  expression describing language consisting of  all strings  of length  $1$ over $\Sigma$\\
    $*$ then $\circ$ then $\cup$   & precedence order, unless parentheses are used to change it\\
    $R_1R_2$ & shorthand  for  $R_1  \circ R_2$ (concatenation symbol is implicit) \\
    $R^+$ & shorthand for $R^* \circ R$ \\
    $R^k$ & shorthand for $R$ concatenated with itself $k$ times, where $k$ is a (specific) natural number\\
    & \\
    \hline
    \end{tabular}
\end{center}

\vfill 

{\bf Caution: many programming languages that support regular expressions build in functionality
that is more powerful than the ``pure'' definition of regular expressions given here. }

Regular expressions are everywhere (once you start looking for them).

Software tools and languages often have built-in support for regular expressions to describe
{\bf patterns} that we want to match (e.g. Excel/ Sheets, grep, Perl, python, Java, Ruby).

Under the hood, the first phase of {\bf compilers} is to transform the strings we write 
in code to tokens (keywords, operators, identifiers, literals). Compilers use regular expressions
to describe the sets of strings that can be used for each token type.

Next time: we'll start to see how to build machines that decide whether strings match the pattern
described by a regular expression.

\newpage

Practice with the regular expressions over $\{a,b\}$ below.

For example: Which regular expression(s) below describe a language that includes the string $a$ as an element?

$a^* b^*$ 

\vfill

$a(ba)^* b$

\vfill

$a^* \cup b^*$

\vfill

$(aaa)^*$

\vfill

$(\varepsilon \cup a) b$

\vfill



\newpage

\subsection*{Week 2 at a glance}

\vspace{-15pt}

\subsubsection*{Textbook reading: Sections 1.1, 1.2}

\vspace{-15pt}

Before Monday, read Figure 1.4 and Definition 1.5 (definition of finite automata) on pages 34-35.

Before Wednesday, read pages 41-43 (Figures 1.18, 1.19, 1.20) for examples of automata and languages. Wednesday's class will be asynchronous. Please read over the annotated notes and watch the relevant supplementary videos. Ask any followup questions in the discussion forum or in office hours.

Before Friday, read pages 48-50 (Figures 1.27, 1.29) which introduces nondeterminism.

No class on Week 3 Monday in observance of Martin Luther King Jr. Day.

For Week 3 Wednesday: read the definition of the union, concatenation, and star operations for languages,  given 
as Definition 1.23 on page 44 and a useful example is Example 1.24.

\vspace{-20pt}

\subsubsection*{We will be learning and practicing to:}
%Define decision problem, Formal definition of automata, Informal definition of automata, Find example languages, 
%Describe and use models of computation that don't involve state machines.
\begin{itemize}
\item Clearly and unambiguously communicate computational ideas using appropriate formalism. Translate across levels of abstraction.
\begin{itemize}
   \item Give examples of sets that are regular (and prove that they are).
   \begin{itemize}
      \item {\bf State the definition of the class of regular languages}
      \item {\bf Give examples of regular languages, using each of the three equivalent models of computation for proving regularity.}
   \end{itemize}
   \item Describe and use models of computation that don't involve state machines.
   \begin{itemize}
      \item {\bf Given a DFA, find a regular expression that describes its language.}
      \item {\bf Given a regular expression, find a DFA that recognizes its language.}
   \end{itemize}
   \item Use precise notation to formally define the state diagram of finite automata.
   \item Use clear English to describe computations of finite automata informally.
      \begin{itemize}
         \item {\bf State the formal definition of (deterministic) finite automata}   
         \item {\bf Trace the computation of a finite automaton on a given string using its state diagram}
         \item {\bf Translate between a state diagram and a formal definition}
         \item {\bf Determine if a given string is in the language recognized by a finite automaton}
         \item {\bf Design an automaton that recognizes a given language}
         \item {\bf Specify a general construction for DFA based on parameters}
         \item {\bf Design general constructions for DFA}
   \end{itemize}
\end{itemize}

\end{itemize}
\newpage
\subsubsection*{TODO:}
\begin{list}{\itemsep-10pt}
   \item \#FinAid Assignment on Canvas (complete as soon as possible) and read syllabus on Canvas
   \item Schedule your Test 1 Attempt 1, Test 2 Attempt 1, Test 1 Attempt 2, and Test 2 Attempt 2 times 
   at PrairieTest (http://us.prairietest.com)
   \item Review Quiz 1 on PrairieLearn (http://us.prairielearn.com), complete by 1/15/25
   \item Create a homework group, possibly by using the Piazza (https://piazza.com/) find-a-teammate tool
   \item Homework 1 submitted via Gradescope (https://www.gradescope.com/), due 1/16/25
   \item Review Quiz 2 on PrairieLearn (http://us.prairielearn.com), complete by 1/22/25
\end{list}


\newpage

\subsection*{Week 2 Monday: Finite automata}

%! app: Regular Languages 
%! outcome: Regular expressions, Formal definition of automata, Informal definition of automata


**This definition was in the pre-class reading**
A finite automaton (FA) is specified by  $M = (Q, \Sigma, \delta, q_0, F)$.
This $5$-tuple is called the {\bf formal definition} of the FA. The FA can also 
be represented by its state diagram: with nodes for the state, labelled edges specifying the 
transition function, and decorations on nodes denoting the start and accept states.

\begin{quote}
Finite set of states $Q$ can be labelled by any collection of distinct names. Often
we use default state labels $q0, q1, \ldots$ 
\end{quote}

\begin{quote}  
The alphabet $\Sigma$ determines the possible inputs to the automaton. 
Each input to the automaton is a string over  $\Sigma$, and the automaton ``processes'' the input
one symbol (or character) at a time.
\end{quote}

\begin{quote}
The transition function $\delta$ gives the next state of the automaton based on the current state of 
the machine and on the next input symbol.
\end{quote}

\begin{quote}
The start state $q_0$ is an element of $Q$.  Each computation of the machine starts at the  start  state.
\end{quote}

\begin{quote}
The accept (final) states $F$ form a subset of the states of the automaton, $F \subseteq  Q$. 
These states are used to flag if the machine accepts or rejects an input string.
\end{quote}


\begin{quote}
The computation of a machine on an input string is a sequence of states
in the machine,  starting with the start state, determined by transitions 
of the machine as it reads successive input symbols.
\end{quote}

\begin{quote}
The finite automaton $M$ accepts the given input string exactly when the computation of $M$ on the input string
ends in an accept state. $M$ rejects the given input string exactly when the computation of 
$M$ on the input string ends in a nonaccept state, that is, a state that is not in $F$.
\end{quote}

\begin{quote} 
The language of $M$, $L(M)$, is defined as the set of  all strings that are each accepted 
by the machine $M$. Each string that is rejected by $M$ is not in $L(M)$.
The language of $M$ is also called the language recognized by $M$.
\end{quote}   
   
What is {\bf finite} about all finite automata? (Select all that apply)
\begin{itemize}
   \item[$\square$] The size of the machine (number of states, number of arrows)
   \item[$\square$] The length of each computation of the machine
   \item[$\square$] The number of strings that are accepted by the machine
\end{itemize}
\newpage
  
\begin{center}
\begin{tikzpicture}[->,>=stealth',shorten >=1pt, auto, node distance=2cm, semithick]
   \tikzstyle{every state}=[text=black, fill=none]
   
   \node[initial,state,accepting] (q0)          {$q0$};
   \node[state,accepting]         (q1) [above right of=q0, xshift=20pt] {$q1$};
   \node[state,accepting]         (q2) [below right of=q0, xshift=20pt] {$q2$};
   \node[state]                   (q3) [below right of=q1, xshift=40pt] {$q3$};
   
   \path (q0) edge  [bend left=0] node {$a$} (q1)
       (q1) edge [loop above] node {$a$} (q1)
       (q1) edge [bend left=0] node {$b$} (q3)
       (q0) edge [bend left=0] node {$b$} (q2)
       (q2) edge [loop above] node {$b$} (q2)
       (q2) edge [bend left=0] node {$a$} (q3)
       (q3) edge [loop above] node {$a,b$} (q3)
   ;
\end{tikzpicture}
\end{center}
The formal definition of this FA is
   
\vfill
\vfill
   

Classify each string $a, aa, ab, ba, bb, \varepsilon$ as accepted by the FA or rejected by the FA.  

{\it Why are these the only two options?}

\vspace{200pt}


The language recognized by this automaton is
  

\vfill

\newpage

\begin{center}
   \begin{tikzpicture}[->,>=stealth',shorten >=1pt, auto, node distance=2cm, semithick]
      \tikzstyle{every state}=[text=black, fill=none]
      
      \node[initial,state,accepting] (q0)          {$q0$};
      \node[state,accepting]         (q1) [right of=q0, xshift=20pt] {$q1$};
      \node[state]         (q2) [right of=q1, xshift=20pt] {$q2$};
      
      \path (q0) edge  [bend left=0] node {$b$} (q1)
          (q0) edge [loop above] node {$a$} (q0)
          (q1) edge [bend left=0] node {$a$} (q2)
          (q1) edge [loop above] node {$b$} (q1)
          (q2) edge [loop above] node {$a,b$} (q2)
      ;
   \end{tikzpicture}
\end{center}

The language recognized by this automaton is
  


\vfill

\hrule

\begin{center}
   \begin{tikzpicture}[->,>=stealth',shorten >=1pt, auto, node distance=2cm, semithick]
      \tikzstyle{every state}=[text=black, fill=none]
      
      \node[initial,state] (q0)          {$q0$};
      \node[state,accepting]         (q1) [above right of=q0, xshift=10pt, yshift=20pt] {$q1$};
      \node[state,accepting]         (q2) [right of=q1, xshift=20pt] {$q2$};
      \node[state]          (q3) [below right of=q2, xshift=10pt, yshift=-20pt] {$q3$};
      
      \path (q0) edge  [bend left=0] node {$a$} (q1)
          (q0) edge [bend left=0] node {$b$} (q3)
          (q1) edge [bend left=0] node {$b$} (q2)
          (q1) edge [bend left=0] node {$a$} (q3)
          (q2) edge [bend left=0] node {$a,b$} (q3)
          (q3) edge [loop right] node {$a,b$} (q3)
      ;
   \end{tikzpicture}
\end{center}

The language recognized by this automaton is
  

\vfill


\subsection*{Week 2 Wednesay: Finite automaton constructions - asynchronous}

%! app: Regular Languages
%! outcome: Regular expressions, Formal definition of automata, Informal definition of automata

{\bf Review}: Formal definition of DFA: $M = (Q, \Sigma, \delta, q_0, F)$ 

\begin{center}
\begin{multicols}{2}
\begin{itemize}
\setlength{\itemsep}{2pt}
\item Finite set of states $Q$
\item Alphabet $\Sigma$
\item Transition function $\delta$
\item Start state $q_0$
\item Accept (final) states $F$
\end{itemize}
\end{multicols}
\end{center}
Quick check: In the state diagram of $M$, how many outgoing arrows are there from each state?



{\bf Note}: We'll see a new kind of finite automaton. It will be helpful to distinguish it from the
machines we've been talking about so we'll use {\bf Deterministic Finite Automaton} (DFA) to refer to the machines 
from Section 1.1.

$M = ( \{ q0, q1, q2\}, \{a,b\}, \delta, q0, \{q0\} )$ 
where $\delta$ is  (rows labelled by states
and columns labelled by symbols):
\begin{center}
\begin{tabular}{c|cc}
$\delta$ & $a$ & $b$ \\
\hline
$q0$ & $q1$ & $q1$ \\
$q1$ & $q2$ & $q2$ \\
$q2$ & $q0$ & $q0$ \\
\end{tabular}
\end{center}

The state diagram for $M$ is 

\vfill


Give two examples of strings that are accepted by $M$ and two examples of strings that are rejected by $M$:

\vfill


A regular expression describing $L(M)$ is


\vfill 

A state diagram for a finite automaton recognizing
$$\{w \mid w~\text{is a string over $\{a,b\}$ whose length is not a multiple of $3$} \}$$

\vfill

Extra example: Let $n$ be an arbitrary positive integer. What is a formal definition for a finite automaton recognizing
\[
\{w \mid w~\text{is a string over $\{0,1\}$ whose length is not a multiple of $n$} \}?
\]

\newpage

Consider the alphabet $\Sigma_1 = \{0,1\}$.

A state diagram for a finite automaton that recognizes $\{w \mid w~\text{contains at most two $1$'s} \}$ is

\vfill
A state diagram for a finite automaton that recognizes $\{w \mid w~\text{contains more than two $1$'s} \}$ is

\vfill

\textbf{Strategy}: Add ``labels" for states in the state diagram, 
e.g. ``have not seen any of desired pattern yet'' or
``sink state''. Then, we can use the analysis of the roles of the states in the 
state diagram to work towards a description of the language recognized
by the finite automaton.


\vfill
\newpage
Or: decompose the language to a simpler one 
that we already know how to recognize with a DFA or NFA.


Textbook Exercise 1.14: 
Suppose $A$ is a language over an alphabet $\Sigma$. 
If there is a DFA $M$ such that $L(M) = A$ then there is another DFA, let's call it $M'$, such that 
$L(M') = \overline{A}$, the complement of $A$, defined as $\{ w \in \Sigma^* \mid w \notin A \}$.


{\bf Proof idea}:


\vfill
A useful bit of terminology: the {\bf iterated transition function} of a finite automaton
$M = (Q, \Sigma, \delta, q_0, F)$ is defined recursively by
\[
\delta^* (~(q,w)~) 
=\begin{cases}
q  \qquad &\text{if $q \in Q, w = \varepsilon$} \\
\delta( ~(q,a)~) \qquad &\text{if $q \in Q$, $w = a \in \Sigma$ } \\
\delta(~(\delta^*(~(q,u)~), a) ~) \qquad &\text{if $q \in Q$, $w = ua$ where $u \in  \Sigma^*$ and $a \in \Sigma$}
\end{cases}
\]

Using  this terminology, $M$ accepts a string $w$ over $\Sigma$ if and only if $\delta^*( ~(q_0,w)~) \in F$.


{\bf Proof}: 
\vfill



\newpage
\subsection*{Week 2 Friday: Nondeterministic automata}

We saw that whenever a language is recognized by a DFA, its
complement is also recognized by some (other) DFA. 

Another way to say this is that the collection of languages
that are each recognizable by a DFA is {\bf closed} under complementation.


Before we continue with more complicated constructions, let's revisit some of 
the assumption of the {\bf D}FA model.

Recall that the computation of {\bf deterministic} finite automaton has exactly one choice for its next step given the current state and the current character to read.


%! app: Regular Languages
%! outcome: Regular expressions, Formal definition of automata, Informal definition of automata

\begin{center}
\begin{tabular}{|ll|}
\hline
\multicolumn{2}{|l|}{{\bf Nondeterministic finite automaton}  (Sipser Page 53) Given as $M = (Q, \Sigma, \delta, q_0, F)$}\\
& \\
Finite set of states $Q$  & Can  be labelled by any collection  of distinct names. Default: $q0, q1, \ldots$  \\
Alphabet $\Sigma$ &  Each input to the automaton is a string over  $\Sigma$. \\
Arrow labels $\Sigma_\varepsilon$ &  $\Sigma_\varepsilon = \Sigma \cup \{ \varepsilon\}$. \\
&  Arrows 
in the state diagram are labelled either by symbols from $\Sigma$ or by $\varepsilon$ \\
Transition function $\delta$  & $\delta: Q \times \Sigma_{\varepsilon} \to \mathcal{P}(Q)$
gives the {\bf set of possible next states} for a transition \\
&  from the current state upon reading a symbol or spontaneously moving.\\
Start state $q_0$ & Element of $Q$.  Each computation of the machine starts at the  start  state.\\
Accept (final) states $F$ & $F \subseteq  Q$.\\
& \\
\multicolumn{2}{|p{\textwidth}|}{$M$ accepts the input string $w \in \Sigma^*$ if and only if {\bf there is} a computation of $M$ on 
$w$ that processes the whole string and ends in an
accept state.}\\
\hline
\end{tabular}
\end{center}

The formal definition of the NFA over $\{0,1\}$ given by this state diagram is: 

\begin{center}
    \begin{tikzpicture}[->,>=stealth',shorten >=1pt, auto, node distance=2cm, semithick]
       \tikzstyle{every state}=[text=black, fill=none]
       
       \node[initial,state] (q0)          {$q0$};
       \node[state,accepting]         (q1) [right of=q0, xshift=10pt]{$q1$};
       
       \path (q0) edge  [loop above] node {$0,1$} (q0)
           (q0) edge [bend left=0] node {$1$} (q1)
       ;
    \end{tikzpicture}
 \end{center}


The language over $\{0,1\}$ recognized by this NFA is:

\vspace{70pt}

{\it Practice}: Change the transition function to get a different NFA which accepts
the empty string (and potentially other strings too).


\newpage

The state diagram of an NFA over $\{a,b\}$ is:

\begin{center}
    \begin{tikzpicture}[->,>=stealth',shorten >=1pt, auto, node distance=2cm, semithick]
       \tikzstyle{every state}=[text=black, fill=none]
       
       \node[initial,state] (q0)          {$q0$};
       \node[state,accepting]         (n) [below right of=q0, xshift=20pt] {$n$};
       \node[state]     (d) [right of=n, xshift=20pt] {$d$};
       \node[state]     (q) [above right of=q0, xshift=10pt] {$q$};
       \node[state]     (r) [right of=q, xshift=10pt] {$r$};
       \node[state,accepting]     (s) [right of=r, xshift=10pt] {$s$};
       
       \path (q0) edge  [bend left=0] node {$\varepsilon$} (q)
           (q0) edge  [bend left=0] node {$\varepsilon$} (n)
           (q) edge [bend left=0] node {$a$} (r)
           (q) edge [loop above] node {$b$} (q)
           (r) edge [bend left=0] node {$b$} (s)
           (r) edge [loop above] node {$a$} (r)
           (s) edge [loop above] node {$a,b$} (s)
           (n) edge [bend left=20] node {$a,b$} (d)
           (d) edge [bend left=20] node {$a,b$} (n)
       ;
    \end{tikzpicture}
 \end{center}

 The formal definition of this NFA is:

 \vfill

\newpage

Suppose $A_1, A_2$ are languages over an alphabet $\Sigma$.
{\bf Claim:} if there is a NFA $N_1$ such that $L(N_1) = A_1$ and 
NFA $N_2$ such that $L(N_2) = A_2$, then there is another NFA, let's call it $N$, such that 
$L(N) = A_1 \cup A_2$.

{\bf Proof idea}: Use nondeterminism to choose which of $N_1$, $N_2$ to run.

\vfill
\begin{comment}
    Draw schematic
\end{comment}

{\bf Formal construction}: Let 
$N_1 = (Q_1, \Sigma, \delta_1, q_1, F_1)$ and $N_2 = (Q_2, \Sigma, \delta_2,q_2, F_2)$
and assume $Q_1 \cap Q_2 = \emptyset$ and that $q_0 \notin Q_1 \cup Q_2$.
Construct $N = (Q, \Sigma, \delta, q_0, F_1 \cup F_2)$ where
\begin{itemize}
    \item $Q = $
    \item $\delta: Q \times \Sigma_\varepsilon \to \mathcal{P}(Q)$ is defined by, for $q \in Q$ and $x \in \Sigma_{\varepsilon}$:
        \[
            \phantom{\delta((q,x))=\begin{cases}  \delta_1 ((q,x)) &\qquad\text{if } q\in Q_1 \\ \delta_2 ((q,x)) &\qquad\text{if } q\in Q_2 \\ \{q1,q2\} &\qquad\text{if } q = q_0, x = \varepsilon \\ \emptyset\text{if } q= q_0, x \neq \varepsilon \end{cases}}
        \]
\end{itemize}


\vfill
{\it Proof of correctness would prove that $L(N) = A_1 \cup A_2$ by considering
an arbitrary string accepted by $N$, tracing an accepting computation of $N$ on it, and using 
that trace to prove the string is in at least one of $A_1$, $A_2$; then, taking an arbitrary 
string in $A_1 \cup A_2$ and proving that it is accepted by $N$. Details left for extra practice.}




\newpage

\subsection*{Week 3 at a glance}


\subsubsection*{Textbook reading: Chapter 1}

\vspace{-15pt}

No class on Week 3 Monday in observance of Martin Luther King Jr. Day.

Before Wednesday: read the definition of the union, concatenation, and star operations for languages,  given 
as Definition 1.23 on page 44 and a useful example is Example 1.24.

Before Friday,  read pages 45-46 (Theorem 1.25) that we'll refer to as a ``closure proof".

For Week 4 Monday, read Introduction to Section 1.4 (page 77) which introduces nonregularity.


\vspace{-20pt}

\subsubsection*{We will be learning and practicing to:}

\vspace{-15pt}

%Define decision problem, Formal definition of automata, Informal definition of automata, Find example languages, 
%Describe and use models of computation that don't involve state machines.
\begin{itemize}
\item Clearly and unambiguously communicate computational ideas using appropriate formalism. Translate across levels of abstraction.
\begin{itemize}
   \item Use precise notation to formally define the state diagram of finite automata.
   \item Use clear English to describe computations of finite automata informally.
      \begin{itemize}
         \item {\bf Motivate the use of nondeterminism}
         \item {\bf State the formal definition of NFA}   
         \item {\bf Trace the computation(s) of a NFA on a given string using its state diagram}
         \item {\bf Determine if a given string is in the language recognized by a NFA}
         \item {\bf Translate between a state diagram and a formal definition of a NFA}
      \end{itemize}
   \item Give examples of sets that are regular (and prove that they are).
   \begin{itemize}
      \item {\bf State the definition of the class of regular languages}
      \item {\bf Give examples of regular languages, using each of the three equivalent models of computation for proving regularity.}
      \item {\bf Choose between multiple models to prove that a language is regular}
      \item {\bf Explain the limits of the class of regular languages}
   \end{itemize}
   \item Describe and use models of computation that don't involve state machines.
   \begin{itemize}
      \item {\bf Given a DFA or NFA, find a regular expression that describes its language.}
      \item {\bf Given a regular expression, find a DFA or NFA that recognizes its language.}
   \end{itemize}
\end{itemize}


\item Understand, guide, shape impact of computing on society/the world. Connect the role of Theory CS classes to other applications (in undergraduate CS curriculum and beyond). Model problems using appropriate mathematical concepts.
 \begin{itemize}
     \item Explain nondeterminism and describe tools for simulating it with deterministic computation.
     \begin{itemize}
       \item {\bf Given a NFA, find a DFA that recognizes its language.}
       \item {\bf Convert between regular expressions and automata}
     \end{itemize}
 \end{itemize}

\end{itemize}

\vspace{-20pt}

\subsubsection*{TODO:}
\begin{list}{\itemsep-10pt}
   \item Schedule your Test 1 Attempt 1, Test 2 Attempt 1, Test 1 Attempt 2, and Test 2 Attempt 2 times 
   at PrairieTest (http://us.prairietest.com)
   \item Review Quiz 3 on PrairieLearn (http://us.prairielearn.com), due 1/29/2025
   \item Homework 2 submitted via Gradescope (https://www.gradescope.com/), due Tuesday 1/30/2025
\end{list}


\vfill

In Computer Science, we operationalize ``hardest'' as ``requires most resources'', where
resources might be memory, time, parallelism, randomness, power, etc.
To be able to compare ``hardness'' of problems, we use a consistent description of problems

{\bf Input}: String

{\bf Output}: Yes/ No, where Yes means that the input string matches the pattern or property described by the problem.

So far: we saw that regular expressions are convenient ways of describring patterns in strings.
{\bf Finite automata} give a model of computation for processing strings and and classifying them into Yes (accepted)
or No (rejected). We will see that each set of strings is described by a regular expression if and only 
if there is a FA that recognizes it.  Another way of thinking about it: properties described by regular
expressions require exactly the computational power of these finite automata.

\newpage
\subsection*{Wednesday: Automata constructions}

%! app: Regular Languages
%! outcome: Formal definition of automata, Informal definition of automata, Nondeterminism

{\bf Review}: The language recognized by the NFA over $\{a,b\}$ with state diagram


    \begin{tikzpicture}[->,>=stealth',shorten >=1pt, auto, node distance=2cm, semithick]
    \tikzstyle{every state}=[text=black, fill=none]
    
    \node[initial,state] (q0)          {$q_0$};
    \node[state]         (q) [above right of=q0, xshift=20pt] {$q$};
    \node[state]         (r) [right of=q, xshift=20pt] {$r$};
    \node[state, accepting]         (s) [right of=r, xshift=20pt] {$s$};
    \node[state, accepting]         (n) [below right of=q0, xshift=20pt] {$n$};
    \node[state]         (d) [right of=n, xshift=20pt] {$d$};
    
    \path (q0) edge  [bend left=0, near start] node {$\varepsilon$} (q)
            edge [bend right=0, near start] node {$\varepsilon$} (n)
        (q) edge [bend left=0] node {$a$} (r)
            edge [loop above, near start] node {$b$} (q)
        (r) edge [bend left=0] node {$b$} (s)
            edge [loop above, near start] node {$a,b$} (r)
        (n) edge [bend left=20] node {$a,b$} (d)
        (d) edge [bend left=20] node {$a,b$} (n)
    ;
    \end{tikzpicture}
is:


\vfill

So far, we know: 
\begin{itemize}
\item The collection of languages that are each recognizable by a 
DFA is {\bf closed} under complementation.

{\it Could we do the same construction with NFA?}

\vspace{100pt}

\item The collection of languages that are each recognizable by a 
NFA is {\bf closed} under union.

{\it Could we do the same construction with DFA?}

\vspace{100pt}

\end{itemize}


\newpage

\vspace{50pt}

Happily, though, an analogous claim is true!

Suppose $A_1, A_2$ are languages over an alphabet $\Sigma$.
{\bf Claim:} if there is a DFA $M_1$ such that $L(M_1) = A_1$ and 
DFA $M_2$ such that $L(M_2) = A_2$, then there is another DFA, let's call it $M$, such that 
$L(M) = A_1 \cup A_2$. {\it Theorem 1.25 in Sipser, page 45}
    
    {\bf Proof idea}:
    
    
    {\bf Formal construction}: 
    
    \vfill

    
    {\bf Example}:  When $A_1 = \{w \mid w~\text{has an $a$ and ends in $b$} \}$ and 
    $A_2 = \{ w \mid w~\text{is of even length} \}$.
    
    \begin{tikzpicture}[->,>=stealth',shorten >=1pt, auto, node distance=2cm, semithick]
        \tikzstyle{every state}=[text=black, fill=none]
        
        \node[initial,state,accepting] (qn)          {$(q,n)$};
        \node[state]         (qd) [below of=qn, yshift=-40pt] {$(q,d)$};
        \node[state]         (rd) [right of=qn, xshift=20pt] {$(r,d)$};
        \node[state,accepting]         (rn) [right of=qd, xshift=20pt] {$(r,n)$};
        \node[state,accepting]         (sn) [right of=rd, xshift=20pt] {$(s,n)$};
        \node[state,accepting]         (sd) [right of=rn, xshift=20pt] {$(s,d)$};
        
        \path (qn) edge  [bend left=20, near start] node {$b$} (qd)
                edge [bend left=20, near start] node {$a$} (rd)
            (qd) edge [bend left=20, near start] node {$b$} (qn)
                edge [bend right=20, near start] node {$a$} (rn)
            (rn) edge [bend left=20, near start] node {$a$} (rd)
                edge [bend left=20, near start] node {$b$} (sd)
            (rd) edge [bend left=20, near start] node {$a$} (rn)
                edge [bend left=20, near start] node {$b$} (sn)
            (sn) edge [bend left=20, near start] node {$a$} (rd)
                edge [bend left=20, near start] node {$b$} (sd)
            (sd) edge [bend left=20, near start] node {$a$} (rn)
                edge [bend left=20, near start] node {$b$} (sn)
        ;
        \end{tikzpicture}
    
    \newpage
    
    Suppose $A_1, A_2$ are languages over an alphabet $\Sigma$.
    {\bf Claim:} if there is a DFA $M_1$ such that $L(M_1) = A_1$ and 
    DFA $M_2$ such that $L(M_2) = A_2$, then there is another DFA, let's call it $M$, such that 
    $L(M) = A_1 \cap A_2$.  {\it Footnote to Sipser Theorem 1.25, page 46}
    
    {\bf Proof idea}:
    
    
    {\bf Formal construction}: 
    
    \vspace{70pt}


    


\newpage
\subsection*{Friday: Regular langauges}

%! app: Regular Languages
%! outcome: Regular expressions, Formal definition of automata, Informal definition of automata, Nondeterminism

So far we have that: 
\begin{itemize}
\item If there is a DFA recognizing a language, there is a DFA recognizing its complement.
\item If there are NFA recognizing two languages, there is a NFA recognizing their union.
\item If there are DFA recognizing two languages, there is a DFA recognizing their union.
\item If there are DFA recognizing two languages, there is a DFA recognizing their intersection.
\end{itemize}

Our goals for today are (1) prove similar results about other set operations, (2) prove that 
NFA and DFA are equally expressive, and therefore (3) define an important class of languages.

\vfill

\newpage
Suppose $A_1, A_2$ are languages over an alphabet $\Sigma$.
{\bf Claim:} if there is a NFA $N_1$ such that $L(N_1) = A_1$ and 
NFA $N_2$ such that $L(N_2) = A_2$, then there is another NFA, let's call it $N$, such that 
$L(N) = A_1 \circ A_2$.

{\bf Proof idea}: Allow computation to move between $N_1$ and $N_2$ ``spontaneously" when reach an accepting state of 
$N_1$, guessing that we've reached the point where the two parts of the string in the set-wise concatenation 
are glued together.


{\bf Formal construction}: Let 
$N_1 = (Q_1, \Sigma, \delta_1, q_1, F_1)$ and $N_2 = (Q_2, \Sigma, \delta_2,q_2, F_2)$
and assume $Q_1 \cap Q_2 = \emptyset$.
Construct $N = (Q, \Sigma, \delta, q_0, F)$ where
\begin{itemize}
    \item $Q = $
    \item $q_0 = $
    \item $F = $
    \item $\delta: Q \times \Sigma_\varepsilon \to \mathcal{P}(Q)$ is defined by, for $q \in Q$ and $a \in \Sigma_{\varepsilon}$:
        \[
            \delta((q,a))=\begin{cases}  
                \delta_1 ((q,a)) &\qquad\text{if } q\in Q_1 \textrm{ and } q \notin F_1\\ 
                \delta_1 ((q,a)) &\qquad\text{if } q\in F_1 \textrm{ and } a \in \Sigma\\ 
                \delta_1 ((q,a)) \cup \{q_2\} &\qquad\text{if } q\in F_1 \textrm{ and } a = \varepsilon\\ 
                \delta_2 ((q,a)) &\qquad\text{if } q\in Q_2
            \end{cases}
        \]
\end{itemize}

\vfill

{\it Proof of correctness would prove that $L(N) = A_1 \circ A_2$ by considering
an arbitrary string accepted by $N$, tracing an accepting computation of $N$ on it, and using 
that trace to prove the string can be written as the result of concatenating two strings, 
the first in $A_1$ and the second in $A_2$; then, taking an arbitrary 
string in $A_1 \circ A_2$ and proving that it is accepted by $N$. Details left for extra practice.}

\newpage



Suppose $A$ is a language over an alphabet $\Sigma$.
{\bf Claim:} if there is a NFA $N$ such that $L(N) = A$, then there is another NFA, let's call it $N'$, such that 
$L(N') = A^*$.

{\bf Proof idea}: Add a fresh start state, which is an accept state. Add spontaneous 
moves from each (old) accept state to the old start state.

{\bf Formal construction}: Let 
$N = (Q, \Sigma, \delta, q_1, F)$ and assume $q_0 \notin Q$.
Construct $N' = (Q', \Sigma, \delta', q_0, F')$ where
\begin{itemize}
    \item $Q' = Q \cup \{q_0\}$
    \item $F' = F \cup \{q_0\}$
    \item $\delta': Q' \times \Sigma_\varepsilon \to \mathcal{P}(Q')$ is defined by, for $q \in Q'$ and $a \in \Sigma_{\varepsilon}$:
        \[
            \delta'((q,a))=\begin{cases}  
                \delta ((q,a)) &\qquad\text{if } q\in Q \textrm{ and } q \notin F\\ 
                \delta ((q,a)) &\qquad\text{if } q\in F \textrm{ and } a \in \Sigma\\ 
                \delta ((q,a)) \cup \{q_1\} &\qquad\text{if } q\in F \textrm{ and } a = \varepsilon\\ 
                \{q_1\} &\qquad\text{if } q = q_0 \textrm{ and } a = \varepsilon \\
                \emptyset &\qquad\text{if } q = q_0 \textrm { and } a \in \Sigma
            \end{cases}
        \]
\end{itemize}


{\it Proof of correctness would prove that $L(N') = A^*$ by considering
an arbitrary string accepted by $N'$, tracing an accepting computation of $N'$ on it, and using 
that trace to prove the string can be written as the result of concatenating some number of strings, 
each of which is in $A$; then, taking an arbitrary 
string in $A^*$ and proving that it is accepted by $N'$. Details left for extra practice.}


{\bf Application}: A state diagram for a NFA over $\Sigma = \{a,b\}$ 
that recognizes $L (( a^*b)^* )$:

\vfill
\newpage
Suppose $A$ is a language over an alphabet $\Sigma$.
{\bf Claim:} if there is a NFA $N$ such that $L(N) = A$ then 
there is a DFA $M$ such that $L(M) = A$.

{\bf Proof idea}: States in $M$ are ``macro-states" -- collections of states from $N$ -- 
that represent the set of possible states a computation of $N$ might be in.


{\bf Formal construction}: Let $N = (Q, \Sigma, \delta, q_0, F)$.  Define 
\[
M = (~ \mathcal{P}(Q), \Sigma, \delta', q',  \{ X \subseteq Q \mid X \cap F \neq \emptyset \}~ )
\]
where $q' = \{ q \in Q \mid \text{$q = q_0$ or is accessible from $q_0$ by spontaneous moves in $N$} \}$
and 
\[
    \delta' (~(X, x)~) = \{ q \in Q \mid q \in \delta( ~(r,x)~) ~\text{for some $r \in X$ or is accessible 
from such an $r$ by spontaneous moves in $N$} \}
\]


Consider the state diagram of an NFA over $\{a,b\}$. Use the ``macro-state'' construction 
to find an equivalent DFA.

\begin{tikzpicture}[->,>=stealth',shorten >=1pt, auto, node distance=2cm, semithick]
    \tikzstyle{every state}=[text=black, fill=none]
    
    \node[initial,state] (q0)          {$q_0$};
    \node[state]         (q1) [right of=q0, xshift=20pt] {$q_1$};
    \node[state,accepting]         (q2) [right of=q1, xshift=20pt] {$q_2$};
    
    \path (q0) edge  [loop above] node {$a,b$} (q0)
            edge [bend left=0] node {$a$} (q1)
        (q1) edge [loop above] node {$a,b$} (q1)
            edge [bend left=0] node {$b$} (q2)
    ;
    \end{tikzpicture}

\vfill

Consider the state diagram of an NFA over $\{0,1\}$. Use the ``macro-state'' construction 
to find an equivalent DFA.

\begin{tikzpicture}[->,>=stealth',shorten >=1pt, auto, node distance=2cm, semithick]
    \tikzstyle{every state}=[text=black, fill=none]
    
    \node[initial,state] (q0)          {$q_0$};
    \node[state,accepting]         (q1) [above right of=q0, xshift=20pt] {$q_1$};
    \node[state,accepting]         (q2) [below right of=q0, xshift=20pt] {$q_2$};
    
    \path (q0)  edge [bend left=0] node {$\varepsilon$} (q1)
        (q0) edge [bend left = 0] node {$\varepsilon$} (q2)
        (q1) edge [loop above] node {$0$} (q1)
        (q2) edge [loop above] node {$1$} (q2)
    ;
    \end{tikzpicture}

\vfill

Note: We can often prune the DFAs that result from the ``macro-state'' constructions to get an 
equivalent DFA with fewer states (e.g.\ only the ``macro-states" reachable from the start state).

\newpage


{\bf The class of regular languages}

Fix an alphabet $\Sigma$. For each language $L$ over $\Sigma$:
\begin{center}
\begin{tabular}{cc}
    {\bf There is a DFA over $\Sigma$ that recognizes $L$}&$\exists M ~(M \textrm{ is a DFA and } L(M) = A)$\\
    {\it if and only if}&\\
    {\bf There is a NFA over $\Sigma$ that recognizes $L$}&$\exists N ~(N \textrm{ is a NFA and } L(N) = A)$\\
    {\it if and only if}&\\
    {\bf There is a regular expression over $\Sigma$ that describes $L$} &$\exists R ~(R \textrm{ is a regular expression and } L(R) = A)$\\
\end{tabular}
\end{center}

A language is called {\bf regular} when any (hence all) of the above three conditions are met.

We already proved that DFAs and NFAs are equally expressive. It remains to prove that regular expressions 
are too.

Part 1: Suppose $A$ is a language over an alphabet $\Sigma$.
If there is a regular expression $R$ such that $L(R) = A$, then there is a NFA, let's call it $N$, such that 
$L(N) = A$.

{\bf Structural induction}: Regular expression is built from basis regular expressions using inductive steps
(union, concatenation, Kleene star symbols). Use constructions to mirror these in NFAs.


{\bf Application}: A state diagram for a NFA over $\{a,b\}$ that recognizes $L(a^* (ab)^*)$:

\vfill

Part 2: Suppose $A$ is a language over an alphabet $\Sigma$.
If there is a DFA $M$ such that $L(M) = A$, then there is a regular expression, let's call it $R$, such that 
$L(R) = A$.

{\bf Proof idea}: Trace all possible paths from start state to accept state.  Express labels of these paths
as regular expressions, and union them all.

\begin{enumerate}
\item Add new start state with $\varepsilon$ arrow to old start state.
\item Add new accept state with $\varepsilon$ arrow from old accept states.  Make old accept states
non-accept.
\item Remove one (of the old) states at a time: modify regular expressions on arrows that went through removed
state to restore language recognized by machine.
\end{enumerate}

{\bf Application}: Find a regular expression describing the language recognized by the DFA with 
state diagram

\begin{tikzpicture}[->,>=stealth',shorten >=1pt, auto, node distance=2cm, semithick]
    \tikzstyle{every state}=[text=black, fill=none]
    
    \node[initial,state,accepting] (q0)          {$q_0$};
    \node[state,accepting]         (q1) [above right of=q0, xshift=20pt] {$q_1$};
    \node[state,accepting]         (q2) [below right of=q0, xshift=20pt] {$q_2$};
    \node[state]                   (q3) [below right of=q1, xshift=20pt] {$q_3$};
    
    \path (q0)  edge [bend left=0] node {$a$} (q1)
        (q0) edge [bend left = 0] node {$b$} (q2)
        (q1) edge [loop above] node {$a$} (q1)
        (q1) edge [bend left=0] node {$b$} (q3)
        (q2) edge [loop below] node {$b$} (q2)
        (q2) edge [bend left=0] node {$a$} (q3)
        (q3) edge [loop above] node {$a,b$} (q3)
    ;
    \end{tikzpicture}

\vfill

\newpage

\subsection*{Week 4 at a glance}

\subsubsection*{Textbook reading: Section 1.4, 2.2, 2.1.}

\vspace{-15pt}

Before Monday, read Introduction to Section 1.4 (page 77) which introduces nonregularity.

Before Wednesday, read Definition 2.13 (page 111-112) introducing Pushdown Automata.

Before Friday, read Example 2.18 (page 114).

For Week 5 Monday: read Introduction to Section 2.1 (pages 101-102).

\vspace{-20pt}

\subsubsection*{We will be learning and practicing to:}

\begin{itemize}
    \item Clearly and unambiguously communicate computational ideas using appropriate formalism. Translate across levels of abstraction.
    \begin{itemize}
       \item Give examples of sets that are regular (and prove that they are).
       \begin{itemize}
          \item {\bf State the definition of the class of regular languages}
          \item {\bf Explain the limits of the class of regular languages}
          \item {\bf Identify some regular sets and some nonregular sets}
       \end{itemize}
       \item Use precise notation to formally define the state diagram of a PDA
       \item Use clear English to describe computations of PDA informally.
       \begin{itemize}
           \item {\bf Define push-down automata informally and formally}
           \item {\bf State the formal definition of a PDA}
           \item {\bf Trace the computation(s) of a PDA on a given string using its state diagram}
           \item {\bf Determine if a given string is in the language recognized by a PDA}
           \item {\bf Translate between a state diagram and a formal definition of a PDA}
           \item {\bf Determine the language recognized by a given PDA}
        \end{itemize}

    \end{itemize}

    \item Know, select and apply appropriate computing knowledge and problem-solving techniques. 
    \begin{itemize}
    \item Apply classical techniques including pumping lemma, determinization, diagonalization, and reduction to analyze the complexity of languages and problems.
    \begin{itemize}
        \item {\bf Justify why the Pumping Lemma is true.}
        \item {\bf Use the pumping lemma to prove that a given language is not regular.}
    \end{itemize}
    \end{itemize}
\end{itemize}

\vspace{-20pt}

\subsubsection*{TODO:}
\begin{list}{\itemsep-10pt}
   \item Schedule your Test 1 Attempt 1, Test 2 Attempt 1, Test 1 Attempt 2, and Test 2 Attempt 2 times 
   at PrairieTest (http://us.prairietest.com)
   \item Review Quiz 3 on PrairieLearn (http://us.prairielearn.com), due 1/29/2025
   \item Homework 2 submitted via Gradescope (https://www.gradescope.com/), due 1/30/2025
   \item Review Quiz 4 on PrairieLearn (http://us.prairielearn.com), due 2/5/2025
\end{list}

\newpage
\subsection*{Monday: Pumping Lemma}

%! app: Regular Languages
%! outcome: Classify language, Find example languages, Pumping Lemma

{\bf Definition and Theorem}: For an alphabet $\Sigma$, a language $L$ over $\Sigma$ is called {\bf regular}
exactly when $L$ is recognized by some DFA, which happens exactly when $L$ is recognized by some NFA, 
and happens exactly when $L$ is described by some regular expression

{\bf We saw that}: The class of regular languages is closed under complementation, union, 
intersection, set-wise concatenation, and Kleene star.

{\it Extra practice}: 

{\bf Disprove}: There is some alphabet $\Sigma$ for which there is 
some language recognized by an NFA but not by any DFA.

\vfill

{\bf Disprove}: There is some alphabet $\Sigma$ for which there is 
some finite language not described by any regular expression over $\Sigma$.

\vfill

{\bf Disprove}: If a language is recognized by an NFA 
then the complement of this language is not recognized by any DFA.

\vfill


{\bf Fix alphabet $\Sigma$. Is every language $L$ over $\Sigma$ regular?}

\begin{center}
\begin{tabular}{c|c}
Set & Cardinality \\
\hline
& \\
$\{0,1\}$ & \\
& \\
$\{0,1\}^*$ & \\
& \\
$\mathcal{P}( \{0,1\})$ & \\
& \\
The set of all languages over $\{0,1\}$ & \\
& \\
The set of all regular expressions over $\{0,1\}$ & \\
& \\
The set of all regular languages over $\{0,1\}$ & \\
& \\
\end{tabular}
\end{center}

\newpage

Strategy: Find an {\bf invariant} property that is true of all regular languages. When analyzing 
a given language, if the invariant is not true about it, then the language is not regular.

\vfill

{\bf Pumping Lemma} (Sipser Theorem 1.70): If $A$ is a regular language, then there
is a number $p$ (a {\it pumping length}) where, if $s$ is any string in $A$ of length at least $p$, 
then $s$ may be divided into three pieces, $s = xyz$ such that
\vspace{-10pt}
\begin{itemize}
\item $|y| > 0$
\item for each $i \geq 0$, $xy^i z \in A$
\item $|xy| \leq p$.
\end{itemize}
\vfill

{\bf Proof idea}: In DFA, the only memory available is in the states. 
Automata can only
``remember'' finitely far in the past and finitely much information, because
they can have only finitely many states. If a computation path of a DFA visits 
the same state more than once, the machine can't tell the difference between 
the first time and future times it visits this state. Thus, if 
a DFA accepts one long string, then it must accept (infinitely) many 
similar strings.
\vfill

{\bf Proof illustration}


\vfill
\vfill


\newpage

{\bf True or False}: A pumping length for $A = \{ 0,1 \}^*$ is $p = 5$.

\vfill


{\bf True or False}: A pumping length for $A = \{ 0,1 \}^*$ is $p = 2$.

\vfill



{\bf True or False}: A pumping length for $A = \{ 0,1 \}^*$ is $p = 105$.

\vfill



Restating {\bf Pumping Lemma}: If $L$ is a regular language, then it  has
a pumping length.


{\bf Contrapositive}: If $L$ has no pumping length, then  it is nonregular.

\vfill

{\Large The Pumping Lemma {\it cannot} be used to prove that a language {\it is} regular.} 

{\Large The Pumping Lemma {\bf can} be used to prove that a language {\it is not} regular.}

{\it Extra practice}: Exercise 1.49 in the book.


\vfill

{\bf Proof strategy}: To prove that a language $L$ is {\bf not} regular, 
\begin{itemize}
    \item Consider an arbitrary positive integer $p$
    \item Prove that $p$ is not a pumping length for $L$
    \item Conclude that $L$ does not have {\it any} pumping length, and therefore it is not regular.
\end{itemize}


{\bf Negation}: A positive integer  $p$  is {\bf not a pumping length} of a language  $L$ over  $\Sigma$  iff
\[
\exists s \left(~  |s| \geq  p \wedge s \in L \wedge \forall x \forall y \forall z  \left( ~\left( s = xyz \wedge 
|y| > 0 \wedge |xy| \leq p~ \right) \to \exists i  (  i \geq 0  \wedge xy^iz  \notin L ) \right) ~\right) 
\]

\newpage
\subsection*{Wednesday: Proving nonregularity, and beyond}

%! app: Regular Languages
%! outcome: Classify language, Find example languages, Pumping Lemma


{\bf Proof strategy}: To prove that a language $L$ is {\bf not} regular, 
\begin{itemize}
    \item Consider an arbitrary positive integer $p$
    \item Prove that $p$ is not a pumping length for $L$. A positive integer  $p$  is {\bf not a pumping length} of a language  $L$ over  $\Sigma$  iff
    \[
    \exists s \left(~  |s| \geq  p \wedge s \in L \wedge \forall x \forall y \forall z  \left( ~\left( s = xyz \wedge 
    |y| > 0 \wedge |xy| \leq p~ \right) \to \exists i  (  i \geq 0  \wedge xy^iz  \notin L ) \right) ~\right) 
    \]
    {\it Informally: }
    
\vfill

    \item Conclude that $L$ does not have {\it any} pumping length, and therefore it is not regular.
\end{itemize}

\vfill
{\bf Example}: $\Sigma  =  \{0,1\}$, $L = \{ 0^n 1^n \mid n  \geq 0\}$.

Fix $p$ an arbitrary positive integer. List strings that are in $L$ and have length  greater than or equal  to $p$:

\vspace{20pt}

Pick $s = $


Suppose $s = xyz$ with  $|xy|  \leq  p$ and $|y| > 0$.
\begin{center}
\begin{tabular}{|c|}
\hline
 \\
\hspace{4in} \\
\hline
\end{tabular}
\end{center}

Then when $i = \hspace{1in}$, $xy^i z  = \hspace{1in}$

\vfill

\newpage

{\bf Example}: $\Sigma  =  \{0,1\}$, $L = \{w w^{\mathcal{R}} \mid w \in \{0,1\}^*\}$.
Remember that the reverse of a string $w$ is denoted $w^\mathcal{R}$  
and means to write $w$  in  the opposite order, if $w = w_1 \cdots  w_n$ then $w^\mathcal{R} = w_n \cdots  w_1$. Note: $\varepsilon^\mathcal{R} = \varepsilon$.


Fix $p$ an arbitrary positive integer. List strings that are in $L$ and have length  greater than or equal  to $p$:

\vspace{10pt}

Pick $s = $

Suppose $s = xyz$ with  $|xy|  \leq  p$ and $|y| > 0$.
\begin{center}
\begin{tabular}{|c|}
\hline
 \\
\hspace{4in} \\
\hline
\end{tabular}
\end{center}
Then when $i = \hspace{1in}$, $xy^i z  = \hspace{1in}$


\vfill


{\bf Example}: $\Sigma  =  \{0,1\}$, $L = \{0^j1^k  \mid j \geq k  \geq 0\}$.

Fix $p$ an arbitrary positive integer. List strings that are in $L$ and have length  greater than or equal  to $p$:

\vspace{10pt}

Pick $s = $


Suppose $s = xyz$ with  $|xy|  \leq  p$ and $|y| > 0$.
\begin{center}
\begin{tabular}{|c|}
\hline
 \\
\hspace{4in} \\
\hline
\end{tabular}
\end{center}
Then when $i = \hspace{1in}$, $xy^i z  = \hspace{1in}$


\vfill


{\bf Example}: $\Sigma  =  \{0,1\}$, $L = \{0^n1^m0^n  \mid m,n  \geq 0\}$.

Fix $p$ an arbitrary positive integer. List strings that are in $L$ and have length  greater than or equal  to $p$:

\vspace{10pt}

Pick $s = $


Suppose $s = xyz$ with  $|xy|  \leq  p$ and $|y| > 0$.
\begin{center}
\begin{tabular}{|c|}
\hline
 \\
\hspace{4in} \\
\hline
\end{tabular}
\end{center}
Then when $i = \hspace{1in}$, $xy^i z  = \hspace{1in}$

\newpage
{\it Extra practice}:


\begin{center}
    \begin{tabular}{c|c| c| c}
    Language & $s \in L$ & $s \notin L$ & Is the language regular or nonregular?  \\
    \hline
     & \hspace{1in} & \hspace{1in}  &  \\
    $\{a^nb^n \mid 0  \leq n  \leq 5 \}$ & & & \\
     & & & \\
    $\{b^n a^n \mid  n  \geq 2\}$  & & & \\
     & & & \\
    $\{a^m b^n \mid  0 \leq m\leq n\}$  & & & \\
     & & & \\
    $\{a^m b^n \mid  m \geq n+3,  n \geq 0\}$  & & & \\
     & & & \\
    $\{b^m a^n \mid  m \geq 1, n \geq  3\}$  & & & \\
     & & & \\
    $\{ w  \in \{a,b\}^* \mid w = w^\mathcal{R} \}$ & & & \\
     & & & \\ 
    $\{ ww^\mathcal{R} \mid w\in \{a,b\}^* \}$ & & & \\
     & & & \\ 
    \end{tabular}
\end{center}
    
    
\newpage
\subsection*{Friday: Pushdown Automata}

%! app: Regular Languages, Context-free Languages
%! outcome: Formal definition of automata, Informal definition of automata, Nondeterminism, Classify language

Regular sets are not the end of the story
\begin{itemize}
    \item Many nice / simple / important sets are not regular
    \item Limitation of the finite-state automaton model: Can't ``count", Can only remember finitely far into the past,
    Can't backtrack, Must make decisions in ``real-time"
    \item We know actual computers are more powerful than this model...
\end{itemize}

The {\bf next} model of computation. Idea: allow some memory of unbounded size. How? 
\begin{itemize}
    \item To generalize regular expressions: {\bf context-free grammars}\\
    \item To generalize NFA: {\bf Pushdown automata}, which is like an NFA with access to a stack: 
    Number of states is fixed, number of entries in stack is unbounded. At each step
    (1) Transition to new state based on current state, letter read, and top letter of stack, then
    (2) (Possibly) push or pop a letter to (or from) top of stack. Accept a string iff
    there is some sequence of states and some sequence of stack contents 
    which helps the PDA processes the entire input string and ends in an accepting state.
\end{itemize}

\vfill

\vfill

Is there a PDA that recognizes the nonregular language $\{0^n1^n \mid n \geq 0 \}$?

\vfill

\newpage

\begin{tikzpicture}[->,>=stealth',shorten >=1pt, auto, node distance=2cm, semithick]
    \tikzstyle{every state}=[text=black, fill=none]
    
    \node[initial,state,accepting] (q0)          {$q0$};
    \node[state]         (q1) [right of=q0, xshift=20pt] {$q1$};
    \node[state]         (q2) [below right of=q1, xshift=20pt] {$q2$};
    \node[state,accepting]         (q3) [left of=q2, xshift=-20pt] {$q3$};
    
    \path (q0) edge [bend left=0] node {$\varepsilon, \varepsilon; \$$} (q1)
        (q1) edge  [loop above] node {$0, \varepsilon; 0$} (q1)
        (q1) edge [bend left=0] node {$1, 0; \varepsilon$} (q2)
        (q2) edge  [loop right] node {$1, 0; \varepsilon$} (q2)
        (q2) edge  [bend left=0] node {$\varepsilon, \$; \varepsilon$} (q3)
    ;
\end{tikzpicture}

The PDA with state diagram above can be informally described as:
\begin{quote}
    Read symbols from the input. As each 0 is read, push it onto the stack. 
    As soon as 1s are seen, pop a 0 off the stack for each 1 read. 
    If the stack becomes empty and we are at the end of the input string, accept the input. 
    If the stack becomes empty and there are 1s left to read, 
    or if 1s are finished while the stack still contains 0s, or if any 0s
    appear in the string following 1s, 
    reject the input.
\end{quote}
    

Trace a computation of this PDA on the input string $01$.

\vfill
    
{\it Extra practice}: Trace the computations of this PDA on the input string $011$.

\vfill

\newpage
A PDA recognizing the set $\{ \hspace{1.5 in} \}$ can be informally described as:
\begin{quote}
    Read symbols from the input. As each 0 is read, push it onto the stack. 
    As soon as 1s are seen, pop a 0 off the stack for each 1 read. 
    If the stack becomes empty and there is exactly one 1 left to read, read that 1 and accept the input. 
    If the stack becomes empty and there are either zero or more than one 1s left to read, 
    or if the 1s are finished while the stack still contains 0s, or if any 0s appear in the input following 1s, 
    reject the input.
\end{quote}
Modify the state diagram below to get a PDA that implements this description:

\begin{tikzpicture}[->,>=stealth',shorten >=1pt, auto, node distance=2cm, semithick]
    \tikzstyle{every state}=[text=black, fill=none]
    
    \node[initial,state,accepting] (q0)          {$q0$};
    \node[state]         (q1) [right of=q0, xshift=20pt] {$q1$};
    \node[state]         (q2) [below right of=q1, xshift=20pt] {$q2$};
    \node[state,accepting]         (q3) [left of=q2, xshift=-20pt] {$q3$};
    
    \path (q0) edge [bend left=0] node {$\varepsilon, \varepsilon; \$$} (q1)
        (q1) edge  [loop above] node {$0, \varepsilon; 0$} (q1)
        (q1) edge [bend left=0] node {$1, 0; \varepsilon$} (q2)
        (q2) edge  [loop right] node {$1, 0; \varepsilon$} (q2)
        (q2) edge  [bend left=0] node {$\varepsilon, \$; \varepsilon$} (q3)
    ;
\end{tikzpicture}
    



\newpage

\subsection*{Week 5 at a glance}

\subsubsection*{Textbook reading: Chapter 2}

\vspace{-20pt}

Before Monday, read Introduction to Section 2.1 (pages 101-102).

Before Wednesday, read Section 2.1

Before Friday, read Theorem 2.20.

For Week 6 Monday: Page 165-166 Introduction to Section 3.1.

\vspace{-20pt}

\subsubsection*{We will be learning and practicing to:}
\vspace{-20pt}

\begin{itemize}
    \item Clearly and unambiguously communicate computational ideas using appropriate formalism. Translate across levels of abstraction.
    \begin{itemize}
        \item Describe and use models of computation that don't involve state machines.
        \begin{itemize}
            \item {\bf Identify the components of a formal definition of a context-free grammar (CFG)}
            \item {\bf Derive strings in the language of a given CFG}
            \item {\bf Determine the language of a given CFG}
            \item {\bf Design a CFG generating a given language}
            \item {\bf Use context-free grammars and relate them to languages and pushdown automata.}
        \end{itemize}
        \item Use precise notation to formally define the state diagram of a Turing machine
        \item Use clear English to describe computations of Turing machines informally.
        \begin{itemize}
            \item {\bf Design a PDA that recognizes a given language.}
         \end{itemize}
       \item Give examples of sets that are context-free (and prove that they are).
       \begin{itemize}
          \item {\bf State the definition of the class of context-free languages}
          \item {\bf Explain the limits of the class of context-free languages}
          \item {\bf Identify some context-free sets and some non-context-free sets}
       \end{itemize}
    \end{itemize}
    \item Know, select and apply appropriate computing knowledge and problem-solving techniques. 
    Reason about computation and systems.
    \begin{itemize}
        \item Describe and prove closure properties of classes of languages under certain operations.
        \begin{itemize}
            \item {\bf Apply a general construction to create a new PDA or CFG from an example one.}
            \item {\bf Formalize a general construction from an informal description of it.}
            \item {\bf Use general constructions to prove closure properties of the class of context-free langugages.}
            \item {\bf Use counterexamples to prove non-closure properties of the class of context-free langugages.}
        \end{itemize}
    \end{itemize}
\end{itemize}

\vspace{-20pt}

\subsubsection*{TODO:}
\begin{list}{\itemsep-10pt}
   \item Schedule your Test 1 Attempt 1, Test 2 Attempt 1, Test 1 Attempt 2, and Test 2 Attempt 2 times
   at PrairieTest (http://us.prairietest.com) . The first Test 1 sessions are next week!
   \item Review Quiz 4 on PrairieLearn (http://us.prairielearn.com), due 2/5/2025
   \item Homework 3 submitted via Gradescope (https://www.gradescope.com/), due 2/6/2025
   \item Review Quiz 5 on PrairieLearn (http://us.prairielearn.com), due 2/12/2025
\end{list}

\newpage



\newpage
\subsection*{Monday: More Pushdown Automata}

%! app: Regular Languages, Context-free Languages
%! outcome: Formal definition of automata, Informal definition of automata, Nondeterminism, Classify language, Find example languages
    

{\bf Definition} A {\bf pushdown automaton} (PDA) is  specified by a  $6$-tuple $(Q, \Sigma, \Gamma, \delta, q_0, F)$
where $Q$ is the finite set of states, $\Sigma$ is the input alphabet,  $\Gamma$ is the stack alphabet,
\[
    \delta: Q \times \Sigma_\varepsilon  \times  \Gamma_\varepsilon \to \mathcal{P}( Q \times \Gamma_\varepsilon)
\]
is the transition function,  $q_0 \in Q$ is the start state, $F \subseteq  Q$ is the set of accept states.


%Draw the state diagram and give the formal definition of a PDA with $\Sigma = \Gamma$.

%\vfill

%Draw the state diagram and give the formal definition of a PDA with $\Sigma \cap \Gamma = \emptyset$.
    
%\vfill

%\newpage
For the PDA state diagrams below, $\Sigma = \{0,1\}$.

\vspace{-15pt}

\begin{center}
\begin{tabular}{c c}
Mathematical description of language & State diagram of PDA recognizing language\\
\hline
& $\Gamma = \{ \$, \#\}$ \hspace{2.3in} \\
& \\
& 
\begin{tikzpicture}[->,>=stealth',shorten >=1pt, auto, node distance=2cm, semithick]
    \tikzstyle{every state}=[text=black, fill=none]
    
    \node[initial,state] (q0)          {$q0$};
    \node[state]         (q1) [right of=q0, xshift=20pt] {$q1$};
    \node[state]         (q2) [below right of=q1, xshift=20pt] {$q2$};
    \node[state]         (q3) [right of=q2, xshift=20pt] {$q3$};
    \node[state,accepting]         (q4) [left of=q2, xshift=-20pt] {$q4$};
    
    \path (q0) edge [bend left=0] node {$\varepsilon, \varepsilon; \$$} (q1)
        (q1) edge  [loop above] node {$0, \varepsilon; \#$} (q1)
        (q1) edge [bend left=0] node {$\varepsilon, \varepsilon; \varepsilon$} (q2)
        (q2) edge  [bend left=20] node [midway, above] {$1, \#; \varepsilon$} (q3)
        (q3) edge  [bend left=20] node [midway, below] {$1, \varepsilon; \varepsilon$} (q2)
        (q2) edge  [bend left=0] node {$\varepsilon, \$; \varepsilon$} (q4)
    ;
\end{tikzpicture}
\\
& \\
& \\
\hline
& $\Gamma = \{ \sun, 1\}$ \hspace{2.3in} \\
& \\
& 
\begin{tikzpicture}[->,>=stealth',shorten >=1pt, auto, node distance=2cm, semithick]
    \tikzstyle{every state}=[text=black, fill=none]
    
    \node[initial,state] (q0)          {$q0$};
    \node[state]         (q1) [right of=q0, xshift=20pt] {$q1$};
    \node[state]         (q2) [below right of=q1, xshift=20pt] {$q2$};
    \node[state]         (q3) [right of=q2, xshift=20pt] {$q3$};
    \node[state,accepting]  (q4) [right of=q3, xshift=20pt] {$q4$};
    \node[state]         (q5) [above right of=q1, xshift=20pt] {$q5$};
    \node[state,accepting]  (q6) [right of=q5, xshift=20pt] {$q6$};
    
    \path (q0) edge [bend left=0] node {$\varepsilon, \varepsilon; \sun$} (q1)
        (q1) edge  [loop above] node {$1, \varepsilon; 1$} (q1)
        (q1) edge [bend left=0] node [below] {$\varepsilon, \varepsilon; \varepsilon$} (q5)
        (q1) edge [bend left=0] node [above]{$\varepsilon, \varepsilon; \varepsilon$} (q2)
        (q5) edge [loop above] node {$0, 1, ; \varepsilon$} (q5)
        (q5) edge [bend left=0] node {$\varepsilon, \sun; \varepsilon$} (q6)
        (q6) edge [loop above] node {$1, \varepsilon; \varepsilon$} (q6)
        (q2) edge [loop below] node {$0, \varepsilon; \varepsilon$} (q2)
        (q2) edge  [bend left=0] node {$\varepsilon, \varepsilon; \varepsilon$} (q3)
        (q3) edge  [loop below] node {$1, 1; \varepsilon$} (q3)
        (q3) edge  [bend left=0] node {$\varepsilon, \sun; \varepsilon$} (q4)
    ;
\end{tikzpicture}
\\
& \\
& \\
\hline
& \\
& \\
& \\
$\{ 0^i 1^j 0^k \mid i,j,k \geq 0 \}$ & \\
& \\
& \\

\end{tabular}
\end{center}
{\it Note: alternate notation is to replace $;$ with $\to$ on arrow labels.}

\newpage
Corollary: for each language $L$ over $\Sigma$, if there is an NFA $N$ with $L(N)=L$
then there is a PDA $M$ with $L(M) = L$

Proof idea: Declare stack alphabet to be $\Gamma = \Sigma$ and then 
don't use stack at all. 


 \vfill

\begin{comment}
{\it Extra practice}: Consider the state diagram of a PDA with input alphabet 
$\Sigma$ and stack alphabet $\Gamma$.

\begin{center}
\begin{tabular}{|c|c|}
\hline
Label & means \\
\hline
$a, b ; c$ when $a \in \Sigma$, $b\in \Gamma$, $c \in \Gamma$ 
& \hspace{3in} \\
& \\
& \\
& \\
& \\
&\\
\hline
$a, \varepsilon ; c$ when $a \in \Sigma$, $c \in \Gamma$ 
& \hspace{3in} \\
& \\
& \\
& \\
& \\
&\\
\hline
$a, b ; \varepsilon$ when $a \in \Sigma$, $b\in \Gamma$
& \hspace{3in} \\
& \\
& \\
& \\
& \\
&\\
\hline
$a, \varepsilon ; \varepsilon$ when $a \in \Sigma$
& \hspace{3in} \\
& \\
& \\
& \\
& \\
&\\
\hline
\end{tabular}
\end{center}


How does the meaning change if $a$ is replaced by $\varepsilon$?
\end{comment}

{\it Big picture}: PDAs are motivated by wanting to add some memory of unbounded size to NFA. How 
do we accomplish a similar enhancement of regular expressions to get a syntactic model that is 
more expressive?

DFA, NFA, PDA: Machines process one input string at a time; the computation of a machine on its input string 
reads the input from left to right.

Regular expressions: Syntactic descriptions of all strings that match a particular pattern; the language 
described by a regular expression is built up recursively according to the expression's syntax

{\bf Context-free grammars}: Rules to produce one string at a time, adding characters from the middle, beginning, 
or end of the final string as the derivation proceeds.\\

\vfill


\newpage
\section*{Wednesday: Context-free Grammars and Languages}

%! app: context-free languages
%! outcome: Classify language, Find example languages, context-free grammars



Definitions below are on pages 101-102.

\vspace{-20pt}

\begin{center}
    \begin{tabular}{|p{2.4in}cp{3.6in}|}
    \hline 
    {\bf Term} & {\bf Typical symbol} & {\bf Meaning} \\
     & or {\bf Notation} & \\
    \hline
    \hline
    {\bf Context-free grammar} (CFG) & $G$ & $G = (V, \Sigma, R, S)$ \\
    The set of {\bf variables}& $V$ & Finite  set of symbols that represent phases in production pattern\\
    The set of {\bf terminals} & $\Sigma$ & Alphabet of symbols of strings generated  by CFG \\
    & & $V \cap \Sigma = \emptyset$ \\
    The set of {\bf rules}& $R$ & Each rule is  $A \to u$ with $A \in V$ and $u  \in (V  \cup \Sigma)^*$\\
    The {\bf start} variable&  $S$  & Usually  on left-hand-side of first/ topmost rule \\
    & &\\
    {\bf Derivation} & $S \Rightarrow \cdots \Rightarrow w$& 
    Sequence  of substitutions in a  CFG (also written $S \Rightarrow^* w$). At each step, we can apply one rule 
    to one occurrence of a variable in the current string by substituting that occurrence of the variable with the 
    right-hand-side of the rule. The derivation must end when the current string has only terminals (no variables)
    because then there are no instances of variables to apply a rule to.\\
    Language {\bf generated} by the context-free grammar $G$ & $L(G)$ &The set of strings for which there is a derivation in $G$. 
    Symbolically: $\{  w \in \Sigma^* \mid S \Rightarrow^* w \}$ i.e. $$\{  w \in \Sigma^* \mid \text{there is  derivation in $G$ that ends
    in $w$} \}$$\\
    {\bf Context-free language} & & A language that is the language generated by some context-free grammar\\
    \hline
    \end{tabular}
\end{center}

\vfill

  
{\bf Examples of context-free grammars, derivations in those grammars, and the languages generated by those grammars}
  
$G_1 =  (\{S\}, \{0\}, R, S)$ with rules
  \begin{align*}
    &S \to 0S\\
    &S \to 0\\
  \end{align*}
  In  $L(G_1)$ \ldots 
  
\vfill

  Not in $L(G_1)$ \ldots 


\vfill

\newpage
  $G_2 =  (\{S\}, \{0,1\}, R, S)$
  \[
  S \to 0S \mid 1S \mid \varepsilon
  \]
  In  $L(G_2)$ \ldots 
  
  \vfill
  
  Not in $L(G_2)$ \ldots 

  \vfill

  $(\{S, T\}, \{0, 1\}, R, S)$ with  rules
  \begin{align*}
  &S \to T1T1T1T \\
  &T \to  0T \mid 1T \mid \varepsilon
  \end{align*}

  In  $L(G_3)$ \ldots 
  
  \vfill
  
  Not in $L(G_3)$ \ldots 

  \vfill

  $G_4 =  (\{A, B\}, \{0, 1\}, R, A)$ with rules
  \[
    A \to 0A0 \mid  0A1 \mid 1A0  \mid 1A1 \mid  1
  \]
  In  $L(G_4)$ \ldots 
  
  \vfill
  
  Not in $L(G_4)$ \ldots 

  \vfill

  
  \newpage
  Design a CFG to generate the  language $\{a^n b^n \mid  n  \geq  0\}$
  
  
  \vfill 


  Design a CFG to generate the  language $\{a^i b^j \mid j \geq i  \geq 0\}$
  
  \vfill
  
  Design a PDA to recognize the  language $\{a^i b^j \mid j \geq i  \geq 0\}$
  
  
  \vfill
  \newpage
    
\newpage
\subsection*{Friday: Context-free and non-context-free languages}

%! app: Regular Languages, context-free languages
%! outcome: Classify language, Find example languages, context-free grammars, Formal definition of automata, Informal definition of automata


{\bf Theorem  2.20}: A language is  generated by some context-free  grammar
if  and only if it is recognized by some push-down automaton.

Definition: a language is called {\bf context-free} if it is the language generated by a context-free grammar.
The class of all context-free language over a given alphabet $\Sigma$ is called {\bf CFL}.

Consequences:
\begin{itemize}
    \item Quick proof that every regular language is context free 
    \item To prove closure of the class of context-free languages under a given operation, we can choose 
    either of two modes 
    of proof (via CFGs or PDAs) depending on which is easier
    \item To fully specify a PDA we could give its $6$-tuple formal definition or we could give its input 
alphabet, stack alphabet, and state diagram.
An informal description of a PDA is a step-by-step description of how its computations 
would process input strings; the reader should be able to reconstruct the state diagram or formal 
definition precisely from such a descripton. The informal description of a PDA can refer to some 
common modules or subroutines that are computable by PDAs:
\begin{itemize}
  \item PDAs can ``test for emptiness of stack'' without providing details. 
  {\it How?} We can always push a special end-of-stack symbol, $\$$, at the start, before processing
  any input, and then use this symbol as a flag.
  \item PDAs can ``test for end of input'' without providing details.
  {\it How?} We can transform a PDA to one where accepting states are only those reachable 
  when there are no more input symbols.
\end{itemize}

\end{itemize}




\vfill

\newpage
Suppose $L_1$ and $L_2$ are context-free languages over $\Sigma$.  {\bf Goal}:  $L_1 \cup L_2$  is  also context-free.

{\it Approach 1: with  PDAs}

Let $M_1 = ( Q_1, \Sigma, \Gamma_1, \delta_1, q_1, F_1)$ and
$M_2 = ( Q_2, \Sigma, \Gamma_2, \delta_2, q_2, F_2)$ be PDAs with 
$L(M_1) =  L_1$  and  $L(M_2) = L_2$.

Define $M = $

\vfill

{\it Approach  2: with CFGs}

Let $G_1 = (V_1, \Sigma, R_1, S_1)$  and   $G_2 = (V_2, \Sigma, R_2, S_2)$  be CFGs  with
$L(G_1) =  L_1$  and  $L(G_2) = L_2$.

Define $G = $

\vfill

\newpage
Suppose $L_1$ and $L_2$ are context-free languages over $\Sigma$.  {\bf Goal}:  $L_1 \circ L_2$  is  also context-free.


{\it Approach 1: with  PDAs}

Let $M_1 = ( Q_1, \Sigma, \Gamma_1, \delta_1, q_1, F_1)$ and
$M_2 = ( Q_2, \Sigma, \Gamma_2, \delta_2, q_2, F_2)$ be PDAs with 
$L(M_1) =  L_1$  and  $L(M_2) = L_2$.

Define $M = $

\vfill

{\it Approach  2: with CFGs}

Let $G_1 = (V_1, \Sigma, R_1, S_1)$  and   $G_2 = (V_2, \Sigma, R_2, S_2)$  be CFGs  with
$L(G_1) =  L_1$  and  $L(G_2) = L_2$.

Define $G = $

\vfill
\newpage

{\it Summary}

Over a fixed alphabet $\Sigma$, a language $L$ is {\bf regular}

\vspace{-20pt}
\begin{center}
    iff it is described by some regular expression \\
    iff it is recognized by some DFA\\
    iff it is recognized by some NFA
\end{center}

Over a fixed alphabet $\Sigma$, a language $L$ is {\bf context-free}

\vspace{-20pt}
\begin{center}
    iff it is generated by some CFG\\
    iff it is recognized by some PDA
\end{center}

\vfill

{\bf Fact}: Every regular language is a context-free language.

\vfill

{\bf Fact}: There are context-free languages that are nonregular.

\vfill

{\bf Fact}: There are countably many regular languages.

\vfill

{\bf Fact}: There are countably infinitely many context-free languages.

\vfill

{\it Consequence}: Most languages are {\bf not} context-free!

\vfill

\newpage
{\bf Examples  of non-context-free languages}

\begin{align*}
    &\{ a^n b^n c^n \mid 0 \leq n , n \in \mathbb{Z}\}\\
    &\{ a^i b^j c^k \mid 0 \leq i \leq j \leq k , i \in \mathbb{Z}, j \in \mathbb{Z}, k \in \mathbb{Z}\}\\
    &\{ ww \mid w \in \{0,1\}^* \}
\end{align*}
(Sipser Ex 2.36, Ex 2.37, 2.38)

There is a Pumping Lemma for CFL that can be used to prove a specific language is non-context-free: 
If $A$ is a context-free language, there
is a number $p$ where, if $s$ is any string in $A$ of length at least $p$, then $s$ may be divided 
into five pieces $s = uvxyz$ where (1) for each $i \geq 0$, $uv^ixy^iz \in A$, (2) $|uv|>0$, (3) $|vxy| \leq p$.
{\it We will not go into the details of the proof or application of Pumping Lemma for CFLs this quarter.}


Recall: A set $X$ is said to be {\bf closed} under an operation $OP$ if, for any elements in $X$, applying 
$OP$ to them gives an element in $X$.  


\begin{center}
\begin{tabular}{|c|l|}
\hline
True/False & Closure claim \\
\hline
True &  The set of integers is closed under multiplication. \\
& $\forall x \forall y \left( ~(x \in \mathbb{Z} \wedge y \in \mathbb{Z})\to xy \in \mathbb{Z}~\right)$ \\
\hline
True & For each set $A$, the power set of $A$ is closed under intersection. \\
& $\forall A_1 \forall A_2 \left( ~(A_1 \in \mathcal{P}(A) \wedge A_2 \in \mathcal{P}(A) \in \mathbb{Z}) \to A_1 \cap A_2 \in \mathcal{P}(A)~\right)$ \\
\hline
  & The class of regular languages over $\Sigma$ is closed under complementation. \\
  & \\
 \hline
  & The class of regular languages over $\Sigma$ is closed under union. \\
  & \\
 \hline
  & The class of regular languages over $\Sigma$ is closed under intersection. \\
  & \\
  \hline
  & The class of regular languages over $\Sigma$ is closed under concatenation. \\
  & \\
 \hline
  & The class of regular languages over $\Sigma$ is closed under Kleene star. \\
  & \\
\hline
    & The class of context-free languages over $\Sigma$ is closed under complementation. \\
  & \\
\hline
    & The class of context-free languages over $\Sigma$ is closed under union. \\
  & \\
\hline
    & The class of context-free languages over $\Sigma$ is closed under intersection. \\
  & \\
\hline
    & The class of context-free languages over $\Sigma$ is closed under concatenation. \\
  & \\
\hline
    & The class of context-free languages over $\Sigma$ is closed under Kleene star. \\
  & \\
\hline
\end{tabular}
\end{center}





\newpage

\subsection*{Week 6 at a glance}

\vspace{-10pt}

\subsubsection*{Textbook reading: Chapter 3}

\vspace{-10pt}

Before Monday, Page 165-166 Introduction to Section 3.1.

\vspace{-10pt}

Before Wednesday, Example 3.9 on page 173.

\vspace{-10pt}

Before Friday, Page 184-185 Terminology for describing Turing machines.

\vspace{-10pt}

For Week 7 Monday: Introduction to Chapter 4.

\vspace{-20pt}

\subsubsection*{We will be learning and practicing to:}
\vspace{-20pt}

\begin{itemize}
    \item Clearly and unambiguously communicate computational ideas using appropriate formalism. Translate across levels of abstraction.
    \begin{itemize}
        \item Use precise notation to formally define the state diagram of a Turing machine
        \item Use clear English to describe computations of Turing machines informally.
        \begin{itemize}
                \item {\bf Motivate the definition of a Turing machine}
                \item {\bf Trace the computation of a Turing machine on given input}
                \item {\bf Describe the language recognized by a Turing machine}
                \item {\bf Determine if a Turing machine is a decider}
                \item {\bf Given an implementation-level description of a Turing machine}
                \item {\bf Use high-level descriptions to define and trace Turing machines}
                \item {\bf Apply dovetailing in high-level definitions of machines}
         \end{itemize}
       \item Give examples of sets that are recognizable and decidable (and prove that they are).
       \begin{itemize}
          \item {\bf State the definition of the class of recognizable languages}
          \item {\bf State the definition of the class of decidable languages}
          \item {\bf State the definition of the class of co-recognizable languages}
       \end{itemize}
    \end{itemize}
    \item Know, select and apply appropriate computing knowledge and problem-solving techniques. 
    Reason about computation and systems.
    \item Describe and prove closure properties of classes of languages under certain operations.
    \begin{itemize}
        \item {\bf Apply a general construction to create a new Turing machine from an example one.}
        \item {\bf Formalize a general construction from an informal description of it.}
        \item {\bf Use general constructions to prove closure properties of the class of decidable langugages.}
        \item {\bf Use general constructions to prove closure properties of the class of recognizable langugages.}
    \end{itemize}
\end{itemize}

\vspace{-20pt}

\subsubsection*{TODO:}
\begin{list}{\itemsep -20pt}
   \item This week: Test 1 Attempt 1 in CBTF. 
   \item Review Quiz 5 on PrairieLearn (http://us.prairielearn.com), due 2/12/2025
   \item Mid quarter feedback survey on Canvas.
\end{list}

\newpage

\section*{Monday: Descriptions of Turing machines}

%! app: Decidable Languages, Undecidable Languages
%! outcome: Formal definition of automata, Informal definition of automata, Classify language, Find example languages

We are ready to introduce a formal model that will capture a notion of general purpose computation.
\begin{itemize}
\item {\it Similar to DFA, NFA, PDA}: input will be an arbitrary string over a fixed alphabet.
\item {\it Different from NFA, PDA}: machine is deterministic.
\item {\it Different from DFA, NFA, PDA}: read-write head can move both to the left and to the right,
and can extend to the right past the original input.
\item {\it Similar to DFA, NFA, PDA}: transition function drives computation one step at a time 
by moving within a finite set of states, always starting at designated start state.
\item {\it Different from DFA, NFA, PDA}: the special states for rejecting and accepting take effect immediately.
\end{itemize}

\vspace{-10pt}

(See more details: Sipser p. 166)

\vfill

Formally: a  Turing machine is $M= (Q, \Sigma, \Gamma, \delta, q_0, q_{accept}, q_{reject})$ 
where $\delta$ is the {\bf transition function} 
\[
  \delta: Q\times \Gamma \to Q \times \Gamma \times \{L, R\}
\]
The {\bf computation} of $M$ on a string $w$ over $\Sigma$  is:

\vspace{-10pt}

\begin{itemize}
\setlength{\itemsep}{0pt}
\item Read/write head starts at leftmost position on tape. 
\item Input string is written on $|w|$-many leftmost cells of tape, 
rest of  the tape cells have  the blank symbol. {\bf Tape alphabet} 
is $\Gamma$ with $\textvisiblespace\in \Gamma$ and $\Sigma \subseteq \Gamma$.
The blank symbol $\textvisiblespace \notin \Sigma$.
\item Given current state of machine and current symbol being read at the tape head, 
the machine transitions to next state, writes a symbol to the current position  of the 
tape  head (overwriting existing symbol), and moves the tape head L or R (if possible). 
\item Computation ends {\bf if and when} machine enters either the accept or the reject state.
This is called {\bf halting}.
Note: $q_{accept} \neq q_{reject}$.
\end{itemize}

The {\bf language recognized by the  Turing machine} $M$,  is  $L(M) = \{ w \in \Sigma^* \mid w \textrm{ is accepted by } M\}$,
which is defined as
\[
  \{ w \in \Sigma^* \mid \textrm{computation of $M$ on $w$ halts after entering the accept state}\}
\]




\newpage
\begin{multicols}{2}
\begin{tikzpicture}[->,>=stealth',shorten >=1pt, auto, node distance=2cm, semithick]
  \tikzstyle{every state}=[text=black, fill=none]
  
  \node[initial,state] (q0)          {$q0$};
  \node[state]         (q1) [right of=q0, xshift=40pt] {$q1$};
  \node[state,accepting]         (qacc) [above right of=q0, yshift=20pt] {$q_{acc}$};
  \node[state]         (qrej) [above right of=q1,yshift=20pt] {$q_{rej}$};
  
  \path (q0) edge [bend left=0] node {$\square; \square, R$} (qacc)
      (q0) edge  [bend left=20] node {$0; \square, R$} (q1)
      (q1) edge [bend left=20] node {$0; \square, R$} (q0)
      (q1) edge [bend left=0] node {$\square; \square, R$} (qrej)
      (qacc) edge  [loop above] node {\parbox{1cm}{$0; \square, R$\newline $\square; \square, R$}} (qacc)
      (qrej) edge  [loop above] node {\parbox{1cm}{$0; \square, R$\newline $\square; \square, R$}}  (qrej)
  ;
\end{tikzpicture}
\columnbreak
Formal definition:

\vspace{10pt}

Sample computation: 

\begin{tabular}{|c|c|c|c|c|c|c|}
\hline
\multicolumn{1}{|c}{$q0\downarrow$} &  \multicolumn{6}{c|}{\phantom{A}}\\
\hline
$0$ & $0$  & $0$ & $\textvisiblespace $& $\textvisiblespace $& $\textvisiblespace $&  $\textvisiblespace $\\
\hline
\multicolumn{7}{|c|}{\phantom{A}}\\
\hline
\phantom{AA} & \phantom{AA}& \phantom{AA}& \phantom{AA}& \phantom{AA}& \phantom{AA}& \phantom{AA} \\
\hline
\multicolumn{7}{|c|}{\phantom{A}}\\
\hline
\phantom{AA} & \phantom{AA}& \phantom{AA}& \phantom{AA}& \phantom{AA}& \phantom{AA}& \phantom{AA} \\
\hline
\multicolumn{7}{|c|}{\phantom{A}}\\
\hline
\phantom{AA} & \phantom{AA}& \phantom{AA}& \phantom{AA}& \phantom{AA}& \phantom{AA}& \phantom{AA} \\
\hline
\multicolumn{7}{|c|}{\phantom{A}}\\
\hline
\phantom{AA} & \phantom{AA}& \phantom{AA}& \phantom{AA}& \phantom{AA}& \phantom{AA}& \phantom{AA} \\
\hline
\end{tabular}
\end{multicols}
\vfill

The language recognized by this machine is \ldots

\vfill
 

{\bf Describing  Turing machines} (Sipser p. 185) To define a Turing machine, we could give a 
\begin{itemize}
\item {\bf Formal definition}: the $7$-tuple of parameters including set of states, 
input alphabet, tape alphabet, transition function, start state, accept state, and reject state; or,
\item {\bf Implementation-level definition}: English prose that describes the Turing machine head 
movements relative to contents of tape, and conditions for accepting / rejecting based on those contents.
\item {\bf High-level description}: description of algorithm (precise sequence of instructions), 
without implementation details of machine. As part of this description, can ``call" and run 
another TM as a subroutine.
\end{itemize}
  
\newpage
Fix $\Sigma = \{0,1\}$, $\Gamma = \{ 0, 1, \textvisiblespace\}$ for the Turing machines with  the following state diagrams:
  
\begin{center}
  \begin{tikzpicture}[->,>=stealth',shorten >=1pt, auto, node distance=2cm, semithick]
    \tikzstyle{every state}=[text=black, fill=none]
    
    \node[initial,state] (q0)          {$q0$};
    \node[state,accepting]         (qacc) [right of=q0, xshift=20pt] {$q_{acc}$};
    
    \path (q0) edge  [loop above] node {\parbox{1cm}{$\square; \square, R$}} (q0)
    ;
  \end{tikzpicture}
\end{center}

Example of string accepted: \\
Example of string rejected: \\


Implementation-level description

\vfill

High-level description

\vfill

\begin{center}
  \begin{tikzpicture}[->,>=stealth',shorten >=1pt, auto, node distance=2cm, semithick]
    \tikzstyle{every state}=[text=black, fill=none]
    
    \node[initial,state] (qrej)          {$q_{rej}$};
    \node[state,accepting]         (qacc) [right of=q0, xshift=20pt] {$q_{acc}$};
  \end{tikzpicture}
\end{center}

Example of string accepted: \\
Example of string rejected: \\


Implementation-level description

\vfill

High-level description

\vfill

\newpage
\begin{center}
  \begin{tikzpicture}[->,>=stealth',shorten >=1pt, auto, node distance=2cm, semithick]
    \tikzstyle{every state}=[text=black, fill=none]
    
    \node[initial,state] (q0)          {$q0$};
    \node[state,accepting]         (qacc) [right of=q0, xshift=20pt] {$q_{acc}$};
    
    \path (q0) edge  [bend left=0] node {\parbox{1cm}{$\square; \square, R$}} (qacc)
    ;
  \end{tikzpicture}
\end{center}

Example of string accepted: \\
Example of string rejected: \\


Implementation-level description

\vfill

High-level description

\vfill

\begin{center}
  \begin{tikzpicture}[->,>=stealth',shorten >=1pt, auto, node distance=2cm, semithick]
    \tikzstyle{every state}=[text=black, fill=none]
    
    \node[initial,state] (q0)          {$q0$};
    \node[state,accepting]         (qacc) [right of=q0, xshift=20pt] {$q_{acc}$};
    
    \path (q0) edge  [loop above] node {\parbox{1cm}{$1; \square, R$\\$0; \square, R$\\$\square; \square, R$}} (q0)
    ;
  \end{tikzpicture}
\end{center}

Example of string accepted: \\
Example of string rejected: \\


Implementation-level description

\vfill

High-level description

\vfill

\newpage

    
\newpage
\subsection*{Wednesday: Recognizable and decidable languages}

%! app: Decidable Languages, Undecidable Languages
%! outcome: Formal definition of automata, Informal definition of automata, Classify language, Find example languages
    

{\it Sipser Figure  3.10}

{\bf Conventions in state diagram of TM}: $b \to R$ label means $b \to b, R$ and
all arrows missing from diagram represent transitions with output $(q_{reject}, \textvisiblespace , R)$

\begin{multicols}{2}
\vspace{-20pt}
\begin{center}
\includegraphics[width=4in]{../../resources/machines/Lect13TM3.png}
\end{center}

Implementation level description of this machine:
\begin{quote}
Zig-zag across tape to corresponding positions on either side of $\#$ to check whether the 
characters in these positions agree. If they do not, or if there is no $\#$, reject. If they 
do, cross them off.

Once all symbols to the left of the $\#$ are crossed off, check for any un-crossed-off symbols 
to the right of $\#$; if there are any, reject; if there aren't, accept.
\end{quote}

The language recognized by this machine is
\[
  \{ w \# w \mid w \in \{0,1\}^* \}
\]

\columnbreak

Computation on  input  string  $01\#01$

\begin{tabular}{|c|c|c|c|c|c|c|}
\hline
\multicolumn{1}{|c}{$q_1 \downarrow$} &  \multicolumn{6}{c|}{\phantom{A}}\\
\hline
$0$ & $1$  & $\#$  & $0$ & $1$ & $\textvisiblespace $& $\textvisiblespace $\\
\hline
  \multicolumn{7}{|c|}{\phantom{A}}\\
  \hline
  \phantom{AA} & \phantom{AA}& \phantom{AA}& \phantom{AA}& \phantom{AA}& \phantom{AA}& \phantom{AA} \\
  \hline
  \multicolumn{7}{|c|}{\phantom{A}}\\
  \hline
  \phantom{AA} & \phantom{AA}& \phantom{AA}& \phantom{AA}& \phantom{AA}& \phantom{AA}& \phantom{AA} \\
  \hline
  \multicolumn{7}{|c|}{\phantom{A}}\\
  \hline
  \phantom{AA} & \phantom{AA}& \phantom{AA}& \phantom{AA}& \phantom{AA}& \phantom{AA}& \phantom{AA} \\
  \hline
  \multicolumn{7}{|c|}{\phantom{A}}\\
  \hline
  \phantom{AA} & \phantom{AA}& \phantom{AA}& \phantom{AA}& \phantom{AA}& \phantom{AA}& \phantom{AA} \\
  \hline
  \multicolumn{7}{|c|}{\phantom{A}}\\
  \hline
  \phantom{AA} & \phantom{AA}& \phantom{AA}& \phantom{AA}& \phantom{AA}& \phantom{AA}& \phantom{AA} \\
  \hline
  \multicolumn{7}{|c|}{\phantom{A}}\\
  \hline
  \phantom{AA} & \phantom{AA}& \phantom{AA}& \phantom{AA}& \phantom{AA}& \phantom{AA}& \phantom{AA} \\
  \hline
  \multicolumn{7}{|c|}{\phantom{A}}\\
  \hline
  \phantom{AA} & \phantom{AA}& \phantom{AA}& \phantom{AA}& \phantom{AA}& \phantom{AA}& \phantom{AA} \\
  \hline
  \multicolumn{7}{|c|}{\phantom{A}}\\
  \hline
  \phantom{AA} & \phantom{AA}& \phantom{AA}& \phantom{AA}& \phantom{AA}& \phantom{AA}& \phantom{AA} \\
  \hline
  \multicolumn{7}{|c|}{\phantom{A}}\\
  \hline
  \phantom{AA} & \phantom{AA}& \phantom{AA}& \phantom{AA}& \phantom{AA}& \phantom{AA}& \phantom{AA} \\
  \hline
  \multicolumn{7}{|c|}{\phantom{A}}\\
  \hline
  \phantom{AA} & \phantom{AA}& \phantom{AA}& \phantom{AA}& \phantom{AA}& \phantom{AA}& \phantom{AA} \\
  \hline
  \multicolumn{7}{|c|}{\phantom{A}}\\
  \hline
  \phantom{AA} & \phantom{AA}& \phantom{AA}& \phantom{AA}& \phantom{AA}& \phantom{AA}& \phantom{AA} \\
  \hline
  \multicolumn{7}{|c|}{\phantom{A}}\\
  \hline
  \phantom{AA} & \phantom{AA}& \phantom{AA}& \phantom{AA}& \phantom{AA}& \phantom{AA}& \phantom{AA} \\
  \hline
  \multicolumn{7}{|c|}{\phantom{A}}\\
  \hline
  \phantom{AA} & \phantom{AA}& \phantom{AA}& \phantom{AA}& \phantom{AA}& \phantom{AA}& \phantom{AA} \\
  \hline
  \multicolumn{7}{|c|}{\phantom{A}}\\
  \hline
  \phantom{AA} & \phantom{AA}& \phantom{AA}& \phantom{AA}& \phantom{AA}& \phantom{AA}& \phantom{AA} \\
  \hline
  \multicolumn{7}{|c|}{\phantom{A}}\\
  \hline
  \phantom{AA} & \phantom{AA}& \phantom{AA}& \phantom{AA}& \phantom{AA}& \phantom{AA}& \phantom{AA} \\
  \hline
  \multicolumn{7}{|c|}{\phantom{A}}\\
  \hline
  \phantom{AA} & \phantom{AA}& \phantom{AA}& \phantom{AA}& \phantom{AA}& \phantom{AA}& \phantom{AA} \\
  \hline
  \multicolumn{7}{|c|}{\phantom{A}}\\
  \hline
  \phantom{AA} & \phantom{AA}& \phantom{AA}& \phantom{AA}& \phantom{AA}& \phantom{AA}& \phantom{AA} \\
  \hline
  \multicolumn{7}{|c|}{\phantom{A}}\\
  \hline
  \phantom{AA} & \phantom{AA}& \phantom{AA}& \phantom{AA}& \phantom{AA}& \phantom{AA}& \phantom{AA} \\
  \hline
  \end{tabular}
\end{multicols}

\begin{multicols}{2}
High-level description of this machine is

\vfill


{\it Recall:} 
High-level descriptions of  Turing machine algorithms are written as indented text within quotation marks.   
Stages of the algorithm are typically numbered consecutively.
The first line specifies the input to the machine, which must be a string.

\columnbreak

{\it Extra practice}

Computation on  input  string  $01\#1$

\begin{tabular}{|c|c|c|c|c|c|c|}
\hline
\multicolumn{1}{|c}{$q_1\downarrow$} &  \multicolumn{6}{c|}{\phantom{A}}\\
\hline
$0$ & $1$  & $\#$  & $1$ & $\textvisiblespace $& $\textvisiblespace $&  $\textvisiblespace $\\
\hline
\multicolumn{7}{|c|}{\phantom{A}}\\
\hline
\phantom{AA} & \phantom{AA}& \phantom{AA}& \phantom{AA}& \phantom{AA}& \phantom{AA}& \phantom{AA} \\
\hline
\multicolumn{7}{|c|}{\phantom{A}}\\
\hline
\phantom{AA} & \phantom{AA}& \phantom{AA}& \phantom{AA}& \phantom{AA}& \phantom{AA}& \phantom{AA} \\
\hline
\multicolumn{7}{|c|}{\phantom{A}}\\
\hline
\phantom{AA} & \phantom{AA}& \phantom{AA}& \phantom{AA}& \phantom{AA}& \phantom{AA}& \phantom{AA} \\
\hline
\multicolumn{7}{|c|}{\phantom{A}}\\
\hline
\phantom{AA} & \phantom{AA}& \phantom{AA}& \phantom{AA}& \phantom{AA}& \phantom{AA}& \phantom{AA} \\
\hline
\multicolumn{7}{|c|}{\phantom{A}}\\
\hline
\phantom{AA} & \phantom{AA}& \phantom{AA}& \phantom{AA}& \phantom{AA}& \phantom{AA}& \phantom{AA} \\
\hline
\multicolumn{7}{|c|}{\phantom{A}}\\
\hline
\phantom{AA} & \phantom{AA}& \phantom{AA}& \phantom{AA}& \phantom{AA}& \phantom{AA}& \phantom{AA} \\
\hline
\multicolumn{7}{|c|}{\phantom{A}}\\
\hline
\phantom{AA} & \phantom{AA}& \phantom{AA}& \phantom{AA}& \phantom{AA}& \phantom{AA}& \phantom{AA} \\
\hline
\multicolumn{7}{|c|}{\phantom{A}}\\
\hline
\phantom{AA} & \phantom{AA}& \phantom{AA}& \phantom{AA}& \phantom{AA}& \phantom{AA}& \phantom{AA} \\
\hline
\multicolumn{7}{|c|}{\phantom{A}}\\
\hline
\phantom{AA} & \phantom{AA}& \phantom{AA}& \phantom{AA}& \phantom{AA}& \phantom{AA}& \phantom{AA} \\
\hline
\multicolumn{7}{|c|}{\phantom{A}}\\
\hline
\phantom{AA} & \phantom{AA}& \phantom{AA}& \phantom{AA}& \phantom{AA}& \phantom{AA}& \phantom{AA} \\
\hline
\multicolumn{7}{|c|}{\phantom{A}}\\
\hline
\phantom{AA} & \phantom{AA}& \phantom{AA}& \phantom{AA}& \phantom{AA}& \phantom{AA}& \phantom{AA} \\
\hline
\multicolumn{7}{|c|}{\phantom{A}}\\
\hline
\phantom{AA} & \phantom{AA}& \phantom{AA}& \phantom{AA}& \phantom{AA}& \phantom{AA}& \phantom{AA} \\
\hline
\multicolumn{7}{|c|}{\phantom{A}}\\
\hline
\phantom{AA} & \phantom{AA}& \phantom{AA}& \phantom{AA}& \phantom{AA}& \phantom{AA}& \phantom{AA} \\
\hline
\multicolumn{7}{|c|}{\phantom{A}}\\
\hline
\phantom{AA} & \phantom{AA}& \phantom{AA}& \phantom{AA}& \phantom{AA}& \phantom{AA}& \phantom{AA} \\
\hline
\multicolumn{7}{|c|}{\phantom{A}}\\
\hline
\phantom{AA} & \phantom{AA}& \phantom{AA}& \phantom{AA}& \phantom{AA}& \phantom{AA}& \phantom{AA} \\
\hline
\multicolumn{7}{|c|}{\phantom{A}}\\
\hline
\phantom{AA} & \phantom{AA}& \phantom{AA}& \phantom{AA}& \phantom{AA}& \phantom{AA}& \phantom{AA} \\
\hline
\multicolumn{7}{|c|}{\phantom{A}}\\
\hline
\phantom{AA} & \phantom{AA}& \phantom{AA}& \phantom{AA}& \phantom{AA}& \phantom{AA}& \phantom{AA} \\
\hline
\multicolumn{7}{|c|}{\phantom{A}}\\
\hline
\phantom{AA} & \phantom{AA}& \phantom{AA}& \phantom{AA}& \phantom{AA}& \phantom{AA}& \phantom{AA} \\
\hline
\multicolumn{7}{|c|}{\phantom{A}}\\
\hline
\phantom{AA} & \phantom{AA}& \phantom{AA}& \phantom{AA}& \phantom{AA}& \phantom{AA}& \phantom{AA} \\
\hline
\multicolumn{7}{|c|}{\phantom{A}}\\
\hline
\phantom{AA} & \phantom{AA}& \phantom{AA}& \phantom{AA}& \phantom{AA}& \phantom{AA}& \phantom{AA} \\
\hline
\end{tabular}

\end{multicols}
\newpage



A language $L$ is {\bf recognized by} a Turing machine $M$ means

\vfill

A Turing  machine  $M$ {\bf  recognizes} a language $L$ means

\vfill

A Turing machine $M$ is a {\bf decider}  means

\vfill

A language  $L$ is {\bf decided by} a Turing  machine  $M$  means

\vfill

A  Turing machine $M$ {\bf decides} a language $L$ means

\vfill

Fix $\Sigma = \{0,1\}$, $\Gamma = \{ 0, 1, \textvisiblespace\}$ for the Turing machines with  the following state diagrams:
  
  \begin{center}
  \begin{tabular}{|c|c|}
  \hline
  \hspace{0.8in}
  \begin{tikzpicture}[->,>=stealth',shorten >=1pt, auto, node distance=2cm, semithick]
    \tikzstyle{every state}=[text=black, fill=none]
    
    \node[initial,state] (q0)          {$q0$};
    \node[state,accepting]         (qacc) [right of=q0, xshift=20pt] {$q_{acc}$};
    
    \path (q0) edge  [loop above] node {$\square; \square, R$} (qacc)
    ;
  \end{tikzpicture}
%  \includegraphics[width=2in]{../../resources/machines/Lect14TM1.png} 
  \phantom{\hspace{0.8in}}&\hspace{0.8in} 
  \begin{tikzpicture}[->,>=stealth',shorten >=1pt, auto, node distance=2cm, semithick]
    \tikzstyle{every state}=[text=black, fill=none]
    
    \node[initial,state] (qrej)          {$q_{rej}$};
    \node[state,accepting]         (qacc) [right of=qrej, xshift=20pt] {$q_{acc}$};
  \end{tikzpicture}
  \phantom{\hspace{0.8in}}\\
  Decider? Yes~~~/ ~~~No
  &Decider? Yes~~~/ ~~~No\\
  & \\
  \hline

  \begin{tikzpicture}[->,>=stealth',shorten >=1pt, auto, node distance=2cm, semithick]
    \tikzstyle{every state}=[text=black, fill=none]
    
    \node[initial,state] (q0)          {$q0$};
    \node[state,accepting]         (qacc) [right of=q0, xshift=20pt] {$q_{acc}$};

    \path (q0) edge [bend left=0] node {$\square; \square, R$} (qacc)
    ;
  \end{tikzpicture} & 
  
  \begin{tikzpicture}[->,>=stealth',shorten >=1pt, auto, node distance=2cm, semithick]
    \tikzstyle{every state}=[text=black, fill=none]
    
    \node[initial,state] (q0)          {$q0$};
    \node[state,accepting]         (qacc) [right of=q0, xshift=20pt] {$q_{acc}$};
    
    \path (q0) edge  [loop above] node {\parbox{1cm}{$0; \square, R$\newline$1; \square, R$\newline $\square; \square, R$}} (q0)
    ;
  \end{tikzpicture}\\
  Decider? Yes~~~/ ~~~No
  &Decider? Yes~~~/ ~~~No\\
  & \\
  
  \hline
  \end{tabular}
  \end{center}
  \newpage




\newpage
\subsection*{Friday: Closure for the classes of recognizable and decidable languages}

%! app: Decidable Languages, Undecidable Languages
%! outcome: Formal definition of automata, Informal definition of automata, Classify language, Find example languages



A {\bf Turing-recognizable} language is a set of strings that 
is the language recognized by some Turing machine. We also 
say that such languages are recognizable.

A {\bf Turing-decidable} language is a set of strings that 
is the language recognized by some decider. We also 
say that such languages are decidable.


An {\bf unrecognizable} language is a language that is not Turing-recognizable.

An {\bf undecidable} language is a language that is not Turing-decidable.

\vfill

{\bf  True} or {\bf False}: Any  decidable language  is  also  recognizable.

\vfill

{\bf  True} or {\bf False}: Any  recognizable language  is  also  decidable.

\vfill

{\bf  True} or {\bf False}: Any  undecidable language  is  also  unrecognizable.

\vfill

{\bf  True} or {\bf False}: Any  unrecognizable language  is  also  undecidable.

\vfill

\newpage


{\bf True} or {\bf False}: The class of Turing-decidable languages is closed under complementation.

\vfill

Using formal definition:
\vfill

Using high-level description:
\vfill


{\bf  Church-Turing Thesis} (Sipser p. 183): The informal notion of algorithm is formalized completely  and correctly by the 
formal definition of a  Turing machine. In other words: all reasonably expressive models of 
computation are equally expressive with the standard Turing machine.




\newpage

\subsection*{Week 7 at a glance}

\subsubsection*{Textbook reading: Chapter 4}

\vspace{-20pt}
No class on Monday in observance of UCSD holiday.

Before Wednesday, Introduction to Chapter 4.

Before Friday, Decidable problems concerning regular languages, 
Sipser pages 194-196.

For Week 8 Monday: An undecidable language, Sipser pages 207-209.

\vspace{-20pt}

\subsubsection*{We will be learning and practicing to:}

\vspace{-20pt}

\begin{itemize}
    \item Clearly and unambiguously communicate computational ideas using appropriate formalism. Translate across levels of abstraction.
    \begin{itemize}
        \item Use clear English to describe computations of Turing machines informally.
        \begin{itemize}
                \item {\bf Use high-level descriptions to define and trace Turing machines}
                \item {\bf Apply dovetailing in high-level definitions of machines}
        \end{itemize}
        \item Give examples of sets that are regular, context-free, decidable, or recognizable (and prove that they are).
        \begin{itemize}
          \item {\bf Give examples of sets that are decidable.}
          \item {\bf Give examples of sets that are recognizable.}
       \end{itemize}
    \end{itemize}
    \item Know, select and apply appropriate computing knowledge and problem-solving techniques. Reason about computation and systems.
    \begin{itemize}
        \item Translate a decision problem to a set of strings coding the problem.
        \begin{itemize}
        \item {\bf Connect languages and computational problems}
        \item {\bf Describe and use the encoding of objects as inputs to Turing machines}
        \item {\bf Trace high-level descriptions of algorithms for computational problems}
        \end{itemize}
    \item Classify the computational complexity of a set of strings by determining whether it is regular, context-free, decidable, or recognizable.
    \begin{itemize}
    \item {\bf Describe common computational problems with respect to DFA, NFA, regular expressions, PDA, and context-free grammars.}
    \item {\bf Give high-level descriptions of Turing machines that decide common computational problems with respect to DFA, NFA, regular expressions, PDA, and context-free grammars.}
\end{itemize}
\end{itemize}
\end{itemize}

\vspace{-20pt}

\subsubsection*{TODO:}
\begin{list}{\itemsep-10pt}
    \item Review Quiz 6 on PrairieLearn (http://us.prairielearn.com), due 2/19/2025
    \item Homework 4 submitted via Gradescope (https://www.gradescope.com/), due 2/20/2025
    \item Review Quiz 7 on PrairieLearn (http://us.prairielearn.com), due 2/26/2025
\end{list}

\newpage

\subsection*{Monday: No class, in observance of UCSD holiday}
\subsection*{Wednesday: General constructions for Turing machines}


%! app: Decidable Languages, Undecidable Languages
%! outcome: Formal definition of automata, Informal definition of automata, Classify language, Find example languages


Definition: A language $L$ over an  alphabet $\Sigma$ is called {\bf co-recognizable} if its complement,  defined
as $\Sigma^* \setminus L  = \{ x  \in  \Sigma^* \mid x \notin  L \}$, is Turing-recognizable.

{\it Notation}: The complement  of a set $X$ is denoted with  a superscript $c$, $X^c$, or an overline,  $\overline{X}$.

{\bf  Theorem} (Sipser Theorem 4.22): A  language is Turing-decidable if and only if both  it and its complement
are Turing-recognizable.

{\bf Proof, first direction:}  Suppose  language  $L$ is  Turing-decidable.   WTS  that both it and its complement 
are Turing-recognizable.

\vfill

{\bf Proof, second direction:}  Suppose  language  $L$ is  Turing-recognizable, and  so is  its complement.   WTS  that $L$
is Turing-decidable.
\vfill




\newpage

\vfill

\vfill

{\bf Dovetailing}: interleaving progress on multiple computations by 
limiting the number of steps each computation makes in each round.
\newpage

{\bf Claim}: If two languages  (over a fixed alphabet  $\Sigma$) are Turing-decidable, then  their union  is  as well.

{\bf Proof}:


\vfill
\newpage

{\bf Claim}: If two languages  (over a fixed alphabet  $\Sigma$) are Turing-recognizable, then  their union  is  as well.

{\bf Proof}:




\newpage
\subsection*{Friday: Decidable problems about regular languages}

%! app: Decidable Languages, Undecidable Languages
%! outcome: Classify language, Find example languages, Define decision problem, Classify decision problem
    
{\bf The Church-Turing thesis posits that each algorithm can be implemented by some Turing machine.}

{\bf Describing  algorithms} (Sipser p. 185) To define a Turing machine, we could give a 
\begin{itemize}
\item {\bf Formal definition}: the $7$-tuple of parameters including set of states, 
input alphabet, tape alphabet, transition function, start state, accept state, and reject state.
This is the low-level programming view that models the logic computation flow in a processor.
\item {\bf Implementation-level definition}: English prose that describes the Turing machine head 
movements relative to contents of tape, and conditions for accepting / rejecting based on those contents.
This level describes memory management and implementing data access with data structures.
  \begin{itemize}
    \item Mention the tape or its contents (e.g.\ ``Scan the tape from left to right until a blank is seen.'')
    \item Mention the tape head (e.g.\ ``Return the tape head to the left end of the tape.'')
  \end{itemize}
\item {\bf High-level description} of algorithm executed by Turing machine: 
description of algorithm (precise sequence of instructions), 
without implementation details of machine. 
High-level descriptions of  Turing machine algorithms are written as indented text within quotation marks.   
Stages of the algorithm are typically numbered consecutively.
The first line specifies the input to the machine, which must be a string.
  \begin{itemize}
    \item Use other Turing machines as subroutines (e.g.\ ``Run $M$ on $w$'')
    \item Build new machines from existing machines using previously shown results (e.g.\ 
    ``Given NFA $A$ construct an NFA $B$ such that $L(B) = \overline{L(A)}$'')
    \item Use previously shown conversions and constructions (e.g.\ ``Convert regular expression $R$ 
    to an NFA $N$'')
  \end{itemize}
\end{itemize}

{\bf Formatted inputs to Turing machine algorithms}

The input to a Turing machine is always a string. 
The format of the input to a Turing machine can be checked to interpret 
this string as representing structured data (like a csv file, the formal definition of a DFA, another Turing machine, etc.)


This string may be the encoding of some object or  list of  objects.  

{\bf Notation:} $\langle O \rangle$ is the string that encodes the object $O$.
$\langle O_1, \ldots, O_n \rangle$ is the string that encodes the list of objects $O_1, \ldots, O_n$.

{\bf Assumption}: There are algorithms (Turing  machines) that can be called as subroutines
to decode the string representations of common objects and  interact with these objects as intended
(data structures). These algorithms are able to ``type-check'' and string representations for different
data structures are unique.
  
\newpage
For example, since there are algorithms to answer each of the following questions,
by Church-Turing thesis, there is a Turing machine that accepts exactly those strings for which the 
answer to the question is ``yes''
\begin{itemize}
    \item Does a string over $\{0,1\}$ have even length?

    \item Does a string over $\{0,1\}$ encode a string of ASCII characters?\footnote{An introduction to ASCII 
    is available on the w3 tutorial \href{https://www.w3schools.com/charsets/ref_html_ascii.asp}{here}.}

    \item Does a DFA have a specific number of states?

    \item Do two NFAs have any state names in common?

    \item Do two CFGs have the same start variable?

  \end{itemize}



A {\bf computational problem} is decidable iff language encoding its positive problem instances
is decidable.

The computational problem ``Does a specific DFA accept a given string?'' is encoded by the language
\begin{align*}
  &\{ \textrm{representations of DFAs $M$ and strings $w$ such that $w \in L(M)$}\}  \\
  =& \{ \langle M, w \rangle \mid M \textrm{ is a DFA}, w \textrm{ is a string}, w \in L(M) \}
\end{align*}

The computational problem ``Is the language generated by a CFG empty?'' is encoded by the language
\begin{align*}
  &\{ \textrm{representations of CFGs $G$  such that $L(G) = \emptyset$}\}  \\
  =& \{ \langle G \rangle \mid G \textrm{ is a CFG},  L(G) = \emptyset \}
\end{align*}



The computational problem ``Is the given Turing machine a decider?'' is encoded by the language
\begin{align*}
  &\{ \textrm{representations of TMs $M$  such that $M$ halts on every input}\}  \\
  =& \{ \langle M \rangle \mid M \textrm{ is a TM and for each string } w, \textrm{$M$ halts on $w$} \}
\end{align*}


{\it Note: writing down the language encoding a computational problem is only the first step in 
determining if it's recognizable, decidable, or \ldots }

Deciding a computational problem means building / defining a Turing 
machine that recognizes the language encoding the computational problem, and that 
is a decider.


Some classes of computational problems will 
help us understand the differences between the machine models we've been studying. (Sipser Section 4.1)

    \begin{center}
    \begin{tabular}{|lcl|}
    \hline
    \multicolumn{3}{|l|}{{\bf  Acceptance problem} } \\
    & & \\
    \ldots for DFA & $A_{DFA}$ & $\{ \langle B,w \rangle \mid  \text{$B$ is a  DFA that accepts input 
    string $w$}\}$ \\
    \ldots for NFA & $A_{NFA}$ & $\{ \langle B,w \rangle \mid  \text{$B$ is a  NFA that accepts input 
    string $w$}\}$ \\
    \ldots for regular expressions & $A_{REX}$ & $\{ \langle R,w \rangle \mid  \text{$R$ is a  regular
    expression that generates input string $w$}\}$ \\
    \ldots for CFG & $A_{CFG}$ & $\{ \langle G,w \rangle \mid  \text{$G$ is a context-free grammar 
    that generates input string $w$}\}$ \\
    \ldots for PDA & $A_{PDA}$ & $\{ \langle B,w \rangle \mid  \text{$B$ is a PDA that accepts input string $w$}\}$ \\
    & & \\
    \hline
    \multicolumn{3}{|l|}{{\bf Language emptiness  testing} } \\
    & & \\
    \ldots for DFA & $E_{DFA}$ & $\{ \langle A \rangle \mid  \text{$A$ is a  DFA and  $L(A) = \emptyset$\}}$ \\
    \ldots for NFA & $E_{NFA}$ & $\{ \langle A\rangle \mid  \text{$A$ is a NFA and  $L(A) = \emptyset$\}}$ \\
    \ldots for regular expressions & $E_{REX}$ & $\{ \langle R \rangle \mid  \text{$R$ is a  regular
    expression and  $L(R) = \emptyset$\}}$ \\
    \ldots for CFG & $E_{CFG}$ & $\{ \langle G \rangle \mid  \text{$G$ is a context-free grammar 
    and  $L(G) = \emptyset$\}}$ \\
    \ldots for PDA & $E_{PDA}$ & $\{ \langle A \rangle \mid  \text{$A$ is a PDA and  $L(A) = \emptyset$\}}$ \\
    & & \\
    \hline
    \multicolumn{3}{|l|}{{\bf Language equality testing} } \\
    & & \\
    \ldots for DFA & $EQ_{DFA}$ & $\{ \langle A, B \rangle \mid  \text{$A$ and $B$ are DFAs and  $L(A) =L(B)$\}}$\\
    \ldots for NFA & $EQ_{NFA}$ & $\{ \langle A, B \rangle \mid  \text{$A$ and $B$ are NFAs and  $L(A) =L(B)$\}}$\\
    \ldots for regular expressions & $EQ_{REX}$ & $\{ \langle R, R' \rangle \mid  \text{$R$ and $R'$ are regular
    expressions and  $L(R) =L(R')$\}}$\\
    \ldots for CFG & $EQ_{CFG}$ & $\{ \langle G, G' \rangle \mid  \text{$G$ and $G'$ are CFGs and  $L(G) =L(G')$\}}$ \\
    \ldots for PDA & $EQ_{PDA}$ & $\{ \langle A, B \rangle \mid  \text{$A$ and $B$ are PDAs and  $L(A) =L(B)$\}}$ \\
    \hline
    \end{tabular}
    \end{center}
    

    Example strings in $A_{DFA}$
    
    \vfill
    
    Example strings in  $E_{DFA}$
    
    \vfill
    
    Example strings in  $EQ_{DFA}$
    
    \vfill

    \begin{quote}
      $M_1 = $ ``On input $\langle M,w\rangle$, where $M$ is a DFA and $w$ is a string:
      \begin{enumerate}
      \setcounter{enumi}{-1}
      \item Type check encoding to check input is correct type. If not, reject.
      \item Simulate $M$ on input $w$ (by keeping track of states in $M$, transition function of $M$, etc.) 
      \item If the simulation ends in an accept state of $M$, accept. If it ends in a non-accept state of $M$, reject. "
      \end{enumerate}
      \end{quote}
      
    
    What is $L(M_1)$? 
    
    \vfill
    
    Is $M_1$ a decider?
    
    \vfill
    
    
    {\it Alternate description}:
    Sometimes omit step 0 from listing and do implicit type check.
    
    Synonyms: ``Simulate'', ``run'', ``call''.
    
    \newpage
    
      True / False: $A_{REX} = A_{NFA} = A_{DFA}$
    
      True / False: $A_{REX} \cap A_{NFA} = \emptyset$, $A_{REX} \cap A_{DFA} = \emptyset$, $A_{DFA} \cap A_{NFA} = \emptyset$
    
      
      \vfill
    
      $E_{DFA} = \{ \langle A \rangle \mid  \text{$A$ is a  DFA and  $L(A) = \emptyset$}\}$. 
        A Turing machine that  decides $E_{DFA}$ is
    
      \begin{multicols}{2}
      \begin{quote}
      $M_2 =  $``On  input  $\langle M\rangle$ where $M$ is a  DFA,
      \begin{enumerate}
      \item For integer  $i = 1, 2, \ldots$
      \item \qquad Let  $s_i$ be the  $i$th string over  the alphabet of  $M$ (ordered in  string order).
      \item \qquad Run $M$ on  input  $s_i$.
      \item \qquad If $M$  accepts,  reject.  If $M$  rejects, increment $i$ and keep going."
      \end{enumerate}
      \end{quote}
      
      \vfill

      \columnbreak
    
    
      \begin{quote}
      $M_3 =  $ `` On  input $\langle M \rangle$ where $M$ is  a  DFA,
      \begin{enumerate}
      \item Mark the start  state  of $M$.
      \item Repeat until no  new states get marked:
      \item \qquad Loop over the states of $M$. 
      \item \qquad Mark any unmarked  state  that  has an incoming  edge  from a marked state.
      \item If  no  accept state of $M$ is  marked, $\underline{\phantom{FILL  IN}}$;  otherwise, 
      $\underline{\phantom{FILL  IN}}$".
      \end{enumerate}
      \end{quote}
    \end{multicols}
      
    \vfill
    
    \newpage

    To build a Turing machine that decides $EQ_{DFA}$, notice that 
    \[
    L_1 = L_2 \qquad\textrm{iff}\qquad (~(L_1 \cap \overline{L_2}) \cup (L_2 \cap \overline L_1)~) = \emptyset  
    \]
    {\it There are no elements that are in one set and not the other}
    
    
    $M_{EQDFA} = $ 
    
    
      \vfill
      
    
      {\bf Summary}:  We can use the decision procedures (Turing machines) of decidable problems
      as subroutines in other algorithms. For example, we have subroutines for deciding each of 
      $A_{DFA}$, $E_{DFA}$, $EQ_{DFA}$.  We can also use algorithms for known constructions
      as subroutines in other algorithms. For example, we have subroutines for: counting the number 
      of states in a state diagram, counting the number of characters in an alphabet, converting DFA
      to a DFA recognizing the complement of the original language or a DFA recognizing the 
      Kleene star of the original language, constructing a DFA or NFA from two DFA or NFA so that 
      we have a machine recognizing the language of the union (or intersection, concatenation)
      of the languages of the original machines; converting regular expressions to equivalent DFA; 
      converting DFA to equivalent regular expressions, etc.
    



\newpage

\subsection*{Week 8 at a glance}


\vspace{-20pt}

\subsubsection*{Textbook reading: Chapter 4, Section 5.3}

\vspace{-20pt}

Before Monday, ``An undecidable language", Sipser pages 207-209.

Before Wednesday, Definition 5.20 and figure 5.21 (page 236) of mapping reduction.

Before Friday, Example 5.24 (page 236).

For Week 9 Monday: Example 5.26 (page 237).

\vspace{-20pt}

\subsubsection*{We will be learning and practicing to:}

\vspace{-20pt}

\begin{itemize}
    \item Clearly and unambiguously communicate computational ideas using appropriate formalism. Translate across levels of abstraction.
    \begin{itemize}
        \item Give examples of sets that are regular, context-free, decidable, or recognizable (and prove that they are).
        \begin{itemize}
          \item {\bf Define and explain the definitions of the computational problem $A_{TM}$}
          \item {\bf Define and explain the definitions of the computational problem $HALT_{TM}$}
       \end{itemize}
    \end{itemize}
    \item Know, select and apply appropriate computing knowledge and problem-solving techniques. Reason about computation and systems.
    \begin{itemize}
        \item Use diagonalization to prove that there are 'hard' languages relative to certain models of computation.
        \begin{itemize}
            \item {\bf Trace the argument that proves $A_{TM}$ is undecidable and explain why it works.}
        \end{itemize}
        \item Use mapping reduction to deduce the complexity of a language by comparing to the complexity of another.
           \begin{itemize}
              \item {\bf Define computable functions, and use them to give mapping reductions between computational problems}
              \item {\bf Build and analyze mapping reductions between computational problems}
              \item {\bf Deduce the decidability or undecidability of a computational problem given mapping reductions between it and other computational problems, or explain when this is not possible.}
           \end{itemize}
    \item Classify the computational complexity of a set of strings by determining whether it is regular, context-free, decidable, or recognizable.
    \begin{itemize}
    \item {\bf State, prove, and use theorems relating decidability, recognizability, and co-recognizability.}
    \item {\bf Prove that a language is decidable or recognizable by defining and analyzing a Turing machines with appropriate properties.}
\end{itemize}
\end{itemize}
\end{itemize}

\vspace{-20pt}

\subsubsection*{TODO:}
\begin{list}{\itemsep-10pt}
   \item Review Quiz 7 on PrairieLearn (http://us.prairielearn.com), due 2/26/2025
   \item Homework 5 submitted via Gradescope (https://www.gradescope.com/), due 2/27/2025
   \item Review Quiz 8 on PrairieLearn (http://us.prairielearn.com), due 3/5/2025
\end{list}

\newpage

\section*{Monday: $A_{TM}$ is recognizable but undecidable}

%! app: Decidable Languages, Undecidable Languages
%! outcome: Classify language, Find example languages, Classify decision problem, Diagonalization, Reduction
    


\begin{center}
    \begin{tabular}{|lcl|}
    \hline
    \multicolumn{3}{|l|}{{\bf  Acceptance problem} } \\
    for Turing  machines  & $A_{TM}$ & $\{ \langle M,w \rangle \mid  \text{$M$ is a Turing machine that accepts input 
    string $w$}\}$ \\
    \hline
    \multicolumn{3}{|l|}{{\bf Language emptiness  testing} } \\
     for Turing machines & $E_{TM}$ & $\{ \langle M \rangle \mid  \text{$M$ is a Turing machine and  $L(M) = \emptyset$\}}$ \\
    \hline
    \multicolumn{3}{|l|}{{\bf Language equality testing} } \\
     for Turing machines& $EQ_{TM}$ & $\{ \langle  M_1, M_2 \rangle \mid  \text{$M_1$ and $M_2$ are Turing machines and  
     $L(M_1) =L(M_2)$\}}$\\
    \hline
    \end{tabular}
    \end{center}
    
    \begin{tabular}{p{3in}p{3in}}
    $M_1$ & $M_2$ \\
    %\includegraphics[width=2in]{../../resources/machines/Week8WarmupTM1.png} 
    \begin{tikzpicture}[->,>=stealth',shorten >=1pt, auto, node distance=2cm, semithick]
        \tikzstyle{every state}=[text=black, fill=none]
        
        \node[initial,state] (q0)          {$q0$};
        \node[state]         (q1) [right of=q0, xshift=80pt] {$q1$};
        \node[state,accepting]   (qacc) [above of=q0] {$q_{acc}$};
        
        \path (q0) edge [bend left=20] node {\parbox{1cm}{$0; 0,R$\newline $1; 1, R$}} (q1)
            (q1) edge [bend left=20] node {\parbox{1cm}{$0; 0,R$\newline $1; 1, R$}} (q0)
            (q0) edge  [bend left=0] node {$\square; \square, R$} (qacc)
        ;
    \end{tikzpicture}
    &
    %\includegraphics[width=2in]{../../resources/machines/Week8WarmupTM2.png}
    \begin{tikzpicture}[->,>=stealth',shorten >=1pt, auto, node distance=2cm, semithick]
        \tikzstyle{every state}=[text=black, fill=none]
        
        \node[initial,state] (q0)          {$q0$};
        \node[state]         (q1) [right of=q0, xshift=40pt] {$q1$};
        \node[state]         (q2) [right of=q1, xshift=20pt] {$q2$};
        \node[state,accepting]   (qacc) [right of=q2, xshift=20pt] {$q_{acc}$};
        
        \path (q0) edge [loop above] node {\parbox{1cm}{$\square; \square, R$}} (q0)
        (q0) edge [bend left=00] node {\parbox{1cm}{$0; \square,R$\newline $1; \square, R$}} (q1)
            (q1) edge [bend left=20] node {\parbox{1cm}{$0; \square,R$\newline $1; \square, R$}} (q2)
            (q2) edge [bend left=20] node {\parbox{1cm}{$0; \square,R$\newline $1; \square, R$}} (q1)
            (q2) edge  [bend left=0] node {$\square; \square, R$} (qacc)
        ;
    \end{tikzpicture}
    
    \end{tabular}
    
    Example strings in $A_{TM}$
    
    \vfill
    
    Example strings in  $E_{TM}$
    
    \vfill
    
    Example strings in  $EQ_{TM}$
    
    \vfill
    
    \newpage
    
    {\bf  Theorem}: $A_{TM}$  is  Turing-recognizable.
    
    
    {\bf  Strategy}:  To prove this theorem, we need  to  define  a Turing  machine  $R_{ATM}$ such that 
    $L(R_{ATM}) = A_{TM}$.
    
    
    Define $R_{ATM} =  $ ``
    
    \vspace{150pt}
    
    
    Proof of correctness: 
    
    
    \vfill
    \vfill
    
    We will show that $A_{TM}$ is undecidable.   {\it First, let's explore what that means.}
    
    \newpage
    
    To prove that a computational problem is {\bf decidable}, we find/ build a Turing 
    machine that recognizes the language encoding the computational problem, and that 
    is a decider.
    
    
    How do we prove a specific problem is {\bf not decidable}?
    
    How would we even find such a computational problem?
    
    
    {\it Counting arguments for the existence of an undecidable language:}
    \begin{itemize}
        \item The set of all Turing machines is countably infinite.
        \item Each recognizable language has at least one Turing machine that recognizes it (by definition), 
        so there can be no more Turing-recognizable
        languages than there are Turing machines. 
        \item Since there are infinitely many Turing-recognizable languages
        (think of the singleton sets), there are countably infinitely 
        many Turing-recognizable languages.
        \item Such the set of Turing-decidable languages is an infinite subset 
        of the set of Turing-recognizable languages, the set of 
        Turing-decidable languages is also countably infinite.
    \end{itemize}
    
    Since there are uncountably many languages (because $\mathcal{P}(\Sigma^*)$
    is uncountable), there are uncountably many unrecognizable languages
    and there are uncountably many undecidable languages.
    
    
    Thus, there's at least one undecidable language!
    
    \vfill
    
    {\bf What's a specific example of a language that is unrecognizable or undecidable?}
    
    To prove that a language is undecidable, we need to prove that there is no Turing machine that decides it.
    
    {\bf Key idea}: proof by contradiction relying on self-referential disagreement.
    
    

{\bf  Theorem}: $A_{TM}$  is  not  Turing-decidable.

{\bf  Proof}: Suppose {\bf towards a  contradiction}  that there  is a Turing machine  that decides $A_{TM}$.  
We call this presumed machine  $M_{ATM}$.

By  assumption, for every  Turing machine  $M$ and every  string $w$

\begin{itemize}
\item If $w \in L(M)$, then  the computation of $M_{ATM}$  on  $\langle M,w \rangle ~~ \underline{\phantom{\hspace{2.5in}}}$
\item If $w \notin L(M)$, then  the computation of $M_{ATM}$  on  $\langle M,w \rangle ~~ \underline{\phantom{\hspace{2.5in}}}$
\end{itemize}


Define  a {\bf new} Turing machine using  the high-level description:
\begin{quote}
$D =  $`` On  input $\langle M \rangle$, where  $M$  is  a Turing machine:
\begin{itemize}
\item[1.] Run  $M_{ATM}$ on  $\langle M, \langle M \rangle  \rangle$.
\item[2.] If $M_{ATM}$ accepts, reject; if  $M_{ATM}$ rejects, accept."
\end{itemize}
\end{quote}


Is $D$ a  Turing machine?

\vspace{30pt}

Is  $D$ a  decider? 

\vspace{30pt}

What is the result of the computation  of $D$  on  $\langle D \rangle$?

\vfill


\newpage

{\bf Summarizing}: 

\begin{itemize}
    \item $A_{TM}$  is recognizable.
    \item $A_{TM}$  is  not  decidable.
\end{itemize}

\vfill

Recall definition: A language $L$ over an  alphabet $\Sigma$ is called {\bf co-recognizable} if its complement,  defined
as $\Sigma^* \setminus L  = \{ x  \in  \Sigma^* \mid x \notin  L \}$, is Turing-recognizable.

and Recall  Theorem (Sipser Theorem 4.22): A  language is Turing-decidable if and only if both  it and its complement
are Turing-recognizable.

\vfill

\begin{itemize}
    \item $A_{TM}$  is recognizable.
    \item $A_{TM}$  is  not  decidable.
    \item $\overline{A_{TM}}$   is  not  recognizable.
    \item $\overline{A_{TM}}$   is  not  decidable.
\end{itemize}

\vfill
    
\newpage
\subsection*{Wednesday: Computable functions and mapping reduction}

%! app: Decidable Languages, Undecidable Languages
%! outcome: Classify language, Classify decision problem, Reduction
    
{\bf Mapping reduction}

Motivation: Proving that $A_{TM}$ is undecidable was hard. How can we leverage that work? 
Can we relate the decidability / undecidability of one problem to another?

\begin{quote}
If problem $X$ is {\bf no harder than} problem $Y$

\ldots and if $Y$ is easy,

\ldots then $X$ must be easy too.
\end{quote}


\begin{quote}
    If problem $X$ is {\bf no harder than} problem $Y$
    
    \ldots and if $X$ is hard,
    
    \ldots then $Y$ must be hard too.
\end{quote}

``Problem $X$ is no harder than problem $Y$'' means 
``Can answer questions about membership in $X$ by converting them to questions about membership in $Y$''.



Definition: For any languages $A$ and $B$, $A$ is  {\bf  mapping  reducible to} $B$  means there is a computable function 
$f : \Sigma^* \to \Sigma^*$ such that {\it for all} strings  $x$ in $\Sigma^*$, 
\[
x  \in  A \qquad \qquad \text{if and  only  if} \qquad \qquad f(x) \in B.
\]
Notation:  when $A$  is mapping reducible to $B$, we write $A  \leq_m B$.

\vfill

{\it Intuition:} $A \leq_m B$ means $A$ is no harder than $B$, i.e. that the level 
of difficulty of $A$ is less than or equal the level of difficulty of $B$.

\vfill

{\bf TODO} 
\begin{enumerate}
\item What is a computable function?
\item How do mapping reductions help establish the computational difficulty of languages?
\end{enumerate}

\vfill
\newpage
{\bf Computable functions}

Definition: A function $f: \Sigma^* \to \Sigma^*$ is a {\bf computable function} means there is some Turing machine such that, 
for each $x$, on input $x$ the Turing machine halts with exactly $f(x)$ followed by all blanks on the tape

\vspace{50pt}


{\it Examples of computable functions}:

The function that maps a string to a string which is one character longer and 
whose value, when interpreted as a fixed-width binary representation of a
nonnegative integer is twice the value of the input string (when interpreted as 
a fixed-width binary representation of a non-negative integer)
\[
f_1: \Sigma^* \to \Sigma^* \qquad  f_1(x)  = x0
\]

To prove $f_1$ is computable function, we define a Turing machine computing it.

{\it High-level description}
\begin{quote}
    ``On input $w$
    
    1. Append $0$ to $w$.
    
    2. Halt.''
\end{quote}

{\it Implementation-level description}
\begin{quote}
    ``On input $w$
    
    1. Sweep read-write head to the right until find first blank cell.
    
    2. Write 0.
    
    3. Halt.''
\end{quote}

{\it Formal definition} $(\{q0, qacc, qrej\}, \{0,1\}, \{0,1,\textvisiblespace\},\delta, q0, qacc, qrej)$
where $\delta$ is specified by the state diagram: 


\newpage


The function that maps a string to the result of repeating the string twice.
\[
f_2: \Sigma^* \to \Sigma^* \qquad f_2( x )  =  xx
\]

\vfill


The function that maps strings that are not the codes of NFAs to the empty 
string and that maps strings that code NFAs to the code of a DFA that recognizes
the language recognized by the NFA produced by the macro-state construction from Chapter 1.


\vfill


The function that maps strings that are not the codes of Turing machines to the empty 
string and that maps strings that code Turing machines to the code of the 
related Turing machine that acts like the Turing machine coded by the input, except
that if this Turing machine coded by the input tries to reject, the 
new machine will go into a loop.
\[
f_4: \Sigma^* \to \Sigma^*  \qquad f_4( x )  =   \begin{cases}  \varepsilon \qquad&\text{if $x$ is not the code of  a TM} \\
\langle (Q \cup \{q_{trap} \}, \Sigma, \Gamma, \delta', q_0, q_{acc}, q_{rej} ) \rangle \qquad&\text{if $x = \langle (Q, \Sigma, \Gamma, \delta, q_0, q_{acc}, q_{rej} )\rangle$}\end{cases}
\]
where $q_{trap} \notin Q$ and 
\[\delta'( (q,x) ) = \begin{cases}
(r,y,d) &\text{if $q \in Q$, $x \in \Gamma$, $\delta ((q,x)) = (r,y,d)$, and  $r \neq  q_{rej}$} \\
(q_{trap}, \textvisiblespace, R) & \text{otherwise}
\end{cases}
\]
\vfill
\vfill

\newpage

Definition:  $A$ is  {\bf  mapping  reducible to} $B$, $A \leq_m B$m  means there is a computable function 
$f : \Sigma^* \to \Sigma^*$ such that {\it for all} strings  $x$ in $\Sigma^*$, 
\[
x  \in  A \qquad \qquad \text{if and  only  if} \qquad \qquad f(x) \in B.
\]
In this case, we say the function $f$ {\bf witnesses} that $A$ is mapping reducible to $B$.

{\it Making intutition precise \ldots}

{\bf Theorem} (Sipser 5.22): If $A \leq_m B$ and $B$ is decidable, then $A$ is decidable.
    
\vfill
\vfill


{\bf Theorem} (Sipser 5.23): If $A \leq_m B$ and $A$ is undecidable, then $B$ is undecidable.
    
\vfill



\newpage
\subsection*{Friday: The Halting problem}

%! app: Decidable Languages, Undecidable Languages
%! outcome: Classify language, Classify decision problem, Reduction
 

Recall definition:  $A$ is  {\bf  mapping  reducible to} $B$  means there is a computable function 
$f : \Sigma^* \to \Sigma^*$ such that {\it for all} strings  $x$ in $\Sigma^*$, 
\[
x  \in  A \qquad \qquad \text{if and  only  if} \qquad \qquad f(x) \in B.
\]
Notation:  when $A$  is mapping reducible to $B$, we write $A  \leq_m B$.

{\it Intuition:} $A \leq_m B$ means $A$ is no harder than $B$, i.e. that the level 
of difficulty of $A$ is less than or equal the level of difficulty of $B$.


{\it Example}: $A_{TM} \leq_m A_{TM}$  \hfill 
$A_{TM} = \{ \langle M, w \rangle  \mid M \text{ is a TM and } w \text{ is a string and } w \in L(M) \}$

{\it To prove, need a witnessing function $f: \Sigma^* \to \Sigma^*$ that is (1) computable and 
(2) for each $x \in \Sigma^*$, $x \in A_{TM}$ iff $f(x) \in A_{TM}$}


\vfill


{\bf Corollary}: For any language $L$, $L \leq_m L$, as witnessed by 

\newpage

{\it Example}: $A_{DFA} \leq_m \{ ww \mid  w \in \{0,1\}^* \}$ 


{\it To prove, need a witnessing function $f: \Sigma^* \to \Sigma^*$ that is (1) computable and 
(2) for each $x \in \Sigma^*$, $x \in A_{DFA}$ iff $f(x) \in \{ ww \mid  w \in \{0,1\}^* \}$}


\vfill

{\bf Corollary}: For any language decidable language $X$ and any 
set $Y$ with at least one string string in $Y$ and at least one string not in $Y$, $X \leq_m Y$, as witnessed by 

\vfill

%{\it Example}: $EQ_{DFA} \leq_{m} A_{DFA}$
%
%\vfill

%{\it Example}: $\{ 0^i  1^j \mid i  \geq 0, j \geq 0 \} \leq_m A_{TM}$ 
%
%\vfill

Next: consider mapping reductions between potentially undecidable languages.
\newpage

{\bf Halting problem}
    \[
    HALT_{TM} = \{ \langle M, w \rangle \mid \text{$M$ is a  Turing machine, $w$ is  a string, and $M$ halts on $w$} \}
    \]


We know $A_{TM}$ is undecidable. If we could prove that $A_{TM} \leq_m HALT_{TM}$ then we could conclude that $HALT_{TM}$ is undecidable too.

\vfill

Could we adapt our approach from before by tweaking the identity map?

\vfill

\newpage
Define $F: \Sigma^* \to \Sigma^*$ by
    \[
    F(x) =  \begin{cases}
    const_{out} \qquad &\text{if  $x \neq \langle M,w \rangle$ for any Turing machine  $M$ and string  $w$ over the alphabet of $M$} \\
    \langle M'_x, w \rangle \qquad &  \text{if $x = \langle M, w \rangle$ for some Turing machine  $M$ and string $w$ over the alphabet of $M$.}
    \end{cases}
    \]
    where $const_{out}  =  \langle  
        %\includegraphics[width=1.5in]{../../resources/machines/Lect22TM1.png} 
        \begin{tikzpicture}[->,>=stealth',shorten >=1pt, auto, node distance=2cm, semithick]
            \tikzstyle{every state}=[text=black, fill=none]
            
            \node[initial,state] (q0)          {$q0$};
            \node[state,accepting]         (qacc) [right of=q0, xshift=20pt] {$q_{acc}$};
            
            \path (q0) edge  [loop above] node {\parbox{1cm}{$0; \square, R$\newline$1; \square, R$\newline $\square; \square, R$}} (q0)
            ;
          \end{tikzpicture},  \varepsilon  \rangle$
    and  $M'_x$ is a Turing machine that computes like $M$ except, if the computation of $M$ ever were to go to a  reject state,
    $M'_x$ loops instead.   
%    \vfill
%
%    $F( \langle \includegraphics[width=1.5in]{../../resources/machines/Lect22TM1.png} ,  001  \rangle)$ =
%
%   \vfill

    %$F( \langle \includegraphics[width=2.5in]{../../resources/machines/Lect22TM2.png} ,  \varepsilon  \rangle)$ =

    $F( \langle 
        \begin{tikzpicture}[->,>=stealth',shorten >=1pt, auto, node distance=2cm, semithick]
            \tikzstyle{every state}=[text=black, fill=none]
            
            \node[initial,state] (q0)          {$q0$};
            \node[state]         (q1) [right of=q0, xshift=80pt] {$q1$};
            \node[state,accepting]   (qacc) [above of=q0] {$q_{acc}$};
            
            \path (q0) edge [bend left=20] node {\parbox{1cm}{$0; 0,R$\newline $1; 1, R$}} (q1)
                (q1) edge [bend left=20] node {\parbox{1cm}{$0; 0,R$\newline $1; 1, R$}} (q0)
                (q0) edge  [bend left=0] node {$\square; \square, R$} (qacc)
            ;
        \end{tikzpicture}, \varepsilon \rangle)$ = 


    To use this function  to prove that $A_{TM} \leq_m HALT_{TM}$, we need  two claims:

    
    Claim (1): $F$ is computable \phantom{\hspace{2in}}
    
    \vfill

    Claim (2): for every  $x$,  $x \in  A_{TM}$ iff $F(x) \in HALT_{TM}$.  
    
    \vfill
    \vfill
    \vfill

\newpage

\subsection*{Week 9 at a glance}

\vspace{-10pt}

\subsubsection*{Textbook reading: Section 5.3, Section 5.1, Section 3.2}

\vspace{-10pt}

For Monday, Example 5.26 (page 237).

For Wednesday,  Theorem 5.30 (page 238) 

For Friday, skim section 3.2.

{\it For Monday of Week 10}: Definition 7.1 (page 276)
\vspace{-20pt}

\subsubsection*{We will be learning and practicing to:}

\vspace{-20pt}

\begin{itemize}
    \item Clearly and unambiguously communicate computational ideas using appropriate formalism. Translate across levels of abstraction.
    \begin{itemize}
        \item Give examples of sets that are regular, context-free, decidable, or recognizable (and prove that they are).
        \begin{itemize}
          \item {\bf Define and explain computational problems, including $A_{**}$, $E_{**}$, $EQ_{**}$, (for **  DFA or TM) and $HALT_{TM}$}
       \end{itemize}
    \end{itemize}
    \item Know, select and apply appropriate computing knowledge and problem-solving techniques. Reason about computation and systems.
    \begin{itemize}
        \item Use mapping reduction to deduce the complexity of a language by comparing to the complexity of another.
           \begin{itemize}
              \item {\bf Explain what it means for one problem to reduce to another}
              \item {\bf Define computable functions, and use them to give mapping reductions between computational problems}
              \item {\bf Build and analyze mapping reductions between computational problems}
           \end{itemize}
    \item Classify the computational complexity of a set of strings by determining whether it is regular, context-free, decidable, or recognizable.
    \begin{itemize}
         \item {\bf State, prove, and use theorems relating decidability, recognizability, and co-recognizability.}
         \item {\bf Prove that a language is decidable or recognizable by defining and analyzing a Turing machines with appropriate properties.}

   \end{itemize}
   \item  Describe several variants of Turing machines and informally explain why they are equally expressive.
   \begin{itemize}
   \item {\bf Define an enumerator}
   \item {\bf Define nondeterministic Turing machines}
   \item {\bf Use high-level descriptions to define and trace machines (Turing machines and enumerators)}
   \item {\bf Apply dovetailing in high-level definitions of machines}
   \end{itemize}
\end{itemize}
\end{itemize}

\vspace{-20pt}

\subsubsection*{TODO:}
\begin{list}{\itemsep-10pt}
   \item Review Quiz 8 on PrairieLearn (http://us.prairielearn.com), due 3/5/2025
   \item Review Quiz 9 on PrairieLearn (http://us.prairielearn.com), due 3/12/2025
   \item Homework 6 submitted via Gradescope (https://www.gradescope.com/), due 3/13/2025
   \item Project submitted via Gradescope (https://www.gradescope.com/), due 3/19/2025
\end{list}

\newpage

\section*{Monday: Mapping reductions and recognizability}

%! app: Decidable Languages, Undecidable Languages
%! outcome: Classify language, Classify decision problem, Reduction
 

Recall definition:  $A$ is  {\bf  mapping  reducible to} $B$  means there is a computable function 
$f : \Sigma^* \to \Sigma^*$ such that {\it for all} strings  $x$ in $\Sigma^*$, 
\[
x  \in  A \qquad \qquad \text{if and  only  if} \qquad \qquad f(x) \in B.
\]
Notation:  when $A$  is mapping reducible to $B$, we write $A  \leq_m B$.

{\bf Theorem} (Sipser 5.23): If $A \leq_m B$ and $A$ is undecidable, then $B$ is undecidable.
    

{\it Last time} we proved that $A_{TM} \le_m HALT_{TM}$ where
    \[
    HALT_{TM} = \{ \langle M, w \rangle \mid \text{$M$ is a  Turing machine, $w$ is  a string, and $M$ halts on $w$} \}
    \]
and since $A_{TM}$ is undecidable, $HALT_{TM}$ is also undecidable. The function 
witnessing the mapping reduction mapped strings in $A_{TM}$ to strings in $HALT_{TM}$ and 
strings not in $A_{TM}$ to strings not in $HALT_{TM}$ by changing encoded Turing machines to 
ones that had identical computations except looped instead of rejecting.


True or False: $\overline{A_{TM}} \leq_m \overline{HALT_{TM}}$

\vfill

True or False: $HALT_{TM} \leq_m A_{TM}$.

{\bf Proof}: Need computable function  $F: \Sigma^* \to \Sigma^*$  such that  
$x \in HALT_{TM}$ iff $F(x)  \in  A_{TM}$.
Define

\vspace{-15pt}

\begin{quote}
$F =  ``$ On input $x$,
\begin{itemize}
\item[1.] Type-check whether  $x = \langle M, w \rangle$ for some TM $M$ and string $w$. 
If so, move to step 2; if  not, output  $\langle \hspace{2in} \rangle$
\item[2.] Construct the following machine $M'_x$:
\vspace{50pt}
\item[3.] Output $\langle M'_x , w\rangle$."
\end{itemize}
\end{quote}

Verifying correctness: (1) Is function well-defined and computable? (2) Does it have the 
translation property $x \in HALT_{TM}$ iff its image is in $A_{TM}$? 
\begin{center}
\begin{tabular}{|c|c|}
\hline
Input string &  Output string \\
\hline
$\langle M, w \rangle$ where  $M$ halts on $w$ & \phantom{\hspace{4in}} \\
& \\& \\
$\langle M, w \rangle$ where $M$ does not halt on $w$ & \\
& \\&\\
$x$ not encoding any pair of  TM and string   &  \\
& \\
\hline
\end{tabular}
\end{center}

\vfill

\newpage


{\bf Theorem} (Sipser 5.28): If $A \leq_m B$ and $B$ is recognizable, then $A$ is recognizable.

{\bf Proof}: 

\vfill

{\bf Corollary}: If  $A \leq_m B$ and $A$ is unrecognizable, then $B$ is unrecognizable.

\vfill

{\it Strategy}:  

(i) To prove that a recognizable language $R$ is undecidable, prove that $A_{TM} \leq_m R$.


(ii) To prove that a co-recognizable language $U$ is undecidable, prove that $\overline{A_{TM}} \leq_m U$,
 i.e. that $A_{TM} \leq_m \overline{U}$.

 \newpage

\[
E_{TM} = \{ \langle M \rangle \mid \text{$M$ is a Turing machine and $L(M) = \emptyset$} \}
\]

\begin{comment}
Example  string in  $E_{TM}$ is \underline{\phantom{\hspace{1.6in}}} .
Example  string not  in  $E_{TM}$ is \underline{\phantom{\hspace{1.6in}}} .
\end{comment}

Can we find algorithms to recognize

$E_{TM}$  ? 

$\overline{E_{TM}}$ ? 

\vfill


{\bf Claim}: $A_{TM}  \leq_m \overline{E_{TM}}$. {\it And hence also } $\overline{A_{TM}} \leq_m E_{TM}$

{\bf Proof}: Need computable function  $F: \Sigma^* \to \Sigma^*$  such that  $x \in A_{TM}$ iff $F(x)  \notin  E_{TM}$.
Define

\vspace{-15pt}

\begin{quote}
$F =  ``$ On input $x$,
\begin{itemize}
\item[1.] Type-check whether  $x = \langle M, w \rangle$ for some TM $M$ and string $w$. 
If so, move to step 2; if  not, output  $\langle \hspace{2in} \rangle$
\item[2.] Construct the following machine $M'_x$:
\vspace{50pt}
\item[3.] Output $\langle M'_x \rangle$."
\end{itemize}
\end{quote}

Verifying correctness: (1) Is function well-defined and computable? (2) Does it have the 
translation property $x \in A_{TM}$ iff its image is {\bf not} in $E_{TM}$ ? 
\begin{center}
\begin{tabular}{|c|c|}
\hline
Input string &  Output string \\
\hline
$\langle M, w \rangle$ where  $w \in L(M)$ & \phantom{\hspace{4in}} \\
& \\
& \\
& \\
$\langle M, w \rangle$ where $w \notin L(M)$ & \\
& \\
&\\ & \\
$x$ not encoding any pair of  TM and string   &  \\
& \\
& \\
\hline
\end{tabular}
\end{center}

\vfill
    
\newpage
\subsection*{Wednesday: More mapping reductions}

%! app: Decidable Languages, Undecidable Languages
%! outcome: Classify language, Classify decision problem, Reduction
 



Recall:  $A$ is  {\bf  mapping  reducible to} $B$, written $A \leq_m B$,  means there is a computable function 
$f : \Sigma^* \to \Sigma^*$ such that {\it for all} strings  $x$ in $\Sigma^*$, 
\[
x  \in  A \qquad \qquad \text{if and  only  if} \qquad \qquad f(x) \in B.
\]

So far: 
\begin{itemize}
\item $A_{TM}$ is recognizable, undecidable, and not-co-recognizable.
\item $\overline{A_{TM}}$ is unrecognizable, undecidable, and co-recognizable.
\item $HALT_{TM}$ is recognizable, undecidable, and not-co-recognizable.
\item $\overline{HALT_{TM}}$ is unrecognizable, undecidable, and co-recognizable.
\item $E_{TM}$ is unrecognizable, undecidable, and co-recognizable.
\item $\overline{E_{TM}}$ is recognizable, undecidable, and not-co-recognizable.
\end{itemize}


\[
EQ_{TM} = \{ \langle M_1, M_2 \rangle \mid \text{$M_1$ and $M_2$ are both Turing machines and $L(M_1) =L(M_2)$} \}
\]


Can we find algorithms to recognize

$EQ_{TM}$  ? 

$\overline{EQ_{TM}}$ ? 

\vfill

{\it Goal}: Show that $EQ_{TM}$ is not recognizable and that $\overline{EQ_{TM}}$ is not recognizable.

Using Corollary to {\bf Theorem 5.28}: If  $A \leq_m B$ and $A$ is unrecognizable, then $B$ is unrecognizable,
it's enough to prove that 
\begin{itemize}
    \item[] $\overline{HALT_{TM}} \leq_m EQ_{TM}$ \hfill aka $HALT_{TM} \leq_m \overline{EQ_{TM}}$
    \item[] $\overline{HALT_{TM}}  \leq_m \overline{EQ_{TM}}$ \hfill aka $HALT_{TM} \leq_m EQ_{TM}$
\end{itemize}

\vfill

\newpage 
Need computable function  $F_1: \Sigma^* \to \Sigma^*$  such that  $x \in HALT_{TM}$ iff 
$F_1(x)  \notin  EQ_{TM}$.



{\it Strategy}:

\vspace{-15pt}

Map strings $\langle M, w \rangle$ to strings $\langle M'_{x},
\scalebox{0.5}{\begin{tikzpicture}[->,>=stealth',shorten >=1pt, auto, node distance=2cm, semithick]
      \tikzstyle{every state}=[text=black, fill=none]
      \node[initial,state] (q0)                    {$q_0$};
      \node[state,accepting] (qacc) [right of = q0, xshift = 20]{$q_{acc}$};
      \path (q0) edge  [loop above] node {$0, 1, \scalebox{1.5}{\textvisiblespace} \to R$} (q0)
     ;
    \end{tikzpicture}}
    \rangle$ 
    . This image string is not in $EQ_{TM}$ when $L(M'_x) \neq \emptyset$.
    
We will build $M'_x$ so that 
    $L(M'_{x}) = \Sigma^*$ when $M$ halts on $w$ and $L(M'_x) = \emptyset$ when $M$ loops on $w$.


Thus: when $\langle M,w \rangle \in HALT_{TM}$ it gets mapped to a string not in $EQ_{TM}$ and 
when $\langle M,w \rangle \notin HALT_{TM}$ it gets mapped to a string that is in $EQ_{TM}$.

\vfill

Define

\vspace{-15pt}

\begin{quote}
$F_1 =  ``$ On input $x$,
\begin{itemize}
\item[1.] Type-check whether  $x = \langle M, w \rangle$ for some TM $M$ and string $w$. 
If so, move to step 2; if  not, output  $\langle \hspace{2in} \rangle$
\item[2.] Construct the following machine $M'_x$:
\vspace{50pt}
\item[3.] Output $\langle M'_{x},
\scalebox{0.5}{\begin{tikzpicture}[->,>=stealth',shorten >=1pt, auto, node distance=2cm, semithick]
      \tikzstyle{every state}=[text=black, fill=none]
      \node[initial,state] (q0)                    {$q_0$};
      \node[state,accepting] (qacc) [right of = q0, xshift = 20]{$q_{acc}$};
      \path (q0) edge  [loop above] node {$0, 1, \scalebox{1.5}{\textvisiblespace} \to R$} (q0)
     ;
    \end{tikzpicture}}
    \rangle$ "
\end{itemize}
\end{quote}

\vfill

Verifying correctness: (1) Is function well-defined and computable? (2) Does it have the 
translation property $x \in HALT_{TM}$ iff its image is {\bf not} in $EQ_{TM}$ ? 
\begin{center}
\begin{tabular}{|c|c|}
\hline
Input string &  Output string \\
\hline
$\langle M, w \rangle$ where  $M$ halts on $w$ & \phantom{\hspace{4in}} \\
& \\
& \\
& \\
$\langle M, w \rangle$ where $M$ loops on $w$ & \\
& \\
&\\ & \\
$x$ not encoding any pair of  TM and string   &  \\
& \\
& \\
\hline
\end{tabular}
\end{center}


\vfill

Conclude: $HALT_{TM} \leq_m \overline{EQ_{TM}}$
\newpage

\newpage 
Need computable function  $F_2: \Sigma^* \to \Sigma^*$  such that  $x \in HALT_{TM}$ iff 
$F_2(x)  \in  EQ_{TM}$.



{\it Strategy}:

\vspace{-15pt}

Map strings $\langle M, w \rangle$ to strings $\langle M'_{x},
\scalebox{0.5}{\begin{tikzpicture}[->,>=stealth',shorten >=1pt, auto, node distance=2cm, semithick]
      \tikzstyle{every state}=[text=black, fill=none]
      \node[initial,state,accepting] (q0)                    {$q_0$};
     ;
    \end{tikzpicture}}
    \rangle$ 
    . This image string is in $EQ_{TM}$ when $L(M'_x) = \Sigma^*$.
    
We will build $M'_x$ so that 
    $L(M'_{x}) = \Sigma^*$ when $M$ halts on $w$ and $L(M'_x) = \emptyset$ when $M$ loops on $w$.


Thus: when $\langle M,w \rangle \in HALT_{TM}$ it gets mapped to a string  in $EQ_{TM}$ and 
when $\langle M,w \rangle \notin HALT_{TM}$ it gets mapped to a string that is not in $EQ_{TM}$.

\vfill

Define

\vspace{-15pt}

\begin{quote}
$F_2 =  ``$ On input $x$,
\begin{itemize}
\item[1.] Type-check whether  $x = \langle M, w \rangle$ for some TM $M$ and string $w$. 
If so, move to step 2; if  not, output  $\langle \hspace{2in} \rangle$
\item[2.] Construct the following machine $M'_x$:
\vspace{50pt}
\item[3.] Output $\langle M'_{x},
\scalebox{0.5}{\begin{tikzpicture}[->,>=stealth',shorten >=1pt, auto, node distance=2cm, semithick]
      \tikzstyle{every state}=[text=black, fill=none]
      \node[initial,state,accepting] (q0)                    {$q_0$};
     ;
    \end{tikzpicture}}
    \rangle$ "
\end{itemize}
\end{quote}

\vfill

Verifying correctness: (1) Is function well-defined and computable? (2) Does it have the 
translation property $x \in HALT_{TM}$ iff its image is in $EQ_{TM}$ ? 
\begin{center}
\begin{tabular}{|c|c|}
\hline
Input string &  Output string \\
\hline
$\langle M, w \rangle$ where  $M$ halts on $w$ & \phantom{\hspace{4in}} \\
& \\
& \\
& \\
$\langle M, w \rangle$ where $M$ loops on $w$ & \\
& \\
&\\ & \\
$x$ not encoding any pair of  TM and string   &  \\
& \\
& \\
\hline
\end{tabular}
\end{center}


\vfill

Conclude: $HALT_{TM} \leq_m EQ_{TM}$



\vfill
\subsection*{Friday: Other models of computation}
%! app: Decidable Languages, Undecidable Languages
%! outcome: Classify language, Classify decision problem, Reduction, Nondeterminism
 

%! app: Decidable Languages, Undecidable Languages
%! outcome: Formal definition of automata, Informal definition of automata, Classify language, Find example languages

Two models of computation are called {\bf equally expressive} when 
every language recognizable with the first model is recognizable with the second, and vice versa.

True / False: NFAs and PDAs are equally expressive.

True / False: Regular expressions and CFGs are equally expressive.


{\bf  Church-Turing Thesis} (Sipser p. 183): The informal notion of algorithm is formalized completely  and correctly by the 
formal definition of a  Turing machine. In other words: all reasonably expressive models of 
computation are equally expressive with the standard Turing machine.


\begin{center}
{\large \it  Some examples of models that are {\bf equally expressive} with deterministic Turing machines: }
\end{center}

\vfill

\fbox{ {\bf May-stay}  machines }
The May-stay machine model is the same as the usual Turing machine model,  except that
on each transition, the tape head may move L, move R, or Stay. 

Formally: $(Q, \Sigma, \Gamma, \delta, q_0, q_{accept}, q_{reject})$ where 
\[
  \delta: Q \times \Gamma \to Q \times \Gamma \times \{L, R, S\}
\]

{\bf Claim}: Turing machines and May-stay machines are equally expressive. {\it To prove \ldots}

To translate a standard TM to a may-stay machine: never use the direction $S$!


To translate one  of the  may-stay machines to standard TM:
any time TM would Stay, move right  then  left.

\begin{comment}
Formally: suppose $M_S =  (Q, \Sigma, \Gamma, \delta, q_0, q_{acc}, q_{rej})$
has $\delta: Q \times \Gamma \to Q \times \Gamma \times \{L, R, S\}$. Define
the Turing-machine
\[
  M_{new} =  (\phantom{\hspace{2.5in}})
\]

\vfill


\phantom{$M_{new}$ construction here \vspace{400pt}}
\vfill
\end{comment}

\vfill 

\fbox{ {\bf Multitape Turing machine}} A multitape Turing machine with $k$ tapes
can be formally represented as 
$(Q, \Sigma,  \Gamma, \delta, q_0, q_{acc}, q_{rej})$ 
where $Q$ is the finite set of  states,
$\Sigma$ is the  input alphabet with  $\textvisiblespace \notin \Sigma$,
$\Gamma$  is the  tape alphabet with $\Sigma \subsetneq \Gamma$ ,
$\delta: Q\times \Gamma^k\to Q \times \Gamma^k \times \{L,R\}^k$ 
(where $k$ is  the number of  states)


If $M$ is a standard  TM, it is a $1$-tape machine.


To translate a $k$-tape machine  to  a standard TM:
Use a  new symbol to separate the contents of each tape
and keep track of location of  head with  special version of each
tape symbol. {\tiny Sipser Theorem 3.13} 

\includegraphics[width=2.5in]{../../resources/images/Figure314.png}

\newpage
\fbox{ {\bf Enumerators} } Enumerators give a different
model of computation where a language is {\bf produced, one string at a time},
rather than recognized by accepting (or not) individual strings.

Each enumerator machine has finite state control, unlimited work tape, and a printer. The computation proceeds
according to transition function; at any point machine may ``send'' a string to the printer.
\[
E  = (Q, \Sigma, \Gamma, \delta, q_0, q_{print})  
\]
$Q$ is the finite set of states, $\Sigma$ is  the output alphabet, $\Gamma$ is the 
tape alphabet ($\Sigma  \subsetneq\Gamma, 
\textvisiblespace \in \Gamma \setminus \Sigma$), 
\[
\delta:  Q  \times  \Gamma \times \Gamma \to  Q \times  \Gamma \times  \Gamma \times \{L, R\} \times  \{L, R\}
\]
where in state $q$, when the working tape is scanning character $x$ and the printer tape is scanning character $y$,
$\delta( (q,x,y) ) = (q', x', y', d_w, d_p)$ means transition to control state $q'$, write $x'$ on 
the working tape, write $y'$ on the printer tape, move in direction $d_w$ on the working tape, and move in direction 
$d_p$ on the printer tape. The computation starts in $q_0$ and each time the computation enters $q_{print}$
the string from the leftmost edge of the printer tape to the first blank cell is considered to be printed.

The language  {\bf  enumerated} by  $E$, $L(E)$, is $\{ w \in \Sigma^* \mid \text{$E$ eventually, at finite  time, 
prints $w$} \}$.

\begin{comment}
\begin{center}
\begin{tabular}{cc}
\includegraphics[width=3.5in]{../../resources/machines/Lec15enumerator.png}  & 
\begin{tabular}{|c|c|c|c|c|c|c|}
\hline
\multicolumn{1}{|c}{$q0$} &  \multicolumn{6}{c|}{\phantom{A}}\\
\hline
$\textvisiblespace ~*$& $\textvisiblespace$  & $\textvisiblespace$ & $\textvisiblespace$& $\textvisiblespace$& $\textvisiblespace$&  $\textvisiblespace$\\
\hline
$\textvisiblespace  ~*$& $\textvisiblespace$  & $\textvisiblespace$ & $\textvisiblespace$& $\textvisiblespace$& $\textvisiblespace$&  $\textvisiblespace$\\
\hline\hline
\multicolumn{7}{|c|}{\phantom{A}}\\
\hline
\phantom{AA} & \phantom{AA}& \phantom{AA}& \phantom{AA}& \phantom{AA}& \phantom{AA}& \phantom{AA} \\
\hline
\phantom{AA} & \phantom{AA}& \phantom{AA}& \phantom{AA}& \phantom{AA}& \phantom{AA}& \phantom{AA} \\
\hline
\hline
\multicolumn{7}{|c|}{\phantom{A}}\\
\hline
\phantom{AA} & \phantom{AA}& \phantom{AA}& \phantom{AA}& \phantom{AA}& \phantom{AA}& \phantom{AA} \\
\hline
\phantom{AA} & \phantom{AA}& \phantom{AA}& \phantom{AA}& \phantom{AA}& \phantom{AA}& \phantom{AA} \\
\hline
\hline
\multicolumn{7}{|c|}{\phantom{A}}\\
\hline
\phantom{AA} & \phantom{AA}& \phantom{AA}& \phantom{AA}& \phantom{AA}& \phantom{AA}& \phantom{AA} \\
\hline
\phantom{AA} & \phantom{AA}& \phantom{AA}& \phantom{AA}& \phantom{AA}& \phantom{AA}& \phantom{AA} \\
\hline
\hline
\multicolumn{7}{|c|}{\phantom{A}}\\
\hline
\phantom{AA} & \phantom{AA}& \phantom{AA}& \phantom{AA}& \phantom{AA}& \phantom{AA}& \phantom{AA} \\
\hline
\phantom{AA} & \phantom{AA}& \phantom{AA}& \phantom{AA}& \phantom{AA}& \phantom{AA}& \phantom{AA} \\
\hline
\end{tabular}
\end{tabular}
\end{center}


\newpage
\end{comment}

{\bf Theorem 3.21} A language is Turing-recognizable iff some enumerator enumerates it.

{\bf Proof, part 1}: Assume $L$ is enumerated by some enumerator, $E$, so $L = L(E)$. We'll use $E$ in a subroutine
within a high-level description of a new Turing machine that we will build to recognize $L$.

{\bf Goal}: build Turing machine $M_E$ with $L(M_E) = L(E)$.

Define $M_E$ as follows: $M_E = $ ``On input $w$,
\begin{enumerate}
\item Run $E$. For each string $x$ printed by $E$.
\item \qquad Check if $x = w$. If so, accept (and halt); otherwise, continue."
\end{enumerate}

{\bf Proof, part 2}: Assume $L$ is Turing-recognizable and there 
is a Turing  machine  $M$ with  $L = L(M)$. We'll use $M$ in a subroutine
within a high-level description of an enumerator that we will build to enumerate $L$.

{\bf Goal}: build enumerator $E_M$ with $L(E_M) = L(M)$.

{\bf Idea}: check each string in turn to see if it is in $L$.

{\it How?} Run computation of $M$ on each string.  {\it But}: need to be careful 
about computations that don't halt.

{\it Recall} String order for $\Sigma = \{0,1\}$: $s_1 = \varepsilon$, $s_2 = 0$, $s_3 = 1$, $s_4 = 00$, $s_5 = 01$, $s_6  = 10$, 
$s_7  =  11$, $s_8 = 000$, \ldots

Define $E_M$ as follows: $E_{M} = $ `` {\it ignore any input.} Repeat the following for $i=1, 2, 3, \ldots$
\begin{enumerate}
  \item Run the computations of $M$ on $s_1$, $s_2$, \ldots, $s_i$ for (at most) $i$ steps each
  \item For each of these $i$ computations that accept during the (at most) $i$ steps, print
  out the accepted string."
\end{enumerate}

\vfill

\fbox{ {\bf Nondeterministic Turing machine}}

At any point in the computation, the nondeterministic machine may proceed according to 
several possibilities: $(Q, \Sigma, \Gamma, \delta, q_0, q_{acc}, q_{rej})$ where 
\[
\delta: Q \times \Gamma \to \mathcal{P}(Q \times \Gamma \times \{L, R\})  
\]
The computation of a nondeterministic Turing machine is a tree with branching
when the next step of the computation has multiple possibilities. A nondeterministic
Turing machine accepts a string exactly when some branch of the computation tree 
enters the accept state.

Given a nondeterministic machine, we can use a $3$-tape Turing machine to 
simulate it by doing a breadth-first search of computation tree: one tape 
is ``read-only'' input tape, one tape simulates the tape of the nondeterministic
computation, and one tape tracks nondeterministic branching. {\tiny Sipser page 178} 

\vfill

{\bf Summary}

Two models of computation are called {\bf equally expressive} when 
every language recognizable with the first model is recognizable with the second, and vice versa.

To prove the existence of a Turing machine that decides / recognizes some language, 
it's enough to construct an example using any of the equally expressive models.

But: some of the {\bf performance} properties of these models are not equivalent.

\vfill

\newpage
\end{document}