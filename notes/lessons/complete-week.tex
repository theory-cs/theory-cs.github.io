\documentclass[12pt, oneside]{article}

\usepackage[letterpaper, scale=0.89, centering]{geometry}
\usepackage{fancyhdr}
\setlength{\parindent}{0em}
\setlength{\parskip}{1em}

\pagestyle{fancy}
\fancyhf{}
\rfoot{\href{https://creativecommons.org/licenses/by-nc-sa/2.0/}{CC BY-NC-SA 2.0} Version \today~(\thepage.)}

\input{../../resources/CSE20packages}

\begin{document}
\begin{flushright}
    \StrBefore{\currfilename}{.}
\end{flushright}

\section*{Let's get started}

We want you to be successful. 

We will work together to build an 
environment in CSE 105 that supports your learning
in a way that respects your
perspectives, experiences, and identities (including race, ethnicity, heritage, gender, sex, 
class, sexuality, religion, ability, age, educational background, etc.).  
Our goal is for you to  engage
with interesting and challenging concepts and 
feel comfortable exploring, asking questions, and thriving.\\

If you are skipping and stretching meals, or having difficulties affording or accessing food, you may be eligible for CalFresh, California's Supplemental Nutrition Assistance Program, that can provide up to \$292 a month in free money on a debit card to buy food. Students can apply at \url{benefitscal.com/r/ucsandiegocalfresh}.
The Hub Basic Needs Center empowers all students by connecting them to resources for food, stable housing and financial literacy. Visit their site at \url{basicneeds.ucsd.edu}.

Financial aid resources, the possibility of emergency grant funding, and off-campus housing referral 
resources are available: see your College Dean of Student Affairs.

If you find yourself in an uncomfortable situation, ask for help. 
We are committed to upholding University policies regarding nondiscrimination, sexual violence and sexual harassment.
Here are some campus contacts that could provide this help: 
Counseling and Psychological Services (CAPS) at 858 534-3755 or \href{http://caps.ucsd.edu}{http://caps.ucsd.edu}; 
OPHD at 858 534-8298 or ophd@ucsd.edu , \href{http://ophd.ucsd.edu}{http://ophd.ucsd.edu};
CARE at Sexual Assault Resource Center at 858 534-5793 or sarc@ucsd.edu , \href{http://care.ucsd.edu}{http://care.ucsd.edu}.


Please reach out (minnes@ucsd.edu) if you need support with extenuating circumstances affecting CSE 105.

\vfill

\section*{Introductions}
Class website on Canvas \href{https://canvas.ucsd.edu/}{https://canvas.ucsd.edu}


Instructor: Prof. Mia Minnes {\tiny{"Minnes" rhymes with Guinness}}, minnes@ucsd.edu, 
\href{http://cseweb.ucsd.edu/~minnes}{http://cseweb.ucsd.edu/~minnes}


Our team: One instructor + two TAs and five tutors + all of you

Fill in contact info for students around you, if you'd like:


\vfill

\newpage 
Welcome to CSE 105: Introduction to Theory of Computation in Winter 2025!

\section*{CSE 105's Big Questions}
\begin{itemize}
   \item What problems are computers capable of solving?
   \item What resources are needed to solve a problem?
   \item Are some problems harder than others?
\end{itemize}

In this context, a {\bf problem} is defined as: ``Making a decision or computing a value based on some input"

Consider the following problems: 
\begin{itemize}
   \item Find a file on your computer
   \item Determine if your code will compile
   \item Find a run-time error in your code
   \item Certify that your system is un-hackable
\end{itemize}

Which of these is hardest?

\vfill

In Computer Science, we operationalize ``hardest'' as ``requires most resources'', where
resources might be memory, time, parallelism, randomness, power, etc.

To be able to compare ``hardness'' of problems, we use a consistent description of problems

{\bf Input}: String

{\bf Output}: Yes/ No, where Yes means that the input string matches the pattern or property described by the problem.

\newpage

\subsection*{Weeks 0 and 1 at a glance}

\subsubsection*{Textbook reading: Chapter 0, Sections 1.3, 1.1}

\vspace{-15pt}

Before Monday, review class syllabus on Canvas (https://canvas.ucsd.edu/).

Before Wednesday, read Example 1.51.

{\it Notice}: we are jumping to Section 1.3 and then will come back 
to Section 1.1 next week.

Before Friday, read Definition 1.52 (definition of regular expressions) on page 64.

For Week 2 Monday: Figure 1.4 and Definition 1.5 (definition of finite automata) on pages 34-35.

{\it Textbook references: Within a chapter, each item is numbered consecutively. Example 1.51 is the fifty-first numbered item in chapter one.}


\subsubsection*{We will be learning and practicing to:}
%Define decision problem, Regular expressions
\begin{itemize}
\item Clearly and unambiguously communicate computational ideas using appropriate formalism. Translate across levels of abstraction.
\begin{itemize}
   \item Translate a decision problem to a set of strings coding the problem.
   \begin{itemize}
      \item {\bf Distinguish between alphabet, language, sets, and strings}
   \end{itemize}
   \item Use regular expressions and relate them to languages and automata.
   \begin{itemize}
      \item {\bf Write and debug regular expressions using correct syntax}
      \item {\bf Determine if a given string is in the language described by a regular expression}
   \end{itemize}
\end{itemize}
\end{itemize}

\subsubsection*{TODO:}
\begin{list}{\itemsep-10pt}
   \item \#FinAid Assignment on Canvas (complete as soon as possible) and read syllabus on Canvas
   \item Schedule your Test 1 Attempt 1, Test 2 Attempt 1, Test 1 Attempt 2, and Test 2 Attempt 2 times 
   at PrairieTest (http://us.prairietest.com)
   \item Review Quiz 1 on PrairieLearn (http://us.prairielearn.com), complete by 1/15/25
   \item Create a homework group, possibly by using the Piazza (https://piazza.com/) find-a-teammate tool
   \item Homework 1 submitted via Gradescope (https://www.gradescope.com/), due 1/16/25
\end{list}

\newpage

\section*{Week 1 Monday: Terminology and Notation}

\input{../activity-snippets/day1.tex}
    
\newpage

\section*{Week 1 Wednesday: Regular expressions}

\input{../activity-snippets/day2.tex}
    
\newpage

\subsection*{Week 1 Friday: Regular expressions conventions}

%! app: Regular Languages
%! outcome: Regular expressions


{\bf Review}: Determine whether each statement below about regular expressions
over the alphabet $\{a,b,c\}$ is true or false:

\begin{comment}
True or False: \qquad 
   $a  \in L(~(a \cup b )~\cup c)$
\end{comment}

True or False: \qquad 
   $ab  \in L(~ (a \cup b)^*  ~)$
   
True or False: \qquad    
   $ba \in L( ~ a^* b^* ~)$
   
True or False: \qquad 
   $\varepsilon  \in L(a \cup b \cup c)$
   
True or False: \qquad 
   $\varepsilon  \in L(~ (a \cup b)^*  ~)$

True or False: \qquad 
   $\varepsilon \in L( ~ aa^* \cup bb^* ~)$

\vfill

{\it Shorthand and conventions} (Sipser pages 63-65)

\vspace{-20pt}

\begin{center}
    \begin{tabular}{|ll|}
    \hline
    & \\
    \multicolumn{2}{|l|}{Assuming $\Sigma$ is the alphabet, we use the following conventions}\\
    & \\
    $\Sigma$   & regular  expression describing language consisting of  all strings  of length  $1$ over $\Sigma$\\
    $*$ then $\circ$ then $\cup$   & precedence order, unless parentheses are used to change it\\
    $R_1R_2$ & shorthand  for  $R_1  \circ R_2$ (concatenation symbol is implicit) \\
    $R^+$ & shorthand for $R^* \circ R$ \\
    $R^k$ & shorthand for $R$ concatenated with itself $k$ times, where $k$ is a (specific) natural number\\
    & \\
    \hline
    \end{tabular}
\end{center}

\vfill 

{\bf Caution: many programming languages that support regular expressions build in functionality
that is more powerful than the ``pure'' definition of regular expressions given here. }

Regular expressions are everywhere (once you start looking for them).

Software tools and languages often have built-in support for regular expressions to describe
{\bf patterns} that we want to match (e.g. Excel/ Sheets, grep, Perl, python, Java, Ruby).

Under the hood, the first phase of {\bf compilers} is to transform the strings we write 
in code to tokens (keywords, operators, identifiers, literals). Compilers use regular expressions
to describe the sets of strings that can be used for each token type.

Next time: we'll start to see how to build machines that decide whether strings match the pattern
described by a regular expression.

\newpage

Practice with the regular expressions over $\{a,b\}$ below.

For example: Which regular expression(s) below describe a language that includes the string $a$ as an element?

$a^* b^*$ 

\vfill

$a(ba)^* b$

\vfill

$a^* \cup b^*$

\vfill

$(aaa)^*$

\vfill

$(\varepsilon \cup a) b$

\vfill



\newpage

\subsection*{Week 2 at a glance}

\vspace{-15pt}

\subsubsection*{Textbook reading: Sections 1.1, 1.2}

\vspace{-15pt}

Before Monday, read Figure 1.4 and Definition 1.5 (definition of finite automata) on pages 34-35.

Before Wednesday, read pages 41-43 (Figures 1.18, 1.19, 1.20) for examples of automata and languages. Wednesday's class will be asynchronous. Please read over the annotated notes and watch the relevant supplementary videos. Ask any followup questions in the discussion forum or in office hours.

Before Friday, read pages 48-50 (Figures 1.27, 1.29) which introduces nondeterminism.

No class on Week 3 Monday in observance of Martin Luther King Jr. Day.

For Week 3 Wednesday: read the definition of the union, concatenation, and star operations for languages,  given 
as Definition 1.23 on page 44 and a useful example is Example 1.24.

\vspace{-20pt}

\subsubsection*{We will be learning and practicing to:}
%Define decision problem, Formal definition of automata, Informal definition of automata, Find example languages, 
%Describe and use models of computation that don't involve state machines.
\begin{itemize}
\item Clearly and unambiguously communicate computational ideas using appropriate formalism. Translate across levels of abstraction.
\begin{itemize}
   \item Give examples of sets that are regular (and prove that they are).
   \begin{itemize}
      \item {\bf State the definition of the class of regular languages}
      \item {\bf Give examples of regular languages, using each of the three equivalent models of computation for proving regularity.}
   \end{itemize}
   \item Describe and use models of computation that don't involve state machines.
   \begin{itemize}
      \item {\bf Given a DFA, find a regular expression that describes its language.}
      \item {\bf Given a regular expression, find a DFA that recognizes its language.}
   \end{itemize}
   \item Use precise notation to formally define the state diagram of finite automata.
   \item Use clear English to describe computations of finite automata informally.
      \begin{itemize}
         \item {\bf State the formal definition of (deterministic) finite automata}   
         \item {\bf Trace the computation of a finite automaton on a given string using its state diagram}
         \item {\bf Translate between a state diagram and a formal definition}
         \item {\bf Determine if a given string is in the language recognized by a finite automaton}
         \item {\bf Design an automaton that recognizes a given language}
         \item {\bf Specify a general construction for DFA based on parameters}
         \item {\bf Design general constructions for DFA}
   \end{itemize}
\end{itemize}

\end{itemize}
\newpage
\subsubsection*{TODO:}
\begin{list}{\itemsep-10pt}
   \item \#FinAid Assignment on Canvas (complete as soon as possible) and read syllabus on Canvas
   \item Schedule your Test 1 Attempt 1, Test 2 Attempt 1, Test 1 Attempt 2, and Test 2 Attempt 2 times 
   at PrairieTest (http://us.prairietest.com)
   \item Review Quiz 1 on PrairieLearn (http://us.prairielearn.com), complete by 1/15/25
   \item Create a homework group, possibly by using the Piazza (https://piazza.com/) find-a-teammate tool
   \item Homework 1 submitted via Gradescope (https://www.gradescope.com/), due 1/16/25
   \item Review Quiz 2 on PrairieLearn (http://us.prairielearn.com), complete by 1/22/25
\end{list}


\newpage

\subsection*{Week 2 Monday: Finite automata}

\input{../activity-snippets/day4.tex}


\subsection*{Week 2 Wednesay: Finite automaton constructions - asynchronous}

\input{../activity-snippets/day5.tex}

\newpage
\subsection*{Week 2 Friday: Nondeterministic automata}

We saw that whenever a language is recognized by a DFA, its
complement is also recognized by some (other) DFA. 

Another way to say this is that the collection of languages
that are each recognizable by a DFA is {\bf closed} under complementation.


Before we continue with more complicated constructions, let's revisit some of 
the assumption of the {\bf D}FA model.

Recall that the computation of {\bf deterministic} finite automaton has exactly one choice for its next step given the current state and the current character to read.


%! app: Regular Languages
%! outcome: Regular expressions, Formal definition of automata, Informal definition of automata

\begin{center}
\begin{tabular}{|ll|}
\hline
\multicolumn{2}{|l|}{{\bf Nondeterministic finite automaton}  (Sipser Page 53) Given as $M = (Q, \Sigma, \delta, q_0, F)$}\\
& \\
Finite set of states $Q$  & Can  be labelled by any collection  of distinct names. Default: $q0, q1, \ldots$  \\
Alphabet $\Sigma$ &  Each input to the automaton is a string over  $\Sigma$. \\
Arrow labels $\Sigma_\varepsilon$ &  $\Sigma_\varepsilon = \Sigma \cup \{ \varepsilon\}$. \\
&  Arrows 
in the state diagram are labelled either by symbols from $\Sigma$ or by $\varepsilon$ \\
Transition function $\delta$  & $\delta: Q \times \Sigma_{\varepsilon} \to \mathcal{P}(Q)$
gives the {\bf set of possible next states} for a transition \\
&  from the current state upon reading a symbol or spontaneously moving.\\
Start state $q_0$ & Element of $Q$.  Each computation of the machine starts at the  start  state.\\
Accept (final) states $F$ & $F \subseteq  Q$.\\
& \\
\multicolumn{2}{|p{\textwidth}|}{$M$ accepts the input string $w \in \Sigma^*$ if and only if {\bf there is} a computation of $M$ on 
$w$ that processes the whole string and ends in an
accept state.}\\
\hline
\end{tabular}
\end{center}

The formal definition of the NFA over $\{0,1\}$ given by this state diagram is: 

\begin{center}
    \begin{tikzpicture}[->,>=stealth',shorten >=1pt, auto, node distance=2cm, semithick]
       \tikzstyle{every state}=[text=black, fill=none]
       
       \node[initial,state] (q0)          {$q0$};
       \node[state,accepting]         (q1) [right of=q0, xshift=10pt]{$q1$};
       
       \path (q0) edge  [loop above] node {$0,1$} (q0)
           (q0) edge [bend left=0] node {$1$} (q1)
       ;
    \end{tikzpicture}
 \end{center}


The language over $\{0,1\}$ recognized by this NFA is:

\vspace{70pt}

{\it Practice}: Change the transition function to get a different NFA which accepts
the empty string (and potentially other strings too).


\newpage

The state diagram of an NFA over $\{a,b\}$ is:

\begin{center}
    \begin{tikzpicture}[->,>=stealth',shorten >=1pt, auto, node distance=2cm, semithick]
       \tikzstyle{every state}=[text=black, fill=none]
       
       \node[initial,state] (q0)          {$q0$};
       \node[state,accepting]         (n) [below right of=q0, xshift=20pt] {$n$};
       \node[state]     (d) [right of=n, xshift=20pt] {$d$};
       \node[state]     (q) [above right of=q0, xshift=10pt] {$q$};
       \node[state]     (r) [right of=q, xshift=10pt] {$r$};
       \node[state,accepting]     (s) [right of=r, xshift=10pt] {$s$};
       
       \path (q0) edge  [bend left=0] node {$\varepsilon$} (q)
           (q0) edge  [bend left=0] node {$\varepsilon$} (n)
           (q) edge [bend left=0] node {$a$} (r)
           (q) edge [loop above] node {$b$} (q)
           (r) edge [bend left=0] node {$b$} (s)
           (r) edge [loop above] node {$a$} (r)
           (s) edge [loop above] node {$a,b$} (s)
           (n) edge [bend left=20] node {$a,b$} (d)
           (d) edge [bend left=20] node {$a,b$} (n)
       ;
    \end{tikzpicture}
 \end{center}

 The formal definition of this NFA is:

 \vfill

\newpage

Suppose $A_1, A_2$ are languages over an alphabet $\Sigma$.
{\bf Claim:} if there is a NFA $N_1$ such that $L(N_1) = A_1$ and 
NFA $N_2$ such that $L(N_2) = A_2$, then there is another NFA, let's call it $N$, such that 
$L(N) = A_1 \cup A_2$.

{\bf Proof idea}: Use nondeterminism to choose which of $N_1$, $N_2$ to run.

\vfill
\begin{comment}
    Draw schematic
\end{comment}

{\bf Formal construction}: Let 
$N_1 = (Q_1, \Sigma, \delta_1, q_1, F_1)$ and $N_2 = (Q_2, \Sigma, \delta_2,q_2, F_2)$
and assume $Q_1 \cap Q_2 = \emptyset$ and that $q_0 \notin Q_1 \cup Q_2$.
Construct $N = (Q, \Sigma, \delta, q_0, F_1 \cup F_2)$ where
\begin{itemize}
    \item $Q = $
    \item $\delta: Q \times \Sigma_\varepsilon \to \mathcal{P}(Q)$ is defined by, for $q \in Q$ and $x \in \Sigma_{\varepsilon}$:
        \[
            \phantom{\delta((q,x))=\begin{cases}  \delta_1 ((q,x)) &\qquad\text{if } q\in Q_1 \\ \delta_2 ((q,x)) &\qquad\text{if } q\in Q_2 \\ \{q1,q2\} &\qquad\text{if } q = q_0, x = \varepsilon \\ \emptyset\text{if } q= q_0, x \neq \varepsilon \end{cases}}
        \]
\end{itemize}


\vfill
{\it Proof of correctness would prove that $L(N) = A_1 \cup A_2$ by considering
an arbitrary string accepted by $N$, tracing an accepting computation of $N$ on it, and using 
that trace to prove the string is in at least one of $A_1$, $A_2$; then, taking an arbitrary 
string in $A_1 \cup A_2$ and proving that it is accepted by $N$. Details left for extra practice.}




\newpage

\subsection*{Week 3 at a glance}


\subsubsection*{Textbook reading: Chapter 1}

\vspace{-15pt}

No class on Week 3 Monday in observance of Martin Luther King Jr. Day.

Before Wednesday: read the definition of the union, concatenation, and star operations for languages,  given 
as Definition 1.23 on page 44 and a useful example is Example 1.24.

Before Friday,  read pages 45-46 (Theorem 1.25) that we'll refer to as a ``closure proof".

For Week 4 Monday, read Introduction to Section 1.4 (page 77) which introduces nonregularity.


\vspace{-20pt}

\subsubsection*{We will be learning and practicing to:}

\vspace{-15pt}

%Define decision problem, Formal definition of automata, Informal definition of automata, Find example languages, 
%Describe and use models of computation that don't involve state machines.
\begin{itemize}
\item Clearly and unambiguously communicate computational ideas using appropriate formalism. Translate across levels of abstraction.
\begin{itemize}
   \item Use precise notation to formally define the state diagram of finite automata.
   \item Use clear English to describe computations of finite automata informally.
      \begin{itemize}
         \item {\bf Motivate the use of nondeterminism}
         \item {\bf State the formal definition of NFA}   
         \item {\bf Trace the computation(s) of a NFA on a given string using its state diagram}
         \item {\bf Determine if a given string is in the language recognized by a NFA}
         \item {\bf Translate between a state diagram and a formal definition of a NFA}
      \end{itemize}
   \item Give examples of sets that are regular (and prove that they are).
   \begin{itemize}
      \item {\bf State the definition of the class of regular languages}
      \item {\bf Give examples of regular languages, using each of the three equivalent models of computation for proving regularity.}
      \item {\bf Choose between multiple models to prove that a language is regular}
      \item {\bf Explain the limits of the class of regular languages}
   \end{itemize}
   \item Describe and use models of computation that don't involve state machines.
   \begin{itemize}
      \item {\bf Given a DFA or NFA, find a regular expression that describes its language.}
      \item {\bf Given a regular expression, find a DFA or NFA that recognizes its language.}
   \end{itemize}
\end{itemize}


\item Understand, guide, shape impact of computing on society/the world. Connect the role of Theory CS classes to other applications (in undergraduate CS curriculum and beyond). Model problems using appropriate mathematical concepts.
 \begin{itemize}
     \item Explain nondeterminism and describe tools for simulating it with deterministic computation.
     \begin{itemize}
       \item {\bf Given a NFA, find a DFA that recognizes its language.}
       \item {\bf Convert between regular expressions and automata}
     \end{itemize}
 \end{itemize}

\end{itemize}

\vspace{-20pt}

\subsubsection*{TODO:}
\begin{list}{\itemsep-10pt}
   \item Schedule your Test 1 Attempt 1, Test 2 Attempt 1, Test 1 Attempt 2, and Test 2 Attempt 2 times 
   at PrairieTest (http://us.prairietest.com)
   \item Review Quiz 3 on PrairieLearn (http://us.prairielearn.com), due 1/29/2025
   \item Homework 2 submitted via Gradescope (https://www.gradescope.com/), due Tuesday 1/30/2025
\end{list}


\vfill

In Computer Science, we operationalize ``hardest'' as ``requires most resources'', where
resources might be memory, time, parallelism, randomness, power, etc.
To be able to compare ``hardness'' of problems, we use a consistent description of problems

{\bf Input}: String

{\bf Output}: Yes/ No, where Yes means that the input string matches the pattern or property described by the problem.

So far: we saw that regular expressions are convenient ways of describring patterns in strings.
{\bf Finite automata} give a model of computation for processing strings and and classifying them into Yes (accepted)
or No (rejected). We will see that each set of strings is described by a regular expression if and only 
if there is a FA that recognizes it.  Another way of thinking about it: properties described by regular
expressions require exactly the computational power of these finite automata.

\newpage
\subsection*{Wednesday: Automata constructions}

\input{../activity-snippets/day7.tex}

\newpage
\subsection*{Friday: Regular langauges}

\input{../activity-snippets/day8.tex}

\newpage

\subsection*{Week 4 at a glance}

\subsubsection*{Textbook reading: Section 1.4, 2.2, 2.1.}

\vspace{-15pt}

Before Monday, read Introduction to Section 1.4 (page 77) which introduces nonregularity.

Before Wednesday, read Definition 2.13 (page 111-112) introducing Pushdown Automata.

Before Friday, read Example 2.18 (page 114).

For Week 5 Monday: read Introduction to Section 2.1 (pages 101-102).

\vspace{-20pt}

\subsubsection*{We will be learning and practicing to:}

\begin{itemize}
    \item Clearly and unambiguously communicate computational ideas using appropriate formalism. Translate across levels of abstraction.
    \begin{itemize}
       \item Give examples of sets that are regular (and prove that they are).
       \begin{itemize}
          \item {\bf State the definition of the class of regular languages}
          \item {\bf Explain the limits of the class of regular languages}
          \item {\bf Identify some regular sets and some nonregular sets}
       \end{itemize}
       \item Use precise notation to formally define the state diagram of a PDA
       \item Use clear English to describe computations of PDA informally.
       \begin{itemize}
           \item {\bf Define push-down automata informally and formally}
           \item {\bf State the formal definition of a PDA}
           \item {\bf Trace the computation(s) of a PDA on a given string using its state diagram}
           \item {\bf Determine if a given string is in the language recognized by a PDA}
           \item {\bf Translate between a state diagram and a formal definition of a PDA}
           \item {\bf Determine the language recognized by a given PDA}
        \end{itemize}

    \end{itemize}

    \item Know, select and apply appropriate computing knowledge and problem-solving techniques. 
    \begin{itemize}
    \item Apply classical techniques including pumping lemma, determinization, diagonalization, and reduction to analyze the complexity of languages and problems.
    \begin{itemize}
        \item {\bf Justify why the Pumping Lemma is true.}
        \item {\bf Use the pumping lemma to prove that a given language is not regular.}
    \end{itemize}
    \end{itemize}
\end{itemize}

\vspace{-20pt}

\subsubsection*{TODO:}
\begin{list}{\itemsep-10pt}
   \item Schedule your Test 1 Attempt 1, Test 2 Attempt 1, Test 1 Attempt 2, and Test 2 Attempt 2 times 
   at PrairieTest (http://us.prairietest.com)
   \item Review Quiz 3 on PrairieLearn (http://us.prairielearn.com), due 1/29/2025
   \item Homework 2 submitted via Gradescope (https://www.gradescope.com/), due 1/30/2025
   \item Review Quiz 4 on PrairieLearn (http://us.prairielearn.com), due 2/5/2025
\end{list}

\newpage
\subsection*{Monday: Pumping Lemma}

%! app: Regular Languages
%! outcome: Classify language, Find example languages, Pumping Lemma

{\bf Definition and Theorem}: For an alphabet $\Sigma$, a language $L$ over $\Sigma$ is called {\bf regular}
exactly when $L$ is recognized by some DFA, which happens exactly when $L$ is recognized by some NFA, 
and happens exactly when $L$ is described by some regular expression

{\bf We saw that}: The class of regular languages is closed under complementation, union, 
intersection, set-wise concatenation, and Kleene star.

{\it Extra practice}: 

{\bf Disprove}: There is some alphabet $\Sigma$ for which there is 
some language recognized by an NFA but not by any DFA.

\vfill

{\bf Disprove}: There is some alphabet $\Sigma$ for which there is 
some finite language not described by any regular expression over $\Sigma$.

\vfill

{\bf Disprove}: If a language is recognized by an NFA 
then the complement of this language is not recognized by any DFA.

\vfill


{\bf Fix alphabet $\Sigma$. Is every language $L$ over $\Sigma$ regular?}

\begin{center}
\begin{tabular}{c|c}
Set & Cardinality \\
\hline
& \\
$\{0,1\}$ & \\
& \\
$\{0,1\}^*$ & \\
& \\
$\mathcal{P}( \{0,1\})$ & \\
& \\
The set of all languages over $\{0,1\}$ & \\
& \\
The set of all regular expressions over $\{0,1\}$ & \\
& \\
The set of all regular languages over $\{0,1\}$ & \\
& \\
\end{tabular}
\end{center}

\newpage

Strategy: Find an {\bf invariant} property that is true of all regular languages. When analyzing 
a given language, if the invariant is not true about it, then the language is not regular.

\vfill

{\bf Pumping Lemma} (Sipser Theorem 1.70): If $A$ is a regular language, then there
is a number $p$ (a {\it pumping length}) where, if $s$ is any string in $A$ of length at least $p$, 
then $s$ may be divided into three pieces, $s = xyz$ such that
\vspace{-10pt}
\begin{itemize}
\item $|y| > 0$
\item for each $i \geq 0$, $xy^i z \in A$
\item $|xy| \leq p$.
\end{itemize}
\vfill

{\bf Proof idea}: In DFA, the only memory available is in the states. 
Automata can only
``remember'' finitely far in the past and finitely much information, because
they can have only finitely many states. If a computation path of a DFA visits 
the same state more than once, the machine can't tell the difference between 
the first time and future times it visits this state. Thus, if 
a DFA accepts one long string, then it must accept (infinitely) many 
similar strings.
\vfill

{\bf Proof illustration}


\vfill
\vfill


\newpage

{\bf True or False}: A pumping length for $A = \{ 0,1 \}^*$ is $p = 5$.

\vfill


{\bf True or False}: A pumping length for $A = \{ 0,1 \}^*$ is $p = 2$.

\vfill



{\bf True or False}: A pumping length for $A = \{ 0,1 \}^*$ is $p = 105$.

\vfill



Restating {\bf Pumping Lemma}: If $L$ is a regular language, then it  has
a pumping length.


{\bf Contrapositive}: If $L$ has no pumping length, then  it is nonregular.

\vfill

{\Large The Pumping Lemma {\it cannot} be used to prove that a language {\it is} regular.} 

{\Large The Pumping Lemma {\bf can} be used to prove that a language {\it is not} regular.}

{\it Extra practice}: Exercise 1.49 in the book.


\vfill

{\bf Proof strategy}: To prove that a language $L$ is {\bf not} regular, 
\begin{itemize}
    \item Consider an arbitrary positive integer $p$
    \item Prove that $p$ is not a pumping length for $L$
    \item Conclude that $L$ does not have {\it any} pumping length, and therefore it is not regular.
\end{itemize}


{\bf Negation}: A positive integer  $p$  is {\bf not a pumping length} of a language  $L$ over  $\Sigma$  iff
\[
\exists s \left(~  |s| \geq  p \wedge s \in L \wedge \forall x \forall y \forall z  \left( ~\left( s = xyz \wedge 
|y| > 0 \wedge |xy| \leq p~ \right) \to \exists i  (  i \geq 0  \wedge xy^iz  \notin L ) \right) ~\right) 
\]

\newpage
\subsection*{Wednesday: Proving nonregularity, and beyond}

\input{../activity-snippets/day10.tex}
    
\newpage
\subsection*{Friday: Pushdown Automata}

\input{../activity-snippets/day11.tex}


\newpage

\subsection*{Week 5 at a glance}

\subsubsection*{Textbook reading: Chapter 2}

\vspace{-20pt}

Before Monday, read Introduction to Section 2.1 (pages 101-102).

Before Wednesday, read Section 2.1

Before Friday, read Theorem 2.20.

For Week 6 Monday: Page 165-166 Introduction to Section 3.1.

\vspace{-20pt}

\subsubsection*{We will be learning and practicing to:}
\vspace{-20pt}

\begin{itemize}
    \item Clearly and unambiguously communicate computational ideas using appropriate formalism. Translate across levels of abstraction.
    \begin{itemize}
        \item Describe and use models of computation that don't involve state machines.
        \begin{itemize}
            \item {\bf Identify the components of a formal definition of a context-free grammar (CFG)}
            \item {\bf Derive strings in the language of a given CFG}
            \item {\bf Determine the language of a given CFG}
            \item {\bf Design a CFG generating a given language}
            \item {\bf Use context-free grammars and relate them to languages and pushdown automata.}
        \end{itemize}
        \item Use precise notation to formally define the state diagram of a Turing machine
        \item Use clear English to describe computations of Turing machines informally.
        \begin{itemize}
            \item {\bf Design a PDA that recognizes a given language.}
         \end{itemize}
       \item Give examples of sets that are context-free (and prove that they are).
       \begin{itemize}
          \item {\bf State the definition of the class of context-free languages}
          \item {\bf Explain the limits of the class of context-free languages}
          \item {\bf Identify some context-free sets and some non-context-free sets}
       \end{itemize}
    \end{itemize}
    \item Know, select and apply appropriate computing knowledge and problem-solving techniques. 
    Reason about computation and systems.
    \begin{itemize}
        \item Describe and prove closure properties of classes of languages under certain operations.
        \begin{itemize}
            \item {\bf Apply a general construction to create a new PDA or CFG from an example one.}
            \item {\bf Formalize a general construction from an informal description of it.}
            \item {\bf Use general constructions to prove closure properties of the class of context-free langugages.}
            \item {\bf Use counterexamples to prove non-closure properties of the class of context-free langugages.}
        \end{itemize}
    \end{itemize}
\end{itemize}

\vspace{-20pt}

\subsubsection*{TODO:}
\begin{list}{\itemsep-10pt}
   \item Schedule your Test 1 Attempt 1, Test 2 Attempt 1, Test 1 Attempt 2, and Test 2 Attempt 2 times
   at PrairieTest (http://us.prairietest.com) . The first Test 1 sessions are next week!
   \item Review Quiz 4 on PrairieLearn (http://us.prairielearn.com), due 2/5/2025
   \item Homework 3 submitted via Gradescope (https://www.gradescope.com/), due 2/6/2025
   \item Review Quiz 5 on PrairieLearn (http://us.prairielearn.com), due 2/12/2025
\end{list}

\newpage



\newpage
\subsection*{Monday: More Pushdown Automata}

%! app: Regular Languages, Context-free Languages
%! outcome: Formal definition of automata, Informal definition of automata, Nondeterminism, Classify language, Find example languages
    

{\bf Definition} A {\bf pushdown automaton} (PDA) is  specified by a  $6$-tuple $(Q, \Sigma, \Gamma, \delta, q_0, F)$
where $Q$ is the finite set of states, $\Sigma$ is the input alphabet,  $\Gamma$ is the stack alphabet,
\[
    \delta: Q \times \Sigma_\varepsilon  \times  \Gamma_\varepsilon \to \mathcal{P}( Q \times \Gamma_\varepsilon)
\]
is the transition function,  $q_0 \in Q$ is the start state, $F \subseteq  Q$ is the set of accept states.


%Draw the state diagram and give the formal definition of a PDA with $\Sigma = \Gamma$.

%\vfill

%Draw the state diagram and give the formal definition of a PDA with $\Sigma \cap \Gamma = \emptyset$.
    
%\vfill

%\newpage
For the PDA state diagrams below, $\Sigma = \{0,1\}$.

\vspace{-15pt}

\begin{center}
\begin{tabular}{c c}
Mathematical description of language & State diagram of PDA recognizing language\\
\hline
& $\Gamma = \{ \$, \#\}$ \hspace{2.3in} \\
& \\
& 
\begin{tikzpicture}[->,>=stealth',shorten >=1pt, auto, node distance=2cm, semithick]
    \tikzstyle{every state}=[text=black, fill=none]
    
    \node[initial,state] (q0)          {$q0$};
    \node[state]         (q1) [right of=q0, xshift=20pt] {$q1$};
    \node[state]         (q2) [below right of=q1, xshift=20pt] {$q2$};
    \node[state]         (q3) [right of=q2, xshift=20pt] {$q3$};
    \node[state,accepting]         (q4) [left of=q2, xshift=-20pt] {$q4$};
    
    \path (q0) edge [bend left=0] node {$\varepsilon, \varepsilon; \$$} (q1)
        (q1) edge  [loop above] node {$0, \varepsilon; \#$} (q1)
        (q1) edge [bend left=0] node {$\varepsilon, \varepsilon; \varepsilon$} (q2)
        (q2) edge  [bend left=20] node [midway, above] {$1, \#; \varepsilon$} (q3)
        (q3) edge  [bend left=20] node [midway, below] {$1, \varepsilon; \varepsilon$} (q2)
        (q2) edge  [bend left=0] node {$\varepsilon, \$; \varepsilon$} (q4)
    ;
\end{tikzpicture}
\\
& \\
& \\
\hline
& $\Gamma = \{ \sun, 1\}$ \hspace{2.3in} \\
& \\
& 
\begin{tikzpicture}[->,>=stealth',shorten >=1pt, auto, node distance=2cm, semithick]
    \tikzstyle{every state}=[text=black, fill=none]
    
    \node[initial,state] (q0)          {$q0$};
    \node[state]         (q1) [right of=q0, xshift=20pt] {$q1$};
    \node[state]         (q2) [below right of=q1, xshift=20pt] {$q2$};
    \node[state]         (q3) [right of=q2, xshift=20pt] {$q3$};
    \node[state,accepting]  (q4) [right of=q3, xshift=20pt] {$q4$};
    \node[state]         (q5) [above right of=q1, xshift=20pt] {$q5$};
    \node[state,accepting]  (q6) [right of=q5, xshift=20pt] {$q6$};
    
    \path (q0) edge [bend left=0] node {$\varepsilon, \varepsilon; \sun$} (q1)
        (q1) edge  [loop above] node {$1, \varepsilon; 1$} (q1)
        (q1) edge [bend left=0] node [below] {$\varepsilon, \varepsilon; \varepsilon$} (q5)
        (q1) edge [bend left=0] node [above]{$\varepsilon, \varepsilon; \varepsilon$} (q2)
        (q5) edge [loop above] node {$0, 1, ; \varepsilon$} (q5)
        (q5) edge [bend left=0] node {$\varepsilon, \sun; \varepsilon$} (q6)
        (q6) edge [loop above] node {$1, \varepsilon; \varepsilon$} (q6)
        (q2) edge [loop below] node {$0, \varepsilon; \varepsilon$} (q2)
        (q2) edge  [bend left=0] node {$\varepsilon, \varepsilon; \varepsilon$} (q3)
        (q3) edge  [loop below] node {$1, 1; \varepsilon$} (q3)
        (q3) edge  [bend left=0] node {$\varepsilon, \sun; \varepsilon$} (q4)
    ;
\end{tikzpicture}
\\
& \\
& \\
\hline
& \\
& \\
& \\
$\{ 0^i 1^j 0^k \mid i,j,k \geq 0 \}$ & \\
& \\
& \\

\end{tabular}
\end{center}
{\it Note: alternate notation is to replace $;$ with $\to$ on arrow labels.}

\newpage
Corollary: for each language $L$ over $\Sigma$, if there is an NFA $N$ with $L(N)=L$
then there is a PDA $M$ with $L(M) = L$

Proof idea: Declare stack alphabet to be $\Gamma = \Sigma$ and then 
don't use stack at all. 


 \vfill

\begin{comment}
{\it Extra practice}: Consider the state diagram of a PDA with input alphabet 
$\Sigma$ and stack alphabet $\Gamma$.

\begin{center}
\begin{tabular}{|c|c|}
\hline
Label & means \\
\hline
$a, b ; c$ when $a \in \Sigma$, $b\in \Gamma$, $c \in \Gamma$ 
& \hspace{3in} \\
& \\
& \\
& \\
& \\
&\\
\hline
$a, \varepsilon ; c$ when $a \in \Sigma$, $c \in \Gamma$ 
& \hspace{3in} \\
& \\
& \\
& \\
& \\
&\\
\hline
$a, b ; \varepsilon$ when $a \in \Sigma$, $b\in \Gamma$
& \hspace{3in} \\
& \\
& \\
& \\
& \\
&\\
\hline
$a, \varepsilon ; \varepsilon$ when $a \in \Sigma$
& \hspace{3in} \\
& \\
& \\
& \\
& \\
&\\
\hline
\end{tabular}
\end{center}


How does the meaning change if $a$ is replaced by $\varepsilon$?
\end{comment}

{\it Big picture}: PDAs are motivated by wanting to add some memory of unbounded size to NFA. How 
do we accomplish a similar enhancement of regular expressions to get a syntactic model that is 
more expressive?

DFA, NFA, PDA: Machines process one input string at a time; the computation of a machine on its input string 
reads the input from left to right.

Regular expressions: Syntactic descriptions of all strings that match a particular pattern; the language 
described by a regular expression is built up recursively according to the expression's syntax

{\bf Context-free grammars}: Rules to produce one string at a time, adding characters from the middle, beginning, 
or end of the final string as the derivation proceeds.\\

\vfill


\newpage
\section*{Wednesday: Context-free Grammars and Languages}

\input{../activity-snippets/day13.tex}
    
\newpage
\subsection*{Friday: Context-free and non-context-free languages}

\input{../activity-snippets/day14.tex}




\newpage

\subsection*{Week 6 at a glance}

\vspace{-10pt}

\subsubsection*{Textbook reading: Chapter 3}

\vspace{-10pt}

Before Monday, Page 165-166 Introduction to Section 3.1.

\vspace{-10pt}

Before Wednesday, Example 3.9 on page 173.

\vspace{-10pt}

Before Friday, Page 184-185 Terminology for describing Turing machines.

\vspace{-10pt}

For Week 7 Monday: Introduction to Chapter 4.

\vspace{-20pt}

\subsubsection*{We will be learning and practicing to:}
\vspace{-20pt}

\begin{itemize}
    \item Clearly and unambiguously communicate computational ideas using appropriate formalism. Translate across levels of abstraction.
    \begin{itemize}
        \item Use precise notation to formally define the state diagram of a Turing machine
        \item Use clear English to describe computations of Turing machines informally.
        \begin{itemize}
                \item {\bf Motivate the definition of a Turing machine}
                \item {\bf Trace the computation of a Turing machine on given input}
                \item {\bf Describe the language recognized by a Turing machine}
                \item {\bf Determine if a Turing machine is a decider}
                \item {\bf Given an implementation-level description of a Turing machine}
                \item {\bf Use high-level descriptions to define and trace Turing machines}
                \item {\bf Apply dovetailing in high-level definitions of machines}
         \end{itemize}
       \item Give examples of sets that are recognizable and decidable (and prove that they are).
       \begin{itemize}
          \item {\bf State the definition of the class of recognizable languages}
          \item {\bf State the definition of the class of decidable languages}
          \item {\bf State the definition of the class of co-recognizable languages}
       \end{itemize}
    \end{itemize}
    \item Know, select and apply appropriate computing knowledge and problem-solving techniques. 
    Reason about computation and systems.
    \item Describe and prove closure properties of classes of languages under certain operations.
    \begin{itemize}
        \item {\bf Apply a general construction to create a new Turing machine from an example one.}
        \item {\bf Formalize a general construction from an informal description of it.}
        \item {\bf Use general constructions to prove closure properties of the class of decidable langugages.}
        \item {\bf Use general constructions to prove closure properties of the class of recognizable langugages.}
    \end{itemize}
\end{itemize}

\vspace{-20pt}

\subsubsection*{TODO:}
\begin{list}{\itemsep -20pt}
   \item This week: Test 1 Attempt 1 in CBTF. 
   \item Review Quiz 5 on PrairieLearn (http://us.prairielearn.com), due 2/12/2025
   \item Mid quarter feedback survey on Canvas.
\end{list}

\newpage

\section*{Monday: Descriptions of Turing machines}

\input{../activity-snippets/day15.tex}
    
\newpage
\subsection*{Wednesday: Recognizable and decidable languages}

\input{../activity-snippets/day16.tex}


\newpage
\subsection*{Friday: Closure for the classes of recognizable and decidable languages}

\input{../activity-snippets/day17.tex}



\newpage

\subsection*{Week 7 at a glance}

\subsubsection*{Textbook reading: Chapter 4}

\vspace{-20pt}
No class on Monday in observance of UCSD holiday.

Before Wednesday, Introduction to Chapter 4.

Before Friday, Decidable problems concerning regular languages, 
Sipser pages 194-196.

For Week 8 Monday: An undecidable language, Sipser pages 207-209.

\vspace{-20pt}

\subsubsection*{We will be learning and practicing to:}

\vspace{-20pt}

\begin{itemize}
    \item Clearly and unambiguously communicate computational ideas using appropriate formalism. Translate across levels of abstraction.
    \begin{itemize}
        \item Use clear English to describe computations of Turing machines informally.
        \begin{itemize}
                \item {\bf Use high-level descriptions to define and trace Turing machines}
                \item {\bf Apply dovetailing in high-level definitions of machines}
        \end{itemize}
        \item Give examples of sets that are regular, context-free, decidable, or recognizable (and prove that they are).
        \begin{itemize}
          \item {\bf Give examples of sets that are decidable.}
          \item {\bf Give examples of sets that are recognizable.}
       \end{itemize}
    \end{itemize}
    \item Know, select and apply appropriate computing knowledge and problem-solving techniques. Reason about computation and systems.
    \begin{itemize}
        \item Translate a decision problem to a set of strings coding the problem.
        \begin{itemize}
        \item {\bf Connect languages and computational problems}
        \item {\bf Describe and use the encoding of objects as inputs to Turing machines}
        \item {\bf Trace high-level descriptions of algorithms for computational problems}
        \end{itemize}
    \item Classify the computational complexity of a set of strings by determining whether it is regular, context-free, decidable, or recognizable.
    \begin{itemize}
    \item {\bf Describe common computational problems with respect to DFA, NFA, regular expressions, PDA, and context-free grammars.}
    \item {\bf Give high-level descriptions of Turing machines that decide common computational problems with respect to DFA, NFA, regular expressions, PDA, and context-free grammars.}
\end{itemize}
\end{itemize}
\end{itemize}

\vspace{-20pt}

\subsubsection*{TODO:}
\begin{list}{\itemsep-10pt}
    \item Review Quiz 6 on PrairieLearn (http://us.prairielearn.com), due 2/19/2025
    \item Homework 4 submitted via Gradescope (https://www.gradescope.com/), due 2/20/2025
    \item Review Quiz 7 on PrairieLearn (http://us.prairielearn.com), due 2/26/2025
\end{list}

\newpage

\subsection*{Monday: No class, in observance of UCSD holiday}
\subsection*{Wednesday: General constructions for Turing machines}


\input{../activity-snippets/day18.tex}


\newpage
\subsection*{Friday: Decidable problems about regular languages}

\input{../activity-snippets/day19.tex}



\newpage

\subsection*{Week 8 at a glance}


\vspace{-20pt}

\subsubsection*{Textbook reading: Chapter 4, Section 5.3}

\vspace{-20pt}

Before Monday, ``An undecidable language", Sipser pages 207-209.

Before Wednesday, Definition 5.20 and figure 5.21 (page 236) of mapping reduction.

Before Friday, Example 5.24 (page 236).

For Week 9 Monday: Example 5.26 (page 237).

\vspace{-20pt}

\subsubsection*{We will be learning and practicing to:}

\vspace{-20pt}

\begin{itemize}
    \item Clearly and unambiguously communicate computational ideas using appropriate formalism. Translate across levels of abstraction.
    \begin{itemize}
        \item Give examples of sets that are regular, context-free, decidable, or recognizable (and prove that they are).
        \begin{itemize}
          \item {\bf Define and explain the definitions of the computational problem $A_{TM}$}
          \item {\bf Define and explain the definitions of the computational problem $HALT_{TM}$}
       \end{itemize}
    \end{itemize}
    \item Know, select and apply appropriate computing knowledge and problem-solving techniques. Reason about computation and systems.
    \begin{itemize}
        \item Use diagonalization to prove that there are 'hard' languages relative to certain models of computation.
        \begin{itemize}
            \item {\bf Trace the argument that proves $A_{TM}$ is undecidable and explain why it works.}
        \end{itemize}
        \item Use mapping reduction to deduce the complexity of a language by comparing to the complexity of another.
           \begin{itemize}
              \item {\bf Define computable functions, and use them to give mapping reductions between computational problems}
              \item {\bf Build and analyze mapping reductions between computational problems}
              \item {\bf Deduce the decidability or undecidability of a computational problem given mapping reductions between it and other computational problems, or explain when this is not possible.}
           \end{itemize}
    \item Classify the computational complexity of a set of strings by determining whether it is regular, context-free, decidable, or recognizable.
    \begin{itemize}
    \item {\bf State, prove, and use theorems relating decidability, recognizability, and co-recognizability.}
    \item {\bf Prove that a language is decidable or recognizable by defining and analyzing a Turing machines with appropriate properties.}
\end{itemize}
\end{itemize}
\end{itemize}

\vspace{-20pt}

\subsubsection*{TODO:}
\begin{list}{\itemsep-10pt}
   \item Review Quiz 7 on PrairieLearn (http://us.prairielearn.com), due 2/26/2025
   \item Homework 5 submitted via Gradescope (https://www.gradescope.com/), due 2/27/2025
   \item Review Quiz 8 on PrairieLearn (http://us.prairielearn.com), due 3/5/2025
\end{list}

\newpage

\section*{Monday: $A_{TM}$ is recognizable but undecidable}

\input{../activity-snippets/day20.tex}
    
\newpage
\subsection*{Wednesday: Computable functions and mapping reduction}

\input{../activity-snippets/day21.tex}


\newpage
\subsection*{Friday: The Halting problem}

\input{../activity-snippets/day22.tex}

\newpage

\subsection*{Week 9 at a glance}

\vspace{-10pt}

\subsubsection*{Textbook reading: Section 5.3, Section 5.1, Section 3.2}

\vspace{-10pt}

For Monday, Example 5.26 (page 237).

For Wednesday,  Theorem 5.30 (page 238) 

For Friday, skim section 3.2.

{\it For Monday of Week 10}: Definition 7.1 (page 276)
\vspace{-20pt}

\subsubsection*{We will be learning and practicing to:}

\vspace{-20pt}

\begin{itemize}
    \item Clearly and unambiguously communicate computational ideas using appropriate formalism. Translate across levels of abstraction.
    \begin{itemize}
        \item Give examples of sets that are regular, context-free, decidable, or recognizable (and prove that they are).
        \begin{itemize}
          \item {\bf Define and explain computational problems, including $A_{**}$, $E_{**}$, $EQ_{**}$, (for **  DFA or TM) and $HALT_{TM}$}
       \end{itemize}
    \end{itemize}
    \item Know, select and apply appropriate computing knowledge and problem-solving techniques. Reason about computation and systems.
    \begin{itemize}
        \item Use mapping reduction to deduce the complexity of a language by comparing to the complexity of another.
           \begin{itemize}
              \item {\bf Explain what it means for one problem to reduce to another}
              \item {\bf Define computable functions, and use them to give mapping reductions between computational problems}
              \item {\bf Build and analyze mapping reductions between computational problems}
           \end{itemize}
    \item Classify the computational complexity of a set of strings by determining whether it is regular, context-free, decidable, or recognizable.
    \begin{itemize}
         \item {\bf State, prove, and use theorems relating decidability, recognizability, and co-recognizability.}
         \item {\bf Prove that a language is decidable or recognizable by defining and analyzing a Turing machines with appropriate properties.}

   \end{itemize}
   \item  Describe several variants of Turing machines and informally explain why they are equally expressive.
   \begin{itemize}
   \item {\bf Define an enumerator}
   \item {\bf Define nondeterministic Turing machines}
   \item {\bf Use high-level descriptions to define and trace machines (Turing machines and enumerators)}
   \item {\bf Apply dovetailing in high-level definitions of machines}
   \end{itemize}
\end{itemize}
\end{itemize}

\vspace{-20pt}

\subsubsection*{TODO:}
\begin{list}{\itemsep-10pt}
   \item Review Quiz 8 on PrairieLearn (http://us.prairielearn.com), due 3/5/2025
   \item Review Quiz 9 on PrairieLearn (http://us.prairielearn.com), due 3/12/2025
   \item Homework 6 submitted via Gradescope (https://www.gradescope.com/), due 3/13/2025
   \item Project submitted via Gradescope (https://www.gradescope.com/), due 3/19/2025
\end{list}

\newpage

\section*{Monday: Mapping reductions and recognizability}

\input{../activity-snippets/day23.tex}
    
\newpage
\subsection*{Wednesday: More mapping reductions}

\input{../activity-snippets/day24.tex}


\vfill
\subsection*{Friday: Other models of computation}
\input{../activity-snippets/day25.tex}

\newpage
\end{document}