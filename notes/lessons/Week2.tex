\documentclass[12pt, oneside]{article}

\usepackage[letterpaper, scale=0.89, centering]{geometry}
\usepackage{fancyhdr}
\setlength{\parindent}{0em}
\setlength{\parskip}{1em}

\pagestyle{fancy}
\fancyhf{}
\rfoot{\href{https://creativecommons.org/licenses/by-nc-sa/2.0/}{CC BY-NC-SA 2.0} Version \today~(\thepage.)}

\input{../../resources/CSE20packages}

\begin{document}
\begin{flushright}
    \StrBefore{\currfilename}{.}
\end{flushright}


In Computer Science, we operationalize ``hardest'' as ``requires most resources'', where
resources might be memory, time, parallelism, randomness, power, etc.
To be able to compare ``hardness'' of problems, we use a consistent description of problems

{\bf Input}: String

{\bf Output}: Yes/ No, where Yes means that the input string matches the pattern or property described by the problem.

So far: we saw that regular expressions are convenient ways of describring patterns in strings.
{\bf Finite automata} give a model of computation for processing strings and and classifying them into Yes (accepted)
or No (rejected). We will see that each set of strings is described by a regular expression if and only 
if there is a FA that recognizes it.  Another way of thinking about it: properties described by regular
expressions require exactly the computational power of these finite automata.


\subsection*{Wednesday: Finite automaton constructions}

%! app: Regular Languages
%! outcome: Regular expressions, Formal definition of automata, Informal definition of automata

\begin{center}
\begin{tabular}{|ll|}
\hline
\multicolumn{2}{|l|}{{\bf Nondeterministic finite automaton}  (Sipser Page 53) Given as $M = (Q, \Sigma, \delta, q_0, F)$}\\
& \\
Finite set of states $Q$  & Can  be labelled by any collection  of distinct names. Default: $q0, q1, \ldots$  \\
Alphabet $\Sigma$ &  Each input to the automaton is a string over  $\Sigma$. \\
Arrow labels $\Sigma_\varepsilon$ &  $\Sigma_\varepsilon = \Sigma \cup \{ \varepsilon\}$. \\
&  Arrows 
in the state diagram are labelled either by symbols from $\Sigma$ or by $\varepsilon$ \\
Transition function $\delta$  & $\delta: Q \times \Sigma_{\varepsilon} \to \mathcal{P}(Q)$
gives the {\bf set of possible next states} for a transition \\
&  from the current state upon reading a symbol or spontaneously moving.\\
Start state $q_0$ & Element of $Q$.  Each computation of the machine starts at the  start  state.\\
Accept (final) states $F$ & $F \subseteq  Q$.\\
& \\
\multicolumn{2}{|p{\textwidth}|}{$M$ accepts the input string $w \in \Sigma^*$ if and only if {\bf there is} a computation of $M$ on 
$w$ that processes the whole string and ends in an
accept state.}\\
\hline
\end{tabular}
\end{center}

The formal definition of the NFA over $\{0,1\}$ given by this state diagram is: 

\includegraphics[width=2in]{../../resources/machines/Lect4NFA1.png}

The language over $\{0,1\}$ recognized by this NFA is:

\vspace{70pt}

Change the transition function to get a different NFA which accepts
the empty string (and potentially other strings too).


\newpage

The state diagram of an NFA over $\{a,b\}$ is below.  The formal definition of this NFA is:

\vspace{-30pt}

\includegraphics[width=2.5in]{../../resources/machines/Lect5NFA1.png}


Suppose $A_1, A_2$ are languages over an alphabet $\Sigma$.
{\bf Claim:} if there is a NFA $N_1$ such that $L(N_1) = A_1$ and 
NFA $N_2$ such that $L(N_2) = A_2$, then there is another NFA, let's call it $N$, such that 
$L(N) = A_1 \cup A_2$.

{\bf Proof idea}: Use nondeterminism to choose which of $N_1$, $N_2$ to run.

\vfill
\begin{comment}
    Draw schematic
\end{comment}

{\bf Formal construction}: Let 
$N_1 = (Q_1, \Sigma, \delta_1, q_1, F_1)$ and $N_2 = (Q_2, \Sigma, \delta_2,q_2, F_2)$
and assume $Q_1 \cap Q_2 = \emptyset$ and that $q_0 \notin Q_1 \cup Q_2$.
Construct $N = (Q, \Sigma, \delta, q_0, F_1 \cup F_2)$ where
\begin{itemize}
    \item $Q = $
    \item $\delta: Q \times \Sigma_\varepsilon \to \mathcal{P}(Q)$ is defined by, for $q \in Q$ and $x \in \Sigma_{\varepsilon}$:
        \[
            \phantom{\delta((q,x))=\begin{cases}  \delta_1 ((q,x)) &\qquad\text{if } q\in Q_1 \\ \delta_2 ((q,x)) &\qquad\text{if } q\in Q_2 \\ \{q1,q2\} &\qquad\text{if } q = q_0, x = \varepsilon \\ \emptyset\text{if } q= q_0, x \neq \varepsilon \end{cases}}
        \]
\end{itemize}


\vfill
{\it Proof of correctness would prove that $L(N) = A_1 \cup A_2$ by considering
an arbitrary string accepted by $N$, tracing an accepting computation of $N$ on it, and using 
that trace to prove the string is in at least one of $A_1$, $A_2$; then, taking an arbitrary 
string in $A_1 \cup A_2$ and proving that it is accepted by $N$. Details left for extra practice.}



\newpage
\subsection*{Friday: Nondeterministic automata}

%! app: Regular Languages
%! outcome: Formal definition of automata, Informal definition of automata, Nondeterminism


\begin{center}
\begin{tabular}{|ll|}
\hline
\multicolumn{2}{|l|}{{\bf Nondeterministic finite automaton} $M = (Q, \Sigma, \delta, q_0, F)$} \\
Finite set of states $Q$  & Can  be labelled by any collection  of distinct names. Default: $q0, q1, \ldots$  \\
Alphabet $\Sigma$ &  Each input to the automaton is a string over  $\Sigma$. \\
Arrow labels $\Sigma_\varepsilon$ &  $\Sigma_\varepsilon = \Sigma \cup \{ \varepsilon\}$. \\
&  Arrows 
in the state diagram are labelled either by symbols from $\Sigma$ or by $\varepsilon$ \\
Transition function $\delta$  & $\delta: Q \times \Sigma_{\varepsilon} \to \mathcal{P}(Q)$
gives the {\bf set of possible next states} for a transition \\
&  from the current state upon reading a symbol or spontaneously moving.\\
Start state $q_0$ & Element of $Q$.  Each computation of the machine starts at the  start  state.\\
Accept (final) states $F$ & $F \subseteq  Q$.\\
$M$ accepts the input string & if and only if {\bf there is} a computation of $M$ on the input string\\
&  that 
processes the whole string and ends in an
accept state.\\
\hline
{\it Page 53}& \\
\hline
\end{tabular}
\end{center}

The formal definition of the NFA over $\{0,1\}$ given by this state diagram is: 

\includegraphics[width=2in]{../../resources/machines/Lect4NFA1.png}

The language over $\{0,1\}$ recognized by this NFA is:

\vspace{70pt}

Change the transition function to get a different NFA which accepts
the empty string.


\newpage

The state diagram of an NFA over $\{a,b\}$ is below.  The formal definition of this NFA is:

\vspace{-30pt}

\includegraphics[width=2.5in]{../../resources/machines/Lect5NFA1.png}


\vspace{-10pt}

The language recognized by this NFA is: 



\newpage
\subsection*{Week 2 at a glance}

\subsubsection*{Textbook reading: Section 1.1, 1.2}

{\it For Wednesday}: Pages 41-43 (Figures 1.18, 1.19, 1.20) (examples of automata and languages).

{\it For Friday}: Pages 48-50 (Figures 1.27, 1.29) (introduction to nondeterminism).

{\it For Week 3 Monday}: Pages 60-61 Theorem 1.47 and Theorem 1.48 (closure proofs).


\subsubsection*{Make sure you can:}

\begin{itemize}
\item Use regular expressions and relate them to languages and automata
\begin{itemize}

   \item Write and debug regular expressions using correct syntax

\end{itemize}

\item Use precise notation to formally define the state diagram of DFA, NFA and 
use clear English to describe computations of DFA, NFA informally.

\begin{itemize}
   \item Design an automaton that recognizes a given language

   \item Specify a general construction for DFA based on parameters

   \item Design general constructions for DFA

   \item Motivate the use of nondeterminism

   \item Trace the computation(s) of a nondeterministic finite automaton

\end{itemize}






\end{itemize}

\subsubsection*{TODO:}
\begin{list}
   {\itemsep2pt}
   \item \#FinAid Assignment on Canvas https://canvas.ucsd.edu/courses/51649/quizzes/158899
   \item Review quizzes based on class material each day. 
   \item Homework assignment 1 due Thursday.
\end{list}

\end{document}