\input{../../resources/lesson-head.tex}

\subsection*{Week 2 at a glance}

\vspace{-15pt}

\subsubsection*{Textbook reading: Sections 1.1, 1.2}

\vspace{-15pt}

Before Monday, read pages 41-43 (Figures 1.18, 1.19, 1.20) for examples of automata and languages.

Before Wednesday, read pages 48-50 (Figures 1.27, 1.29) which introduces nondeterminism.

Before Friday, read pages 45-46 (Theorem 1.25) that we'll refer to as a ``closure proof".

For Week 3 Monday: Theorem 1.47 + 1.48, Theorem 1.39 ``Proof Idea'', Example 1.41, Example 1.56.

\vspace{-20pt}

\subsubsection*{We will be learning and practicing to:}
%Define decision problem, Formal definition of automata, Informal definition of automata, Find example languages, 
%Describe and use models of computation that don't involve state machines.
\begin{itemize}
\item Clearly and unambiguously communicate computational ideas using appropriate formalism. Translate across levels of abstraction.
\begin{itemize}
   \item Give examples of sets that are regular (and prove that they are).
   \begin{itemize}
      \item {\bf State the definition of the class of regular languages}
      \item {\bf Give examples of regular languages, using each of the three equivalent models of computation for proving regularity.}
   \end{itemize}
   \item Describe and use models of computation that don't involve state machines.
   \begin{itemize}
      \item {\bf Given a DFA or NFA, find a regular expression that describes its language.}
      \item {\bf Given a regular expression, find a DFA or NFA that recognizes its language.}
   \end{itemize}
   \item Use precise notation to formally define the state diagram of finite automata.
   \item Use clear English to describe computations of finite automata TM informally.
   \begin{itemize}
      \item {\bf Design an automaton that recognizes a given language}
      \item {\bf Specify a general construction for DFA based on parameters}
      \item {\bf Design general constructions for DFA}
      \item {\bf Motivate the use of nondeterminism}
      \item {\bf State the formal definition of NFA}   
      \item {\bf Trace the computation(s) of a NFA on a given string using its state diagram}
      \item {\bf Determine if a given string is in the language recognized by a NFA}
      \item {\bf Translate between a state diagram and a formal definition of a NFA}
   \end{itemize}
\end{itemize}

%\item Know, select and apply appropriate computing knowledge and problem-solving techniques. Reason about computation and systems. Use mathematical techniques to solve problems. Determine appropriate conceptual tools to apply to new situations. Know when tools do not apply and try different approaches. Critically analyze and evaluate candidate solutions.
%\begin{itemize}
%   \item 
%\end{itemize}

% \item Understand, guide, shape impact of computing on society/the world. Connect the role of Theory CS classes to other applications (in undergraduate CS curriculum and beyond). Model problems using appropriate mathematical concepts.
% \begin{itemize}
%     \item Explain nondeterminism and describe tools for simulating it with deterministic computation.
%     \begin{itemize}
%       \item {\bf Given a nondeterministic finite automaton, find a deterministic finite automaton that recognizes its language.}
%    \end{itemize}
% \end{itemize}

\end{itemize}

\vspace{-50pt}

\subsubsection*{TODO:}
\begin{list}{\itemsep-10pt}
   \item \#FinAid Assignment on Canvas (complete as soon as possible) and read syllabus on Canvas
   \item Schedule your Test 1 Attempt 1, Test 2 Attempt 1, Test 1 Attempt 2, and Test 2 Attempt 2 times 
   at PrairieTest (http://us.prairietest.com)
   \item Homework 1 submitted via Gradescope (https://www.gradescope.com/), due Tuesday 10/8/2024
   \item Review Quiz 2 on PrairieLearn (http://us.prairielearn.com), complete by Sunday 10/13/2024
\end{list}


\newpage
\subsection*{Monday: Finite automaton constructions}

%! app: Regular Languages
%! outcome: Regular expressions, Formal definition of automata, Informal definition of automata

{\bf Review}: Formal definition of DFA: $M = (Q, \Sigma, \delta, q_0, F)$ 

\begin{center}
\begin{multicols}{2}
\begin{itemize}
\setlength{\itemsep}{2pt}
\item Finite set of states $Q$
\item Alphabet $\Sigma$
\item Transition function $\delta$
\item Start state $q_0$
\item Accept (final) states $F$
\end{itemize}
\end{multicols}
\end{center}
In the state diagram of $M$, how many outgoing arrows are there from each state?

$M = ( \{ q, r, s\}, \{a,b\}, \delta, q, \{s\} )$ 
where $\delta$ is  (rows labelled by states
and columns labelled by symbols):
\begin{center}
\begin{tabular}{c|cc}
$\delta$ & $a$ & $b$ \\
\hline
$q$ & $r$ & $q$ \\
$r$ & $r$ & $s$ \\
$s$ & $s$ & $s$ \\
\end{tabular}
\end{center}

The state diagram for $M$ is 

\vfill



Give two examples of strings that are accepted by $M$ and two examples of strings that are rejected by $M$:

\vfill

Add ``labels" for states in the state diagram, e.g. ``have not seen any of desired pattern yet'' or
``sink state''.
\newpage

We can use the analysis of the roles of the states in the state diagram to describe the language
recognized by the DFA. 


$L(M) = $

A regular expression describing $L(M)$ is




\vspace{300pt}

Let the alphabet be $\Sigma_1 = \{0,1\}$.

A state diagram for a DFA that recognizes $\{w \mid w~\text{contains at most two $1$'s} \}$ is

\vspace{70pt}

A state diagram for a DFA that recognizes $\{w \mid w~\text{contains more than two $1$'s} \}$ is

\vspace{70pt}


\newpage
{\it Extra example:} A state diagram for DFA recognizing
$$\{w \mid w~\text{is a string over $\{0,1\}$ whose length is not a multiple of $3$} \}$$

\vspace{70pt}


Let $n$ be an arbitrary positive integer. What is a formal definition for a DFA recognizing
\[
\{w \mid w~\text{is a string over $\{0,1\}$ whose length is not a multiple of $n$} \}?
\]

\vfill

{\bf Note}: On Wednesday, we'll see a new kind of finite automaton. It will be helpful to distinguish it from the
machines we've been talking about so we'll use {\bf Deterministic Finite Automaton} (DFA) to refer to the machines 
from Section 1.1.

\newpage
\subsection*{Wednesday: Nondeterministic automata}

We saw that whenever a language is recognized by a DFA, its
complement is also recognized by some (other) DFA. 

Another way to say this is that the collection of languages
that are each recognizable by a DFA is {\bf closed} under complementation.


%! app: Regular Languages
%! outcome: Regular expressions, Formal definition of automata, Informal definition of automata

\begin{center}
\begin{tabular}{|ll|}
\hline
\multicolumn{2}{|l|}{{\bf Nondeterministic finite automaton}  (Sipser Page 53) Given as $M = (Q, \Sigma, \delta, q_0, F)$}\\
& \\
Finite set of states $Q$  & Can  be labelled by any collection  of distinct names. Default: $q0, q1, \ldots$  \\
Alphabet $\Sigma$ &  Each input to the automaton is a string over  $\Sigma$. \\
Arrow labels $\Sigma_\varepsilon$ &  $\Sigma_\varepsilon = \Sigma \cup \{ \varepsilon\}$. \\
&  Arrows 
in the state diagram are labelled either by symbols from $\Sigma$ or by $\varepsilon$ \\
Transition function $\delta$  & $\delta: Q \times \Sigma_{\varepsilon} \to \mathcal{P}(Q)$
gives the {\bf set of possible next states} for a transition \\
&  from the current state upon reading a symbol or spontaneously moving.\\
Start state $q_0$ & Element of $Q$.  Each computation of the machine starts at the  start  state.\\
Accept (final) states $F$ & $F \subseteq  Q$.\\
& \\
\multicolumn{2}{|p{\textwidth}|}{$M$ accepts the input string $w \in \Sigma^*$ if and only if {\bf there is} a computation of $M$ on 
$w$ that processes the whole string and ends in an
accept state.}\\
\hline
\end{tabular}
\end{center}

The formal definition of the NFA over $\{0,1\}$ given by this state diagram is: 

\includegraphics[width=2in]{../../resources/machines/Lect4NFA1.png}

The language over $\{0,1\}$ recognized by this NFA is:

\vspace{70pt}

Change the transition function to get a different NFA which accepts
the empty string (and potentially other strings too).


\newpage

The state diagram of an NFA over $\{a,b\}$ is below.  The formal definition of this NFA is:

\vspace{-30pt}

\includegraphics[width=2.5in]{../../resources/machines/Lect5NFA1.png}


Suppose $A_1, A_2$ are languages over an alphabet $\Sigma$.
{\bf Claim:} if there is a NFA $N_1$ such that $L(N_1) = A_1$ and 
NFA $N_2$ such that $L(N_2) = A_2$, then there is another NFA, let's call it $N$, such that 
$L(N) = A_1 \cup A_2$.

{\bf Proof idea}: Use nondeterminism to choose which of $N_1$, $N_2$ to run.

\vfill
\begin{comment}
    Draw schematic
\end{comment}

{\bf Formal construction}: Let 
$N_1 = (Q_1, \Sigma, \delta_1, q_1, F_1)$ and $N_2 = (Q_2, \Sigma, \delta_2,q_2, F_2)$
and assume $Q_1 \cap Q_2 = \emptyset$ and that $q_0 \notin Q_1 \cup Q_2$.
Construct $N = (Q, \Sigma, \delta, q_0, F_1 \cup F_2)$ where
\begin{itemize}
    \item $Q = $
    \item $\delta: Q \times \Sigma_\varepsilon \to \mathcal{P}(Q)$ is defined by, for $q \in Q$ and $x \in \Sigma_{\varepsilon}$:
        \[
            \phantom{\delta((q,x))=\begin{cases}  \delta_1 ((q,x)) &\qquad\text{if } q\in Q_1 \\ \delta_2 ((q,x)) &\qquad\text{if } q\in Q_2 \\ \{q1,q2\} &\qquad\text{if } q = q_0, x = \varepsilon \\ \emptyset\text{if } q= q_0, x \neq \varepsilon \end{cases}}
        \]
\end{itemize}


\vfill
{\it Proof of correctness would prove that $L(N) = A_1 \cup A_2$ by considering
an arbitrary string accepted by $N$, tracing an accepting computation of $N$ on it, and using 
that trace to prove the string is in at least one of $A_1$, $A_2$; then, taking an arbitrary 
string in $A_1 \cup A_2$ and proving that it is accepted by $N$. Details left for extra practice.}


\newpage
\subsection*{Friday: Automata constructions}

%! app: Regular Languages
%! outcome: Formal definition of automata, Informal definition of automata, Nondeterminism
{\bf Warmup}: Design a DFA (deterministic finite automaton) and an NFA (nondeterministic
finite automaton) that each recognize each of the following languages over $\{a,b\}$
\[
    \{ w \mid \text{$w$ has an $a$ and ends in $b$}\}
\]

\vfill

\[
    \{ w \mid \text{$w$ has an $a$ or ends in $b$}\}
\]

\vfill



\textbf{Strategy}: To design DFA or NFA for a given language,  
identify patterns that can be built up as we process strings and create states
for intermediate stages. Or: decompose the language to a simpler one 
that we already know how to recognize with a DFA or NFA.


{\it Recall} (from Wednesday of last week, and in textbook Exercise 1.14): 
if there is a DFA $M$ such that $L(M) = A$ then there is another DFA, let's call it $M'$, such that 
$L(M') = \overline{A}$, the complement of $A$, defined as $\{ w \in \Sigma^* \mid w \notin A \}$.


Let's practice defining automata constructions by coming up with other ways to get new automata from old.
\newpage

Suppose $A_1, A_2$ are languages over an alphabet $\Sigma$.
{\bf Claim:} if there is a NFA $N_1$ such that $L(N_1) = A_1$ and 
NFA $N_2$ such that $L(N_2) = A_2$, then there is another NFA, let's call it $N$, such that 
$L(N) = A_1 \cup A_2$.

{\bf Proof idea}: Use nondeterminism to choose which of $N_1$, $N_2$ to run.

\vfill
\begin{comment}
    Draw schematic
\end{comment}

{\bf Formal construction}: Let 
$N_1 = (Q_1, \Sigma, \delta_1, q_1, F_1)$ and $N_2 = (Q_2, \Sigma, \delta_2,q_2, F_2)$
and assume $Q_1 \cap Q_2 = \emptyset$ and that $q_0 \notin Q_1 \cup Q_2$.
Construct $N = (Q, \Sigma, \delta, q_0, F_1 \cup F_2)$ where
\begin{itemize}
    \item $Q = $
    \item $\delta: Q \times \Sigma_\varepsilon \to \mathcal{P}(Q)$ is defined by, for $q \in Q$ and $x \in \Sigma_{\varepsilon}$:
        \[
            \phantom{\delta((q,x))=\begin{cases}  \delta_1 ((q,x)) &\qquad\text{if } q\in Q_1 \\ \delta_2 ((q,x)) &\qquad\text{if } q\in Q_2 \\ \{q1,q2\} &\qquad\text{if } q = q_0, x = \varepsilon \\ \emptyset\text{if } q= q_0, x \neq \varepsilon \end{cases}}
        \]
\end{itemize}


\vfill
{\it Proof of correctness would prove that $L(N) = A_1 \cup A_2$ by considering
an arbitrary string accepted by $N$, tracing an accepting computation of $N$ on it, and using 
that trace to prove the string is in at least one of $A_1$, $A_2$; then, taking an arbitrary 
string in $A_1 \cup A_2$ and proving that it is accepted by $N$. Details left for extra practice.}


{\bf Example}: The language recognized by the NFA over $\{a,b\}$ with state diagram


    \begin{tikzpicture}[->,>=stealth',shorten >=1pt, auto, node distance=2cm, semithick]
    \tikzstyle{every state}=[text=black, fill=yellow!40]
    
    \node[initial,state] (q0)          {$q_0$};
    \node[state]         (q) [above right of=q0, xshift=20pt] {$q$};
    \node[state]         (r) [right of=q, xshift=20pt] {$r$};
    \node[state, accepting]         (s) [right of=r, xshift=20pt] {$s$};
    \node[state, accepting]         (n) [below right of=q0, xshift=20pt] {$n$};
    \node[state]         (d) [right of=n, xshift=20pt] {$d$};
    
    \path (q0) edge  [bend left=0, near start] node {$\varepsilon$} (q)
            edge [bend right=0, near start] node {$\varepsilon$} (n)
        (q) edge [bend left=0] node {$a$} (r)
            edge [loop above, near start] node {$b$} (q)
        (r) edge [bend left=0] node {$b$} (s)
            edge [loop above, near start] node {$a,b$} (r)
        (n) edge [bend left=20] node {$a,b$} (d)
        (d) edge [bend left=20] node {$a,b$} (n)
    ;
    \end{tikzpicture}
is:


\newpage

Could we do the same construction with DFA?

\vspace{50pt}

Happily, though, an analogous claim is true!

Suppose $A_1, A_2$ are languages over an alphabet $\Sigma$.
{\bf Claim:} if there is a DFA $M_1$ such that $L(M_1) = A_1$ and 
DFA $M_2$ such that $L(M_2) = A_2$, then there is another DFA, let's call it $M$, such that 
$L(M) = A_1 \cup A_2$. {\it Theorem 1.25 in Sipser, page 45}
    
    {\bf Proof idea}:
    
    
    {\bf Formal construction}: 
    
    \vfill

    
    {\bf Example}:  When $A_1 = \{w \mid w~\text{has an $a$ and ends in $b$} \}$ and 
    $A_2 = \{ w \mid w~\text{is of even length} \}$.
    
    \begin{tikzpicture}[->,>=stealth',shorten >=1pt, auto, node distance=2cm, semithick]
        \tikzstyle{every state}=[text=black, fill=yellow!40]
        
        \node[initial,state,accepting] (qn)          {$(q,n)$};
        \node[state]         (qd) [below of=qn, yshift=-40pt] {$(q,d)$};
        \node[state]         (rd) [right of=qn, xshift=20pt] {$(r,d)$};
        \node[state,accepting]         (rn) [right of=qd, xshift=20pt] {$(r,n)$};
        \node[state,accepting]         (sn) [right of=rd, xshift=20pt] {$(s,n)$};
        \node[state,accepting]         (sd) [right of=rn, xshift=20pt] {$(s,d)$};
        
        \path (qn) edge  [bend left=20, near start] node {$b$} (qd)
                edge [bend left=20, near start] node {$a$} (rd)
            (qd) edge [bend left=20, near start] node {$b$} (qn)
                edge [bend right=20, near start] node {$a$} (rn)
            (rn) edge [bend left=20, near start] node {$a$} (rd)
                edge [bend left=20, near start] node {$b$} (sd)
            (rd) edge [bend left=20, near start] node {$a$} (rn)
                edge [bend left=20, near start] node {$b$} (sn)
            (sn) edge [bend left=20, near start] node {$a$} (rd)
                edge [bend left=20, near start] node {$b$} (sd)
            (sd) edge [bend left=20, near start] node {$a$} (rn)
                edge [bend left=20, near start] node {$b$} (sn)
        ;
        \end{tikzpicture}
    
    \newpage
    
    Suppose $A_1, A_2$ are languages over an alphabet $\Sigma$.
    {\bf Claim:} if there is a DFA $M_1$ such that $L(M_1) = A_1$ and 
    DFA $M_2$ such that $L(M_2) = A_2$, then there is another DFA, let's call it $M$, such that 
    $L(M) = A_1 \cap A_2$.  {\it Sipser Theorem 1.25, page 45}
    
    {\bf Proof idea}:
    
    
    {\bf Formal construction}: 
    
    \vspace{70pt}


    



\end{document}