\input{../../resources/lesson-head.tex}

\subsection*{Week 9 at a glance}

\vspace{-20pt}

\subsubsection*{Textbook reading: Section 5.3, Section 5.1, Section 3.2}

\vspace{-20pt}

For Monday, Example 5.26 (page 237).

For Wednesday,  Theorem 5.30 (page 238) and skim section 3.2.

Friday: no class in observance of Thanksgiving holiday.

{\it For Monday of Week 10}: Definition 7.1 (page 276)
\vspace{-20pt}

\subsubsection*{We will be learning and practicing to:}

\vspace{-20pt}

\begin{itemize}
    \item Clearly and unambiguously communicate computational ideas using appropriate formalism. Translate across levels of abstraction.
    \begin{itemize}
        \item Give examples of sets that are regular, context-free, decidable, or recognizable (and prove that they are).
        \begin{itemize}
          \item {\bf Define and explain computational problems, including $A$**, $E$**, $EQ$**, (for **  DFA or TM) and $HALT_{TM}$}
       \end{itemize}
    \end{itemize}
    \item Know, select and apply appropriate computing knowledge and problem-solving techniques. Reason about computation and systems.
    \begin{itemize}
        \item Use mapping reduction to deduce the complexity of a language by comparing to the complexity of another.
           \begin{itemize}
              \item {\bf Explain what it means for one problem to reduce to another}
              \item {\bf Define computable functions, and use them to give mapping reductions between computational problems}
              \item {\bf Build and analyze mapping reductions between computational problems}
           \end{itemize}
    \item Classify the computational complexity of a set of strings by determining whether it is regular, context-free, decidable, or recognizable.
    \begin{itemize}
         \item {\bf State, prove, and use theorems relating decidability, recognizability, and co-recognizability.}
         \item {\bf Prove that a language is decidable or recognizable by defining and analyzing a Turing machines with appropriate properties.}

   \end{itemize}
   \item  Describe several variants of Turing machines and informally explain why they are equally expressive.
   \begin{itemize}
   \item {\bf Define an enumerator}
   \item {\bf Define nondeterministic Turing machines}
   \item {\bf Use high-level descriptions to define and trace machines (Turing machines and enumerators)}
   \item {\bf Apply dovetailing in high-level definitions of machines}
   \end{itemize}
\end{itemize}
\end{itemize}

\vspace{-20pt}

\subsubsection*{TODO:}
\begin{list}{\itemsep-10pt}
   \item Review Quiz 9 on PrairieLearn (http://us.prairielearn.com), complete by Sunday 12/1/2024
\end{list}

\newpage

\section*{Monday: Mapping reductions and recognizability}

%! app: Decidable Languages, Undecidable Languages
%! outcome: Classify language, Classify decision problem, Reduction
 

Recall definition:  $A$ is  {\bf  mapping  reducible to} $B$  means there is a computable function 
$f : \Sigma^* \to \Sigma^*$ such that {\it for all} strings  $x$ in $\Sigma^*$, 
\[
x  \in  A \qquad \qquad \text{if and  only  if} \qquad \qquad f(x) \in B.
\]
Notation:  when $A$  is mapping reducible to $B$, we write $A  \leq_m B$.

{\bf Theorem} (Sipser 5.23): If $A \leq_m B$ and $A$ is undecidable, then $B$ is undecidable.
    

{\it Last time} we proved that $A_{TM} \le_m HALT_{TM}$ where
    \[
    HALT_{TM} = \{ \langle M, w \rangle \mid \text{$M$ is a  Turing machine, $w$ is  a string, and $M$ halts on $w$} \}
    \]
and since $A_{TM}$ is undecidable, $HALT_{TM}$ is also undecidable. The function 
witnessing the mapping reduction mapped strings in $A_{TM}$ to strings in $HALT_{TM}$ and 
strings not in $A_{TM}$ to strings not in $HALT_{TM}$ by changing encoded Turing machines to 
ones that had identical computations except looped instead of rejecting.

\begin{comment}
Define $F: \Sigma^* \to \Sigma^*$ by
    \[
    F(x) =  \begin{cases}
    const_{out} \qquad &\text{if  $x \neq \langle M,w \rangle$ for any Turing machine  $M$ and string  $w$ over the alphabet of $M$} \\
    \langle M', w \rangle \qquad &  \text{if $x = \langle M, w \rangle$ for some Turing machine  $M$ and string $w$ over the alphabet of $M$.}
    \end{cases}
    \]
    where $const_{out}  =  \langle  \includegraphics[width=1.5in]{../../resources/machines/Lect22TM1.png} ,  \varepsilon  \rangle$
    and  $M'$ is a Turing machine that computes like $M$ except, if the computation ever were to go to a  reject state,
    $M'$ loops instead.
    
    \vfill

    $F( \langle \includegraphics[width=1.5in]{../../resources/machines/Lect22TM1.png} ,  001  \rangle)$ =

    \vfill

    $F( \langle \includegraphics[width=2.5in]{../../resources/machines/Lect22TM2.png} ,  1  \rangle)$ =

    \vfill
    
    \newpage
    To use this function  to prove that $A_{TM} \leq_m HALT_{TM}$, we need  two claims:

    
    Claim (1): $F$ is computable \phantom{\hspace{2in}}
    
    \vfill

    Claim (2): for every  $x$,  $x \in  A_{TM}$ iff $F(x) \in HALT_{TM}$.  
    
    \vfill
\end{comment}

True or False: $\overline{A_{TM}} \leq_m \overline{HALT_{TM}}$

\vfill

True or False: $HALT_{TM} \leq_m A_{TM}$.

{\bf Proof}: Need computable function  $F: \Sigma^* \to \Sigma^*$  such that  
$x \in HALT_{TM}$ iff $F(x)  \in  A_{TM}$.
Define

\vspace{-15pt}

\begin{quote}
$F =  ``$ On input $x$,
\begin{itemize}
\item[1.] Type-check whether  $x = \langle M, w \rangle$ for some TM $M$ and string $w$. 
If so, move to step 2; if  not, output  $\langle \hspace{2in} \rangle$
\item[2.] Construct the following machine $M'_x$:
\vspace{50pt}
\item[3.] Output $\langle M'_x , w\rangle$."
\end{itemize}
\end{quote}

Verifying correctness: (1) Is function well-defined and computable? (2) Does it have the 
translation property $x \in HALT_{TM}$ iff its image is in $A_{TM}$? 
\begin{center}
\begin{tabular}{|c|c|}
\hline
Input string &  Output string \\
\hline
$\langle M, w \rangle$ where  $M$ halts on $w$ & \phantom{\hspace{4in}} \\
& \\& \\
$\langle M, w \rangle$ where $M$ does not halt on $w$ & \\
& \\&\\
$x$ not encoding any pair of  TM and string   &  \\
& \\
\hline
\end{tabular}
\end{center}

\vfill

\newpage


{\bf Theorem} (Sipser 5.28): If $A \leq_m B$ and $B$ is recognizable, then $A$ is recognizable.

{\bf Proof}: 

\vfill

{\bf Corollary}: If  $A \leq_m B$ and $A$ is unrecognizable, then $B$ is unrecognizable.

\vfill

{\it Strategy}:  

(i) To prove that a recognizable language $R$ is undecidable, prove that $A_{TM} \leq_m R$.


(ii) To prove that a co-recognizable language $U$ is undecidable, prove that $\overline{A_{TM}} \leq_m U$,
 i.e. that $A_{TM} \leq_m \overline{U}$.

 \newpage

\[
E_{TM} = \{ \langle M \rangle \mid \text{$M$ is a Turing machine and $L(M) = \emptyset$} \}
\]

\begin{comment}
Example  string in  $E_{TM}$ is \underline{\phantom{\hspace{1.6in}}} .
Example  string not  in  $E_{TM}$ is \underline{\phantom{\hspace{1.6in}}} .
\end{comment}

Can we find algorithms to recognize

$E_{TM}$  ? 

$\overline{E_{TM}}$ ? 

\vfill


{\bf Claim}: $A_{TM}  \leq_m \overline{E_{TM}}$. {\it And hence also } $\overline{A_{TM}} \leq_m E_{TM}$

{\bf Proof}: Need computable function  $F: \Sigma^* \to \Sigma^*$  such that  $x \in A_{TM}$ iff $F(x)  \notin  E_{TM}$.
Define

\vspace{-15pt}

\begin{quote}
$F =  ``$ On input $x$,
\begin{itemize}
\item[1.] Type-check whether  $x = \langle M, w \rangle$ for some TM $M$ and string $w$. 
If so, move to step 2; if  not, output  $\langle \hspace{2in} \rangle$
\item[2.] Construct the following machine $M'_x$:
\vspace{50pt}
\item[3.] Output $\langle M'_x \rangle$."
\end{itemize}
\end{quote}

Verifying correctness: (1) Is function well-defined and computable? (2) Does it have the 
translation property $x \in A_{TM}$ iff its image is {\bf not} in $E_{TM}$ ? 
\begin{center}
\begin{tabular}{|c|c|}
\hline
Input string &  Output string \\
\hline
$\langle M, w \rangle$ where  $w \in L(M)$ & \phantom{\hspace{4in}} \\
& \\
& \\
& \\
$\langle M, w \rangle$ where $w \notin L(M)$ & \\
& \\
&\\ & \\
$x$ not encoding any pair of  TM and string   &  \\
& \\
& \\
\hline
\end{tabular}
\end{center}

\vfill
    
\newpage
\subsection*{Wednesday: More mapping reductions and other models of computation}

%! app: Decidable Languages, Undecidable Languages
%! outcome: Classify language, Classify decision problem, Reduction
 



Recall:  $A$ is  {\bf  mapping  reducible to} $B$, written $A \leq_m B$,  means there is a computable function 
$f : \Sigma^* \to \Sigma^*$ such that {\it for all} strings  $x$ in $\Sigma^*$, 
\[
x  \in  A \qquad \qquad \text{if and  only  if} \qquad \qquad f(x) \in B.
\]

So far: 
\begin{itemize}
\item $A_{TM}$ is recognizable, undecidable, and not-co-recognizable.
\item $\overline{A_{TM}}$ is unrecognizable, undecidable, and co-recognizable.
\item $HALT_{TM}$ is recognizable, undecidable, and not-co-recognizable.
\item $\overline{HALT_{TM}}$ is unrecognizable, undecidable, and co-recognizable.
\item $E_{TM}$ is unrecognizable, undecidable, and co-recognizable.
\item $\overline{E_{TM}}$ is recognizable, undecidable, and not-co-recognizable.
\end{itemize}


\[
EQ_{TM} = \{ \langle M, M' \rangle \mid \text{$M$ and $M'$ are both Turing machines and $L(M) =L(M')$} \}
\]


Can we find algorithms to recognize

$EQ_{TM}$  ? 

$\overline{EQ_{TM}}$ ? 

\vfill

{\it Goal}: Show that $EQ_{TM}$ is not recognizable and that $\overline{EQ_{TM}}$ is not recognizable.

Using Corollary to {\bf Theorem 5.28}: If  $A \leq_m B$ and $A$ is unrecognizable, then $B$ is unrecognizable,
it's enough to prove that 
\begin{itemize}
    \item[] $\overline{HALT_{TM}} \leq_m EQ_{TM}$ \hfill aka $HALT_{TM} \leq_m \overline{EQ_{TM}}$
    \item[] $\overline{HALT_{TM}}  \leq_m \overline{EQ_{TM}}$ \hfill aka $HALT_{TM} \leq_m EQ_{TM}$
\end{itemize}

\vfill

\newpage 
Need computable function  $F_1: \Sigma^* \to \Sigma^*$  such that  $x \in HALT_{TM}$ iff 
$F_1(x)  \notin  EQ_{TM}$.



{\it Strategy}:

\vspace{-15pt}

Map strings $\langle M, w \rangle$ to strings $\langle M'_{x},
\scalebox{0.5}{\begin{tikzpicture}[->,>=stealth',shorten >=1pt, auto, node distance=2cm, semithick]
      \tikzstyle{every state}=[text=black, fill=yellow!40]
      \node[initial,state] (q0)                    {$q_0$};
      \node[state,accepting] (qacc) [right of = q0, xshift = 20]{$q_{acc}$};
      \path (q0) edge  [loop above] node {$0, 1, \scalebox{1.5}{\textvisiblespace} \to R$} (q0)
     ;
    \end{tikzpicture}}
    \rangle$ 
    . This image string is not in $EQ_{TM}$ when $L(M'_x) \neq \emptyset$.
    
We will build $M'_x$ so that 
    $L(M'_{x}) = \Sigma^*$ when $M$ halts on $w$ and $L(M'_x) = \emptyset$ when $M$ loops on $w$.


Thus: when $\langle M,w \rangle \in HALT_{TM}$ it gets mapped to a string not in $EQ_{TM}$ and 
when $\langle M,w \rangle \notin HALT_{TM}$ it gets mapped to a string that is in $EQ_{TM}$.

\vfill

Define

\vspace{-15pt}

\begin{quote}
$F_1 =  ``$ On input $x$,
\begin{itemize}
\item[1.] Type-check whether  $x = \langle M, w \rangle$ for some TM $M$ and string $w$. 
If so, move to step 2; if  not, output  $\langle \hspace{2in} \rangle$
\item[2.] Construct the following machine $M'_x$:
\vspace{50pt}
\item[3.] Output $\langle M'_{x},
\scalebox{0.5}{\begin{tikzpicture}[->,>=stealth',shorten >=1pt, auto, node distance=2cm, semithick]
      \tikzstyle{every state}=[text=black, fill=yellow!40]
      \node[initial,state] (q0)                    {$q_0$};
      \node[state,accepting] (qacc) [right of = q0, xshift = 20]{$q_{acc}$};
      \path (q0) edge  [loop above] node {$0, 1, \scalebox{1.5}{\textvisiblespace} \to R$} (q0)
     ;
    \end{tikzpicture}}
    \rangle$ "
\end{itemize}
\end{quote}

\vfill

Verifying correctness: (1) Is function well-defined and computable? (2) Does it have the 
translation property $x \in HALT_{TM}$ iff its image is {\bf not} in $EQ_{TM}$ ? 
\begin{center}
\begin{tabular}{|c|c|}
\hline
Input string &  Output string \\
\hline
$\langle M, w \rangle$ where  $M$ halts on $w$ & \phantom{\hspace{4in}} \\
& \\
& \\
& \\
$\langle M, w \rangle$ where $M$ loops on $w$ & \\
& \\
&\\ & \\
$x$ not encoding any pair of  TM and string   &  \\
& \\
& \\
\hline
\end{tabular}
\end{center}


\vfill

Conclude: $HALT_{TM} \leq_m \overline{EQ_{TM}}$
\newpage

\newpage 
Need computable function  $F_2: \Sigma^* \to \Sigma^*$  such that  $x \in HALT_{TM}$ iff 
$F_2(x)  \in  EQ_{TM}$.



{\it Strategy}:

\vspace{-15pt}

Map strings $\langle M, w \rangle$ to strings $\langle M'_{x},
\scalebox{0.5}{\begin{tikzpicture}[->,>=stealth',shorten >=1pt, auto, node distance=2cm, semithick]
      \tikzstyle{every state}=[text=black, fill=yellow!40]
      \node[initial,state,accepting] (q0)                    {$q_0$};
     ;
    \end{tikzpicture}}
    \rangle$ 
    . This image string is in $EQ_{TM}$ when $L(M'_x) = \Sigma^*$.
    
We will build $M'_x$ so that 
    $L(M'_{x}) = \Sigma^*$ when $M$ halts on $w$ and $L(M'_x) = \emptyset$ when $M$ loops on $w$.


Thus: when $\langle M,w \rangle \in HALT_{TM}$ it gets mapped to a string  in $EQ_{TM}$ and 
when $\langle M,w \rangle \notin HALT_{TM}$ it gets mapped to a string that is not in $EQ_{TM}$.

\vfill

Define

\vspace{-15pt}

\begin{quote}
$F_2 =  ``$ On input $x$,
\begin{itemize}
\item[1.] Type-check whether  $x = \langle M, w \rangle$ for some TM $M$ and string $w$. 
If so, move to step 2; if  not, output  $\langle \hspace{2in} \rangle$
\item[2.] Construct the following machine $M'_x$:
\vspace{50pt}
\item[3.] Output $\langle M'_{x},
\scalebox{0.5}{\begin{tikzpicture}[->,>=stealth',shorten >=1pt, auto, node distance=2cm, semithick]
      \tikzstyle{every state}=[text=black, fill=yellow!40]
      \node[initial,state,accepting] (q0)                    {$q_0$};
     ;
    \end{tikzpicture}}
    \rangle$ "
\end{itemize}
\end{quote}

\vfill

Verifying correctness: (1) Is function well-defined and computable? (2) Does it have the 
translation property $x \in HALT_{TM}$ iff its image is in $EQ_{TM}$ ? 
\begin{center}
\begin{tabular}{|c|c|}
\hline
Input string &  Output string \\
\hline
$\langle M, w \rangle$ where  $M$ halts on $w$ & \phantom{\hspace{4in}} \\
& \\
& \\
& \\
$\langle M, w \rangle$ where $M$ loops on $w$ & \\
& \\
&\\ & \\
$x$ not encoding any pair of  TM and string   &  \\
& \\
& \\
\hline
\end{tabular}
\end{center}


\vfill

Conclude: $HALT_{TM} \leq_m EQ_{TM}$



\vfill
\subsection*{Friday: No class in observance of Thanksgiving Holiday}
\end{document}
