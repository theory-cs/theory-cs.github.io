\documentclass[12pt, oneside]{article}

\usepackage[letterpaper, scale=0.89, centering]{geometry}
\usepackage{fancyhdr}
\setlength{\parindent}{0em}
\setlength{\parskip}{1em}

\pagestyle{fancy}
\fancyhf{}
\rfoot{\href{https://creativecommons.org/licenses/by-nc-sa/2.0/}{CC BY-NC-SA 2.0} Version \today~(\thepage.)}

\input{../../resources/CSE20packages}

\begin{document}
\begin{flushright}
    \StrBefore{\currfilename}{.}
\end{flushright}

\subsection*{Week 9 at a glance}

\vspace{-10pt}

\subsubsection*{Textbook reading: Section 5.3, Section 5.1, Section 3.2}

\vspace{-10pt}

For Monday, Example 5.26 (page 237).

For Wednesday,  Theorem 5.30 (page 238) 

For Friday, skim section 3.2.

{\it For Monday of Week 10}: Definition 7.1 (page 276)
\vspace{-20pt}

\subsubsection*{We will be learning and practicing to:}

\vspace{-20pt}

\begin{itemize}
    \item Clearly and unambiguously communicate computational ideas using appropriate formalism. Translate across levels of abstraction.
    \begin{itemize}
        \item Give examples of sets that are regular, context-free, decidable, or recognizable (and prove that they are).
        \begin{itemize}
          \item {\bf Define and explain computational problems, including $A_{**}$, $E_{**}$, $EQ_{**}$, (for **  DFA or TM) and $HALT_{TM}$}
       \end{itemize}
    \end{itemize}
    \item Know, select and apply appropriate computing knowledge and problem-solving techniques. Reason about computation and systems.
    \begin{itemize}
        \item Use mapping reduction to deduce the complexity of a language by comparing to the complexity of another.
           \begin{itemize}
              \item {\bf Explain what it means for one problem to reduce to another}
              \item {\bf Define computable functions, and use them to give mapping reductions between computational problems}
              \item {\bf Build and analyze mapping reductions between computational problems}
           \end{itemize}
    \item Classify the computational complexity of a set of strings by determining whether it is regular, context-free, decidable, or recognizable.
    \begin{itemize}
         \item {\bf State, prove, and use theorems relating decidability, recognizability, and co-recognizability.}
         \item {\bf Prove that a language is decidable or recognizable by defining and analyzing a Turing machines with appropriate properties.}

   \end{itemize}
   \item  Describe several variants of Turing machines and informally explain why they are equally expressive.
   \begin{itemize}
   \item {\bf Define an enumerator}
   \item {\bf Define nondeterministic Turing machines}
   \item {\bf Use high-level descriptions to define and trace machines (Turing machines and enumerators)}
   \item {\bf Apply dovetailing in high-level definitions of machines}
   \end{itemize}
\end{itemize}
\end{itemize}

\vspace{-20pt}

\subsubsection*{TODO:}
\begin{list}{\itemsep-10pt}
   \item Review Quiz 8 on PrairieLearn (http://us.prairielearn.com), due 3/5/2025
   \item Review Quiz 9 on PrairieLearn (http://us.prairielearn.com), due 3/12/2025
   \item Homework 6 submitted via Gradescope (https://www.gradescope.com/), due 3/13/2025
   \item Project submitted via Gradescope (https://www.gradescope.com/), due 3/19/2025
\end{list}

\newpage

\section*{Monday: Mapping reductions and recognizability}

\input{../activity-snippets/day23.tex}
    
\newpage
\subsection*{Wednesday: More mapping reductions}

\input{../activity-snippets/day24.tex}


\vfill
\subsection*{Friday: Other models of computation}
\input{../activity-snippets/day25.tex}
\end{document}
