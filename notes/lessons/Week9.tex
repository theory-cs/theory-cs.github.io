
\documentclass[12pt, oneside]{article}

\usepackage[letterpaper, scale=0.89, centering]{geometry}
\usepackage{fancyhdr}
\setlength{\parindent}{0em}
\setlength{\parskip}{1em}

\pagestyle{fancy}
\fancyhf{}
\rfoot{\href{https://creativecommons.org/licenses/by-nc-sa/2.0/}{CC BY-NC-SA 2.0} Version \today~(\thepage.)}

\input{../../resources/CSE20packages}

\begin{document}
\begin{flushright}
    \StrBefore{\currfilename}{.}
\end{flushright}

\section*{Monday: Mapping reductions and recognizability}

%! app: Decidable Languages, Undecidable Languages
%! outcome: Classify language, Classify decision problem, Reduction
 

Recall definition:  $A$ is  {\bf  mapping  reducible to} $B$  means there is a computable function 
$f : \Sigma^* \to \Sigma^*$ such that {\it for all} strings  $x$ in $\Sigma^*$, 
\[
x  \in  A \qquad \qquad \text{if and  only  if} \qquad \qquad f(x) \in B.
\]
Notation:  when $A$  is mapping reducible to $B$, we write $A  \leq_m B$.

{\bf Theorem} (Sipser 5.23): If $A \leq_m B$ and $A$ is undecidable, then $B$ is undecidable.
    
{\bf Halting problem}
    \[
    HALT_{TM} = \{ \langle M, w \rangle \mid \text{$M$ is a  Turing machine, $w$ is  a string, and $M$ halts on $w$} \}
    \]

We will define a computable function that witnesses the mapping reduction $A_{TM} \leq_m HALT_{TM}$.

Using Theorem 5.23, we can then conclude that $HALT_{TM}$ is undecidable.

Define $F: \Sigma^* \to \Sigma^*$ by
    \[
    F(x) =  \begin{cases}
    const_{out} \qquad &\text{if  $x \neq \langle M,w \rangle$ for any Turing machine  $M$ and string  $w$ over the alphabet of $M$} \\
    \langle M', w \rangle \qquad &  \text{if $x = \langle M, w \rangle$ for some Turing machine  $M$ and string $w$ over the alphabet of $M$.}
    \end{cases}
    \]
    where $const_{out}  =  \langle  \includegraphics[width=1.5in]{../../resources/machines/Lect22TM1.png} ,  \varepsilon  \rangle$
    and  $M'$ is a Turing machine that computes like $M$ except, if the computation ever were to go to a  reject state,
    $M'$ loops instead.
    
    \vfill

    $F( \langle \includegraphics[width=1.5in]{../../resources/machines/Lect22TM1.png} ,  001  \rangle)$ =

    \vfill

    $F( \langle \includegraphics[width=2.5in]{../../resources/machines/Lect22TM2.png} ,  1  \rangle)$ =

    \vfill
    
    \newpage
    To use this function  to prove that $A_{TM} \leq_m HALT_{TM}$, we need  two claims:

    
    Claim (1): $F$ is computable \phantom{\hspace{2in}}
    
    \vfill

    Claim (2): for every  $x$,  $x \in  A_{TM}$ iff $F(x) \in HALT_{TM}$.  
    
    \vfill
   
    
\newpage
True or False: $\overline{A_{TM}} \leq_m \overline{HALT_{TM}}$

\vfill

True or False: $HALT_{TM} \leq_m A_{TM}$.

\vfill

\newpage
    
\newpage
\subsection*{Wednesday: More mapping reductions}

%! app: Decidable Languages, Undecidable Languages
%! outcome: Classify language, Classify decision problem, Reduction
 



Recall:  $A$ is  {\bf  mapping  reducible to} $B$, written $A \leq_m B$,  means there is a computable function 
$f : \Sigma^* \to \Sigma^*$ such that {\it for all} strings  $x$ in $\Sigma^*$, 
\[
x  \in  A \qquad \qquad \text{if and  only  if} \qquad \qquad f(x) \in B.
\]

So far: 
\begin{itemize}
\item $A_{TM}$ is recognizable, undecidable, and not-co-recognizable.
\item $\overline{A_{TM}}$ is unrecognizable, undecidable, and co-recognizable.
\item $HALT_{TM}$ is recognizable, undecidable, and not-co-recognizable.
\item $\overline{HALT_{TM}}$ is unrecognizable, undecidable, and co-recognizable.
\item $E_{TM}$ is unrecognizable, undecidable, and co-recognizable.
\item $\overline{E_{TM}}$ is recognizable, undecidable, and not-co-recognizable.
\end{itemize}


\[
EQ_{TM} = \{ \langle M, M' \rangle \mid \text{$M$ and $M'$ are both Turing machines and $L(M) =L(M')$} \}
\]


Can we find algorithms to recognize

$EQ_{TM}$  ? 

$\overline{EQ_{TM}}$ ? 

\vfill

{\it Goal}: Show that $EQ_{TM}$ is not recognizable and that $\overline{EQ_{TM}}$ is not recognizable.

Using Corollary to {\bf Theorem 5.28}: If  $A \leq_m B$ and $A$ is unrecognizable, then $B$ is unrecognizable,
it's enough to prove that 
\begin{itemize}
    \item[] $\overline{HALT_{TM}} \leq_m EQ_{TM}$ \hfill aka $HALT_{TM} \leq_m \overline{EQ_{TM}}$
    \item[] $\overline{HALT_{TM}}  \leq_m \overline{EQ_{TM}}$ \hfill aka $HALT_{TM} \leq_m EQ_{TM}$
\end{itemize}

\vfill

\newpage 
Need computable function  $F_1: \Sigma^* \to \Sigma^*$  such that  $x \in HALT_{TM}$ iff 
$F_1(x)  \notin  EQ_{TM}$.



{\it Strategy}:

\vspace{-15pt}

Map strings $\langle M, w \rangle$ to strings $\langle M'_{x},
\scalebox{0.5}{\begin{tikzpicture}[->,>=stealth',shorten >=1pt, auto, node distance=2cm, semithick]
      \tikzstyle{every state}=[text=black, fill=yellow!40]
      \node[initial,state] (q0)                    {$q_0$};
      \node[state,accepting] (qacc) [right of = q0, xshift = 20]{$q_{acc}$};
      \path (q0) edge  [loop above] node {$0, 1, \scalebox{1.5}{\textvisiblespace} \to R$} (q0)
     ;
    \end{tikzpicture}}
    \rangle$ 
    . This image string is not in $EQ_{TM}$ when $L(M'_x) \neq \emptyset$.
    
We will build $M'_x$ so that 
    $L(M'_{x}) = \Sigma^*$ when $M$ halts on $w$ and $L(M'_x) = \emptyset$ when $M$ loops on $w$.


Thus: when $\langle M,w \rangle \in HALT_{TM}$ it gets mapped to a string not in $EQ_{TM}$ and 
when $\langle M,w \rangle \notin HALT_{TM}$ it gets mapped to a string that is in $EQ_{TM}$.

\vfill

Define

\vspace{-15pt}

\begin{quote}
$F_1 =  ``$ On input $x$,
\begin{itemize}
\item[1.] Type-check whether  $x = \langle M, w \rangle$ for some TM $M$ and string $w$. 
If so, move to step 2; if  not, output  $\langle \hspace{2in} \rangle$
\item[2.] Construct the following machine $M'_x$:
\vspace{50pt}
\item[3.] Output $\langle M'_{x},
\scalebox{0.5}{\begin{tikzpicture}[->,>=stealth',shorten >=1pt, auto, node distance=2cm, semithick]
      \tikzstyle{every state}=[text=black, fill=yellow!40]
      \node[initial,state] (q0)                    {$q_0$};
      \node[state,accepting] (qacc) [right of = q0, xshift = 20]{$q_{acc}$};
      \path (q0) edge  [loop above] node {$0, 1, \scalebox{1.5}{\textvisiblespace} \to R$} (q0)
     ;
    \end{tikzpicture}}
    \rangle$ "
\end{itemize}
\end{quote}

\vfill

Verifying correctness: (1) Is function well-defined and computable? (2) Does it have the 
translation property $x \in HALT_{TM}$ iff its image is {\bf not} in $EQ_{TM}$ ? 
\begin{center}
\begin{tabular}{|c|c|}
\hline
Input string &  Output string \\
\hline
$\langle M, w \rangle$ where  $M$ halts on $w$ & \phantom{\hspace{4in}} \\
& \\
& \\
& \\
$\langle M, w \rangle$ where $M$ loops on $w$ & \\
& \\
&\\ & \\
$x$ not encoding any pair of  TM and string   &  \\
& \\
& \\
\hline
\end{tabular}
\end{center}


\vfill

Conclude: $HALT_{TM} \leq_m \overline{EQ_{TM}}$
\newpage

\newpage 
Need computable function  $F_2: \Sigma^* \to \Sigma^*$  such that  $x \in HALT_{TM}$ iff 
$F_2(x)  \in  EQ_{TM}$.



{\it Strategy}:

\vspace{-15pt}

Map strings $\langle M, w \rangle$ to strings $\langle M'_{x},
\scalebox{0.5}{\begin{tikzpicture}[->,>=stealth',shorten >=1pt, auto, node distance=2cm, semithick]
      \tikzstyle{every state}=[text=black, fill=yellow!40]
      \node[initial,state,accepting] (q0)                    {$q_0$};
     ;
    \end{tikzpicture}}
    \rangle$ 
    . This image string is in $EQ_{TM}$ when $L(M'_x) = \Sigma^*$.
    
We will build $M'_x$ so that 
    $L(M'_{x}) = \Sigma^*$ when $M$ halts on $w$ and $L(M'_x) = \emptyset$ when $M$ loops on $w$.


Thus: when $\langle M,w \rangle \in HALT_{TM}$ it gets mapped to a string  in $EQ_{TM}$ and 
when $\langle M,w \rangle \notin HALT_{TM}$ it gets mapped to a string that is not in $EQ_{TM}$.

\vfill

Define

\vspace{-15pt}

\begin{quote}
$F_2 =  ``$ On input $x$,
\begin{itemize}
\item[1.] Type-check whether  $x = \langle M, w \rangle$ for some TM $M$ and string $w$. 
If so, move to step 2; if  not, output  $\langle \hspace{2in} \rangle$
\item[2.] Construct the following machine $M'_x$:
\vspace{50pt}
\item[3.] Output $\langle M'_{x},
\scalebox{0.5}{\begin{tikzpicture}[->,>=stealth',shorten >=1pt, auto, node distance=2cm, semithick]
      \tikzstyle{every state}=[text=black, fill=yellow!40]
      \node[initial,state,accepting] (q0)                    {$q_0$};
     ;
    \end{tikzpicture}}
    \rangle$ "
\end{itemize}
\end{quote}

\vfill

Verifying correctness: (1) Is function well-defined and computable? (2) Does it have the 
translation property $x \in HALT_{TM}$ iff its image is in $EQ_{TM}$ ? 
\begin{center}
\begin{tabular}{|c|c|}
\hline
Input string &  Output string \\
\hline
$\langle M, w \rangle$ where  $M$ halts on $w$ & \phantom{\hspace{4in}} \\
& \\
& \\
& \\
$\langle M, w \rangle$ where $M$ loops on $w$ & \\
& \\
&\\ & \\
$x$ not encoding any pair of  TM and string   &  \\
& \\
& \\
\hline
\end{tabular}
\end{center}


\vfill

Conclude: $HALT_{TM} \leq_m EQ_{TM}$



\newpage
\subsection*{Friday: Other models of computation}

%! app: Decidable Languages, Undecidable Languages
%! outcome: Classify language, Classify decision problem, Reduction, Nondeterminism
 

In practice, computers (and Turing machines) don't have infinite tape, 
and we can't afford to wait unboundedly long for an answer.
``Decidable" isn't good enough - we want ``Efficiently decidable".

For a given algorithm working on a given input, how long do we need to wait for an answer? 
How does the running time depend on the input in the worst-case? average-case? 
We expect to have to spend more time on computations with larger inputs.


A language is {\bf recognizable} if \underline{\phantom{\hspace{4.5in}}}

A language is {\bf decidable} if \underline{\phantom{\hspace{4.7in}}}

A language is {\bf efficiently  decidable} if \underline{\phantom{\hspace{4in}}}

A function is {\bf computable} if \underline{\phantom{\hspace{4.7in}}}

A function is {\bf efficiently computable} if \underline{\phantom{\hspace{4in}}}\\

\vfill

Definition  (Sipser 7.1): For  $M$ a deterministic decider, its {\bf running time} is the function  $f: \mathbb{N} \to \mathbb{N}$
given  by
\[
f(n) =  \text{max number of  steps $M$ takes before halting, over all inputs  of length $n$}
\]

Definition (Sipser 7.7): For each function $t(n)$, the {\bf time complexity class}  $TIME(t(n))$, is defined  by
\[
TIME( t(n)) = \{ L \mid \text{$L$ is decidable by  a Turing machine with running time in  $O(t(n))$} \}
\]

An example of an element of  $TIME(  1  )$ is 

An example of an element of  $TIME(  n  )$ is 


Note: $TIME( 1) \subseteq TIME (n)  \subseteq TIME(n^2)$

\vfill

Definition (Sipser 7.12) : $P$ is the class of languages that  are decidable in polynomial time on 
a deterministic 1-tape  Turing  machine
\[
P  =  \bigcup_k TIME(n^k)
\]

{\it Compare to exponential time: brute-force search.}


Theorem (Sipser 7.8): Let $t(n)$ be a  function with  $t(n)  \geq n$.  Then every $t(n)$ time deterministic 
multitape Turing machine has an equivalent $O(t^2(n))$ time deterministic 1-tape Turing machine.



\newpage


Definition (Sipser  7.9): For $N$ a nodeterministic decider.  
The {\bf running time} of $N$ is the function $f: \mathbb{N} \to \mathbb{N}$ given  by
\[
f(n) =  \text{max number of  steps $N$ takes on  any branch before halting, over all inputs  of length $n$}
\]

Definition (Sipser 7.21): For each function $t(n)$, the {\bf nondeterministic time complexity class}  
$NTIME(t(n))$, is defined  by
\[
NTIME( t(n)) = \{ L \mid \text{$L$ is decidable by a nondeterministic Turing machine with running time in $O(t(n))$} \}
\]
\[
NP = \bigcup_k NTIME(n^k)
\]


{\bf True} or {\bf False}: $TIME(n^2) \subseteq NTIME(n^2)$

\vfill

{\bf True} or {\bf False}: $NTIME(n^2) \subseteq DTIME(n^2)$

\vfill

{\bf Examples in $P$ }

{\it Can't use nondeterminism; Can use multiple tapes; Often need to be “more clever” than naïve / brute force approach}
\[
    PATH = \{\langle G,s,t\rangle \mid \textrm{$G$ is digraph with $n$ nodes there is path from s to t}\}
\]
Use breadth first search to show in $P$
\[
    RELPRIME = \{ \langle x,y\rangle \mid \textrm{$x$ and $y$ are relatively prime integers}\}
\]
Use Euclidean Algorithm to show in $P$
\[
    L(G) = \{w \mid \textrm{$w$ is generated by $G$}\} 
\]
(where $G$ is a context-free grammar). Use dynamic programming to show in $P$.

\vfill
{\bf Examples in $NP$}

{\it ``Verifiable" i.e. NP, Can be decided by a nondeterministic TM in polynomial time,
best known deterministic solution may be brute-force, 
solution can be verified by a deterministic TM in polynomial time.}

\[
    HAMPATH = \{\langle G,s,t \rangle \mid \textrm{$G$ is digraph with $n$ nodes, there is path
from $s$ to $t$ that goes through every node exactly once}\}
\]
\[
    VERTEX-COVER = \{ \langle G,k\rangle \mid \textrm{$G$ is an undirected graph with $n$
nodes that has a $k$-node vertex cover}\}
\]
\[
    CLIQUE = \{ \langle G,k\rangle \mid \textrm{$G$ is an undirected graph with $n$ nodes that has a $k$-clique}\}
\]
\[
    SAT =\{ \langle X \rangle \mid \textrm{$X$ is a satisfiable Boolean formula with $n$ variables}\}
\]

\newpage

\newpage

\subsection*{Week 9 at a glance}

\subsubsection*{Textbook reading: Section 5.3, 5.1, 3.2}

{\it For Monday}: Example 5.26 (page 237)

{\it For Wednesday}: Theorem 5.30 (page 238)

{\it For Friday}: Skim Section 3.2

{\it For Monday of Week 10}: Definition 7.1 (page 276)

\subsubsection*{Make sure you can:}
\begin{itemize}
\item Classify the computational complexity of a set of strings by determining whether it is decidable or undecidable and recognizable or unrecognizable.
\begin{itemize}
   \item State, prove, and use theorems relating decidability, recognizability, and co-recognizability.
   \item Prove that a language is decidable or recognizable by defining and analyzing a Turing machines with appropriate properties.
   \item Define and explain core examples of computational problems, including $A$**, $E$**, $EQ$**, (for ** either DFA or TM) and $HALT_{TM}$
\end{itemize}
\item Use diagonalization to prove that there are 'hard' languages relative to certain models of computation.
\item Use mapping reduction to deduce the complexity of a language by comparing to the complexity of another.
   \begin{itemize}
      \item Explain what it means for one problem to reduce to another
      \item Define computable functions, and use them to give mapping reductions between computational problems
      \item Define and explain $A_{TM}$ and $HALT_{TM}$
      \item Build and analyze mapping reductions between computational problems
      \item Distinguish between computability and complexity
      \item Articulate motivating questions of complexity
      \item Use appropriate reduction (e.g. mapping, Turing, polynomial-time) to deduce the complexity of a language by comparing to the complexity of another.
   \end{itemize}
\item  Describe several variants of Turing machines and informally explain why they are equally expressive.
   \begin{itemize}
   \item Define an enumerator
   \item Describe the language enumerated by an enumerator
   \item Use high-level descriptions to define and trace machines (Turing machines and enumerators)
   \item Apply dovetailing in high-level definitions of machines
   \item Define nondeterministic Turing machines
   \item State and use the Church-Turing thesis
   \end{itemize}
\end{itemize}


\subsubsection*{TODO:}
\begin{list}
   {\itemsep2pt}
   \item Student Evaluations of Teaching forms: Evaluations are open for completion anytime BEFORE 8AM on Saturday, March 16.
   Access your SETs from the Evaluations site
   \begin{quote}
        \url{https://academicaffairs.ucsd.edu/Modules/Evals}
   \end{quote}
   You will separately evaluate each of your listed instructors for each enrolled course. 

   **NEW** WINTER 2024 SET INCENTIVE LOTTERY: In Winter 2024, students who complete all of their student 
   evaluation forms for their undergraduate course will be entered into a lottery to win one of 
   5 \$100 Visa gift cards! To be entered into the lottery, students must complete at 
   least one instructor evaluation for EACH of their undergraduate courses. 
   They will be automatically entered when they have completed an instructor evaluation for 
   all of their undergraduate courses.

   \item Review quizzes based on class material each day.
   \item Test this Friday in Discussion section.
   \item Homework assignment 5 due next Thursday.
\end{list}

\end{document}
