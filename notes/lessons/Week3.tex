\documentclass[12pt, oneside]{article}

\usepackage[letterpaper, scale=0.89, centering]{geometry}
\usepackage{fancyhdr}
\setlength{\parindent}{0em}
\setlength{\parskip}{1em}

\pagestyle{fancy}
\fancyhf{}
\rfoot{\href{https://creativecommons.org/licenses/by-nc-sa/2.0/}{CC BY-NC-SA 2.0} Version \today~(\thepage.)}

\input{../../resources/CSE20packages}

\begin{document}
\begin{flushright}
    \StrBefore{\currfilename}{.}
\end{flushright}

\subsection*{Week 3 at a glance}


\subsubsection*{Textbook reading: Chapter 1}

\vspace{-15pt}

No class on Week 3 Monday in observance of Martin Luther King Jr. Day.

Before Wednesday: read the definition of the union, concatenation, and star operations for languages,  given 
as Definition 1.23 on page 44 and a useful example is Example 1.24.

Before Friday,  read pages 45-46 (Theorem 1.25) that we'll refer to as a ``closure proof".

For Week 4 Monday, read Introduction to Section 1.4 (page 77) which introduces nonregularity.


\vspace{-20pt}

\subsubsection*{We will be learning and practicing to:}

\vspace{-15pt}

%Define decision problem, Formal definition of automata, Informal definition of automata, Find example languages, 
%Describe and use models of computation that don't involve state machines.
\begin{itemize}
\item Clearly and unambiguously communicate computational ideas using appropriate formalism. Translate across levels of abstraction.
\begin{itemize}
   \item Use precise notation to formally define the state diagram of finite automata.
   \item Use clear English to describe computations of finite automata informally.
      \begin{itemize}
         \item {\bf Motivate the use of nondeterminism}
         \item {\bf State the formal definition of NFA}   
         \item {\bf Trace the computation(s) of a NFA on a given string using its state diagram}
         \item {\bf Determine if a given string is in the language recognized by a NFA}
         \item {\bf Translate between a state diagram and a formal definition of a NFA}
      \end{itemize}
   \item Give examples of sets that are regular (and prove that they are).
   \begin{itemize}
      \item {\bf State the definition of the class of regular languages}
      \item {\bf Give examples of regular languages, using each of the three equivalent models of computation for proving regularity.}
      \item {\bf Choose between multiple models to prove that a language is regular}
      \item {\bf Explain the limits of the class of regular languages}
   \end{itemize}
   \item Describe and use models of computation that don't involve state machines.
   \begin{itemize}
      \item {\bf Given a DFA or NFA, find a regular expression that describes its language.}
      \item {\bf Given a regular expression, find a DFA or NFA that recognizes its language.}
   \end{itemize}
\end{itemize}


\item Understand, guide, shape impact of computing on society/the world. Connect the role of Theory CS classes to other applications (in undergraduate CS curriculum and beyond). Model problems using appropriate mathematical concepts.
 \begin{itemize}
     \item Explain nondeterminism and describe tools for simulating it with deterministic computation.
     \begin{itemize}
       \item {\bf Given a NFA, find a DFA that recognizes its language.}
       \item {\bf Convert between regular expressions and automata}
     \end{itemize}
 \end{itemize}

\end{itemize}

\vspace{-20pt}

\subsubsection*{TODO:}
\begin{list}{\itemsep-10pt}
   \item Schedule your Test 1 Attempt 1, Test 2 Attempt 1, Test 1 Attempt 2, and Test 2 Attempt 2 times 
   at PrairieTest (http://us.prairietest.com)
   \item Review Quiz 3 on PrairieLearn (http://us.prairielearn.com), due 1/29/2025
   \item Homework 2 submitted via Gradescope (https://www.gradescope.com/), due Tuesday 1/30/2025
\end{list}


\vfill

In Computer Science, we operationalize ``hardest'' as ``requires most resources'', where
resources might be memory, time, parallelism, randomness, power, etc.
To be able to compare ``hardness'' of problems, we use a consistent description of problems

{\bf Input}: String

{\bf Output}: Yes/ No, where Yes means that the input string matches the pattern or property described by the problem.

So far: we saw that regular expressions are convenient ways of describring patterns in strings.
{\bf Finite automata} give a model of computation for processing strings and and classifying them into Yes (accepted)
or No (rejected). We will see that each set of strings is described by a regular expression if and only 
if there is a FA that recognizes it.  Another way of thinking about it: properties described by regular
expressions require exactly the computational power of these finite automata.

\newpage
\subsection*{Wednesday: Automata constructions}

\input{../activity-snippets/day7.tex}

\newpage
\subsection*{Friday: Regular langauges}

\input{../activity-snippets/day8.tex}

\end{document}