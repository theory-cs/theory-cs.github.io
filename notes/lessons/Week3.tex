\input{../../resources/lesson-head.tex}

\subsection*{Week 3 at a glance}


\subsubsection*{Textbook reading: Chapter 1}

\vspace{-15pt}

No class on Week 3 Monday in observance of Martin Luther King Jr. Day.

Before Wednesday: read the definition of the union, concatenation, and star operations for languages,  given 
as Definition 1.23 on page 44 and a useful example is Example 1.24.

Before Friday,  read pages 45-46 (Theorem 1.25) that we'll refer to as a ``closure proof".

For Week 4 Monday, read Introduction to Section 1.4 (page 77) which introduces nonregularity.


\vspace{-20pt}

\subsubsection*{We will be learning and practicing to:}

\vspace{-15pt}

%Define decision problem, Formal definition of automata, Informal definition of automata, Find example languages, 
%Describe and use models of computation that don't involve state machines.
\begin{itemize}
\item Clearly and unambiguously communicate computational ideas using appropriate formalism. Translate across levels of abstraction.
\begin{itemize}
   \item Use precise notation to formally define the state diagram of finite automata.
   \item Use clear English to describe computations of finite automata informally.
      \begin{itemize}
         \item {\bf Motivate the use of nondeterminism}
         \item {\bf State the formal definition of NFA}   
         \item {\bf Trace the computation(s) of a NFA on a given string using its state diagram}
         \item {\bf Determine if a given string is in the language recognized by a NFA}
         \item {\bf Translate between a state diagram and a formal definition of a NFA}
      \end{itemize}
   \item Give examples of sets that are regular (and prove that they are).
   \begin{itemize}
      \item {\bf State the definition of the class of regular languages}
      \item {\bf Give examples of regular languages, using each of the three equivalent models of computation for proving regularity.}
      \item {\bf Choose between multiple models to prove that a language is regular}
      \item {\bf Explain the limits of the class of regular languages}
   \end{itemize}
   \item Describe and use models of computation that don't involve state machines.
   \begin{itemize}
      \item {\bf Given a DFA or NFA, find a regular expression that describes its language.}
      \item {\bf Given a regular expression, find a DFA or NFA that recognizes its language.}
   \end{itemize}
\end{itemize}


\item Understand, guide, shape impact of computing on society/the world. Connect the role of Theory CS classes to other applications (in undergraduate CS curriculum and beyond). Model problems using appropriate mathematical concepts.
 \begin{itemize}
     \item Explain nondeterminism and describe tools for simulating it with deterministic computation.
     \begin{itemize}
       \item {\bf Given a NFA, find a DFA that recognizes its language.}
       \item {\bf Convert between regular expressions and automata}
     \end{itemize}
 \end{itemize}

\end{itemize}

\vspace{-20pt}

\subsubsection*{TODO:}
\begin{list}{\itemsep-10pt}
   \item Schedule your Test 1 Attempt 1, Test 2 Attempt 1, Test 1 Attempt 2, and Test 2 Attempt 2 times 
   at PrairieTest (http://us.prairietest.com)
   \item Review Quiz 3 on PrairieLearn (http://us.prairielearn.com), due 1/29/2025
   \item Homework 2 submitted via Gradescope (https://www.gradescope.com/), due Tuesday 1/30/2025
\end{list}


\vfill

In Computer Science, we operationalize ``hardest'' as ``requires most resources'', where
resources might be memory, time, parallelism, randomness, power, etc.
To be able to compare ``hardness'' of problems, we use a consistent description of problems

{\bf Input}: String

{\bf Output}: Yes/ No, where Yes means that the input string matches the pattern or property described by the problem.

So far: we saw that regular expressions are convenient ways of describring patterns in strings.
{\bf Finite automata} give a model of computation for processing strings and and classifying them into Yes (accepted)
or No (rejected). We will see that each set of strings is described by a regular expression if and only 
if there is a FA that recognizes it.  Another way of thinking about it: properties described by regular
expressions require exactly the computational power of these finite automata.

\newpage
\subsection*{Wednesday: Automata constructions}

%! app: Regular Languages
%! outcome: Formal definition of automata, Informal definition of automata, Nondeterminism

{\bf Review}: The language recognized by the NFA over $\{a,b\}$ with state diagram


    \begin{tikzpicture}[->,>=stealth',shorten >=1pt, auto, node distance=2cm, semithick]
    \tikzstyle{every state}=[text=black, fill=none]
    
    \node[initial,state] (q0)          {$q_0$};
    \node[state]         (q) [above right of=q0, xshift=20pt] {$q$};
    \node[state]         (r) [right of=q, xshift=20pt] {$r$};
    \node[state, accepting]         (s) [right of=r, xshift=20pt] {$s$};
    \node[state, accepting]         (n) [below right of=q0, xshift=20pt] {$n$};
    \node[state]         (d) [right of=n, xshift=20pt] {$d$};
    
    \path (q0) edge  [bend left=0, near start] node {$\varepsilon$} (q)
            edge [bend right=0, near start] node {$\varepsilon$} (n)
        (q) edge [bend left=0] node {$a$} (r)
            edge [loop above, near start] node {$b$} (q)
        (r) edge [bend left=0] node {$b$} (s)
            edge [loop above, near start] node {$a,b$} (r)
        (n) edge [bend left=20] node {$a,b$} (d)
        (d) edge [bend left=20] node {$a,b$} (n)
    ;
    \end{tikzpicture}
is:


\vfill

So far, we know: 
\begin{itemize}
\item The collection of languages that are each recognizable by a 
DFA is {\bf closed} under complementation.

{\it Could we do the same construction with NFA?}

\vspace{100pt}

\item The collection of languages that are each recognizable by a 
NFA is {\bf closed} under union.

{\it Could we do the same construction with DFA?}

\vspace{100pt}

\end{itemize}


\newpage

\vspace{50pt}

Happily, though, an analogous claim is true!

Suppose $A_1, A_2$ are languages over an alphabet $\Sigma$.
{\bf Claim:} if there is a DFA $M_1$ such that $L(M_1) = A_1$ and 
DFA $M_2$ such that $L(M_2) = A_2$, then there is another DFA, let's call it $M$, such that 
$L(M) = A_1 \cup A_2$. {\it Theorem 1.25 in Sipser, page 45}
    
    {\bf Proof idea}:
    
    
    {\bf Formal construction}: 
    
    \vfill

    
    {\bf Example}:  When $A_1 = \{w \mid w~\text{has an $a$ and ends in $b$} \}$ and 
    $A_2 = \{ w \mid w~\text{is of even length} \}$.
    
    \begin{tikzpicture}[->,>=stealth',shorten >=1pt, auto, node distance=2cm, semithick]
        \tikzstyle{every state}=[text=black, fill=none]
        
        \node[initial,state,accepting] (qn)          {$(q,n)$};
        \node[state]         (qd) [below of=qn, yshift=-40pt] {$(q,d)$};
        \node[state]         (rd) [right of=qn, xshift=20pt] {$(r,d)$};
        \node[state,accepting]         (rn) [right of=qd, xshift=20pt] {$(r,n)$};
        \node[state,accepting]         (sn) [right of=rd, xshift=20pt] {$(s,n)$};
        \node[state,accepting]         (sd) [right of=rn, xshift=20pt] {$(s,d)$};
        
        \path (qn) edge  [bend left=20, near start] node {$b$} (qd)
                edge [bend left=20, near start] node {$a$} (rd)
            (qd) edge [bend left=20, near start] node {$b$} (qn)
                edge [bend right=20, near start] node {$a$} (rn)
            (rn) edge [bend left=20, near start] node {$a$} (rd)
                edge [bend left=20, near start] node {$b$} (sd)
            (rd) edge [bend left=20, near start] node {$a$} (rn)
                edge [bend left=20, near start] node {$b$} (sn)
            (sn) edge [bend left=20, near start] node {$a$} (rd)
                edge [bend left=20, near start] node {$b$} (sd)
            (sd) edge [bend left=20, near start] node {$a$} (rn)
                edge [bend left=20, near start] node {$b$} (sn)
        ;
        \end{tikzpicture}
    
    \newpage
    
    Suppose $A_1, A_2$ are languages over an alphabet $\Sigma$.
    {\bf Claim:} if there is a DFA $M_1$ such that $L(M_1) = A_1$ and 
    DFA $M_2$ such that $L(M_2) = A_2$, then there is another DFA, let's call it $M$, such that 
    $L(M) = A_1 \cap A_2$.  {\it Footnote to Sipser Theorem 1.25, page 46}
    
    {\bf Proof idea}:
    
    
    {\bf Formal construction}: 
    
    \vspace{70pt}


    


\newpage
\subsection*{Friday: Regular langauges}

%! app: Regular Languages
%! outcome: Regular expressions, Formal definition of automata, Informal definition of automata, Nondeterminism

So far we have that: 
\begin{itemize}
\item If there is a DFA recognizing a language, there is a DFA recognizing its complement.
\item If there are NFA recognizing two languages, there is a NFA recognizing their union.
\item If there are DFA recognizing two languages, there is a DFA recognizing their union.
\item If there are DFA recognizing two languages, there is a DFA recognizing their intersection.
\end{itemize}

Our goals for today are (1) prove similar results about other set operations, (2) prove that 
NFA and DFA are equally expressive, and therefore (3) define an important class of languages.

\vfill

\newpage
Suppose $A_1, A_2$ are languages over an alphabet $\Sigma$.
{\bf Claim:} if there is a NFA $N_1$ such that $L(N_1) = A_1$ and 
NFA $N_2$ such that $L(N_2) = A_2$, then there is another NFA, let's call it $N$, such that 
$L(N) = A_1 \circ A_2$.

{\bf Proof idea}: Allow computation to move between $N_1$ and $N_2$ ``spontaneously" when reach an accepting state of 
$N_1$, guessing that we've reached the point where the two parts of the string in the set-wise concatenation 
are glued together.


{\bf Formal construction}: Let 
$N_1 = (Q_1, \Sigma, \delta_1, q_1, F_1)$ and $N_2 = (Q_2, \Sigma, \delta_2,q_2, F_2)$
and assume $Q_1 \cap Q_2 = \emptyset$.
Construct $N = (Q, \Sigma, \delta, q_0, F)$ where
\begin{itemize}
    \item $Q = $
    \item $q_0 = $
    \item $F = $
    \item $\delta: Q \times \Sigma_\varepsilon \to \mathcal{P}(Q)$ is defined by, for $q \in Q$ and $a \in \Sigma_{\varepsilon}$:
        \[
            \delta((q,a))=\begin{cases}  
                \delta_1 ((q,a)) &\qquad\text{if } q\in Q_1 \textrm{ and } q \notin F_1\\ 
                \delta_1 ((q,a)) &\qquad\text{if } q\in F_1 \textrm{ and } a \in \Sigma\\ 
                \delta_1 ((q,a)) \cup \{q_2\} &\qquad\text{if } q\in F_1 \textrm{ and } a = \varepsilon\\ 
                \delta_2 ((q,a)) &\qquad\text{if } q\in Q_2
            \end{cases}
        \]
\end{itemize}

\vfill

{\it Proof of correctness would prove that $L(N) = A_1 \circ A_2$ by considering
an arbitrary string accepted by $N$, tracing an accepting computation of $N$ on it, and using 
that trace to prove the string can be written as the result of concatenating two strings, 
the first in $A_1$ and the second in $A_2$; then, taking an arbitrary 
string in $A_1 \circ A_2$ and proving that it is accepted by $N$. Details left for extra practice.}

\newpage



Suppose $A$ is a language over an alphabet $\Sigma$.
{\bf Claim:} if there is a NFA $N$ such that $L(N) = A$, then there is another NFA, let's call it $N'$, such that 
$L(N') = A^*$.

{\bf Proof idea}: Add a fresh start state, which is an accept state. Add spontaneous 
moves from each (old) accept state to the old start state.

{\bf Formal construction}: Let 
$N = (Q, \Sigma, \delta, q_1, F)$ and assume $q_0 \notin Q$.
Construct $N' = (Q', \Sigma, \delta', q_0, F')$ where
\begin{itemize}
    \item $Q' = Q \cup \{q_0\}$
    \item $F' = F \cup \{q_0\}$
    \item $\delta': Q' \times \Sigma_\varepsilon \to \mathcal{P}(Q')$ is defined by, for $q \in Q'$ and $a \in \Sigma_{\varepsilon}$:
        \[
            \delta'((q,a))=\begin{cases}  
                \delta ((q,a)) &\qquad\text{if } q\in Q \textrm{ and } q \notin F\\ 
                \delta ((q,a)) &\qquad\text{if } q\in F \textrm{ and } a \in \Sigma\\ 
                \delta ((q,a)) \cup \{q_1\} &\qquad\text{if } q\in F \textrm{ and } a = \varepsilon\\ 
                \{q_1\} &\qquad\text{if } q = q_0 \textrm{ and } a = \varepsilon \\
                \emptyset &\qquad\text{if } q = q_0 \textrm { and } a \in \Sigma
            \end{cases}
        \]
\end{itemize}


{\it Proof of correctness would prove that $L(N') = A^*$ by considering
an arbitrary string accepted by $N'$, tracing an accepting computation of $N'$ on it, and using 
that trace to prove the string can be written as the result of concatenating some number of strings, 
each of which is in $A$; then, taking an arbitrary 
string in $A^*$ and proving that it is accepted by $N'$. Details left for extra practice.}


{\bf Application}: A state diagram for a NFA over $\Sigma = \{a,b\}$ 
that recognizes $L (( a^*b)^* )$:

\vfill
\newpage
Suppose $A$ is a language over an alphabet $\Sigma$.
{\bf Claim:} if there is a NFA $N$ such that $L(N) = A$ then 
there is a DFA $M$ such that $L(M) = A$.

{\bf Proof idea}: States in $M$ are ``macro-states" -- collections of states from $N$ -- 
that represent the set of possible states a computation of $N$ might be in.


{\bf Formal construction}: Let $N = (Q, \Sigma, \delta, q_0, F)$.  Define 
\[
M = (~ \mathcal{P}(Q), \Sigma, \delta', q',  \{ X \subseteq Q \mid X \cap F \neq \emptyset \}~ )
\]
where $q' = \{ q \in Q \mid \text{$q = q_0$ or is accessible from $q_0$ by spontaneous moves in $N$} \}$
and 
\[
    \delta' (~(X, x)~) = \{ q \in Q \mid q \in \delta( ~(r,x)~) ~\text{for some $r \in X$ or is accessible 
from such an $r$ by spontaneous moves in $N$} \}
\]


Consider the state diagram of an NFA over $\{a,b\}$. Use the ``macro-state'' construction 
to find an equivalent DFA.

\begin{tikzpicture}[->,>=stealth',shorten >=1pt, auto, node distance=2cm, semithick]
    \tikzstyle{every state}=[text=black, fill=none]
    
    \node[initial,state] (q0)          {$q_0$};
    \node[state]         (q1) [right of=q0, xshift=20pt] {$q_1$};
    \node[state,accepting]         (q2) [right of=q1, xshift=20pt] {$q_2$};
    
    \path (q0) edge  [loop above] node {$a,b$} (q0)
            edge [bend left=0] node {$a$} (q1)
        (q1) edge [loop above] node {$a,b$} (q1)
            edge [bend left=0] node {$b$} (q2)
    ;
    \end{tikzpicture}

\vfill

Consider the state diagram of an NFA over $\{0,1\}$. Use the ``macro-state'' construction 
to find an equivalent DFA.

\begin{tikzpicture}[->,>=stealth',shorten >=1pt, auto, node distance=2cm, semithick]
    \tikzstyle{every state}=[text=black, fill=none]
    
    \node[initial,state] (q0)          {$q_0$};
    \node[state,accepting]         (q1) [above right of=q0, xshift=20pt] {$q_1$};
    \node[state,accepting]         (q2) [below right of=q0, xshift=20pt] {$q_2$};
    
    \path (q0)  edge [bend left=0] node {$\varepsilon$} (q1)
        (q0) edge [bend left = 0] node {$\varepsilon$} (q2)
        (q1) edge [loop above] node {$0$} (q1)
        (q2) edge [loop above] node {$1$} (q2)
    ;
    \end{tikzpicture}

\vfill

Note: We can often prune the DFAs that result from the ``macro-state'' constructions to get an 
equivalent DFA with fewer states (e.g.\ only the ``macro-states" reachable from the start state).

\newpage


{\bf The class of regular languages}

Fix an alphabet $\Sigma$. For each language $L$ over $\Sigma$:
\begin{center}
\begin{tabular}{cc}
    {\bf There is a DFA over $\Sigma$ that recognizes $L$}&$\exists M ~(M \textrm{ is a DFA and } L(M) = A)$\\
    {\it if and only if}&\\
    {\bf There is a NFA over $\Sigma$ that recognizes $L$}&$\exists N ~(N \textrm{ is a NFA and } L(N) = A)$\\
    {\it if and only if}&\\
    {\bf There is a regular expression over $\Sigma$ that describes $L$} &$\exists R ~(R \textrm{ is a regular expression and } L(R) = A)$\\
\end{tabular}
\end{center}

A language is called {\bf regular} when any (hence all) of the above three conditions are met.

We already proved that DFAs and NFAs are equally expressive. It remains to prove that regular expressions 
are too.

Part 1: Suppose $A$ is a language over an alphabet $\Sigma$.
If there is a regular expression $R$ such that $L(R) = A$, then there is a NFA, let's call it $N$, such that 
$L(N) = A$.

{\bf Structural induction}: Regular expression is built from basis regular expressions using inductive steps
(union, concatenation, Kleene star symbols). Use constructions to mirror these in NFAs.


{\bf Application}: A state diagram for a NFA over $\{a,b\}$ that recognizes $L(a^* (ab)^*)$:

\vfill

Part 2: Suppose $A$ is a language over an alphabet $\Sigma$.
If there is a DFA $M$ such that $L(M) = A$, then there is a regular expression, let's call it $R$, such that 
$L(R) = A$.

{\bf Proof idea}: Trace all possible paths from start state to accept state.  Express labels of these paths
as regular expressions, and union them all.

\begin{enumerate}
\item Add new start state with $\varepsilon$ arrow to old start state.
\item Add new accept state with $\varepsilon$ arrow from old accept states.  Make old accept states
non-accept.
\item Remove one (of the old) states at a time: modify regular expressions on arrows that went through removed
state to restore language recognized by machine.
\end{enumerate}

{\bf Application}: Find a regular expression describing the language recognized by the DFA with 
state diagram

\begin{tikzpicture}[->,>=stealth',shorten >=1pt, auto, node distance=2cm, semithick]
    \tikzstyle{every state}=[text=black, fill=none]
    
    \node[initial,state,accepting] (q0)          {$q_0$};
    \node[state,accepting]         (q1) [above right of=q0, xshift=20pt] {$q_1$};
    \node[state,accepting]         (q2) [below right of=q0, xshift=20pt] {$q_2$};
    \node[state]                   (q3) [below right of=q1, xshift=20pt] {$q_3$};
    
    \path (q0)  edge [bend left=0] node {$a$} (q1)
        (q0) edge [bend left = 0] node {$b$} (q2)
        (q1) edge [loop above] node {$a$} (q1)
        (q1) edge [bend left=0] node {$b$} (q3)
        (q2) edge [loop below] node {$b$} (q2)
        (q2) edge [bend left=0] node {$a$} (q3)
        (q3) edge [loop above] node {$a,b$} (q3)
    ;
    \end{tikzpicture}

\vfill

\end{document}