\documentclass[12pt, oneside]{article}

\usepackage[letterpaper, scale=0.89, centering]{geometry}
\usepackage{fancyhdr}
\setlength{\parindent}{0em}
\setlength{\parskip}{1em}

\pagestyle{fancy}
\fancyhf{}
\rfoot{\href{https://creativecommons.org/licenses/by-nc-sa/2.0/}{CC BY-NC-SA 2.0} Version \today~(\thepage.)}

\input{../../resources/CSE20packages}

\begin{document}
\begin{flushright}
    \StrBefore{\currfilename}{.}
\end{flushright}

\section*{Before we start}

We are committed to fostering a learning environment for this course that supports a diversity of thoughts, 
perspectives and experiences, and respects your identities (including race, ethnicity, heritage, gender, sex, 
class, sexuality, religion, ability, age, educational background, etc.).  
Our goal is to create a diverse and inclusive learning environment where all students feel comfortable and can thrive. 

If you or someone you know is suffering from food and/or housing insecurities 
there are UCSD resources here to help:

Basic Needs Office: \href{https://basicneeds.ucsd.edu/}{https://basicneeds.ucsd.edu/}

Triton Food Pantry (in the old Student Center)
is free and anonymous, and includes produce: 

\href{https://www.facebook.com/tritonfoodpantry/}{https://www.facebook.com/tritonfoodpantry/}

Mutual Aid UCSD: \href{https://mutualaiducsd.wordpress.com/}{https://mutualaiducsd.wordpress.com/}

Financial aid resources, the possibility of emergency grant funding, and off-campus housing referral 
resources are available. See CAPS and your college dean.

If you find yourself in an uncomfortable situation, ask for help. 
We are committed to upholding University policies regarding nondiscrimination, sexual violence and sexual harassment.

Counseling and Psychological Services (CAPS) at 858 5343755 or \href{http://caps.ucsd.edu}{http://caps.ucsd.edu}


OPHD at (858) 534-8298, ophd@ucsd.edu , \href{http://ophd.ucsd.edu}{http://ophd.ucsd.edu}. 
CARE at Sexual Assault Resource Center at 858 5345793 sarc@ucsd.edu \href{http://care.ucsd.edu}{http://care.ucsd.edu}

\subsection*{Spring quarter philosophy}
Spring 2022 is still a transition quarter so please be patient with us as we do our best 
to deliver a great learning opportunity that meets the needs of all students and adheres to the university guidelines. 

Please do not come to class if you are sick or even think you might be sick.

Please reach out (minnes@eng.ucsd.edu) if you need support with extenuating circumstances.

Based on current UCSD guidelines (as of March 21), masks are required in class. 
All students who attend class must also be fully vaccinated against COVID-19
unless they have a university-approved exemption.
We will continue to follow the campus guidelines as updated on https://returntolearn.ucsd.edu/ .


\newpage

\section*{Introductions}
Class website: \href{https://cseweb.ucsd.edu/classes/sp22/cse105-a/}{https://cseweb.ucsd.edu/classes/sp22/cse105-a/}

{\bf Pro-tip}: the URL structure is your map to finding your course website for other CSE classes.

Instructor: Prof. Mia Minnes {\tiny{"Minnes" rhymes with Guinness}}, minnes@eng.ucsd.edu, 
\href{http://cseweb.ucsd.edu/~minnes}{http://cseweb.ucsd.edu/~minnes}

Our team: Six TAs and five tutors + all of you

Fill in contact info for students around you, if you'd like:

\vfill


On a typical week: {\bf MWF} Lectures + review quizzes, {\bf M} Discussion, {\bf Th} Homework / project due.
Office hours and Q+A on Piazza available throughout the week.
All dates are on \href{https://canvas.ucsd.edu/}{Canvas (click for link)} and details are on
 \href{https://theory-cs.github.io/website/overview_calendar.html}{course calendar (click for link)}.

\newpage Welcome to CSE 105: Introduction to Theory of Computation in Spring 2022!

\section*{CSE 105's Big Questions}
\begin{itemize}
   \item What problems are computers capable of solving?
   \item What resources are needed to solve a problem?
   \item Are some problems harder than others?
\end{itemize}

In this context, a {\bf problem} is defined as: ``Making a decision or computing a value based on some input"

Consider the following problems: 
\begin{itemize}
   \item Find a file on your computer
   \item Determine if your code will compile
   \item Find a run-time error in your code
   \item Certify that your system is un-hackable
\end{itemize}

Which of these is hardest?

\vfill

In Computer Science, we operationalize ``hardest'' as ``requires most resources'', where
resources might be memory, time, parallelism, randomness, power, etc.

To be able to compare ``hardness'' of problems, we use a consistent description of problems

{\bf Input}: String

{\bf Output}: Yes/ No, where Yes means that the input string matches the pattern or property described by the problem.


\newpage

Class content for Monday, Wednesday, Friday to be available shortly.

\end{document}