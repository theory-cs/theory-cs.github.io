\documentclass[12pt, oneside]{article}

\usepackage[letterpaper, scale=0.89, centering]{geometry}
\usepackage{fancyhdr}
\setlength{\parindent}{0em}
\setlength{\parskip}{1em}

\pagestyle{fancy}
\fancyhf{}
\rfoot{\href{https://creativecommons.org/licenses/by-nc-sa/2.0/}{CC BY-NC-SA 2.0} Version \today~(\thepage.)}

\input{../../resources/CSE20packages}

\begin{document}
\begin{flushright}
    \StrBefore{\currfilename}{.}
\end{flushright}

\subsection*{Week 4 at a glance}

\subsubsection*{Textbook reading: Section 1.4, 2.2, 2.1.}

\vspace{-15pt}

Before Monday, read Introduction to Section 1.4 (page 77) which introduces nonregularity.

Before Wednesday, read Definition 2.13 (page 111-112) introducing Pushdown Automata.

Before Friday, read Example 2.18 (page 114).

For Week 5 Monday: read Introduction to Section 2.1 (pages 101-102).

\vspace{-20pt}

\subsubsection*{We will be learning and practicing to:}

\begin{itemize}
    \item Clearly and unambiguously communicate computational ideas using appropriate formalism. Translate across levels of abstraction.
    \begin{itemize}
       \item Give examples of sets that are regular (and prove that they are).
       \begin{itemize}
          \item {\bf State the definition of the class of regular languages}
          \item {\bf Explain the limits of the class of regular languages}
          \item {\bf Identify some regular sets and some nonregular sets}
       \end{itemize}
       \item Use precise notation to formally define the state diagram of a PDA
       \item Use clear English to describe computations of PDA informally.
       \begin{itemize}
           \item {\bf Define push-down automata informally and formally}
           \item {\bf State the formal definition of a PDA}
           \item {\bf Trace the computation(s) of a PDA on a given string using its state diagram}
           \item {\bf Determine if a given string is in the language recognized by a PDA}
           \item {\bf Translate between a state diagram and a formal definition of a PDA}
           \item {\bf Determine the language recognized by a given PDA}
        \end{itemize}

    \end{itemize}

    \item Know, select and apply appropriate computing knowledge and problem-solving techniques. 
    \begin{itemize}
    \item Apply classical techniques including pumping lemma, determinization, diagonalization, and reduction to analyze the complexity of languages and problems.
    \begin{itemize}
        \item {\bf Justify why the Pumping Lemma is true.}
        \item {\bf Use the pumping lemma to prove that a given language is not regular.}
    \end{itemize}
    \end{itemize}
\end{itemize}

\vspace{-20pt}

\subsubsection*{TODO:}
\begin{list}{\itemsep-10pt}
   \item Schedule your Test 1 Attempt 1, Test 2 Attempt 1, Test 1 Attempt 2, and Test 2 Attempt 2 times 
   at PrairieTest (http://us.prairietest.com)
   \item Review Quiz 3 on PrairieLearn (http://us.prairielearn.com), due 1/29/2025
   \item Homework 2 submitted via Gradescope (https://www.gradescope.com/), due 1/30/2025
   \item Review Quiz 4 on PrairieLearn (http://us.prairielearn.com), due 2/5/2025
\end{list}

\newpage
\subsection*{Monday: Pumping Lemma}

%! app: Regular Languages
%! outcome: Classify language, Find example languages, Pumping Lemma

{\bf Definition and Theorem}: For an alphabet $\Sigma$, a language $L$ over $\Sigma$ is called {\bf regular}
exactly when $L$ is recognized by some DFA, which happens exactly when $L$ is recognized by some NFA, 
and happens exactly when $L$ is described by some regular expression

{\bf We saw that}: The class of regular languages is closed under complementation, union, 
intersection, set-wise concatenation, and Kleene star.

{\it Extra practice}: 

{\bf Disprove}: There is some alphabet $\Sigma$ for which there is 
some language recognized by an NFA but not by any DFA.

\vfill

{\bf Disprove}: There is some alphabet $\Sigma$ for which there is 
some finite language not described by any regular expression over $\Sigma$.

\vfill

{\bf Disprove}: If a language is recognized by an NFA 
then the complement of this language is not recognized by any DFA.

\vfill


{\bf Fix alphabet $\Sigma$. Is every language $L$ over $\Sigma$ regular?}

\begin{center}
\begin{tabular}{c|c}
Set & Cardinality \\
\hline
& \\
$\{0,1\}$ & \\
& \\
$\{0,1\}^*$ & \\
& \\
$\mathcal{P}( \{0,1\})$ & \\
& \\
The set of all languages over $\{0,1\}$ & \\
& \\
The set of all regular expressions over $\{0,1\}$ & \\
& \\
The set of all regular languages over $\{0,1\}$ & \\
& \\
\end{tabular}
\end{center}

\newpage

Strategy: Find an {\bf invariant} property that is true of all regular languages. When analyzing 
a given language, if the invariant is not true about it, then the language is not regular.

\vfill

{\bf Pumping Lemma} (Sipser Theorem 1.70): If $A$ is a regular language, then there
is a number $p$ (a {\it pumping length}) where, if $s$ is any string in $A$ of length at least $p$, 
then $s$ may be divided into three pieces, $s = xyz$ such that
\vspace{-10pt}
\begin{itemize}
\item $|y| > 0$
\item for each $i \geq 0$, $xy^i z \in A$
\item $|xy| \leq p$.
\end{itemize}
\vfill

{\bf Proof idea}: In DFA, the only memory available is in the states. 
Automata can only
``remember'' finitely far in the past and finitely much information, because
they can have only finitely many states. If a computation path of a DFA visits 
the same state more than once, the machine can't tell the difference between 
the first time and future times it visits this state. Thus, if 
a DFA accepts one long string, then it must accept (infinitely) many 
similar strings.
\vfill

{\bf Proof illustration}


\vfill
\vfill


\newpage

{\bf True or False}: A pumping length for $A = \{ 0,1 \}^*$ is $p = 5$.

\vfill


{\bf True or False}: A pumping length for $A = \{ 0,1 \}^*$ is $p = 2$.

\vfill



{\bf True or False}: A pumping length for $A = \{ 0,1 \}^*$ is $p = 105$.

\vfill



Restating {\bf Pumping Lemma}: If $L$ is a regular language, then it  has
a pumping length.


{\bf Contrapositive}: If $L$ has no pumping length, then  it is nonregular.

\vfill

{\Large The Pumping Lemma {\it cannot} be used to prove that a language {\it is} regular.} 

{\Large The Pumping Lemma {\bf can} be used to prove that a language {\it is not} regular.}

{\it Extra practice}: Exercise 1.49 in the book.


\vfill

{\bf Proof strategy}: To prove that a language $L$ is {\bf not} regular, 
\begin{itemize}
    \item Consider an arbitrary positive integer $p$
    \item Prove that $p$ is not a pumping length for $L$
    \item Conclude that $L$ does not have {\it any} pumping length, and therefore it is not regular.
\end{itemize}


{\bf Negation}: A positive integer  $p$  is {\bf not a pumping length} of a language  $L$ over  $\Sigma$  iff
\[
\exists s \left(~  |s| \geq  p \wedge s \in L \wedge \forall x \forall y \forall z  \left( ~\left( s = xyz \wedge 
|y| > 0 \wedge |xy| \leq p~ \right) \to \exists i  (  i \geq 0  \wedge xy^iz  \notin L ) \right) ~\right) 
\]

\newpage
\subsection*{Wednesday: Proving nonregularity, and beyond}

\input{../activity-snippets/day10.tex}
    
\newpage
\subsection*{Friday: Pushdown Automata}

\input{../activity-snippets/day11.tex}


\end{document}