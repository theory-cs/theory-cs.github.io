
\input{../../resources/lesson-head.tex}

\section*{Monday: Time Complexity}

%! app: Decidable Languages, Undecidable Languages
%! outcome: Classify language, Classify decision problem, Reduction, Nondeterminism


In practice, computers (and Turing machines) don't have infinite tape, 
and we can't afford to wait unboundedly long for an answer.
``Decidable" isn't good enough - we want ``Efficiently decidable".

For a given algorithm working on a given input, how long do we need to wait for an answer? 
How does the running time depend on the input in the worst-case? average-case? 
We expect to have to spend more time on computations with larger inputs.


A language is {\bf recognizable} if \underline{\phantom{\hspace{4.5in}}}

A language is {\bf decidable} if \underline{\phantom{\hspace{4.7in}}}

A language is {\bf efficiently  decidable} if \underline{\phantom{\hspace{4in}}}

A function is {\bf computable} if \underline{\phantom{\hspace{4.7in}}}

A function is {\bf efficiently computable} if \underline{\phantom{\hspace{4in}}}\\

\vfill
\newpage

Definition  (Sipser 7.1): For  $M$ a deterministic decider, its {\bf running time} is the function  $f: \mathbb{N} \to \mathbb{N}$
given  by
\[
f(n) =  \text{max number of  steps $M$ takes before halting, over all inputs  of length $n$}
\]

Definition (Sipser 7.7): For each function $t(n)$, the {\bf time complexity class}  $TIME(t(n))$, is defined  by
\[
TIME( t(n)) = \{ L \mid \text{$L$ is decidable by  a Turing machine with running time in  $O(t(n))$} \}
\]

An example of an element of  $TIME(  1  )$ is 

An example of an element of  $TIME(  n  )$ is 


Note: $TIME( 1) \subseteq TIME (n)  \subseteq TIME(n^2)$

\vfill

Definition (Sipser 7.12) : $P$ is the class of languages that  are decidable in polynomial time on 
a deterministic 1-tape  Turing  machine
\[
P  =  \bigcup_k TIME(n^k)
\]


\vfill

Theorem (Sipser 7.8): Let $t(n)$ be a  function with  $t(n)  \geq n$.  Then every $t(n)$ time deterministic 
multitape Turing machine has an equivalent $O(t^2(n))$ time deterministic 1-tape Turing machine.

\vfill

\newpage

\textcolor{gray}{Definitions (Sipser 7.1, 7.7, 7.12): For  $M$ a deterministic decider, its {\bf running time} is the function  $f: \mathbb{N} \to \mathbb{N}$
given  by
\[
f(n) =  \text{max number of  steps $M$ takes before halting, over all inputs  of length $n$}
\]
For each function $t(n)$, the {\bf time complexity class}  $TIME(t(n))$, is defined  by
\[
TIME( t(n)) = \{ L \mid \text{$L$ is decidable by  a Turing machine with running time in  $O(t(n))$} \}
\]
$P$ is the class of languages that  are decidable in polynomial time on 
a deterministic 1-tape  Turing  machine
\[
P  =  \bigcup_k TIME(n^k)
\]}

Definition (Sipser  7.9): For $N$ a nodeterministic decider.  
The {\bf running time} of $N$ is the function $f: \mathbb{N} \to \mathbb{N}$ given  by
\[
f(n) =  \text{max number of  steps $N$ takes on  any branch before halting, over all inputs  of length $n$}
\]

\vfill

Definition (Sipser 7.21): For each function $t(n)$, the {\bf nondeterministic time complexity class}  
$NTIME(t(n))$, is defined  by
\[
NTIME( t(n)) = \{ L \mid \text{$L$ is decidable by a nondeterministic Turing machine with running time in $O(t(n))$} \}
\]

\vfill

\[
NP = \bigcup_k NTIME(n^k)
\]

\vfill

{\bf True} or {\bf False}: $TIME(n^2) \subseteq NTIME(n^2)$

\vfill

{\bf True} or {\bf False}: $NTIME(n^2) \subseteq TIME(n^2)$

\vfill

{\bf Every problem in NP is decidable with an exponential-time algorithm}

Nondeterministic approach: guess a possible solution, verify that it works.

Brute-force (worst-case exponential time) approach: iterate over all possible solutions, for each 
one, check if it works.



%\vfill
%
    
\newpage
\subsection*{Wednesday: P vs. NP}

%! app: Decidable Languages, Undecidable Languages
%! outcome: Classify language, Classify decision problem, Reduction, Nondeterminism
 

Definition (Sipser 7.29) Language  $A$ is {\bf polynomial-time mapping reducible} to language $B$, written $A \leq_P B$,
means there is a polynomial-time computable function $f: \Sigma^* \to \Sigma^*$  such that for every $x \in \Sigma^*$
\[
x \in A \qquad \text{iff} \qquad f(x) \in B.
\]
The  function $f$ is  called the  polynomial time reduction of $A$ to $B$.

{\bf  Theorem}  (Sipser 7.31):  If $A \leq_P B$ and $B  \in P$ then $A \in P$.

Proof: 

\vfill

Definition (Sipser 7.34; based in Stephen Cook and Leonid Levin's work in the 1970s): 
A language $B$ is {\bf  NP-complete} means (1) $B$ is in NP {\bf and}  (2) every language
$A$ in $NP$ is polynomial time reducible to $B$.

{\bf  Theorem}  (Sipser 7.35):  If $B$ is NP-complete and $B \in P$ then $P = NP$.

Proof: 

\vfill

\newpage

{\bf 3SAT}: A literal is a Boolean variable (e.g.  $x$) or a negated Boolean variable (e.g.  $\bar{x}$).  
A Boolean formula is a {\bf  3cnf-formula} if it is a Boolean formula in conjunctive normal form (a conjunction  
of  disjunctive clauses of literals) and each clause  has  three literals.
\[
3SAT  = \{  \langle  \phi \rangle \mid \text{$\phi$ is  a  satisfiable 3cnf-formula} \}
\]


Example string  in $3SAT$
\[
   \langle (x \vee \bar{y} \vee {\bar z}) \wedge (\bar{x}  \vee y  \vee  z) \wedge (x \vee y  \vee z) \rangle
\]



Example  string not  in $3SAT$
\[
   \langle (x \vee y \vee z) \wedge 
    (x \vee y \vee{\bar z}) \wedge
    (x \vee \bar{y} \vee z) \wedge
    (x \vee \bar{y} \vee \bar{z}) \wedge
    (\bar{x} \vee y \vee z) \wedge
    (\bar{x} \vee y \vee{\bar z}) \wedge
    (\bar{x} \vee \bar{y} \vee z) \wedge
    (\bar{x} \vee \bar{y} \vee \bar{z}) \rangle
\]



{\bf Cook-Levin Theorem}: $3SAT$ is $NP$-complete.


{\it Are there other $NP$-complete problems?} To prove that $X$ is $NP$-complete
\begin{itemize}
\item {\it From scratch}: prove $X$ is in $NP$ and that all $NP$ problems are polynomial-time
reducible to $X$.
\item {\it Using reduction}: prove $X$ is in $NP$ and that a known-to-be $NP$-complete problem 
is polynomial-time reducible to $X$.
\end{itemize}

\vfill
\vfill


\newpage

{\bf CLIQUE}: A {\bf $k$-clique} in an undirected graph is a maximally connected subgraph with $k$  nodes.
\[
CLIQUE  = \{  \langle G, k \rangle \mid \text{$G$ is an  undirected graph with  a $k$-clique} \}
\]


Example string  in $CLIQUE$

\vfill

Example  string not  in $CLIQUE$

\vfill

Theorem (Sipser 7.32):
\[
3SAT  \leq_P CLIQUE
\]

Given a Boolean formula in conjunctive normal form with $k$ clauses and three literals per clause, we will 
map it to a graph so that the graph has a clique if the original formula is satisfiable and the 
graph does not have a clique if the original formula is not satisfiable.

The graph has $3k$ vertices (one for each literal in each clause) and an edge between all vertices except
\begin{itemize}
    \item vertices for two literals in the same clause
    \item vertices for literals that are negations of one another
\end{itemize}

Example: $(x \vee \bar{y} \vee {\bar z}) \wedge (\bar{x}  \vee y  \vee  z) \wedge (x \vee y  \vee z)$

\vfill

\vfill
\vfill
\newpage



\newpage
\subsection*{Friday: Review}

%! app: Regular Languages, Context-free Languages, Decidable Languages, Undecidable Languages, Complexity
%! outcome: Classify language, Classify decision problem, Reduction, Nondeterminism
 
{\bf NP-Complete Problems}

{\bf 3SAT}: A literal is a Boolean variable (e.g.  $x$) or a negated Boolean variable (e.g.  $\bar{x}$).  
A Boolean formula is a {\bf  3cnf-formula} if it is a Boolean formula in conjunctive normal form (a conjunction  
of  disjunctive clauses of literals) and each clause  has  three literals.
\[
3SAT  = \{  \langle  \phi \rangle \mid \text{$\phi$ is  a  satisfiable 3cnf-formula} \}
\]


Example string  in $3SAT$
\[
   \langle (x \vee \bar{y} \vee {\bar z}) \wedge (\bar{x}  \vee y  \vee  z) \wedge (x \vee y  \vee z) \rangle
\]



Example  string not  in $3SAT$
\[
   \langle (x \vee y \vee z) \wedge 
    (x \vee y \vee{\bar z}) \wedge
    (x \vee \bar{y} \vee z) \wedge
    (x \vee \bar{y} \vee \bar{z}) \wedge
    (\bar{x} \vee y \vee z) \wedge
    (\bar{x} \vee y \vee{\bar z}) \wedge
    (\bar{x} \vee \bar{y} \vee z) \wedge
    (\bar{x} \vee \bar{y} \vee \bar{z}) \rangle
\]



{\bf Cook-Levin Theorem}: $3SAT$ is $NP$-complete.


{\it Are there other $NP$-complete problems?} To prove that $X$ is $NP$-complete
\begin{itemize}
\item {\it From scratch}: prove $X$ is in $NP$ and that all $NP$ problems are polynomial-time
reducible to $X$.
\item {\it Using reduction}: prove $X$ is in $NP$ and that a known-to-be $NP$-complete problem 
is polynomial-time reducible to $X$.
\end{itemize}

\vfill
\vfill


\newpage

{\bf CLIQUE}: A {\bf $k$-clique} in an undirected graph is a maximally connected subgraph with $k$  nodes.
\[
CLIQUE  = \{  \langle G, k \rangle \mid \text{$G$ is an  undirected graph with  a $k$-clique} \}
\]


Example string  in $CLIQUE$

\vfill

Example  string not  in $CLIQUE$

\vfill

Theorem (Sipser 7.32):
\[
3SAT  \leq_P CLIQUE
\]

Given a Boolean formula in conjunctive normal form with $k$ clauses and three literals per clause, we will 
map it to a graph so that the graph has a clique if the original formula is satisfiable and the 
graph does not have a clique if the original formula is not satisfiable.

The graph has $3k$ vertices (one for each literal in each clause) and an edge between all vertices except
\begin{itemize}
    \item vertices for two literals in the same clause
    \item vertices for literals that are negations of one another
\end{itemize}

Example: $(x \vee \bar{y} \vee {\bar z}) \wedge (\bar{x}  \vee y  \vee  z) \wedge (x \vee y  \vee z)$

\vfill

\vfill
\vfill
\newpage

\begin{center}
    \begin{tabular}{|p{4in}|p{3.5in}|}
        \hline
        & \\
        {\bf Model of Computation} & {\bf Class of Languages}\\
        &\\
        \hline
        & \\
        {\bf Deterministic finite automata}:
        formal definition, how to design for a given language, 
        how to describe language of a machine?
        {\bf Nondeterministic finite automata}:
        formal definition, how to design for a given language, 
        how to describe language of a machine?
        {\bf Regular expressions}: formal definition, how to design for a given language, 
        how to describe language of expression?
        {\it Also}: converting between different models. &
        {\bf Class of regular languages}: what are the closure 
        properties of this class? which languages are not in the class?
        using {\bf pumping lemma} to prove nonregularity.\\
        & \\
        \hline
        & \\
        {\bf Push-down automata}:
        formal definition, how to design for a given language, 
        how to describe language of a machine?
        {\bf Context-free grammars}:
        formal definition, how to design for a given language, 
        how to describe language of a grammar? &
        {\bf Class of context-free languages}: what are the closure 
        properties of this class? which languages are not in the class?\\
        & \\
        \hline
        & \\
        Turing machines that always halt in polynomial time
        & $P$ \\
        & \\
        Nondeterministic Turing machines that always halt in polynomial time 
        & $NP$ \\
        & \\
        \hline
        & \\
        {\bf Deciders} (Turing machines that always halt): 
        formal definition, how to design for a given language, 
        how to describe language of a machine? &
        {\bf Class of decidable languages}: what are the closure properties 
        of this class? which languages are not in the class? using diagonalization
        and mapping reduction to show undecidability \\
        & \\
        \hline
        & \\
        {\bf Turing machines}
        formal definition, how to design for a given language, 
        how to describe language of a machine? &
        {\bf Class of recognizable languages}: what are the closure properties 
        of this class? which languages are not in the class? using closure
        and mapping reduction to show unrecognizability \\
        & \\
        \hline
    \end{tabular}
\end{center}

\newpage

{\bf Given a language, prove it is regular}

{\it Strategy 1}: construct DFA recognizing the language and prove it works.

{\it Strategy 2}: construct NFA recognizing the language and prove it works.

{\it Strategy 3}: construct regular expression recognizing the language and prove it works.

{\it ``Prove it works'' means \ldots}

\vspace{100pt}

{\bf Example}: $L  = \{ w \in \{0,1\}^* \mid \textrm{$w$ has odd number of $1$s or starts with $0$}\}$

Using NFA

\vfill

Using regular expressions

\vfill


\newpage

{\bf Example}: Select all and only the options that result in a true statement: ``To show 
a language $A$ is not regular, we can\ldots'' 

\begin{enumerate}
    \item[a.] Show $A$ is finite
    \item[b.] Show there is a CFG generating $A$
    \item[c.] Show $A$ has no pumping length
    \item[d.] Show $A$ is undecidable
\end{enumerate}

\newpage

{\bf Example}: What is the language generated by the CFG with rules
\begin{align*}
    S &\to aSb \mid bY \mid Ya \\
    Y &\to bY \mid Ya \mid \varepsilon 
\end{align*}

\newpage

{\bf Example}: Prove that the language 
$T = \{ \langle M \rangle \mid \textrm{$M$ is a Turing machine and $L(M)$ is infinite}\}$ 
is undecidable.

\newpage

{\bf Example}: Prove that the class of decidable languages is closed under concatenation.


\newpage


\begin{center}
\includegraphics[width=5in]{../../resources/images/wood-951875_960_720.jpeg}
\end{center}



\subsection*{Week 10 at a glance}

\subsubsection*{Textbook reading: Chapter 7}

{\it For Monday}: Definition 7.1 (page 276)

{\it For Wednesday}: Definition 7.7 (page 279)


\subsubsection*{Make sure you can:}
\begin{itemize}
\item Classify the computational complexity of a set of strings by determining whether it is decidable or undecidable and recognizable or unrecognizable.
\begin{itemize}
    \item Distinguish between computability and complexity
    \item Articulate motivating questions of complexity
    \item Define NP-completeness
    \item Give examples of PTIME-decidable, NPTIME-decidable, and NP-complete problems
\end{itemize}
\item Use mapping reduction to deduce the complexity of a language by comparing to the complexity of another.
   \begin{itemize}
      \item Distinguish between computability and complexity
      \item Articulate motivating questions of complexity
      \item Use appropriate reduction (e.g. mapping, Turing, polynomial-time) to deduce the complexity of a language by comparing to the complexity of another.
      \item Use polynomial-time reduction to prove NP-completeness
    \end{itemize}
\end{itemize}

\begin{comment}
\end{comment}

\subsubsection*{TODO:}
\begin{list}
   {\itemsep2pt}
   \item Student Evaluations of Teaching forms: Evaluations are open for completion anytime BEFORE 8AM on Saturday, March 16.
    Access your SETs from the Evaluations site
    \begin{quote}
         \url{https://academicaffairs.ucsd.edu/Modules/Evals}
    \end{quote}
    You will separately evaluate each of your listed instructors for each enrolled course. 

    **NEW** WINTER 2024 SET INCENTIVE LOTTERY: In Winter 2024, students who complete all of their student 
    evaluation forms for their undergraduate course will be entered into a lottery to win one of 
    5 \$100 Visa gift cards! To be entered into the lottery, students must complete at 
    least one instructor evaluation for EACH of their undergraduate courses. 
    They will be automatically entered when they have completed an instructor evaluation for 
    all of their undergraduate courses.

   \item Review quizzes based on class material each day; review quiz for Friday includes opportunity for feedback for course.
   \item Homework assignment 5 due Thursday.
\end{list}


\end{document}
