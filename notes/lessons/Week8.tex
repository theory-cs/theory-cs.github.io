\documentclass[12pt, oneside]{article}

\usepackage[letterpaper, scale=0.89, centering]{geometry}
\usepackage{fancyhdr}
\setlength{\parindent}{0em}
\setlength{\parskip}{1em}

\pagestyle{fancy}
\fancyhf{}
\rfoot{\href{https://creativecommons.org/licenses/by-nc-sa/2.0/}{CC BY-NC-SA 2.0} Version \today~(\thepage.)}

\input{../../resources/CSE20packages}

\begin{document}
\begin{flushright}
    \StrBefore{\currfilename}{.}
\end{flushright}

\subsection*{Week 8 at a glance}


\vspace{-20pt}

\subsubsection*{Textbook reading: Chapter 4, Section 5.3}

\vspace{-20pt}

Before Monday, ``An undecidable language", Sipser pages 207-209.

Before Wednesday, Definition 5.20 and figure 5.21 (page 236) of mapping reduction.

Before Friday, Example 5.24 (page 236).

For Week 9 Monday: Example 5.26 (page 237).

\vspace{-20pt}

\subsubsection*{We will be learning and practicing to:}

\vspace{-20pt}

\begin{itemize}
    \item Clearly and unambiguously communicate computational ideas using appropriate formalism. Translate across levels of abstraction.
    \begin{itemize}
        \item Give examples of sets that are regular, context-free, decidable, or recognizable (and prove that they are).
        \begin{itemize}
          \item {\bf Define and explain the definitions of the computational problem $A_{TM}$}
          \item {\bf Define and explain the definitions of the computational problem $HALT_{TM}$}
       \end{itemize}
    \end{itemize}
    \item Know, select and apply appropriate computing knowledge and problem-solving techniques. Reason about computation and systems.
    \begin{itemize}
        \item Use diagonalization to prove that there are 'hard' languages relative to certain models of computation.
        \begin{itemize}
            \item {\bf Trace the argument that proves $A_{TM}$ is undecidable and explain why it works.}
        \end{itemize}
        \item Use mapping reduction to deduce the complexity of a language by comparing to the complexity of another.
           \begin{itemize}
              \item {\bf Define computable functions, and use them to give mapping reductions between computational problems}
              \item {\bf Build and analyze mapping reductions between computational problems}
              \item {\bf Deduce the decidability or undecidability of a computational problem given mapping reductions between it and other computational problems, or explain when this is not possible.}
           \end{itemize}
    \item Classify the computational complexity of a set of strings by determining whether it is regular, context-free, decidable, or recognizable.
    \begin{itemize}
    \item {\bf State, prove, and use theorems relating decidability, recognizability, and co-recognizability.}
    \item {\bf Prove that a language is decidable or recognizable by defining and analyzing a Turing machines with appropriate properties.}
\end{itemize}
\end{itemize}
\end{itemize}

\vspace{-20pt}

\subsubsection*{TODO:}
\begin{list}{\itemsep-10pt}
   \item Review Quiz 7 on PrairieLearn (http://us.prairielearn.com), due 2/26/2025
   \item Homework 5 submitted via Gradescope (https://www.gradescope.com/), due 2/27/2025
   \item Review Quiz 8 on PrairieLearn (http://us.prairielearn.com), due 3/5/2025
\end{list}

\newpage

\section*{Monday: $A_{TM}$ is recognizable but undecidable}

\input{../activity-snippets/day20.tex}
    
\newpage
\subsection*{Wednesday: Computable functions and mapping reduction}

\input{../activity-snippets/day21.tex}


\newpage
\subsection*{Friday: The Halting problem}

\input{../activity-snippets/day22.tex}
\end{document}
