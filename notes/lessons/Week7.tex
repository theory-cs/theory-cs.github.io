\documentclass[12pt, oneside]{article}

\usepackage[letterpaper, scale=0.89, centering]{geometry}
\usepackage{fancyhdr}
\setlength{\parindent}{0em}
\setlength{\parskip}{1em}

\pagestyle{fancy}
\fancyhf{}
\rfoot{\href{https://creativecommons.org/licenses/by-nc-sa/2.0/}{CC BY-NC-SA 2.0} Version \today~(\thepage.)}

\input{../../resources/CSE20packages}

\begin{document}
\begin{flushright}
    \StrBefore{\currfilename}{.}
\end{flushright}

\subsection*{Week 7 at a glance}

\subsubsection*{Textbook reading: Chapter 4}

\vspace{-20pt}
No class on Monday in observance of UCSD holiday.

Before Wednesday, Introduction to Chapter 4.

Before Friday, Decidable problems concerning regular languages, 
Sipser pages 194-196.

For Week 8 Monday: An undecidable language, Sipser pages 207-209.

\vspace{-20pt}

\subsubsection*{We will be learning and practicing to:}

\vspace{-20pt}

\begin{itemize}
    \item Clearly and unambiguously communicate computational ideas using appropriate formalism. Translate across levels of abstraction.
    \begin{itemize}
        \item Use clear English to describe computations of Turing machines informally.
        \begin{itemize}
                \item {\bf Use high-level descriptions to define and trace Turing machines}
                \item {\bf Apply dovetailing in high-level definitions of machines}
        \end{itemize}
        \item Give examples of sets that are regular, context-free, decidable, or recognizable (and prove that they are).
        \begin{itemize}
          \item {\bf Give examples of sets that are decidable.}
          \item {\bf Give examples of sets that are recognizable.}
       \end{itemize}
    \end{itemize}
    \item Know, select and apply appropriate computing knowledge and problem-solving techniques. Reason about computation and systems.
    \begin{itemize}
        \item Translate a decision problem to a set of strings coding the problem.
        \begin{itemize}
        \item {\bf Connect languages and computational problems}
        \item {\bf Describe and use the encoding of objects as inputs to Turing machines}
        \item {\bf Trace high-level descriptions of algorithms for computational problems}
        \end{itemize}
    \item Classify the computational complexity of a set of strings by determining whether it is regular, context-free, decidable, or recognizable.
    \begin{itemize}
    \item {\bf Describe common computational problems with respect to DFA, NFA, regular expressions, PDA, and context-free grammars.}
    \item {\bf Give high-level descriptions of Turing machines that decide common computational problems with respect to DFA, NFA, regular expressions, PDA, and context-free grammars.}
\end{itemize}
\end{itemize}
\end{itemize}

\vspace{-20pt}

\subsubsection*{TODO:}
\begin{list}{\itemsep-10pt}
    \item Review Quiz 6 on PrairieLearn (http://us.prairielearn.com), due 2/19/2025
    \item Homework 4 submitted via Gradescope (https://www.gradescope.com/), due 2/20/2025
    \item Review Quiz 7 on PrairieLearn (http://us.prairielearn.com), due 2/26/2025
\end{list}

\newpage

\subsection*{Monday: No class, in observance of UCSD holiday}
\subsection*{Wednesday: General constructions for Turing machines}


\input{../activity-snippets/day18.tex}


\newpage
\subsection*{Friday: Decidable problems about regular languages}

\input{../activity-snippets/day19.tex}


\end{document}
