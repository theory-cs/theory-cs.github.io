\documentclass[12pt, oneside]{article}

\usepackage[letterpaper, scale=0.8, centering]{geometry}
\usepackage{fancyhdr}
\setlength{\parindent}{0em}
\setlength{\parskip}{1em}

\pagestyle{fancy}
\fancyhf{}
\renewcommand{\headrulewidth}{0pt}
\rfoot{{\footnotesize Copyright Mia Minnes, 2021, Version \today~(\thepage)}}

\author{CSE20F21}

\newcommand{\instructions}{{\bf For all HW assignments:}

Weekly homework may be done individually or in groups of up to 3 students. 
You may switch HW partners for different HW assignments. 
The lowest HW score will not be included in your overall HW average. 
Please ensure your name(s) and PID(s) are clearly visible on the first page of your homework submission.

All submitted homework for this class must be typed. 
Diagrams may be hand-drawn and scanned and included in the typed document. 
You can use a word processing editor if you like (Microsoft Word, Open Office, Notepad, Vim, Google Docs, etc.) 
but you might find it useful to take this opportunity to learn LaTeX. 
LaTeX is a markup language used widely in computer science and mathematics. 
The homework assignments are typed using LaTeX and you can use the source files 
as templates for typesetting your solutions\footnote{To use this template, copy the source file (extension \texttt{.tex}) 
to your working directory or upload to Overleaf.}.


{\bf Integrity reminders}
\begin{itemize}
\item Problems should be solved together, not divided up between the partners. The homework is
designed to give you practice with the main concepts and techniques of the course, 
while getting to know and learn from your classmates.
\item You may not collaborate on homework with anyone other than your group members.
You may ask questions about the homework in office hours (of the instructor, TAs, and/or tutors) and 
on Piazza (as private notes viewable only to the Instructors).  
You \emph{cannot} use any online resources about the course content other than the class material 
from this quarter -- this is primarily to ensure that we all use consistent notation and
definitions we will use this quarter.
\item Do not share written solutions or partial solutions for homework with 
other students in the class who are not in your group. Doing so would dilute their learning 
experience and detract from their success in the class.
\end{itemize}

}
\usepackage{amssymb,amsmath,pifont,amsfonts,comment,enumerate,enumitem}
\usepackage{currfile,xstring,hyperref,tabularx,graphicx,wasysym}
\usepackage[labelformat=empty]{caption}
\usepackage{xcolor}
\usepackage{multicol,multirow,array,listings,tabularx,lastpage,textcomp,booktabs}

% NOTE(joe): This environment is credit @pnpo (https://tex.stackexchange.com/a/218450)
\lstnewenvironment{algorithm}[1][] %defines the algorithm listing environment
{   
    \lstset{ %this is the stype
        mathescape=true,
        frame=tB,
        numbers=left, 
        numberstyle=\tiny,
        basicstyle=\rmfamily\scriptsize, 
        keywordstyle=\color{black}\bfseries,
        keywords={,procedure, div, for, to, input, output, return, datatype, function, in, if, else, foreach, while, begin, end, }
        numbers=left,
        xleftmargin=.04\textwidth,
        #1
    }
}
{}

\newcommand\abs[1]{\lvert~#1~\rvert}
\newcommand{\st}{\mid}

\newcommand{\cmark}{\ding{51}}
\newcommand{\xmark}{\ding{55}}




\title{HW1CSE105W24: Homework assignment 1}
\date{Due: January 18th at 5pm (no penalty late submission until 8am next morning), via Gradescope}


\begin{document}
\maketitle
\thispagestyle{fancy}

{\bf In this assignment,}

You will practice reading and
applying the definitions of alphabets, strings, languages, Kleene star, and regular expressions.
You will use regular expressions and relate them to languages and finite automata.
You will use precise notation to formally define the state diagram of finite automata,
and you will use clear English to describe computations of finite automata informally.


{\bf Resources}: To review the topics 
for this assignment, see the class material from Week 1.
We will post frequently asked questions and our answers to them in a 
pinned Piazza post.

{\bf Reading and extra practice problems}: Sipser Section 0, 1.3, 1.1.
Chapter 1 exercises 1.1, 1.2, 1.3, 1.18, 1.23.

\instructions

You will submit this assignment via Gradescope
(\href{https://www.gradescope.com}{https://www.gradescope.com}) 
in the assignment called ``hw1CSE105W24''.

{\bf Assigned questions}
\begin{enumerate}[wide, labelwidth=!, labelindent=0pt]
%%%%%%%%%%% PROBLEM 1 %%%%%%%%%%%
\item\gradeCompleteFirst \textbf{Finding examples and edge cases} (12 points):

With $\Sigma_1 = \{0,1\}$ and 
$\Sigma_2 = \{a,b,c,d,e,f,g,h,i,j,k,l,m,n,o,p,q,r,s,t,u,v,w,x,y,z\}$
and $\Gamma = \{0,1,x,y,z\}$

    \begin{enumerate}
    \item Give an example of the shortest string over $\Sigma_1$ that is meaningful to you in some way, and explain 
    why it's meaningful to you.

    \item List all examples of strings of length $1$ over $\Sigma_2$ and explain why your list is exhaustive.

    \item Calculate the number of distinct strings of length 3 over $\Gamma$ and explain your calculation.

    \item With the ordering $x < y < z < 0 < 1$, list the first ten strings over $\Gamma$ in string order.

    \item Give an example of a finite set that is a language over $\Sigma_1$ and over $\Sigma_2$ and over $\Gamma$, 
    or explain why there is no such set.
    
    \item  Give an example of an infinite set that is a language over $\Sigma_1$  and over $\Gamma$, 
    or explain why there is no such set.

    \end{enumerate}

%%%%%%%%%%% PROBLEM 2 %%%%%%%%%%%
\item\gradeComplete \textbf{Regular expressions} (10 points):

    \begin{enumerate}
    \item Give three regular expressions that all describe the set of all strings over $\{a,b\}$ that have 
    even length. Ungraded bonus challenge: Make the expressions as different as possible!

    \item A friend tells you that each regular expression that has a Kleene star ($~^*$) describes an
    infinite language. Are they right? Either help them justify their claim or give a counterexample to disprove it
    and then fix the formula.

    \end{enumerate}

%%%%%%%%%%% PROBLEM 3 %%%%%%%%%%%
\item\gradeCorrectFirst \textbf{Functions over languages} (15 points):

For languages $L_1, L_2$ over the alphabet $\Sigma_1 = \{0,1\}$, we have the 
associated sets of strings
\[
    SUBSTRING(L_1) = \{ w \in \Sigma_1^* ~|~ \text{there exist } a,b \in \Sigma_1^* \text{ such that } awb \in L_1\}
\]
and 
\[
    L_1 \circ L_2 = \{ w \in \Sigma_1^* ~|~ w = uv \text{ for some strings } u \in L_1 \text{ and } v \in L_2 \}
\]
    \begin{enumerate}
    \item Specify an example language $A$ over $\Sigma_1$ such that 
    $A \neq \emptyset$ and yet $\SUBSTRING(A) = \emptyset$, 
    or explain why there is no such example. 
    A complete solution will include either (1) a precise and
    clear description of your example language $A$ 
    and a precise and clear description of
    the result of computing $\SUBSTRING(A)$ using relevant definitions 
    to justify this description and to justify the set equality with 
    $\emptyset$, or (2) a sufficiently general and correct argument
    why there is no such example, referring back to the relevant definitions.

    \item Specify example languages $B, C$ over $\Sigma_1$ such that 
    $B \neq \Sigma_1^*$ and $C \neq \Sigma_1^*$ and yet $B \circ C = \Sigma_1^*$, 
    or explain why there are no such examples. 
    A complete solution will include either (1) a precise and
    clear description of your example languages $B,C$ 
    and a precise and clear description of
    the result of computing $B \circ C$ using relevant definitions 
    to justify this description and to justify the set equality with 
    $\Sigma_1^*$, or (2) a sufficiently general and correct argument
    why there is no such example, referring back to the relevant definitions.

    \item Specify example {\bf finite} languages $L_1, L_2$ over $\Sigma_1$ such that 
    $L_1 \circ L_2 \neq L_1$ but $|L_1 \circ L_2| = |L_1|$, or 
    explain why there are no such examples.
    A complete solution will include either (1) a precise and
    clear description of your example languages $L_1, L_2$ 
    and a precise and clear description of
    the result of computing $L_1 \circ L_2$ using relevant definitions 
    to justify this description and to justify the cardinality claims and set (in)equality claims, 
    or (2) a sufficiently general and correct argument
    why there is no such example, referring back to the relevant definitions.
    \end{enumerate}


%%%%%%%%%%% PROBLEM 4 %%%%%%%%%%%
\item\gradeCorrect \textbf{Finite automata} (13 points):

Consider the finite automaton $(Q, \Sigma, \delta, q_0, F)$ whose state diagram is depicted below
\begin{center}
\begin{tikzpicture}[->,>=stealth',shorten >=1pt, auto, node distance=2cm, semithick]
\tikzstyle{every state}=[text=black, fill=yellow!40]

\node[initial,state, accepting] (q0)          {$q_0$};
\node[state]         (q1) [above right of=q0, xshift=20pt] {$q_1$};
\node[state]         (q2) [below right of=q0, xshift=20pt] {$q_2$};
\node[state]         (q3) [below right of=q1, xshift=20pt] {$q_3$};

\path (q0) edge  [loop above, near start] node {$1$} (q0)
        edge [bend left=20, near end] node {$0$} (q1)
    (q1) edge [bend left=0] node {$0$} (q2)
        edge [bend left=20] node {$1$} (q3)
    (q2) edge [bend left=20] node {$0$} (q0)
        edge [bend right=20] node {$1$} (q3)
    (q3) edge [loop right] node {$0,1$} (q3)
;
\end{tikzpicture}
\end{center}
where $Q = \{q_0, q_1, q_2, q_3\}$, $\Sigma = \{0,1\}$, and $F = \{q_0\}$, and $\delta: Q \times \Sigma \to Q$
is specified by the look-up table
\begin{center}
\begin{tabular}{c|cc}
        & $0$ & $1$ \\
    \hline
  $q_0$ & $q_1$ & $q_0$ \\
  $q_1$ & $q_2$ & $q_3$ \\
  $q_2$ & $q_0$ & $q_3$ \\
  $q_3$ & $q_3$ & $q_3$
\end{tabular}
\end{center}
    \begin{enumerate}
    \item  A friend tries to summarize the transition function with the formula
    \[
        \delta(q_i,x) = \begin{cases}
            q_0 &\text{ when $i=0$ and $x=1$} \\
            q_3 &\text{ when $0<i\leq 3$ and $x=1$}\\
            q_j &\text{ when $j = (i+1) \mod 3$ and $x=0$}
        \end{cases}
    \]
    Are they right? Either help them justify their claim or give a counterexample to disprove it and then 
    fix their formula.

    \item Give a regular expression $R$ so that $L(R)$ is the language 
    recognized by this finite automaton. Justify your answer by referring to the 
    definition of the semantics of regular expressions and computations of finite automata. 
    Include an explanation for why each string in $L(R)$ is accepted by the finite automaton {\it and}
    for why each string not in $L(R)$ is rejected by the finite automaton.

    \item  Keeping the same set of states $Q = \{q_0, q_1, q_2, q_3\}$, alphabet $\Sigma = \{0,1\}$, 
    same start state $q_0$, and same transition 
    function $\delta$, choose a new set of accepting states $F_{new}$ so that the new 
    finite automaton that results accepts at least one string that the original one rejected {\bf and} rejects
    at least one string that the original one accepted, or explain why there is no such choice of $F_{new}$.
    A complete solution will include either (1) a precise and
    clear description of your choice of $F_{new}$
    and a precise and clear the two example strings using relevant definitions 
    to justify them, or (2) a sufficiently general and correct argument
    why there is no such example, referring back to the relevant definitions.

    \end{enumerate}
    
    \end{enumerate}
\end{document}