\documentclass[12pt, oneside]{article}

\usepackage[letterpaper, scale=0.8, centering]{geometry}
\usepackage{fancyhdr}
\setlength{\parindent}{0em}
\setlength{\parskip}{1em}

\pagestyle{fancy}
\fancyhf{}
\renewcommand{\headrulewidth}{0pt}
\rfoot{{\footnotesize Copyright Mia Minnes, 2021, Version \today~(\thepage)}}

\author{CSE20F21}

\newcommand{\instructions}{{\bf For all HW assignments:}

Weekly homework may be done individually or in groups of up to 3 students. 
You may switch HW partners for different HW assignments. 
The lowest HW score will not be included in your overall HW average. 
Please ensure your name(s) and PID(s) are clearly visible on the first page of your homework submission.

All submitted homework for this class must be typed. 
Diagrams may be hand-drawn and scanned and included in the typed document. 
You can use a word processing editor if you like (Microsoft Word, Open Office, Notepad, Vim, Google Docs, etc.) 
but you might find it useful to take this opportunity to learn LaTeX. 
LaTeX is a markup language used widely in computer science and mathematics. 
The homework assignments are typed using LaTeX and you can use the source files 
as templates for typesetting your solutions\footnote{To use this template, copy the source file (extension \texttt{.tex}) 
to your working directory or upload to Overleaf.}.


{\bf Integrity reminders}
\begin{itemize}
\item Problems should be solved together, not divided up between the partners. The homework is
designed to give you practice with the main concepts and techniques of the course, 
while getting to know and learn from your classmates.
\item You may not collaborate on homework with anyone other than your group members.
You may ask questions about the homework in office hours (of the instructor, TAs, and/or tutors) and 
on Piazza (as private notes viewable only to the Instructors).  
You \emph{cannot} use any online resources about the course content other than the class material 
from this quarter -- this is primarily to ensure that we all use consistent notation and
definitions we will use this quarter.
\item Do not share written solutions or partial solutions for homework with 
other students in the class who are not in your group. Doing so would dilute their learning 
experience and detract from their success in the class.
\end{itemize}

}
\usepackage{amssymb,amsmath,pifont,amsfonts,comment,enumerate,enumitem}
\usepackage{currfile,xstring,hyperref,tabularx,graphicx,wasysym}
\usepackage[labelformat=empty]{caption}
\usepackage{xcolor}
\usepackage{multicol,multirow,array,listings,tabularx,lastpage,textcomp,booktabs}

% NOTE(joe): This environment is credit @pnpo (https://tex.stackexchange.com/a/218450)
\lstnewenvironment{algorithm}[1][] %defines the algorithm listing environment
{   
    \lstset{ %this is the stype
        mathescape=true,
        frame=tB,
        numbers=left, 
        numberstyle=\tiny,
        basicstyle=\rmfamily\scriptsize, 
        keywordstyle=\color{black}\bfseries,
        keywords={,procedure, div, for, to, input, output, return, datatype, function, in, if, else, foreach, while, begin, end, }
        numbers=left,
        xleftmargin=.04\textwidth,
        #1
    }
}
{}

\newcommand\abs[1]{\lvert~#1~\rvert}
\newcommand{\st}{\mid}

\newcommand{\cmark}{\ding{51}}
\newcommand{\xmark}{\ding{55}}




\title{HW3 : Nonregular Languages and Pushdown Automata}
\date{Due: April 25th at 5pm (no penalty late submission until 8am next morning), via Gradescope}

\begin{document}
\maketitle
\thispagestyle{fancy}

\textbf{In this assignment:}

You will practice distinguishing between regular and nonregular languages using both 
closure arguments and the pumping lemma.

\textit{Resources}: To review the topics you are working with for this assignment, 
see the class material from Week 2 through Week 4. We will post frequently asked questions 
and our answers to them in a pinned Piazza post.

\textit{Reading and extra practice problems}: Sipser Section 1.4, 2.2. Chapter 1 
exercises 1.29, 1.30. Chapter 1 problems 1.49, 1.50, 1.51.

\textit{Key Concepts:} Pumping lemma, pumping length, regular languages, nonregular languages, 
pushdown automata, stack.

\instructions

You will submit this assignment via Gradescope
(\href{https://www.gradescope.com}{https://www.gradescope.com}) 
in the assignment called ``hw3CSE105Sp23''.

\textbf{Requests from your TAs and tutors}
To help us with grading please 
\begin{itemize}
    \item Start each question on a new page.
    \item Label the start of each solution with {\bf Answer}.
\end{itemize}

\textbf{Assigned questions}

\begin{enumerate} 

%%%%%%%%%%% PROBLEM 1 %%%%%%%%%%%

\item \textbf{Regular or not?} (21 points):\\
Fix $\Sigma = \{0,1\}$ and $\Gamma = \{0,1,2\}$.
For each of the languages listed below, 
prove that it is either regular or nonregular. {\it Note:} You might find it useful to 
explore the definition of each set and 
consider alternate (simpler) ways of stating it.

For each language that is regular, a complete 
solution will include a 
precise definition of a DFA, NFA, or regular 
expression that recognizes or describes it, along 
with a brief justification
of your construction by explaining the role each 
state plays in the machine
or referring back to relevant definitions.

For each language that is nonregular, a complete
solution will use the pumping lemma to prove that it is nonregular, 
including appropriate
justification related to the specific language.

\begin{enumerate}
    \item\gradeCorrectFirst $L_1 = \{0^n x 1^n \mid n \ge 1, x \in \Sigma^*\}$, a language over $\Sigma$.
    \item\gradeCorrect $L_2 = \{0^n 1 x 0 1^n \mid n 
    \ge 1, x \in \Sigma^* \}$, a language over 
    $\Sigma$.
    \item\gradeCorrect 
    Recall that for $L \subseteq \Sigma^*$, we define
    \begin{align*}
    \REP(L) &:= \{ w \in \Gamma^* \mid \text{between every pair of successive $2$'s in $w$ is a string in $L$}\}\\
    &\phantom{:}=\{w \in \Gamma^* \mid \text{for all } v \in \Sigma^* \text{ if } 2v2 \in \SUBSTRING(\{w\})  \text{, then } v \in L\} 
    \end{align*}
    $L_3 = \REP( \{ 0^n 1^n \mid n \geq 1 \} )$, 
    a language over $\Gamma$.
\end{enumerate}


%%%%%%%%%%% PROBLEM 2 %%%%%%%%%%%

\item \textbf{Properties of nonregular languages} (21 points): \\
Prove or disprove each of the following statements. In other words, decide whether 
each statement is true or false and justify your decision. 
Let $\Sigma = \{0,1\}$ and let $\Gamma = \{0,1,2\}$.
\begin{enumerate}
    \item\gradeCorrect For all languages $L, K$ over
    $\Sigma$, if $L$ is nonregular and $K$ is finite, then $L - K$ is nonregular.
    Recall: $L - K = \{ w \in \Sigma^* \mid w \in L \text{ and } w \notin K\}$.
    \item\gradeCorrect Every infinite language over $\Sigma$ 
    where each string in the language has an equal number of $0$'s and 
    $1$'s is nonregular.
    \item\gradeCorrect Recall that for language $K$ over $\Gamma$,
    \[
    \SUBSTRING(K) := \{ w \in \Gamma^* \mid \text{there exist } a,b \in \Gamma^* \text{ such that } awb \in K\}.
    \]
    For every nonregular language $K$ over $\Gamma$, $\SUBSTRING(K)$ is nonregular.
\end{enumerate}

%%%%%%%%%%% PROBLEM 3 %%%%%%%%%%%
\item \textbf{Pumping dilemma} (8 points): \\
Your friend claims that the Pumping Lemma is useless for proving that an infinite language 
$K \subseteq \Sigma^*$ is not regular. Their logic goes like this

\begin{enumerate}[label=(Step~\arabic*), leftmargin=2cm]
\item[(Step 1)] Suppose that $K$ is regular. It can be recognized by a 
DFA $M = (Q, \Sigma, \delta, q_0, F)$.
\item[(Step 2)] For arbitrary DFA $M$, the pumping length $p$ is at least $|Q|$.
\item[(Step 3)] However, for every integer $n \ge |Q|$, there exists a machine 
$M' = (Q', \Sigma, \delta', q_0', F')$ such that $L(M') = L(M) = K$ and $|Q'| = n$.
\item[(Step 4)] Therefore, the Pumping Lemma cannot be used to pump any string 
of finite length since its pumping length might be arbitrarily large.
\end{enumerate}

Below, we will examine the steps above in detail. Justify your answer to each part.
\begin{enumerate}
    \item\gradeCompleteFirst (Step 1): Is this statement true? In other words, just because we're assuming that $K$ is regular a regular language, does it mean we can assume there is a DFA that recognizes it?

    \item\gradeComplete (Step 2): 
    In general, it's true that the smallest the pumping length of a language
    recognized by a DFA with states $Q$ can be is $|Q|$. Prove this by finding a specific infinite language $K$ and a DFA recognizing 
    where $K$ cannot have pumping length smaller than $|Q|$.

    \item\gradeComplete (Step 3): This step is correct;
    prove the stated version of this statement: For every integer $n \ge |Q|$, there 
    exists a machine $M' = (Q', \Sigma, \delta', q_0', F')$ such that $L(M') = L(M)$ and $|Q'| = n$.

    (\textit{Challenge; not graded}): 
    Define a \emph{cycle} to be a sequence of \emph{distinct} states $q_1, q_2, \ldots, q_m$ such that
	\[
	\delta(q_1, \sigma_1) = q_2, \hspace{1cm} \delta(q_2, \sigma_2) = q_3, \hspace{.5cm} \ldots, \hspace{.5cm}\delta(q_m, \sigma_m) = q_1, 
	\]
	where $\sigma_1, \sigma_2, \ldots, \sigma_m \in \Sigma$ are symbols in the alphabet. 
    An objection to the statement in (Step 3) is that the proof of the Pumping Lemma 
    depends on the length of the cycles in the DFA rather than the number of states. That is, 
    increasing the number of states in your DFA might not increase the pumping length because 
    the length of the smallest cycle stays the same. Nevertheless, a version of your friend's statement 
    is still true whenever you impose this additional cycle constraint:  for every integer $n \ge |Q|$, 
    there exists a machine $M' = (Q', \Sigma, \delta', q_0', F')$ such that $L(M') = L(M)$ and the length 
    of the smallest cycle in the $M'$ is at least $n$. \\

	Your task is to show that even this more general statement is true for the simple language $\Sigma^*$ 
    recognized by the DFA below:
	\begin{center}
    \begin{tikzpicture}[->,>=stealth',shorten >=1pt, auto, node distance=2cm, semithick]
    \tikzstyle{every state}=[text=black, fill=yellow!40]

    \node[initial,state, accepting] (q0) {$q_0$};

    \path (q0)  edge [loop right] node {$0,1$} (q0);
    \end{tikzpicture}
    \end{center}
	For all $n \ge 1$, define a DFA for this language where the length of the smallest cycle is $n$.

    \item\gradeComplete (Step 4): Describe why this statement is true/false/misleading.
\end{enumerate}


%%%%%%%%%%% END PROBLEMS  %%%%%%%%%%%
\end{enumerate}

\end{document}