\documentclass[12pt, oneside]{article}

\usepackage[letterpaper, scale=0.8, centering]{geometry}
\usepackage{fancyhdr}
\setlength{\parindent}{0em}
\setlength{\parskip}{1em}

\pagestyle{fancy}
\fancyhf{}
\renewcommand{\headrulewidth}{0pt}
\rfoot{{\footnotesize Copyright Mia Minnes, 2021, Version \today~(\thepage)}}

\author{CSE20F21}

\newcommand{\instructions}{{\bf For all HW assignments:}

Weekly homework may be done individually or in groups of up to 3 students. 
You may switch HW partners for different HW assignments. 
The lowest HW score will not be included in your overall HW average. 
Please ensure your name(s) and PID(s) are clearly visible on the first page of your homework submission.

All submitted homework for this class must be typed. 
Diagrams may be hand-drawn and scanned and included in the typed document. 
You can use a word processing editor if you like (Microsoft Word, Open Office, Notepad, Vim, Google Docs, etc.) 
but you might find it useful to take this opportunity to learn LaTeX. 
LaTeX is a markup language used widely in computer science and mathematics. 
The homework assignments are typed using LaTeX and you can use the source files 
as templates for typesetting your solutions\footnote{To use this template, copy the source file (extension \texttt{.tex}) 
to your working directory or upload to Overleaf.}.


{\bf Integrity reminders}
\begin{itemize}
\item Problems should be solved together, not divided up between the partners. The homework is
designed to give you practice with the main concepts and techniques of the course, 
while getting to know and learn from your classmates.
\item You may not collaborate on homework with anyone other than your group members.
You may ask questions about the homework in office hours (of the instructor, TAs, and/or tutors) and 
on Piazza (as private notes viewable only to the Instructors).  
You \emph{cannot} use any online resources about the course content other than the class material 
from this quarter -- this is primarily to ensure that we all use consistent notation and
definitions we will use this quarter.
\item Do not share written solutions or partial solutions for homework with 
other students in the class who are not in your group. Doing so would dilute their learning 
experience and detract from their success in the class.
\end{itemize}

}
\usepackage{amssymb,amsmath,pifont,amsfonts,comment,enumerate,enumitem}
\usepackage{currfile,xstring,hyperref,tabularx,graphicx,wasysym}
\usepackage[labelformat=empty]{caption}
\usepackage{xcolor}
\usepackage{multicol,multirow,array,listings,tabularx,lastpage,textcomp,booktabs}

% NOTE(joe): This environment is credit @pnpo (https://tex.stackexchange.com/a/218450)
\lstnewenvironment{algorithm}[1][] %defines the algorithm listing environment
{   
    \lstset{ %this is the stype
        mathescape=true,
        frame=tB,
        numbers=left, 
        numberstyle=\tiny,
        basicstyle=\rmfamily\scriptsize, 
        keywordstyle=\color{black}\bfseries,
        keywords={,procedure, div, for, to, input, output, return, datatype, function, in, if, else, foreach, while, begin, end, }
        numbers=left,
        xleftmargin=.04\textwidth,
        #1
    }
}
{}

\newcommand\abs[1]{\lvert~#1~\rvert}
\newcommand{\st}{\mid}

\newcommand{\cmark}{\ding{51}}
\newcommand{\xmark}{\ding{55}}




\title{HW5CSE105W25: Homework assignment 5}
\date{Due: February 27th at 5pm, via Gradescope}

\begin{document}
\maketitle
\thispagestyle{fancy}

{\bf In this assignment,}
You will practice analyzing, designing, and working with Turing machines.
You will use general constructions and specific machines to explore the classes 
of recognizable and decidable languages. 
You will explore various ways to encode machines as strings so that 
computational problems can be recognized and solved.

{\bf Resources}: To review the topics 
for this assignment, see the class material from Weeks 6, 7, and 8.
We will post frequently asked questions and our answers to them in a 
pinned Piazza post. 

{\bf Reading and extra practice problems}:  
Sipser Chapters 3 and 4.
Chapter 3 exercises 3.1, 3.2, 3.5, 3.8.
Chapter 4 exercises 4.1, 4.2, 4.3, 4.4, 4.5.

\instructions

You will submit this assignment via Gradescope
(\href{https://www.gradescope.com}{https://www.gradescope.com}) 
in the assignment called ``hw5CSE105W25''.

{\bf Assigned questions}
\begin{enumerate}[wide, labelwidth=!, labelindent=0pt]

%%%%%%%%%%% PROBLEM 1 %%%%%%%%%%%
\item\textbf{Equally expressive models} (10 points):
The {\bf  Church-Turing Thesis} (Sipser p.~183) says that the informal notion of algorithm is formalized completely  and correctly by the 
formal definition of a  Turing machine. In other words: all reasonably expressive models of 
computation are equally expressive with the standard Turing machine.
In this question, we will give support for this thesis by showing that 
some adaptations of the standard (Chapter 3) Turing machine model still 
gives us a new model that is equally expressive.

\begin{enumerate}
\item\gradeCompleteFirst Let's define a new machine model, and call it the {\bf May-stay}  machine.
The May-stay machine model is the same as the usual Turing machine model,  except that
on each transition, the tape head may move L, move R, or Stay. 

Formally: a May-stay machine is given by the $7$-tuple
$(Q, \Sigma, \Gamma, \delta, q_0, q_{accept}, q_{reject})$ where 
$Q$ is a finite set with $q_0 \in Q$ and $q_{accept} \in Q$
and $q_{reject} \in Q$ and $q_{accept} \neq q_{reject}$, 
$\Sigma$ and $\Gamma$ are alphabets and $\Sigma \subseteq \Gamma$ and $\square \in \Gamma$ and $\square \notin \Sigma$, and the transition function 
has signature
\[
  \delta: Q \times \Gamma \to Q \times \Gamma \times \{L, R, S\}
\]
The notions of computation and acceptance are analogous to that from Turing machines.

Prove that Turing machines and May-stay machines are equally expressive.
A complete proof will use the formal definitions of the machines.

{\it Hint: Include two directions of implications. First, let $M$ be an 
arbitrary Turing machine and prove that there's a May-stay machine 
that recognizes the language recognized by $M$. Next, let $M_S$ 
be an arbitrary May-stay 
machine and prove that there's a Turing machine that recognizes the language
recognized by $M_S$.}

\item\gradeCorrectFirst Let's define a new machine model, and call it the {\bf Double-move}  machine.
The Double-move machine model is the same as the usual Turing machine model,  except that on each transition, the tape head may move L, move R one cell, or 
move R two cells. Formally: a Double-move machine is given by the $7$-tuple 
$(Q, \Sigma, \Gamma, \delta, q_0, q_{accept}, q_{reject})$ where 
$Q$ is a finite set with $q_0 \in Q$ and $q_{accept} \in Q$
and $q_{reject} \in Q$ and $q_{accept} \neq q_{reject}$, 
$\Sigma$ and $\Gamma$ are alphabets and $\Sigma \subseteq \Gamma$ and $\square \in \Gamma$ and $\square \notin \Sigma$, and the transition function 
has signature
\[
  \delta: Q \times \Gamma \to Q \times \Gamma \times \{L, R, T\}
\]
where $L$ means that the read-write head moves to the left one cell (or stays
put if it's at the leftmost cell already), $R$ means that the read-write head 
moves one cell to the right , and $T$ means that the read-write head moves 
two cells to the right. The notion of computation and acceptance are analogous to that from Turing machines.

Prove that Turing machines and Double-move machines are equally expressive.
A complete proof will use the formal definitions of the machines.

{\it Hint: Include two directions of implications. First, let $M$ be an 
arbitrary Turing machine and prove that there's a Double-move machine 
that recognizes the language recognized by $M$. Next, let $M_D$ 
be an arbitrary Double-move
machine and prove that there's a Turing machine that recognizes the language
recognized by $M_D$.}

\item \gradeComplete In your proofs of equal expressivity in the previous 
parts of this question, you proved that a language is recognizable by some
Turing machine if and only if it is recognizable by some May-stay machine
or by some Double-move machine. Do your proofs also prove that a language is 
decidable by some
Turing machine if and only if it is decidable by some May-stay machine
or by some Double-move machine? Justify your answer.
\end{enumerate}

%%%%%%%%%%% PROBLEM 2 %%%%%%%%%%%
\item\textbf{Modifying machines} (12 points)

\begin{enumerate}
\item\gradeCorrect
Suppose a friend suggests that the following construction can be used
to prove that the class of decidable languages is closed under intersection.

Construction: given deciders $M_1$ and $M_2$ build the following machine $M$
\begin{align*}
    M &= ``\text{On input }w:\\
     &\text{1. Run $M_1$ on input $w$.}\\
     &\text{2. If $M_1$ accepts $w$, accept. } \\
     &\text{3. Run $M_2$ on input $w$.} \\
     &\text{4. If $M_2$ accepts $w$, accept.}\\
     &\text{5. If $M_2$ rejects $w$, reject."}\\
 \end{align*}

 Build a counterexample that could be used to convince your friend that this 
 construction doesn't work. A complete counterexample will include (1) a high-level description of $M_1$, (2) a high-level description of $M_2$, (3) a justification for why they provide a counterexample (that references
 the definitions of $M$ , decidable languages, and intersection).

 {\it Ungraded bonus:} Is it possible to change one line of the construction to make it work?

 \item\gradeCorrect
 Suppose a friend suggests that the following construction can be used
 to prove that the class of recognizable languages is closed under intersection.
 
 Construction: given Turing machines $M_1$ and $M_2$ build the following machine $M'$
\begin{align*}
    M' &= ``\text{On input }w:\\
     &\text{1. Run $M_1$ on input $w$.}\\
     &\text{2. If $M_1$ rejects $w$, reject.} \\
     &\text{3. Run $M_2$ on input $w$.} \\
     &\text{4. If $M_2$ rejects $w$, reject.}\\
 \end{align*}

 Build a counterexample that could be used to convince your friend that this 
 construction doesn't work. A complete counterexample will include (1) a high-level description of $M_1$, (2) a high-level description of $M_2$, (3) a justification for why they provide a counterexample (that references the definition of $M'$, recognizable languages, and intersection).


 {\it Ungraded bonus:} Is it possible to change one line of the construction to make it work?
\end{enumerate}

%%%%%%%%%%% PROBLEM 3 %%%%%%%%%%%
\item\textbf{Closure} (12 points):

For each language $L$ over an alphabet $\Sigma$, we have the 
associated sets of strings (also over $\Sigma$)
\[
    L^* = \{ w_1 \cdots w_k \mid k \geq 0 \textrm{ and each } w_i \in L\}
\]
and
\[
    SUBSTRING(L) = \{ w \in \Sigma^* ~|~ \text{there exist } x,y \in \Sigma^* \text{ such that } xwy \in L\}
\]
and 
\[
    EXTEND(L) = \{ w \in \Sigma^* ~|~ w = uv \text{ for some strings } u \in L \text{ and } v \in \Sigma^* \}
\]

\begin{enumerate}
\item[(a)]\gradeCorrect Prove whether this Turing machine construction below 
{\bf can} or {\bf cannot} be used to prove that the
class of recognizable languages over $\Sigma$ is closed under the 
Kleene star operation or the $SUBSTRING$ operation or the $EXTEND$ operation.

Suppose $M$ is a Turing machine over the alphabet $\Sigma$. 
Let $s_1, s_2, \ldots$ be a list of all strings in 
$\Sigma^*$ in string (shortlex) order.
We define a new Turing machine
by giving its high-level description as follows: 
\begin{align*}
   M_{a} &= ``\text{On input }w:\\
    &\text{1. For $n = 1, 2, \ldots$}\\
    &\text{2.~~~For $j = 1, 2, \ldots n$} \\
    &\text{3.~~~~~~For $k = 1, 2, \ldots, n$} \\
    &\text{4.~~~~~~~~~Run the computation of $M$ on $s_jws_k$ for at most $n$ steps}\\
    &\text{5.~~~~~~~~~If that computation halts and accepts within $n$ steps, accept.}\\
    &\text{6.~~~~~~~~~Otherwise, continue with the next iteration of this inner loop"}\\
\end{align*}

A complete and correct answer will either identify which operation works and give the proof of correctness why, for any Turing machine $M$, $L(M_{a})$ is equal to the result of applying this operation to $L(M$); {\bf or} give a counterexample (a recognizable set $A$ and a Turing machine $M$ recognizing $A$
and a description of why $L(M_a)$ where $M_a$ is the result of the construction 
applied to $M$ doesn't equal $A^*$ and doesn't equal $SUBSTRING(A)$ and doesn't equal $EXTEND(A)$.

\item[(b)]\gradeCorrect Prove whether this Turing machine construction below 
{\bf can} or {\bf cannot} be used to prove that the
class of recognizable languages over $\Sigma$ is closed under the 
Kleene star operation or the $SUBSTRING$ operation or the $EXTEND$ operation.

Suppose $M$ is a Turing machine over the alphabet $\Sigma$. 
Let $s_1, s_2, \ldots$ be a list of all strings in 
$\Sigma^*$ in string (shortlex) order.
We define a new Turing machine
by giving its high-level description as follows: 
\begin{align*}
    M_{b} &= ``\text{On input }w:\\
     &\text{1. For $n = 1, 2, \ldots$}\\
     &\text{2.~~~For $j = 0, 2, \ldots |w|$} \\
     &\text{3.~~~~~~Let $u$ be the string consisting of the first $j$ characters of $w$} \\
     &\text{4.~~~~~~Run the computation of $M$ on $u$ for at most $n$ steps}\\
     &\text{5.~~~~~~If that computation halts and accepts within $n$ steps, accept.}\\
     &\text{6.~~~~~~Otherwise, continue with the next iteration of this inner loop"}\\
 \end{align*}

A complete and correct answer will either identify which operation works and give the proof of correctness why, for any Turing machine $M$, $L(M_{b})$ is equal to the result of applying this operation to $L(M$); {\bf or} give a counterexample (a recognizable set $B$ and a Turing machine $M$ recognizing $B$
and a description of why $L(M_b)$ where $M_b$ is the result of the construction 
applied to $M$ doesn't equal equal $B^*$ and doesn't equal $SUBSTRING(B)$ and doesn't equal $EXTEND(B)$.
\end{enumerate}

%%%%%%%%%%% PROBLEM 4 %%%%%%%%%%%
\item \textbf{Computational problems} (8 points):
Recall the definitions of some example computational problems from class

\hspace{-30pt}
    \begin{tabular}{|lcl|}
    \hline
    \multicolumn{3}{|l|}{{\bf  Acceptance problem} } \\
    & & \\
    \ldots for DFA & $A_{DFA}$ & $\{ \langle B,w \rangle \mid  \text{$B$ is a  DFA that accepts input 
    string $w$}\}$ \\
    \ldots for NFA & $A_{NFA}$ & $\{ \langle B,w \rangle \mid  \text{$B$ is a  NFA that accepts input 
    string $w$}\}$ \\
    \ldots for regular expressions & $A_{REX}$ & $\{ \langle R,w \rangle \mid  \text{$R$ is a  regular
    expression that generates input string $w$}\}$ \\
    \ldots for CFG & $A_{CFG}$ & $\{ \langle G,w \rangle \mid  \text{$G$ is a context-free grammar 
    that generates input string $w$}\}$ \\
    \ldots for PDA & $A_{PDA}$ & $\{ \langle B,w \rangle \mid  \text{$B$ is a PDA that accepts input string $w$}\}$ \\
    & & \\
    \hline
    \multicolumn{3}{|l|}{{\bf Language emptiness  testing} } \\
    & & \\
    \ldots for DFA & $E_{DFA}$ & $\{ \langle A \rangle \mid  \text{$A$ is a  DFA and  $L(A) = \emptyset$\}}$ \\
    \ldots for NFA & $E_{NFA}$ & $\{ \langle A\rangle \mid  \text{$A$ is a NFA and  $L(A) = \emptyset$\}}$ \\
    \ldots for regular expressions & $E_{REX}$ & $\{ \langle R \rangle \mid  \text{$R$ is a  regular
    expression and  $L(R) = \emptyset$\}}$ \\
    \ldots for CFG & $E_{CFG}$ & $\{ \langle G \rangle \mid  \text{$G$ is a context-free grammar 
    and  $L(G) = \emptyset$\}}$ \\
    \ldots for PDA & $E_{PDA}$ & $\{ \langle A \rangle \mid  \text{$A$ is a PDA and  $L(A) = \emptyset$\}}$ \\
    & & \\
    \hline
    \multicolumn{3}{|l|}{{\bf Language equality testing} } \\
    & & \\
    \ldots for DFA & $EQ_{DFA}$ & $\{ \langle A, B \rangle \mid  \text{$A$ and $B$ are DFAs and  $L(A) =L(B)$\}}$\\
    \ldots for NFA & $EQ_{NFA}$ & $\{ \langle A, B \rangle \mid  \text{$A$ and $B$ are NFAs and  $L(A) =L(B)$\}}$\\
    \ldots for regular expressions & $EQ_{REX}$ & $\{ \langle R, R' \rangle \mid  \text{$R$ and $R'$ are regular
    expressions and  $L(R) =L(R')$\}}$\\
    \ldots for CFG & $EQ_{CFG}$ & $\{ \langle G, G' \rangle \mid  \text{$G$ and $G'$ are CFGs and  $L(G) =L(G')$\}}$ \\
    \ldots for PDA & $EQ_{PDA}$ & $\{ \langle A, B \rangle \mid  \text{$A$ and $B$ are PDAs and  $L(A) =L(B)$\}}$ \\
    \hline
    \end{tabular}

\begin{enumerate}
    \item[(a)] \gradeComplete Pick three of the computational problems above and give 
    examples (preferably different from the ones we talked about in class) of strings that are
    in each of the corresponding languages. Remember to use the 
    notation $\langle \cdots \rangle$ to denote the string encoding of relevant objects.
    {\it Extension, not for credit:} Explain why it's hard to write a specific string of 
    $0$s and $1$s and make a claim about membership in one of these sets.
    \item[(b)] \gradeComplete Computational problems can also be defined for Turing machines.
    Consider the two high-level descriptions of Turing machines below.
    Reverse-engineer them to define the computational problem that is being
    recognized, where $L(M_{DFA})$ is the language corresponding to this computational
    problem about DFA and $L(M_{TM})$ is the language corresponding to this computational
    problem about Turing machines. {\it Hint}: the computational problem is not acceptance,
    language emptiness, or language equality (but is related to one of them).

    Let $s_1, s_2, \ldots$ be a list of all strings in 
    $\{0,1\}^*$ in string (shortlex) order. Consider the following Turing machines
    \begin{align*}
        M_{DFA} &= ``\text{On input $\langle D \rangle$ where $D$ is a DFA}:\\
         &\text{1. for $i=1, 2, 3, \ldots$} \\
         &\text{2.~~~ Run $D$ on $s_i$} \\
         &\text{3.~~~~If it accepts, accept.}\\
         &\text{4.~~~~If it rejects, go to the next iteration of the loop"}\\
     \end{align*}
     and
     \begin{align*}
        M_{TM} &= ``\text{On input $\langle T \rangle$ where $T$ is a Turing machine}:\\
         &\text{1. for $i=1, 2, 3, \ldots$} \\
         &\text{2.~~~ Run $T$ for $i$ steps on each input $s_1, s_2, \ldots, s_i$ in turn} \\
         &\text{3.~~~~If $T$ has accepted any of these, accept.}\\
         &\text{4.~~~~Otherwise, go to the next iteration of the loop"}\\
     \end{align*}
\end{enumerate}


%%%%%%%%%%% PROBLEM 5 %%%%%%%%%%%
\item\textbf{Computational problems} (8 points):

\begin{enumerate}
    \item\gradeComplete Prove that the language $$\{\langle D \rangle \mid D \text{ is an NFA over $\{0,1\}$ and $D$ accepts at least $3$ strings 
    of length less than $5$ }\}$$
    is decidable.
    \item\gradeCorrect Prove that the language $$\{\langle R \rangle \mid R \text{ is a regular expression over $\{0,1\}$ and } L(R) \text{ has
    infinitely many strings starting with $0$} \}$$ is decidable.
\end{enumerate}

\end{enumerate}
\end{document}