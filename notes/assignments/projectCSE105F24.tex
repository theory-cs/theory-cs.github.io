\documentclass[12pt, oneside]{article}

\usepackage[letterpaper, scale=0.8, centering]{geometry}
\usepackage{fancyhdr}
\setlength{\parindent}{0em}
\setlength{\parskip}{1em}

\pagestyle{fancy}
\fancyhf{}
\renewcommand{\headrulewidth}{0pt}
\rfoot{{\footnotesize Copyright Mia Minnes, 2025, Version \today~(\thepage)}}

\usepackage{titlesec}

\author{CSE105W25}

\newcommand{\instructions}{{\bf For all HW assignments:} Weekly homework 
may be done individually or in groups of up to 3 students. 
You may switch HW partners for different HW assignments. 
Please ensure your name(s) and PID(s) are clearly visible on the first page of your homework submission 
and then upload the PDF to Gradescope. If working in a group, submit only one submission per group: 
one partner uploads the submission through their Gradescope account and then adds the other group member(s) 
to the Gradescope submission by selecting their name(s) in the ``Add Group Members" dialog box. 
You will need to re-add your group member(s) every time you resubmit a new version of your assignment.
 Each homework question will be graded either for correctness (including clear and precise explanations and 
 justifications of all answers) or fair effort completeness. 
 On the ``graded for correctness"
 questions, you may only collaborate with CSE 105 students in your group; if your
 group has questions about a problem, you may ask in drop-in help hours or post a private
post (visible only to the Instructors) on Piazza. On the "graded for completeness" questions, you 
may collaborate with all other CSE 105 students this quarter, and you may make public posts about these questions 
on Piazza.

All submitted homework for this class must be typed. 
You can use a word processing editor if you like (Microsoft Word, Open Office, Notepad, Vim, Google Docs, etc.) 
but you might find it useful to take this opportunity to learn LaTeX. 
LaTeX is a markup language used widely in computer science and mathematics. 
The homework assignments are typed using LaTeX and you can use the source files 
as templates for typesetting your solutions.
To generate state diagrams of machines, you can (1) use the LaTex tikzpicture
environment (see templates in the class notes), or (2) use the software tools Flap.js or
JFLAP described in the class syllabus (and include a screenshot in your PDF), or (3) you can carefully
and clearly hand-draw
the diagram and take a picture and include it in your PDF.
We recommend that you
submit early drafts to Gradescope so that in case of any technical difficulties, at least some of your
work is present. You may update your submission as many times as you'd like up to the deadline.


{\bf Integrity reminders}
\begin{itemize}
\item Problems should be solved together, not divided up between the partners. The homework is
designed to give you practice with the main concepts and techniques of the course, 
while getting to know and learn from your classmates.
\item On the "graded for correctness"
questions, you may only collaborate with CSE 105 students in your group.
You may ask questions about the homework in office hours (of the instructor, TAs, and/or tutors) and 
on Piazza (as private notes viewable only to the Instructors).  
You \emph{cannot} use any online resources about the course content other than the class material 
from this quarter -- this is primarily to ensure that we all use consistent notation and
definitions (aligned with the textbook) and also to protect the learning experience you will have when
the `aha' moments of solving the problem authentically happen.
\item Do not share written solutions or partial solutions for homework with 
other students in the class who are not in your group. Doing so would dilute their learning 
experience and detract from their success in the class.
\end{itemize}

}

\newcommand{\gradeCorrect}{({\it Graded for correctness}) }
\newcommand{\gradeCorrectFirst}{\gradeCorrect\footnote{This means your solution 
will be evaluated not only on the correctness of your answers, but on your ability
to present your ideas clearly and logically. You should explain how you 
arrived at your conclusions, using
mathematically sound reasoning. Whether you use formal proof techniques or 
write a more informal argument
for why something is true, your answers should always be well-supported. 
Your goal should be to convince the
reader that your results and methods are sound.} }
\newcommand{\gradeComplete}{({\it Graded for completeness}) }
\newcommand{\gradeCompleteFirst}{\gradeComplete\footnote{This means you will 
get full credit so long as your submission demonstrates honest effort to 
answer the question. You will not be penalized for incorrect answers. 
To demonstrate your honest effort in answering the question, we 
expect you to include your attempt to answer *each* part of the question. 
If you get stuck with your attempt, you can still demonstrate 
your effort by explaining where you got stuck and what 
you did to try to get unstuck.} }

\usepackage{tikz}
\usetikzlibrary{automata,positioning,arrows}

\input{../../resources/discrete-math-packages}

\newcommand{\SUBSTRING}{\textsc{Substring}}
\newcommand{\REP}{\textsc{Rep}}
\newcommand{\blank}{\scalebox{1.5}{\textvisiblespace}}

\titleformat{\subsubsection}[runin]
   {\normalfont\bfseries}{}{}{}
   
\title{ProjectCSE105F24: Project}
\date{Due December 11, 2024 at 11am}

\begin{document}
\maketitle

\thispagestyle{fancy}


The CSE 105 project is designed for you to go deeper and 
extend your work on assignments 
and to see how some of the abstract notions we discuss can 
be implemented in concrete ways. 
The project is an individual assignment and has two tasks: 

Task 1: Illustrating the decidability of a computational problem, and

Task 2: Illustrating a mapping reduction

\subsubsection*{What resources can you use?} This project must be completed individually, 
without any help from other people, including the course staff (other than logistics support if 
you get stuck with screencast).
You can use any of this quarter's CSE 105 offering (notes, readings, class videos, homework feedback). 
Tools for drawing state diagrams (like Flap.js and JFLAP and the PrairieLearn automata library) can be used to help draw the diagrams 
in the project too.

These resources should be more than enough.
If you are struggling to get started and want to look elsewhere online, 
you must acknowledge this by listing and citing any resources you consult 
(even if you do not explicitly quote them), including any large-language model style resources (ChatGPT, Bard, Co-Pilot, etc.). 
Link directly to them and include the name of the author / video creator, 
any and all search strings or prompts you used, and the reason you consulted this reference. The work you submit for the project needs to be your own. Again, you shouldn't need to look anywhere 
other than this quarter's material and doing so may result in definitions that
conflict with our conventions in this class so think carefully before you go down this path.

If you get stuck on any part of the project, we encourage you to focus on communicating what you think 
the question might mean, including bringing an example from class or homework you think might be relevant, and 
include any submission any aspect where you're unsure. Clear communication about these
theoretical ideas and their applications is one of the main goals of the project.

{\bf Submitting the project} You will submit a PDF plus a video file for the first task and a PDF plus a video fiile for the 
second task. All file submissions will be in Gradescope.

\newpage
\subsubsection*{Your video:} You may produce screencasts 
with any software you choose. 
One option is to record yourself with Zoom; a tutorial on how to use 
Zoom to record a 
screencast (courtesy of Prof. Joe Politz)  is here: 

\url{https://drive.google.com/open?id=1KROMAQuTCk40zwrEFotlYSJJQdcG_GUU}.

The video that was produced from that recording session in Zoom is here:

\url{https://drive.google.com/open?id=1MxJN6CQcXqIbOekDYMxjh7mTt1TyRVMl}

Please send an email to the instructors 
(minnes@ucsd.edu) if you have 
concerns about  the video / screencast components of this project or 
cannot complete projects in this style for some reason.


\subsubsection*{Reference definitions for computational problems from Section 4.1:}
\begin{center}
   \begin{tabular}{|p{1in}cl|}
   \hline
   \multicolumn{3}{|l|}{{\bf  Acceptance problem} } \\
   & & \\
   \ldots for DFA & $A_{DFA}$ & $\{ \langle B,w \rangle \mid  \text{$B$ is a  DFA that accepts input 
   string $w$}\}$ \\
   \ldots for NFA & $A_{NFA}$ & $\{ \langle B,w \rangle \mid  \text{$B$ is a  NFA that accepts input 
   string $w$}\}$ \\
   \ldots for regular expressions & $A_{REX}$ & $\{ \langle R,w \rangle \mid  \text{$R$ is a  regular
   expression that generates input string $w$}\}$ \\
   \ldots for CFG & $A_{CFG}$ & $\{ \langle G,w \rangle \mid  \text{$G$ is a context-free grammar 
   that generates input string $w$}\}$ \\
   \ldots for PDA & $A_{PDA}$ & $\{ \langle B,w \rangle \mid  \text{$B$ is a PDA that accepts input string $w$}\}$ \\
   & & \\
   \hline
   \multicolumn{3}{|l|}{{\bf Language emptiness  testing} } \\
   & & \\
   \ldots for DFA & $E_{DFA}$ & $\{ \langle A \rangle \mid  \text{$A$ is a  DFA and  $L(A) = \emptyset$\}}$ \\
   \ldots for NFA & $E_{NFA}$ & $\{ \langle A\rangle \mid  \text{$A$ is a NFA and  $L(A) = \emptyset$\}}$ \\
   \ldots for regular expressions & $E_{REX}$ & $\{ \langle R \rangle \mid  \text{$R$ is a  regular
   expression and  $L(R) = \emptyset$\}}$ \\
   \ldots for CFG & $E_{CFG}$ & $\{ \langle G \rangle \mid  \text{$G$ is a context-free grammar 
   and  $L(G) = \emptyset$\}}$ \\
   \ldots for PDA & $E_{PDA}$ & $\{ \langle A \rangle \mid  \text{$A$ is a PDA and  $L(A) = \emptyset$\}}$ \\
   & & \\
   \hline
   \multicolumn{3}{|l|}{{\bf Language equality testing} } \\
   & & \\
   \ldots for DFA & $EQ_{DFA}$ & $\{ \langle A, B \rangle \mid  \text{$A$ and $B$ are DFAs and  $L(A) =L(B)$\}}$\\
   \ldots for NFA & $EQ_{NFA}$ & $\{ \langle A, B \rangle \mid  \text{$A$ and $B$ are NFAs and  $L(A) =L(B)$\}}$\\
   \ldots for regular expressions & $EQ_{REX}$ & $\{ \langle R, R' \rangle \mid  \text{$R$ and $R'$ are regular
   expressions and  $L(R) =L(R')$\}}$\\
   \ldots for CFG & $EQ_{CFG}$ & $\{ \langle G, G' \rangle \mid  \text{$G$ and $G'$ are CFGs and  $L(G) =L(G')$\}}$ \\
   \ldots for PDA & $EQ_{PDA}$ & $\{ \langle A, B \rangle \mid  \text{$A$ and $B$ are PDAs and  $L(A) =L(B)$\}}$ \\
   \hline
   \end{tabular}
   \end{center}


\newpage
\subsubsection*{Task 1: Illustrating the decidability of a computational problem}

Many computational problems are decidable, sometimes using beautiful algorithms.
In this part of the project, you'll choose a 
{\bf decidable} computational problem, and demonstrate the proof that it is 
decidable by building a program in a programming language of your choice (aka a high-level description of a Turing machine) that decides it. 
You will then demonstrate  how your construction works for some test examples.

Specifically:

\vspace{-10pt}

\begin{enumerate}
\item Choose a decidable computational problem from Section 4.1. 
{\it Note:
if you'd like to consider a different computational problem instead, please check with Prof. Minnes first. You must do so no later than the start of Week 9.}
\item Write a program in Java, Python, JavaScript, C++ , or another programming language of your choosing that decides this computational problem.  The function input must be a {\bf string} and part of your work in this program 
is to design string representations for arbitrary instances of the model of 
computation the computational problem you picked is about (e.g. DFA, NFA, regular expressions, CFG, or NFA). The function output must be a {\bf boolean} 
true (if the string is in the set representing the computational problem) or 
false (if the string is not in the set representing the computational problem).
\begin{itemize}
   \item You may use our class notes and the textbook for ideas on the algorithm that your program will implement.
   \item If you would like, you may use aids such as co-pilot or ChatGPT to help you write this program. 
   However, you should test the code that is produced and be able to explain what it is doing. Your code needs to be well-organized and well-documented.
   As a header in your code file, include a comment block describing any resources that were used to 
   help generate your code, including any and all prompts used in interactions 
   with LLM coding tools.
\end{itemize}

\item To demonstrate your program, select one string that is in the 
set representing the computational problem, and one string that is not in the 
set representing the computational problem, explain why these strings are valid
examples, and demonstrate running your program on each to get the appropriate 
output.
\end{enumerate}

Presenting your reasoning and demonstrating it via screenshare are important 
skills that also  show us a lot of your learning. Getting practice with this 
style of presentation is a good thing  for you to learn in general and a rich 
way for us to assess your skills. 
To demonstrate your work, you will create a 3-5 minute screencast video 
explaining your code design and demonstrating its functionality.

\newpage
{\bf Checklist for submission} For this task, you will submit a PDF plus a video file.

\vspace{-10pt}

\begin{itemize}
   \item[(PDF)] Writeup includes a clear specification of computational problem being decided.
   \item[(PDF)] Documentation for program deciding this computational problem: 
   include a description of how input strings are parsed to represent
   instances of the computational problem.
   \item[(PDF)] Clear specification of two example strings, explaining which is in the set (and why) and which is not in the set (and why not).
   \item[(PDF)] Project submission includes a printout of code for program implementing algorithm to decide the computational problem, as well as screen shots demonstrating running your program on your 
   example strings.
   \item[(PDF)] Project writeup is typed or clearly hand drawn with precise language and notation for all terms.
   \item[(Video)] Start with your face and your student ID visible for a few seconds at the beginning, and introduce yourself audibly while on screen. You don't have to be on camera for the  rest of the video, though it's fine if you are. We are looking for a brief confirmation that  it's you creating the video and doing the work you submitted.
\item[(Video)] Present the computational problem you will be working with, and 
example strings that you will be using, including explanations of why you chose this problem and these strings (and why one of the strings is in the set 
and why the other is not).
\item[(Video)] Show on the screen and explain the code for your program, including the software design choices you made
(e.g. which data structures are you using, etc.) and any resources you used. The video 
should clearly describe which programming language was chosen 
for the implementation and gives the reasons why.
\item[(Video)] Demonstrate running your code on each of your example inputs. The video should include screencasts of 
running the code live.
Explain why the output of your program is what you would expect, by connecting the output of the the definition of the computational problem and your chosen parsing of input strings.
\item[(Video)] Logistics: video needs to load correctly, be between 3 and 5 minutes, 
show your face and ID, and you introduce yourself 
audibly while on screen.
\end{itemize}

{\bf Note}: Clarity and brevity are both important aspects of your video.  In previous years, we've seen 
students speed up their videos to get below the 5 minute upper bound. This is ok so long as it doesn't 
compromise clarity. If the graders need to slow your video down to understand
it, it may not earn full credit.


\newpage
\subsubsection*{Task 2: Illustrating a mapping reduction}

We can use mapping reductions to prove that interesting computational 
problems are undecidable, building on 
the undecidability of other computational problems.
In this part of the project, you'll choose a specific {\bf mapping reduction}
and implement a computable function that witnesses it
using a  programming language of your choice (aka a high-level description of a Turing machine that computes it).
You will then demonstrate  how your construction works for some test examples.

Specifically:

\vspace{-10pt}

\begin{enumerate}
\item Choose a mapping reduction we discussed in class or in the homework
or in review quizzes or in the textbook where both sets being compared are 
undecidable. {\it Note:
if you'd like to consider a  mapping reduction we have not discussed instead, 
please check with Prof. Minnes first. 
You must do so no later than the start of Week 9.}
\item Write a program in Java, Python, JavaScript, C++ , or another programming language of your choosing that implements a computable function witnessing this mapping reduction.  The function input must be a {\bf string}  and the function 
output must be a {\bf string}. Part of your work in this program 
is to design string representations for arbitrary instances of the model of 
computation the computational problems being compared in the mapping reduction.
Your function will need to be able to process *any* string as input.
\begin{itemize}
   \item You may use our class notes and the textbook for ideas on the algorithm that your program will implement.
   \item If you would like, you may use aids such as co-pilot or ChatGPT to help you write this program. 
   However, you should test the code that is produced and be able to explain what it is doing. Your code needs to be well-organized and well-documented.
   As a header in your code file, include a comment block describing any resources that were used to 
   help generate your code, including any and all prompts used in interactions 
   with LLM coding tools.
\end{itemize}

\item To demonstrate your program, you will need to run it for an example positive and negative instance. That is to say, if you are implementing 
a computable function witnessing $X \leq_m Y$, you will select one string that is in $X$ and one string that is not in $X$, and you will 
 demonstrate running your program on each of these strings and explain why 
 the output of the function is good.
\end{enumerate}

Presenting your reasoning and demonstrating it via screenshare are important 
skills that also  show us a lot of your learning. Getting practice with this 
style of presentation is a good thing  for you to learn in general and a rich 
way for us to assess your skills. 
To demonstrate your work, you will create a 3-5 minute screencast video 
explaining your code design and demonstrating its functionality.

\newpage
{\bf Checklist for submission} For this task, you will submit a PDF plus a video file.

\vspace{-10pt}

\begin{itemize}
   \item[(PDF)] Writeup includes a clear specification of mapping reduction being witnessed, and both sets in the reduction are undecidable.
   \item[(PDF)] Documentation for program computing the function witnessing this mapping reduction:
   include a description of how input strings are parsed and how output strings correspond to input strings.
   \item[(PDF)] Clear specification of two example strings, explaining which is is a positive instance (and why) and which is a negative instance (and why not).
   \item[(PDF)] Project submission includes a printout of code for program computing the function witnessing the mapping reduction, as well as screen shots demonstrating running your program on your 
   example strings.
   \item[(PDF)] Project writeup is typed or clearly hand drawn with precise language and notation for all terms.
   \item[(Video)] Start with your face and your student ID visible for a few seconds at the beginning, and introduce yourself audibly while on screen. You don't have to be on camera for the  rest of the video, though it's fine if you are. We are looking for a brief confirmation that  it's you creating the video and doing the work you submitted.
\item[(Video)] Present the mapping reduction you will be working with, and 
example strings that you will be using, including explanations of why you chose this reduction and these strings (and why one of the strings is a positive instance and the other is a negative instance).
\item[(Video)] Show on the screen and explain the code for your program, including the software design choices you made
(e.g. which data structures are you using, etc.) and any resources you used. The video 
should clearly describe which programming language was chosen 
for the implementation and gives the reasons why.
\item[(Video)] Demonstrate running your code on each of your example inputs. The video should include screencasts of 
running the code live.
Explain why the output of your program is what you would expect, by connecting the output of the 
program to the definition of the mapping reduction and your chosen parsing of input strings.
\item[(Video)] Logistics: video needs to load correctly, be between 3 and 5 minutes, 
show your face and ID, and you introduce yourself 
audibly while on screen.
\end{itemize}

{\bf Note}: Clarity and brevity are both important aspects of your video.  In previous years, we've seen 
students speed up their videos to get below the 5 minute upper bound. This is ok so long as it doesn't 
compromise clarity. If the graders need to slow your video down to understand
it, it may not earn full credit.



\end{document}