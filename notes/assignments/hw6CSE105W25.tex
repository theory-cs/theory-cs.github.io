\documentclass[12pt, oneside]{article}

\usepackage[letterpaper, scale=0.8, centering]{geometry}
\usepackage{fancyhdr}
\setlength{\parindent}{0em}
\setlength{\parskip}{1em}

\pagestyle{fancy}
\fancyhf{}
\renewcommand{\headrulewidth}{0pt}
\rfoot{{\footnotesize Copyright Mia Minnes, 2021, Version \today~(\thepage)}}

\author{CSE20F21}

\newcommand{\instructions}{{\bf For all HW assignments:}

Weekly homework may be done individually or in groups of up to 3 students. 
You may switch HW partners for different HW assignments. 
The lowest HW score will not be included in your overall HW average. 
Please ensure your name(s) and PID(s) are clearly visible on the first page of your homework submission.

All submitted homework for this class must be typed. 
Diagrams may be hand-drawn and scanned and included in the typed document. 
You can use a word processing editor if you like (Microsoft Word, Open Office, Notepad, Vim, Google Docs, etc.) 
but you might find it useful to take this opportunity to learn LaTeX. 
LaTeX is a markup language used widely in computer science and mathematics. 
The homework assignments are typed using LaTeX and you can use the source files 
as templates for typesetting your solutions\footnote{To use this template, copy the source file (extension \texttt{.tex}) 
to your working directory or upload to Overleaf.}.


{\bf Integrity reminders}
\begin{itemize}
\item Problems should be solved together, not divided up between the partners. The homework is
designed to give you practice with the main concepts and techniques of the course, 
while getting to know and learn from your classmates.
\item You may not collaborate on homework with anyone other than your group members.
You may ask questions about the homework in office hours (of the instructor, TAs, and/or tutors) and 
on Piazza (as private notes viewable only to the Instructors).  
You \emph{cannot} use any online resources about the course content other than the class material 
from this quarter -- this is primarily to ensure that we all use consistent notation and
definitions we will use this quarter.
\item Do not share written solutions or partial solutions for homework with 
other students in the class who are not in your group. Doing so would dilute their learning 
experience and detract from their success in the class.
\end{itemize}

}
\usepackage{amssymb,amsmath,pifont,amsfonts,comment,enumerate,enumitem}
\usepackage{currfile,xstring,hyperref,tabularx,graphicx,wasysym}
\usepackage[labelformat=empty]{caption}
\usepackage{xcolor}
\usepackage{multicol,multirow,array,listings,tabularx,lastpage,textcomp,booktabs}

% NOTE(joe): This environment is credit @pnpo (https://tex.stackexchange.com/a/218450)
\lstnewenvironment{algorithm}[1][] %defines the algorithm listing environment
{   
    \lstset{ %this is the stype
        mathescape=true,
        frame=tB,
        numbers=left, 
        numberstyle=\tiny,
        basicstyle=\rmfamily\scriptsize, 
        keywordstyle=\color{black}\bfseries,
        keywords={,procedure, div, for, to, input, output, return, datatype, function, in, if, else, foreach, while, begin, end, }
        numbers=left,
        xleftmargin=.04\textwidth,
        #1
    }
}
{}

\newcommand\abs[1]{\lvert~#1~\rvert}
\newcommand{\st}{\mid}

\newcommand{\cmark}{\ding{51}}
\newcommand{\xmark}{\ding{55}}




\title{HW6CSE105W25: Homework assignment 6}
\date{Due: March 13, 2025 at 5pm, via Gradescope}


\begin{document}
\maketitle
\thispagestyle{fancy}

{\bf In this assignment,}

You will  practice analyzing, designing, and working with reductions to compare 
the difficulty level of computational problems.
You will explore various ways to encode machines as strings so that 
computational problems can be recognized.

{\bf Resources}: To review the topics 
for this assignment, see the class material from Weeks 8 and 9.
We will post frequently asked questions and our answers to them in a 
pinned Piazza post. 

{\bf Reading and extra practice problems}:  
Sipser Sections 4.2, 5.3, 5.1.
Chapter 4 exercises 4.9, 4.12.
Chapter 5 exercises 5.4, 5.5, 5.6, 5.7. 
Chapter 5 problems 5.22, 5.23, 5.24, 5.28

\instructions

You will submit this assignment via Gradescope
(\href{https://www.gradescope.com}{https://www.gradescope.com}) 
in the assignment called ``hw6CSE105W25''.

{\bf Assigned questions}
\begin{enumerate}[wide, labelwidth=!, labelindent=0pt]

%%%%%%%%%%% PROBLEM 1 %%%%%%%%%%%
\item \textbf{What's wrong with these reductions? (if anything)} (16 points):
Suppose your friends are practicing
coming up with mapping reductions $A \leq_m B$ and their witnessing
functions $f: \Sigma^* \to \Sigma^*$. For each of the following 
attempts, determine if it  has error(s) or is correct.
Do so by labelling each attempt with all and 
only the labels below that apply, and justifying
this labelling.
\begin{itemize}
\item \textit{Error Type 1:} The given function 
can't witness the claimed mapping reduction because there
exists an $x \in A$ such that $f(x) \not\in B$.
\item \textit{Error Type 2:} The given function 
can't witness the claimed mapping reduction because there 
exists an $x \not\in A$ such that $f(x) \in B$.
\item \textit{Error Type 3:} The given function 
can't witness the claimed mapping reduction because the specified
function is not computable.
\item \textit{Correct:} The 
claimed mapping reduction is true and 
is witnessed by the given function.
\end{itemize}

Clearly present your answer by
providing a brief (3-4 sentences or so) justification for 
whether {\bf each} of these labels applies to each example.


\begin{enumerate}
\item\gradeCompleteFirst $A_{\mathrm{TM}} \le_m HALT_{\mathrm{TM}}$ and 
\[
f(x) = \begin{cases}
 \scalebox{.7}{$\langle$ \hspace{-.5cm} \raisebox{-.4cm}{
\begin{tikzpicture}[->,>=stealth',shorten >=1pt, auto, node distance=2cm, semithick]
  \tikzstyle{every state}=[text=black, fill=none]
  \node[initial,state,accepting] (q0)                    {$q_{\mathrm{acc}}$};
 ;
\end{tikzpicture}}
$, \varepsilon \rangle$}  
& \text{if } x = \langle M, w \rangle \text{ for a Turing machine $M$ and string $w$}\\
& \qquad \qquad \text{ and } w \in L(M) \\

\scalebox{.7}{$\langle$ \hspace{-.5cm} \raisebox{-.4cm}{
\begin{tikzpicture}[->,>=stealth',shorten >=1pt, auto, node distance=2cm, semithick]
  \tikzstyle{every state}=[text=black, fill=none]
  \node[initial,state] (q0)                    {$q_0$};
  \node[state,accepting] (qacc) [right of = q0, xshift = 20]{$q_{acc}$};
  \path (q0) edge  [loop above] node {$0, 1, \blank \to R$} (q0)
 ;
\end{tikzpicture}}
$\rangle$} 
& \text{otherwise}
\end{cases} 
\]


\item\gradeComplete $A_{\mathrm{TM}} \le_m EQ_{\mathrm{TM}} $ with 
\[
f(x) = \begin{cases}
 \scalebox{.7}{$\langle$ \hspace{-.5cm} \raisebox{-.4cm}{
\begin{tikzpicture}[->,>=stealth',shorten >=1pt, auto, node distance=2cm, semithick]
  \tikzstyle{every state}=[text=black, fill=none]
  \node[initial,state,accepting] (q0)                    {$q_{\mathrm{acc}}$};
 ;
\end{tikzpicture}}
, $M_w \rangle$}  & \text{if } x = \langle M, w \rangle \text{ for a Turing machine $M$ and string $w$}\\\\
\scalebox{.7}{$\langle$ \hspace{-.5cm} \raisebox{-.4cm}{
    \begin{tikzpicture}[->,>=stealth',shorten >=1pt, auto, node distance=2cm, semithick]
      \tikzstyle{every state}=[text=black, fill=none]
      \node[initial,state,accepting] (q0)                    {$q_{\mathrm{acc}}$};
     ;
    \end{tikzpicture}}}
    , ~~~
    \scalebox{.7}{\hspace{-.5cm} \raisebox{-.4cm}{\begin{tikzpicture}[->,>=stealth',shorten >=1pt, auto, node distance=2cm, semithick]
        \tikzstyle{every state}=[text=black, fill=none]
        \node[initial,state] (qrej)                    {$q_{\mathrm{rej}}$};
        \node[state,accepting] (qacc) [right of=qrej]            {$q_{\mathrm{acc}}$};
       ;
      \end{tikzpicture}}$\rangle$ }  & \text{otherwise}.
\end{cases}
\]
Where for each Turing machine $M$, we  define 
\begin{align*}
    M_w = ``&\text{On input } y \\
    &1. \text{   Simulate $M$ on $w$.}\\
    &2. \text{   If it accepts, accept.}\\
    &3. \text{   If it rejects, reject."}
\end{align*}

\item\gradeCorrectFirst $HALT_{\mathrm{TM}} \le_m EQ_{\mathrm{TM}} $ with 
\[
f(x) = \begin{cases}
 \scalebox{.7}{$\langle$ \hspace{-.5cm} \raisebox{-.4cm}{
\begin{tikzpicture}[->,>=stealth',shorten >=1pt, auto, node distance=2cm, semithick]
  \tikzstyle{every state}=[text=black, fill=none]
  \node[initial,state,accepting] (q0)                    {$q_{\mathrm{acc}}$};
 ;
\end{tikzpicture}}
, $M_w \rangle$}  & \text{if } x = \langle M, w \rangle \text{ for a Turing machine $M$ and string $w$}\\\\
\scalebox{.7}{$\langle$ \hspace{-.5cm} \raisebox{-.4cm}{
    \begin{tikzpicture}[->,>=stealth',shorten >=1pt, auto, node distance=2cm, semithick]
      \tikzstyle{every state}=[text=black, fill=none]
      \node[initial,state,accepting] (q0)                    {$q_{\mathrm{acc}}$};
     ;
    \end{tikzpicture}}}
    , ~~~
    \scalebox{.7}{\hspace{-.5cm} \raisebox{-.4cm}{\begin{tikzpicture}[->,>=stealth',shorten >=1pt, auto, node distance=2cm, semithick]
        \tikzstyle{every state}=[text=black, fill=none]
        \node[initial,state] (qrej)                    {$q_{\mathrm{rej}}$};
        \node[state,accepting] (qacc) [right of=qrej]            {$q_{\mathrm{acc}}$};
       ;
      \end{tikzpicture}}$\rangle$ }  & \text{otherwise}.
\end{cases}
\]
Where for each Turing machine $M$, we  define 
\begin{align*}
    M_w = ``&\text{On input } y \\
    &1. \text{   If } y \text{ is not the empty string, accept.}\\
    &2. \text{   Else, simulate $M$ on $w$.}\\
    &3. \text{   If it accepts, accept.}\\
    &4. \text{   If it rejects, reject."}
\end{align*}

\item\gradeCorrect $\{w w \mid w \in \{0,1\}^* \} \leq \Sigma^*$ and
$f(x) = 11$ for each $x \in \{0,1\}^*$.

\item\gradeCorrect $\Sigma^* \le_m \{w w \mid w \in \{0,1\}^* \}$ and
$f(x) = 11$ for each $x \in \{0,1\}^*$.


\end{enumerate}

%%%%%%%%%%% PROBLEM 2 %%%%%%%%%%%
\item\textbf{Using mapping reductions} (14 points):
Consider the following computational problems we've discussed
\begin{align*}
A_{TM} &= \{ \langle M, w \rangle \mid M \text{ is a Turing machine, } w \text{ is a string and $M$ accepts $w$}\} \\
HALT_{TM} &= \{ \langle M, w \rangle \mid M \text{ is a Turing machine, } w \text{ is a string and $M$ halts on $w$}\} \\
E_{TM} &=  \{ \langle M \rangle \mid M \text{ is a Turing machine and } L(M) = \emptyset\} \\
EQ_{TM} &= \{ \langle M_1, M_2 \rangle \mid M_1, M_2 \text{ are both Turing machines and } L(M_1) = L(M_2) \}
\end{align*}
and the new computational problem
\begin{align*}
    IncludesEmptyString_{TM} &= \{ \langle M \rangle \mid M \text{ is a Turing machine and }\\
    &\text{$M$ accepts the empty string (and maybe other strings too)} \}
\end{align*}

\begin{enumerate}
\item[(a)] \gradeCorrect Give an example of a string that is an element of $IncludesEmptyString_{TM}$ and a string that is not an element of
$IncludesEmptyString_{TM}$ and briefly justify your choices.
\item[(b)] \gradeComplete Prove that $IncludesEmptyString_{TM}$ is not decidable by showing that $A_{TM} \leq_m IncludesEmptyString_{TM}$.
\item[(c)] \gradeCorrect Give a different proof that $IncludesEmptyString_{TM}$ is not decidable by showing that $HALT_{TM} \leq_m IncludesEmptyString_{TM}$.
\item[(d)] \gradeComplete Is $IncludesEmptyString_{TM}$ recognizable? Justify your answer.
\end{enumerate}

%%%%%%%%%%% PROBLEM 3 %%%%%%%%%%%
\item\textbf{Using mapping reductions} (14 points):
Consider the following computational problems we've discussed
\begin{align*}
A_{TM} &= \{ \langle M, w \rangle \mid M \text{ is a Turing machine, } w \text{ is a string and $M$ accepts $w$}\} \\
HALT_{TM} &= \{ \langle M, w \rangle \mid M \text{ is a Turing machine, } w \text{ is a string and $M$ halts on $w$}\} \\
E_{TM} &=  \{ \langle M \rangle \mid M \text{ is a Turing machine and } L(M) = \emptyset\} \\
EQ_{TM} &= \{ \langle M_1, M_2 \rangle \mid M_1, M_2 \text{ are both Turing machines and } L(M_1) = L(M_2) \}
\end{align*}
and the new computational problem
\begin{align*}
    NotIncludesEmptyString_{TM} &= \{ \langle M \rangle \mid M \text{ is a Turing machine and $M$ does not accept the empty string} \}
\end{align*}
\begin{enumerate}
\item[(a)] \gradeCorrect Prove that $NotIncludesEmptyString_{TM}$ is not the complement of $E_{TM}$ and is also not the complement of $IncludesEmptyString_{TM}$.
\item[(b)] \gradeComplete Prove that $NotIncludesEmptyString_{TM}$ is not decidable by showing that $\overline{HALT_{TM}} \leq_m NotIncludesEmptyString_{TM}$.
\item[(c)] \gradeCorrect Give a different proof that $NotIncludesEmptyString_{TM}$ is not decidable by showing that $\overline{A_{TM}} \leq_m NotIncludesEmptyString_{TM}$. 
\item[(d)] \gradeComplete Is $NotIncludesEmptyString_{TM}$ recognizable? Justify your answer.
\end{enumerate}


%%%%%%%%%%% PROBLEM 4 %%%%%%%%%%%
\item \textbf{Examples of languages} (6 points):

For each part of the question, use precise mathematical notation or English to define your examples
and then briefly justify why they work.


For each language $L$ over an alphabet $\Sigma$, we have the 
associated sets of strings (also over $\Sigma$)
\[
    L^* = \{ w_1 \cdots w_k \mid k \geq 0 \textrm{ and each } w_i \in L\}
\]
and
\[
    SUBSTRING(L) = \{ w \in \Sigma^* ~|~ \text{there exist } x,y \in \Sigma^* \text{ such that } xwy \in L\}
\]
and 
\[
    EXTEND(L) = \{ w \in \Sigma^* ~|~ w = uv \text{ for some strings } u \in L \text{ and } v \in \Sigma^* \}
\]

\begin{enumerate}
    \item\gradeCorrect Two undecidable languages $L_1$ and $L_2$ over the same alphabet
        whose union $L_1 \cup L_2$ is co-recognizable, or write {\bf NONE}
        if there is no such example (and explain why).
    
    \item\gradeCorrect An unrecognizable
        language $L_3$ for which $EXTEND(L_3)$ is regular
        or write {\bf NONE} if there is no such example (and explain why).
    
    
    \item\gradeComplete A co-recognizable language $L_4$ that is NP-complete,
         or write {\bf NONE} if there is no such example (and explain why).
        Recall the definition: A language $L$ over an  alphabet $\Sigma$ is called {\bf co-recognizable} if its complement,  defined
        as $\Sigma^* \setminus L  = \{ x  \in  \Sigma^* \mid x \notin  L \}$, is Turing-recognizable.

        {\it This part of the question uses definitions from Week 10 of the course.}
        
\end{enumerate}
    

\end{enumerate}
\end{document}