\documentclass[12pt, oneside]{article}

\usepackage[letterpaper, scale=0.8, centering]{geometry}
\usepackage{fancyhdr}
\setlength{\parindent}{0em}
\setlength{\parskip}{1em}

\pagestyle{fancy}
\fancyhf{}
\renewcommand{\headrulewidth}{0pt}
\rfoot{{\footnotesize Copyright Mia Minnes, 2021, Version \today~(\thepage)}}

\author{CSE20F21}

\newcommand{\instructions}{{\bf For all HW assignments:}

Weekly homework may be done individually or in groups of up to 3 students. 
You may switch HW partners for different HW assignments. 
The lowest HW score will not be included in your overall HW average. 
Please ensure your name(s) and PID(s) are clearly visible on the first page of your homework submission.

All submitted homework for this class must be typed. 
Diagrams may be hand-drawn and scanned and included in the typed document. 
You can use a word processing editor if you like (Microsoft Word, Open Office, Notepad, Vim, Google Docs, etc.) 
but you might find it useful to take this opportunity to learn LaTeX. 
LaTeX is a markup language used widely in computer science and mathematics. 
The homework assignments are typed using LaTeX and you can use the source files 
as templates for typesetting your solutions\footnote{To use this template, copy the source file (extension \texttt{.tex}) 
to your working directory or upload to Overleaf.}.


{\bf Integrity reminders}
\begin{itemize}
\item Problems should be solved together, not divided up between the partners. The homework is
designed to give you practice with the main concepts and techniques of the course, 
while getting to know and learn from your classmates.
\item You may not collaborate on homework with anyone other than your group members.
You may ask questions about the homework in office hours (of the instructor, TAs, and/or tutors) and 
on Piazza (as private notes viewable only to the Instructors).  
You \emph{cannot} use any online resources about the course content other than the class material 
from this quarter -- this is primarily to ensure that we all use consistent notation and
definitions we will use this quarter.
\item Do not share written solutions or partial solutions for homework with 
other students in the class who are not in your group. Doing so would dilute their learning 
experience and detract from their success in the class.
\end{itemize}

}
\usepackage{amssymb,amsmath,pifont,amsfonts,comment,enumerate,enumitem}
\usepackage{currfile,xstring,hyperref,tabularx,graphicx,wasysym}
\usepackage[labelformat=empty]{caption}
\usepackage{xcolor}
\usepackage{multicol,multirow,array,listings,tabularx,lastpage,textcomp,booktabs}

% NOTE(joe): This environment is credit @pnpo (https://tex.stackexchange.com/a/218450)
\lstnewenvironment{algorithm}[1][] %defines the algorithm listing environment
{   
    \lstset{ %this is the stype
        mathescape=true,
        frame=tB,
        numbers=left, 
        numberstyle=\tiny,
        basicstyle=\rmfamily\scriptsize, 
        keywordstyle=\color{black}\bfseries,
        keywords={,procedure, div, for, to, input, output, return, datatype, function, in, if, else, foreach, while, begin, end, }
        numbers=left,
        xleftmargin=.04\textwidth,
        #1
    }
}
{}

\newcommand\abs[1]{\lvert~#1~\rvert}
\newcommand{\st}{\mid}

\newcommand{\cmark}{\ding{51}}
\newcommand{\xmark}{\ding{55}}




\title{HW1 : Regular Expressions and Finite Automata}
\date{Due: April 11th at 5pm (no penalty late submission until 8am next morning), via Gradescope}


\begin{document}
\maketitle
\thispagestyle{fancy}


{\bf In this assignment,}

You will practice reading and
applying the definitions of alphabets, strings, languages, Kleene star, and regular expressions.
You will use regular expressions and relate them to languages and finite automata.
You will use precise notation to formally define the state diagram of finite automata,
and you will use clear English to describe computations of finite automata informally.


{\bf Resources}: To review the topics 
for this assignment, see the class material from Week 1.
We will post frequently asked questions and our answers to them in a 
pinned Piazza post.

{\bf Reading and extra practice problems}: Sipser Section 0, 1.3, 1.1.
Chapter 1 exercises 1.1, 1.2, 1.3, 1.18, 1.23.

\instructions

You will submit this assignment via Gradescope
(\href{https://www.gradescope.com}{https://www.gradescope.com}) 
in the assignment called ``hw1CSE105Sp23''.

{\bf Assigned questions}
%%%%%%%%%%% PROBLEM 1 %%%%%%%%%%%

\begin{enumerate}[wide, labelwidth=!, labelindent=0pt]
    \item \textbf{Functions over sets of strings} (17 points): \\
    For this question, fix the alphabets $\Sigma = \{0,1\}$ and
    $\Gamma = \{0,1,2\}$.

    Whenever $K$ is a set of strings over $\Gamma$ and 
    $L$ is a set of strings over $\Sigma$, 
    we can use the following 
    rules to define associated sets of strings:
    \begin{align*}
    \SUBSTRING(K) &:= \{ w \in \Gamma^* \mid \text{there exist } a,b \in \Gamma^* \text{ such that } awb \in K\} \\
    \REP(L) &:= \{ w \in \Gamma^* \mid \text{between every 
    pair of successive $2$s in $w$ is a string in $L$}\}\\
    &\phantom{:}=\{w \in \Gamma^* \mid \text{for all } v \in \Sigma^* \text{ if } 2v2 \in \SUBSTRING(\{w\})  \text{, then } v \in L\}
    \end{align*}
    \textit{Note:} Formally, $\SUBSTRING$ and $\REP$ are functions
    whose domains and codomains are specified as
    $$\SUBSTRING: \mathcal{P}(\Gamma^*) \to \mathcal{P}(\Gamma^*)$$ 
    and 
    $$\REP: \mathcal{P}(\Sigma^*) \to \mathcal{P}(\Gamma^*)$$
    In other words, $\SUBSTRING$ maps sets of strings with characters $\{0,1,2\}$ to associated
    sets of strings with characters $\{0,1,2\}$; and $\REP$ maps sets of strings with characters 
    in $\{0,1\}$ to associated sets of strings with characters in $\{0,1,2\}$.
    \begin{enumerate}
    \item \gradeCorrectFirst Consider $w = 0120$ (which is a string in $\Gamma^*$). 
    List every element of the set $\SUBSTRING(\{w\})$. 
    In other words, fill in the blank
    \[
    \SUBSTRING(\{w\}) = \{ \underline{\phantom{\hspace{3in}}} \}
    \]
    Briefly justify your answer by referring back to the relevant definitions.

    {\it Not graded, but good to think about: Why do we need the curly braces---``$\{$'' and ``$\}$''---around $w$ for the input to $\SUBSTRING$?}

    \item\gradeCorrect Specify an example language $A$ over $\Gamma$ such that 
    $A \neq \Gamma^*$ and yet $\SUBSTRING(A) = \Gamma^*$, 
    or explain why there is no such example. 
    A complete solution will include either (1) a precise and
    clear description of your example language $A$ 
    and a precise and clear description of
    the result of computing $\SUBSTRING(A)$ using relevant definitions 
    to justify this description and to justify the set equality with 
    $\Gamma^*$, or (2) a sufficiently general and correct argument
    why there is no such example, referring back to the relevant definitions.

    \item\gradeCompleteFirst Define the language $B$ to 
    be the language over $\Sigma$ described by the regular expression 
    \[
    \Sigma^* 1 \Sigma^*
    \] 
    In plain English, we might explain that $B$ is the set of all 
    strings of $0$s and $1$s that contain a $1$. Give a plain English 
    explanation for the set of strings $\REP(B)$.
    
    \item\gradeCorrect Prove/disprove:
    For every finite language $L$ over $\Sigma$, $\REP(L)$ is also a finite
    set of strings.  A complete answer will either give a general
    argument starting with an arbitrary finite language and proving 
    that the result of applying $\REP$ is also finite, or will give a 
    counterexample (which is a specific example of a finite language 
    $L$ for which applying $\REP$ gives an infinite language, with 
    justification referring back to the relevant definitions).

    {\it Note: A finite language is a set of finitely many strings. 
    This includes the possibility that $L$ is the empty set!}

    \item\gradeComplete Write a template for a regular expression that describes $\REP(L)$
    when $L$ is described by a regular expression $R$.
    You may use union, concatenation, Kleene star, and 
    $\Sigma$, $\Gamma$, and $R$.  (We're using the shorthand for regular expressions
    describing alphabets from page 64.)

    \end{enumerate}

%%%%%%%%%%% PROBLEM 2 %%%%%%%%%%%
    \item \textbf{Deciphering regular expressions} (22 points): \\
    For this question, let's fix the regular expression over the alphabet $\{0,1\}$
    \[
    R = 0^* (1 \cup 10)^*
    \]

    For each choice of strings of length $3$, 
    $a, b, c \in \{0,1\}^3$ we can define the regular
    expression:
    \[
    X_{a,b,c} = 0 (a \cup b \cup c)^* 
    \]
    \begin{enumerate}
    \item \gradeComplete Give a plain English explanation for the language 
    described by the regular expression $R$. 
    This continues a theme from Problem~1---before trying to prove formal statements about a specific regular expression, it's often 
    good to try to translate it into a form that is more easy to reason about. 
    Typically speaking, the shorter and more concise your plain English 
    description is, the more useful it will be in reasoning about the language.
        
    \item \gradeCorrect Suppose $a = 000$, $b=001$, $c=011$ so 
    \[
    X_{a,b,c} = 0 ( 000 \cup 001 \cup 011)^*
    \]
    Show that $L(R) \not\subseteq L(X_{a,b,c})$ by giving some string in $L(R)$ 
    which is not in $L(X_{a,b,c})$, and justifying this choice referring back to relevant definitions.

    \item \gradeCorrect More generally, prove that 
    \[
    L(R) \not\subseteq L(X_{a,b,c})
    \]
    for \emph{all} possible strings $a, b, c \in \{0,1\}^3$. 
    Hint: What are the possible lengths of strings in $L(R)$ (and why does this help)?

    \item \gradeCorrect Give 
    a specific example of three distinct strings $a, b, c \in \{0,1,2\}^3$ such that 
    \[
    L(X_{a,b,c}) \subseteq L(R)
    \]
    Briefly justify your answer by 
    explaining how an arbitrary element of $L(X_{a,b,c})$ is guaranteed to be an element of 
    $L(R)$.

    \item \gradeCorrect Give a specific 
    example of three distinct strings $a, b, c \in \{0,1,2\}^3$ such that 
    \[
    L(X_{a,b,c}) \not\subseteq L(R)
    \]
    Briefly justify your 
    answer by giving a counterexample string 
    that is in $L(X_{a,b,c})$ and is not in $L(R)$
    (and explaining why using relevant definitions).
    \end{enumerate}

%%%%%%%%%%% PROBLEM 3 %%%%%%%%%%%
    \item \textbf{The right transition function can make or break a DFA} (6 points): \\
    Consider the finite automaton $(Q, \Sigma, \delta, q_0, F)$ depicted below
    \begin{center}
    \begin{tikzpicture}[->,>=stealth',shorten >=1pt, auto, node distance=2cm, semithick]
    \tikzstyle{every state}=[text=black, fill=yellow!40]

    \node[initial,state, accepting] (q0)                    {$q_0$};
    \node[state]         (q1) [right of=q0] {$q_1$};
    \node[state]         (q2) [right of=q1] {$q_2$};

    \path (q0) edge  [loop above] node {$0$} (q0)
            edge [bend left=20] node {$1$} (q1)
        (q1) edge [bend left=20] node {$0$} (q2)
            edge [bend left=20] node {$1$} (q0)
        (q2) edge [loop above] node {$1$} (q2)
            edge [bend left=20] node {$0$} (q1)
    ;
    \end{tikzpicture}
    \end{center}
    where $Q = \{q_0, q_1, q_2\}$, $\Sigma = \{0,1\}$, and $F = \{q_0\}$. 

    \begin{enumerate}
    \item \gradeComplete Find and fix the mistake in the 
    following symbolic description of the transition function $\delta \colon Q \times \Sigma \to Q$: for each $j \in \{0,1\}$
    \[
    \delta(q_0, j) = q_j \hspace{2cm} \delta(q_1, j) = q_{1-j} \hspace{2cm} \delta(q_2, j) = q_{1+j}
    \]
    \item \gradeCorrect Keeping the same set of states $Q = \{q_0, q_1, q_2\}$, alphabet $\Sigma = \{0,1\}$, starting state $q_0$, and set
    of accepting states $F = \{q_0\}$, change the transition function $\delta$ so that the resulting finite
    automaton recognizes the language 
    described by the regular expression
    \[
        0^* \cup \Sigma^* 1000^*
    \]
    Briefly justify why the resulting finite automaton works 
    by describing the role of each state with your new transition function and relating it to a plain English description 
    of the language described by the regular expression.

    Note: with regular expressions $*$ binds more tightly than concatenation 
    so $1000^* = (100)(0^*)$.

    \end{enumerate}
    {\it (Challenge question, not graded) There is a beautiful plain English description of the language recognized by the finite automaton
    with the state diagram depicted at the start of Problem~3. 
    What is it?}

    %%%%%%%%%%% PROBLEM 4 %%%%%%%%%%%
    \item \textbf{Being precise with terminology} (5 points): \\
    For each of the following statements, determine if 
    it is true, false, or if the question doesn't even 
    make sense (because the statement isn't well formed
    or doesn't use terms in ways consistent with definitions from class).

    \begin{enumerate}
    \item\gradeComplete The empty string is in every language.
    \item\gradeComplete $\Sigma^*$ is a language.
    \item\gradeComplete Every language is a regular expression.
    \item\gradeComplete Alphabets are infinite.
    \item\gradeComplete There is a (finite) number $k \in \mathcal N$ such that every DFA has fewer than $k$ states.
    \end{enumerate}
\end{enumerate}
\newpage

\title{HW2 : Regular Languages and Automata Constructions}
\date{Due: April 18th at 5pm (no penalty late submission until 8am next morning), via Gradescope}


\maketitle
\thispagestyle{fancy}

You will practice designing multiple representations of regular languages and working with general 
constructions of automata to demonstrate the richness of the class of regular languages.

\textit{Resources}: To review the topics you are working with for this assignment, see the 
class material from Week 1 and Week 2. We will post frequently asked questions and our answers to them in a pinned Piazza post.

\textit{Reading and extra practice problems}: Sipser Section 1.1, 1.2, 1.3. 
Chapter 1 exercises 1.4, 1.5, 1.6, 1.7, 1.8, 1.9, 1.10, 1.11, 1.12, 1.14, 1.15, 1.16, 1.17, 1.19, 1.20, 1.21, 1.22.

\textit{Key Concepts:} Regular expressions, language described by a regular expression, 
deterministic finite automata (DFAs), regular languages, closure of the class of regular languages under certain operations, 
nondeterministic finite automata (NFA).

\instructions

You will submit this assignment via Gradescope
(\href{https://www.gradescope.com}{https://www.gradescope.com}) 
in the assignment called ``hw2CSE105Sp23''.

\textbf{Assigned questions}

\begin{enumerate} 

%%%%%%%%%%% PROBLEM 1 %%%%%%%%%%%

\item \textbf{It can be hard to give a good complement} (15 points): \\
For any language $L \subseteq \Sigma^*$, recall that we define its \emph{complement} as 
$$\overline{L} := \Sigma^* - L = \{w \in \Sigma^* \mid w \notin L\}$$ That is, the complement of $L$ 
contains all and only those strings which are not in $L$. Our notation for regular expressions does not 
include the complement symbol. However, 
it turns out that the complement of a language described by a regular expression is guaranteed to also be describable by a 
(different) regular expression. For example, over the alphabet $\Sigma = \{0,1\}$, the complement of the language described 
by the regular expression $\Sigma^* 0$ is described by the regular expression $\varepsilon \cup \Sigma^*1$
because any string that does not end in $0$
must either be the empty string or end in $1$.

For each of the regular expressions $R$ over the alphabet $\Sigma = \{0,1\}$ below, write the regular 
expression for~$\overline{L(R)}$. Your regular expressions may use the symbols
$\varnothing$, $\varepsilon$, $0$, $1$, and the 
following operations to combine them: union, concatenation, 
and Kleene star.

Briefly justify why your solution for each part works by giving plain English descriptions of the language 
described by the regular expression and of its complement and connecting them to the regular 
expression via relevant definitions. An English description that is more 
detailed than simply negating the description in the original language will likely be helpful in the justification.

\begin{enumerate}
    \item\gradeCorrectFirst $(\Sigma \Sigma)^*$
    \item\gradeCorrect $\Sigma^* 11 \Sigma^*$
    \item\gradeCorrect $0^* 1 0^* 1 0^*$
\end{enumerate}

%%%%%%%%%%% PROBLEM 2 %%%%%%%%%%%

\item \textbf{Closure of the class of regular languages under intersection} (12 points): \\
For this question, let $\Sigma = \{0,1\}$.
Recall the DFA over $\Sigma$ from the previous homework:
\begin{center}
\begin{tikzpicture}[->,>=stealth',shorten >=1pt, auto, node distance=2cm, semithick]
  \tikzstyle{every state}=[text=black, fill=yellow!40]

  \node[initial,state, accepting] (q0)                    {$q_0$};
  \node[state]         (q1) [right of=q0] {$q_1$};
  \node[state]         (q2) [right of=q1] {$q_2$};

  \path (q0) edge  [loop above] node {$0$} (q0)
  		edge [bend left=20] node {$1$} (q1)
	(q1) edge [bend left=20] node {$0$} (q2)
		edge [bend left=20] node {$1$} (q0)
	(q2) edge [loop above] node {$1$} (q2)
		edge [bend left=20] node {$0$} (q1)
 ;
\end{tikzpicture}
\end{center}
We'll call the language recognized by the DFA above $A$.
Let's also define a new language $B \subseteq \Sigma^*$ to be 
the language recognized by the DFA 
over $\Sigma$ with state diagram below:
\begin{center}
\begin{tikzpicture}[->,>=stealth',shorten >=1pt, auto, node distance=2cm, semithick]
  \tikzstyle{every state}=[text=black, fill=yellow!40]

  \node[initial,state, accepting] (q0)                    {$r_0$};
  \node[state]         (q1) [right of=q0] {$r_1$};

  \path (q0) edge  [loop above] node {$0$} (q0)
  		edge [bend left=20] node {$1$} (q1)
	(q1) edge [loop above] node {$1$} (q1)
		edge [bend left=20] node {$0$} (q0)
 ;
\end{tikzpicture}
\end{center}

\begin{enumerate}
    \item\gradeCorrect Using the construction for the intersection of two regular languages (Sipser page 46), 
    draw the state diagram for a DFA recognizing the intersection of the languages $A$ and $B$. The labels of 
    each one of your states should be the ordered pair of labels for the states from the two machines above. 
    Your diagram should have $6$ states.
    \item\gradeComplete In this part of the problem, you will prove that the general construction for the DFA 
    recognizing intersection of two languages that 
    you used in part (a) does not always produce a DFA
    with the smallest number of states possible. 
    You will do this by giving one counterexample (that combined with your work in part (a), 
    proves the general claim). Your task: design a DFA with exactly 4 states that recognizes 
    the language $A \cap B$. Briefly justify why your design works by describing the role of each state
    of your DFA and relating it to a plain English description of the language resulting from the intersection.
    \item\gradeCorrect  Later in the class we will learn that there are some languages which are not regular, 
    and in fact, we will learn specific techniques to prove that certain languages are not regular. 
    For the moment, however, we can already investigate closure properties of the class of regular 
    languages just by knowing that a non-regular language exists. 

    We know (from the textbook and our work in class) that if $L$ and $K$ are regular languages, 
    then $L \cap K$ is regular (for arbitrary languages $L$ and $K$). Prove that the converse of this statement is false; that is, 
    give a counterexample by giving a specific regular language $L$ so that
    for each non-regular language $X$, $L \cap X$ is regular (even though $X$ isn't).

    In your solution, justify why $L$ is regular 
    and why $L \cap X$ is regular (for arbitrary $X$) using relevant definitions.
\end{enumerate}


{\it (Challenge question, not graded) Prove/disprove: For 
any language $L$ over $\Sigma^*$, $L \cap B$ is regular implies $L$ is regular, where $B$ is the specific language from part (a) 
and (b) of Problem 2.}

%%%%%%%%%%% PROBLEM 3 %%%%%%%%%%%
\item {\bf Closure of the class of regular languages under} \SUBSTRING~(16 points): \\
Let $\Gamma = \{0,1,2\}$. From the previous homework, recall the function $\SUBSTRING$ that 
has domain and codomain
$\mathcal{P}(\Gamma^*)$, 
where, for each language $K$ over $\Gamma$,
$$\SUBSTRING(K) := \{ w \in \Gamma^* \mid \text{there exist } a,b \in \Gamma^* \text{ such that } awb \in K\}$$
\begin{enumerate}
\item \gradeCorrect Consider the NFA over $\Gamma$ with state diagram:
\begin{center}
\begin{tikzpicture}[->,>=stealth',shorten >=1pt, auto, node distance=2cm, semithick]
  \tikzstyle{every state}=[text=black, fill=yellow!40]

  \node[initial,state, accepting] (q0)                    {$q_0$};
  \node		     (c)   [right of=q0] {};
  \node[state]         (q1) [right of=c] {$q_1$};
  
  \path (q0) edge [bend left=20] node {$0$} (q1)
	(q1) 	edge [bend left=20] node {$1$} (q0)
 ;
\end{tikzpicture}
\end{center}
We'll call the language recognized by the NFA above $C$.
Fill in the blanks below: 
\begin{itemize}
    \item An example of a string over $\Gamma$ that is in $C$
    {\bf and} is in $\SUBSTRING(C)$ is \underline{\phantom{\hspace{1in}}} because 
    \underline{\phantom{\hspace{1in}}}
    \item An example of a string over $\Gamma$ that is in $C$
    {\bf and} is {\bf not} in $\SUBSTRING(C)$ is \underline{\phantom{\hspace{1in}}} because 
    \underline{\phantom{\hspace{1in}}}
    \item An example of a string over $\Gamma$ that is {\bf not} in $C$
    {\bf and} is in $\SUBSTRING(C)$ is \underline{\phantom{\hspace{1in}}} because 
    \underline{\phantom{\hspace{1in}}}
    \item An example of a string over $\Gamma$ that is {\bf not} in $C$
    {\bf and} is {\bf not} in $\SUBSTRING(C)$ is \underline{\phantom{\hspace{1in}}} because 
    \underline{\phantom{\hspace{1in}}}
\end{itemize}
For each item, you'll either fill in a specific string 
and a justification that refers back to the relevant 
definitions, or you'll write ``impossible'' for the first 
part of the sentence and justify why it's impossible to find such 
an example referring back to the relevant definitions.
\item\gradeComplete Prove 
that the class of regular languages is closed under the $\SUBSTRING$ operation. Namely, give a general construction 
that takes an arbitrary NFA and constructs
 an NFA that recognizes the result of applying $\SUBSTRING$
 to the language recognized by the original machine. 
You can describe your construction in words and/or
draw a picture to illustrate your construction.
You do not have to write down a formal specification.

\item\gradeComplete Draw the state diagram of an 
NFA over $\Gamma$ that recognizes $\SUBSTRING(C)$ (for 
$C$ the language from part (a) of this Problem), using
your construction from part (b) of this Problem, or manually
constructing it. 
Describe the computation(s) of this NFA for each of the 
sample strings you gave in part (a).
\end{enumerate}

%%%%%%%%%%% PROBLEM 4 %%%%%%%%%%%
\item {\bf Closure of the class of regular star-free languages under} \REP~(7 points):\\
A language is said to be \emph{star-free} whenever it can be described by a regular expression that 
has no Kleene star operations, but where complement operation can be 
incorporated into the expression as many times as you like. 
For example, the language $$\{\varepsilon, 0010\}$$ is star-free because it can be described
by $\varepsilon \cup 0010$ which does not use the Kleene star operation symbol.
\begin{enumerate}
    \item\gradeCorrect Prove that the set of all strings 
    over $\Gamma = \{0,1,2\}$ is star-free. A complete solution
    will give an expression that describes this language 
    that does not use Kleene star but may incoporate the complement
    expression as many times as you like, along 
    with a justification that refers back to relevant definitions.
    \item\gradeComplete Prove that every finite language is star-free.
    \item\gradeComplete Let $\Sigma = \{0,1\}$. From the previous
    homework, recall the function $\REP$ that has domain 
    $\mathcal{P}(\Sigma^*)$ and codomain $\mathcal{P}(\Gamma^*)$, 
    where, for each language $L$ over $\Sigma$,
    \begin{align*}
        \REP(L) &:= \{ w \in \Gamma^* \mid \text{between every 
    pair of successive $2$'s in $w$ is a string in $L$}\}
    \end{align*}
    
    Show that $\REP(L)$ is a regular and star-free language 
    whenever $L$ is a regular and star-free language. That is, 
    given an expression $R$ describing $L$, write a regular 
    expression for $\REP(L)$ using only the regular expressions 
    $R$, $\varnothing$, $\varepsilon$, $0$, $1$, $2$, and the 
    following operations to combine them: union, concatenation, 
    and complement. You may assume that $\overline{R}$
    describes $\Sigma^* -L(R)$, that is, the complement for the regular expression $R$ over the alphabet $\Sigma$ is itself 
    a language over $\Sigma$. 
\end{enumerate}

%%%%%%%%%%% END PROBLEMS  %%%%%%%%%%%
\end{enumerate}

\newpage

\end{document}