\documentclass[12pt, oneside]{article}

\usepackage[letterpaper, scale=0.8, centering]{geometry}
\usepackage{fancyhdr}
\setlength{\parindent}{0em}
\setlength{\parskip}{1em}

\pagestyle{fancy}
\fancyhf{}
\renewcommand{\headrulewidth}{0pt}
\rfoot{{\footnotesize Copyright Mia Minnes, 2021, Version \today~(\thepage)}}

\author{CSE20F21}

\newcommand{\instructions}{{\bf For all HW assignments:}

Weekly homework may be done individually or in groups of up to 3 students. 
You may switch HW partners for different HW assignments. 
The lowest HW score will not be included in your overall HW average. 
Please ensure your name(s) and PID(s) are clearly visible on the first page of your homework submission.

All submitted homework for this class must be typed. 
Diagrams may be hand-drawn and scanned and included in the typed document. 
You can use a word processing editor if you like (Microsoft Word, Open Office, Notepad, Vim, Google Docs, etc.) 
but you might find it useful to take this opportunity to learn LaTeX. 
LaTeX is a markup language used widely in computer science and mathematics. 
The homework assignments are typed using LaTeX and you can use the source files 
as templates for typesetting your solutions\footnote{To use this template, copy the source file (extension \texttt{.tex}) 
to your working directory or upload to Overleaf.}.


{\bf Integrity reminders}
\begin{itemize}
\item Problems should be solved together, not divided up between the partners. The homework is
designed to give you practice with the main concepts and techniques of the course, 
while getting to know and learn from your classmates.
\item You may not collaborate on homework with anyone other than your group members.
You may ask questions about the homework in office hours (of the instructor, TAs, and/or tutors) and 
on Piazza (as private notes viewable only to the Instructors).  
You \emph{cannot} use any online resources about the course content other than the class material 
from this quarter -- this is primarily to ensure that we all use consistent notation and
definitions we will use this quarter.
\item Do not share written solutions or partial solutions for homework with 
other students in the class who are not in your group. Doing so would dilute their learning 
experience and detract from their success in the class.
\end{itemize}

}
\usepackage{amssymb,amsmath,pifont,amsfonts,comment,enumerate,enumitem}
\usepackage{currfile,xstring,hyperref,tabularx,graphicx,wasysym}
\usepackage[labelformat=empty]{caption}
\usepackage{xcolor}
\usepackage{multicol,multirow,array,listings,tabularx,lastpage,textcomp,booktabs}

% NOTE(joe): This environment is credit @pnpo (https://tex.stackexchange.com/a/218450)
\lstnewenvironment{algorithm}[1][] %defines the algorithm listing environment
{   
    \lstset{ %this is the stype
        mathescape=true,
        frame=tB,
        numbers=left, 
        numberstyle=\tiny,
        basicstyle=\rmfamily\scriptsize, 
        keywordstyle=\color{black}\bfseries,
        keywords={,procedure, div, for, to, input, output, return, datatype, function, in, if, else, foreach, while, begin, end, }
        numbers=left,
        xleftmargin=.04\textwidth,
        #1
    }
}
{}

\newcommand\abs[1]{\lvert~#1~\rvert}
\newcommand{\st}{\mid}

\newcommand{\cmark}{\ding{51}}
\newcommand{\xmark}{\ding{55}}




\title{HW1CSE105W25: Homework assignment 1}
\date{Due: January 16th at 5pm, via Gradescope}


\begin{document}
\maketitle
\thispagestyle{fancy}

{\bf In this assignment,}

You will practice reading and
applying the definitions of alphabets, strings, languages, Kleene star, and regular expressions.
You will use regular expressions and relate them to languages.


{\bf Resources}: To review the topics 
for this assignment, see the class material from Weeks 0 and 1 and Review Quiz 1.
We will post frequently asked questions and our answers to them in a 
pinned Piazza post.

{\bf Reading and extra practice problems}: Sipser Section 0, 1.3.
Chapter 0 exercises 0.1, 0.2, 0.3, 0.5, 0.6, 0.9. Chapter 1 exercises 1.19, 1.23.

\instructions

You will submit this assignment via Gradescope
(\href{https://www.gradescope.com}{https://www.gradescope.com}) 
in the assignment called ``hw1CSE105W25''.

{\bf Assigned questions}


\begin{enumerate}[wide, labelwidth=!, labelindent=0pt]
%%%%%%%%%%% PROBLEM 1 %%%%%%%%%%%
\item\textbf{Strings and languages: finding examples and edge cases} (12 points):
    \begin{enumerate}
    \item\gradeCompleteFirst  Give five (different) example alphabets that are meaningful or useful to you in some way. Specify them formally, either with 
    roster notation (which means listing all and only distict elements between $\{$ and $\}$ and separated by commas) or with another approach to precisely define all and only the elements of the alphabet.

    \item\gradeComplete Give an example of a finite set over an alphabet  and an infinite set over an alphabet. You get to choose the alphabet, and you get to choose the sets.  The goal is to practice communicating your choices and definitions with clear and precise notation. One habit that will be useful (for this course, and beyond), is to think of your response for each question as a well-formed paragraph: include all the information that is relevant so that your solution is self-contained, and so that each sentence is grammatically constructed.
    
    \item\gradeCorrectFirst Define an alphabet $\Sigma_1$ and an alphabet $\Sigma_2$ and a language $L_1$ over $\Sigma_1$ that is also a language over $\Sigma_2$ and a language $L_2$ over $\Sigma_2$ that is {\bf not} a language over $\Sigma_1$. 
    A complete and correct answer will use clear and precise notation
    (consistent with the textbook and class notes) and will include a description of why the given example $L_1$
    is a language over both $\Sigma_1$ and $\Sigma_2$ and a description 
    of why the given example $L_2$ is a language over $\Sigma_2$ and not over $\Sigma_1$.

    \end{enumerate}

%%%%%%%%%%% PROBLEM 2 %%%%%%%%%%%
\item\textbf{Regular expressions} (20 points):

    \begin{enumerate}
    \item\gradeComplete  Give three regular expressions that all describe the set of all strings over $\{a,b\}$ that have 
    odd length. Ungraded bonus challenge: Make the expressions as different as possible!

    \item\gradeComplete  A friend tells you that each regular expression that has a Kleene star ($~^*$) describes an
    infinite language. Are they right? Either help them justify their claim or give a counterexample to disprove it
    and explain your counterexample.

    \item\gradeCorrect For this question, the alphabet is $\{a,b,c\}$. A friend is trying to design a regular expression that describes the set of all strings over this alphabet that end in $c$. Classify each of the following attempts as 
    \begin{itemize}
    \item Correct. Explain why.
    \item Error Type 1: Incorrect, because (even though each string that is in the language described by the regular expression ends in $c$) there is a string that ends in $c$ that is not in the language described by the regular expression. Give this example string and explain why it proves we're in this case. 
    \item Error Type 2: Incorrect, because (even though each string that ends in $c$ is in the language described by the regular expression), there is a string in the language described by the regular expression that does not end in $c$. Give this example string and explain why it proves we're in this case. 
    \item Error Type 3: Incorrect, because there are two counterexample strings, one which is a string that ends in $c$ that is not in the language described by the regular expression and one which is in the language described by the regular expression but does not end in $c$.Give both example strings and describe why each has the given property.
    \end{itemize}

    \vfill
    \newpage

    \hrule
    {\it Worked example for reference:} Consider the regular expression $(a\cup b\cup c)^*$. This regular expression has {\bf Error Type 2} because it describes the set of all strings over $\{a,b,c\}$, so even though each string that ends in $c$ is in this language, there is an example, say $ab$ that is a string in the language described by the regular expression (because we consider the string formed as a result of the Kleene star operation which has 2 slots and where the first slot matches the $a$ in $a \cup b \cup c$ and the second slot matches $b$ in $a \cup b \cup c$) but does not end in $c$ (it ends in $b$).
    \hrule
    \begin{enumerate}
        \item The regular expression is   
        \[
        (a\cup b)^* \circ c
        \]
        \item The regular expression is  
        \[
        (a \circ b \circ c)^*
        \]
        \item The regular expression is  
        \[
        a^*c ~\cup~ b^*c ~\cup~ c^*c
        \]
    \end{enumerate}
    \end{enumerate}

%%%%%%%%%%% PROBLEM 3 %%%%%%%%%%%
\item\textbf{Functions over languages} (18 points):

For each language $L$ over an alphabet $\Sigma$, we have the 
associated sets of strings (also over $\Sigma$)
\[
    L^* = \{ w_1 \cdots w_k \mid k \geq 0 \textrm{ and each } w_i \in L\}
\]
and
\[
    SUBSTRING(L) = \{ w \in \Sigma^* ~|~ \text{there exist } x,y \in \Sigma^* \text{ such that } xwy \in L\}
\]
and 
\[
    EXTEND(L) = \{ w \in \Sigma^* ~|~ w = uv \text{ for some strings } u \in L \text{ and } v \in \Sigma^* \}
\]
Also, recall the set operations union and intersection: for any sets $X$ and $Y$
\[
X \cup Y = \{ w \mid w \in X \text{ or } w \in Y \}
\]
\[
X \cap Y = \{ w \mid w \in X \text{ and } w \in Y \}
\]

    \begin{enumerate}
    \item\gradeComplete Specify an example language $A$ over $\{0,1\}$ such that 
    $$SUBSTRING(A) = EXTEND(A)$$
    or explain why there is no such example. 
    A complete solution will include either (1) a precise and
    clear description of your example language $A$ 
    and a precise and clear description of
    the result of computing $SUBSTRING(A)$, $EXTEND(A)$ (using the given definitions)
    to justify this description and to justify the set equality,
    or (2) a sufficiently general and correct argument
    why there is no such example, referring back to the relevant definitions.

    \item\gradeCorrect Specify an example language $B$ over $\{0,1\}$ such that 
    $$SUBSTRING(B) \cap EXTEND(B) = \{\varepsilon\}$$ and $$SUBSTRING(B) \cup EXTEND(B) = \{0,1\}^*$$
    or explain why there is no such example. 
    A complete solution will include either (1) a precise and
    clear description of your example language $B$ 
    and a precise and clear description of
    the result of computing $SUBSTRING(B)$, $EXTEND(B)$ (using the given definitions)
    to justify this description and to justify the set equality with 
    $\{\varepsilon\}$ and $\{0,1\}^*$ (respectively), or (2) a sufficiently general and correct argument
    why there is no such example, referring back to the relevant definitions.

    \item\gradeCorrect Specify an example {\bf infinite} language $C$ over $\{0,1\}$ such that 
    $$SUBSTRING(C) \neq \{0,1\}^*$$ and $$SUBSTRING(C) = C^*$$or 
    explain why there is no such example.
    A complete solution will include either (1) a precise and
    clear description of your example language $C$ 
    and a precise and clear description of
    the result of computing $SUBSTRING(C)$, $C^*$ (using the given definitions)
    to justify this description and to justify the set nonequality claims, 
    or (2) a sufficiently general and correct argument
    why there is no such example, referring back to the relevant definitions.


    \item\gradeCorrect Specify an example {\bf finite} language $D$ over $\{0,1\}$ such that 
    $$EXTEND(D) \neq \{0,1\}^*$$ and $$EXTEND(D) = D^*$$or 
    explain why there is no such example.
    A complete solution will include either (1) a precise and
    clear description of your example language $D$ 
    and a precise and clear description of
    the result of computing $EXTEND(D)$, $D^*$ (using the given definitions)
    to justify this description and to justify the set nonequality claims, 
    or (2) a sufficiently general and correct argument
    why there is no such example, referring back to the relevant definitions.
    \end{enumerate}



    
    \end{enumerate}
\newpage

\title{HW2CSE105W25: Homework assignment 2}
\date{Due: January 30th at 5pm, via Gradescope}



\maketitle
\thispagestyle{fancy}

{\bf In this assignment,}

You will practice designing multiple representations of regular languages and working with 
general constructions of automata to demonstrate the richness of the class of regular languages.


{\bf Resources}: To review the topics 
for this assignment, see the class material from Week 2 and Week 3.
We will post frequently asked questions and our answers to them in a 
pinned Piazza post. 

{\bf Reading and extra practice problems}:  
Sipser Section 1.1, 1.2, 1.3. 
Chapter 1 exercises 1.4, 1.5, 1.6, 1.7, 1.8, 1.9, 1.10, 1.11, 1.12, 1.13, 1.14, 1.15, 
1.16, 1.17, 1.18, 1.19, 1.20, 1.21, 1.22, 1.23. Chapter 1 problem 1.31, 
1.36, 1.37.

\instructions

You will submit this assignment via Gradescope
(\href{https://www.gradescope.com}{https://www.gradescope.com}) 
in the assignment called ``hw2CSE105W25''.

{\bf Assigned questions}
\begin{enumerate}[wide, labelwidth=!, labelindent=0pt]
    %%%%%%%%%%% PROBLEM 4 %%%%%%%%%%%
\item\textbf{Finite automata} (10 points):
Consider the finite automaton $M = (Q, \Sigma, \delta, q_0, F)$ whose state diagram is depicted below
\begin{center}
\begin{tikzpicture}[->,>=stealth',shorten >=1pt, auto, node distance=2cm, semithick]
\tikzstyle{every state}=[text=black, fill=none]

\node[initial,state] (q0)          {$q_0$};
\node[state,accepting]         (q1) [above right of=q0, xshift=20pt] {$q_1$};
\node[state]         (q2) [right of=q1, xshift=20pt] {$q_2$};
\node[state,accepting]         (q3) [right of=q2, xshift=20pt] {$q_3$};
\node[state] (q4) [below of=q1] {$q_4$};

\path (q0) edge  [bend left=0] node {$b$} (q1)
        edge [bend left=0] node{$a$} (q4)
    (q1) edge [loop above] node {$b$} (q1)
        edge [bend left=0] node {$a$} (q2)
    (q2) edge [bend left=0] node {$b$} (q3)
        edge [loop above] node {$a$} (q2)
    (q3) edge [loop right] node {$a,b$} (q3)
    (q4) edge [loop right] node{$a,b$} (q4)
;
\end{tikzpicture}
\end{center}
    \begin{enumerate}
    \item\gradeCompleteFirst Write the formal definition of this automaton. In other words, give the five defining parameters $Q$, $\Sigma$, $\delta$, $q_0$, $F$ so that they are consistent with the state diagram of $M$.

    \item\gradeCorrectFirst Give a regular expression $R$ so that $L(R) = L(M)$. In other words, we want a regular expression that describes the language 
    recognized by this finite automaton. Justify your answer by referring to the 
    definition of the semantics of regular expressions and computations of finite automata. 
    Include an explanation for why each string in $L(R)$ is accepted by the finite automaton {\it and}
    for why each string not in $L(R)$ is rejected by the finite automaton.

    {\it Ungraded bonus: can you find more than one such regular expression?}

    \item\gradeComplete Keeping the same set of states $Q$, 
    input alphabet $\Sigma$, 
    same start state $q_0$, and same transition 
    function $\delta$, choose a new set of accepting states $F_{new1}$ so that the new 
    finite automaton $M_1 = (Q, \Sigma, \delta, q_0, F_{new1})$ that results 
    recognizes a {\bf proper superset} of $L(M)$, or explain 
    why there is no such example. 
    A complete solution will include either (1) a precise and
    clear description of your choice of $F_{new1}$
    {\it and} a precise and clear explanation of why every string that is accepted by 
    $M$ is also accepted by $M_1$ {\it and} an example of a string that is accepted by $M_1$ and is rejected by $M$; or (2) a sufficiently general and correct argument
    why there is no such example, referring back to the relevant definitions.


    \item\gradeCorrect Keeping the same set of states $Q$, 
    input alphabet $\Sigma$, 
    same start state $q_0$, and same transition 
    function $\delta$, choose a new set of accepting states $F_{new2}$ so that the new 
    finite automaton $M_2 = (Q, \Sigma, \delta, q_0, F_{new2})$ that results 
    recognizes a {\bf nonempty proper subset} of $L(M)$, or explain 
    why there is no such example. 
    A complete solution will include either (1) a precise and
    clear description of your choice of $F_{new2}$
    {\it and} an example string accepted by $M_2$ {\it and} a precise and clear explanation of why every string that is accepted by 
    $M_2$ is also accepted by $M$ {\it and} an example of a string that is accepted by $M$ and is rejected by $M_2$; or (2) a sufficiently general and correct argument
    why there is no such example, referring back to the relevant definitions.

    \end{enumerate}

    %%%%%%%%%%% PROBLEM 2 %%%%%%%%%%%
\item \textbf{Automata design} (12 points):
As background to this question, recall that integers can be represented using base $b$ expansions, for 
any convenient choice of base $b$. The precise definition is:
for $b$ an integer greater than $1$ and $n$ a positive integer, 
the {\bf base $b$ expansion of $n$}  is defined to be
\[
(a_{k-1} \cdots a_1 a_0)_b
\]
where $k$ is a positive integer, $a_0, a_1, \ldots, a_{k-1}$ 
are nonnegative integers less than $b$, $a_{k-1} \neq  0$, and
\[
n =  \sum_{i=0}^{k-1} a_{i} b^{i}
\]

Notice: {\it The base $b$ expansion of a positive integer $n$ is a string over the alphabet 
$\{x \in \mathbb{Z} \st 0 \leq x < b\}$
whose leftmost character is nonzero.}

An important property of base $b$ expansions of integers is that, for each integer $b$ greater than $1$,
each positive integer $n = (a_{k-1} \cdots a_1 a_0)_b$, and each nonnegative integer $a$ less than $b$, 
\[
    bn + a = (a_{k-1} \cdots a_1 a_0a)_b
\]
In other words, shifting the base $b$ expansion to the left results in multiplying the integer value by the base.
In this question we'll explore building deterministic finite automata that recognize 
languages that correspond to useful sets of integers.

    \begin{enumerate}
    \item\gradeComplete Design a DFA that recognizes the set of binary (base $2$) expansions of 
    positive integers that are powers of $2$. A complete solution will include the state diagram of your DFA and 
    a brief justification 
    of your construction by explaining the role each state plays in the machine, as well as a brief 
    justification about how the strings accepted and rejected by the machine connect to the specified language.

    {\it Hints}: (1) A power of $2$ is an integer $x$ that can be written as $2^y$ for some nonnegative integer $y$, 
    (2) the DFA should accept the strings $100$, $10$ and $100000$ and should reject the 
    strings $010$, $1101$, and $\varepsilon$ (can you see why?).

    \item\gradeCorrect Design a DFA that recognizes the set of 
    binary (base $2$) expansions of positive integers that are less than $10$. Your DFA must use {\bf fewer than ten states}. 
    A complete solution will include the state diagram of your DFA and 
    a brief justification 
    of your construction by explaining the role each state plays in the machine, as well as a brief 
    justification about how the strings accepted and rejected by the machine connect to the specified language.

    \item \gradeComplete Find a positive integer $B$ greater than $1$ so that 
    there is a DFA that recognizes the set of 
    base $B$ expansions of positive integers that are less than $10$ and it 
    uses as few states as possible. A complete solution will include the state 
    diagram of your DFA and 
    a brief justification of your choice of base.
        

    {\it Hint: sometimes rewriting the defining membership condition for a set in different ways helps us find alternate representations of that set.}
    \end{enumerate}


%%%%%%%%%%% PROBLEM 3 %%%%%%%%%%%
\item \textbf{Nondeterminism} (15 points): For this question, the alphabet is $\{a,b,c\}$.
\begin{enumerate}
\item\gradeComplete Design a NFA that recognizes the language
    \[ 
    L_1 = \{w \in \{a,b,c\}^* \mid w \text{ starts with $a$ {\bf and} ends with $a$}\}
    \]

    A complete solution will include the state diagram of your NFA and 
    a brief justification 
    of your construction that explains the role each state plays in the machine, as well as a brief 
    justification about how the strings accepted and rejected by the machine connect to the specified language.

\item\gradeCorrect Design a NFA that recognizes the language 
    \[ 
    L_2 = \{w \in \{a,b,c\}^* \mid w \text{ has no consecutive repeated characters}\}
    \]
    For example, the empty string, $a$, $bac$, and $abca$ are each elements of this language but $aa$ and $abb$ and $abbc$ are not elements of this language.

    A complete solution will include the state diagram of your NFA and 
    a brief justification 
    of your construction that explains the role each state plays in the machine, as well as a brief 
    justification about how the strings accepted and rejected by the machine connect to the specified language.

\item\gradeComplete Consider the language
\begin{align*}
L_1 \cup L_2 = \{w \in \{a,b,c\}^* \mid w &\text{ starts with $a$ and ends with $a$} \\
&\text{{\bf or} has no consecutive repeated characters}\}
\end{align*}
Give at least two representations of this language among the following: 
\begin{itemize}
\item A regular expression that describes $L_1 \cup L_2$
\item A DFA that recognizes $L_1 \cup L_2$
\item A NFA that recognizes $L_1 \cup L_2$
\end{itemize}

You can design your automata directly or use the constructions from class and chapter 1 in the book to build these automata from automata for the simpler languages.
    
A complete solution will include at least two of the representations
as well as  a brief justification of each construction.


\item\gradeComplete Consider the language
    \begin{align*}
    L_1 \cap L_2 = \{w \in \{a,b,c\}^* \mid w &\text{ starts with $a$ and ends with $a$} \\
    &\text{{\bf and} has no consecutive repeated characters}\}
    \end{align*}
    Give at least two representations of this language among the following: 
    \begin{itemize}
    \item A regular expression that describes $L_1 \cap L_2$
    \item A DFA that recognizes $L_1 \cap L_2$
    \item A NFA that recognizes $L_1 \cap L_2$
    \end{itemize}
    You can design your automata directly or use the constructions from class and chapter 1 in the book to build these automata from automata for the simpler languages.
    
A complete solution will include at least two of the representations
as well as  a brief justification of each construction.

\end{enumerate}

%%%%%%%%%%% PROBLEM 4 %%%%%%%%%%%
\item\textbf{General constructions} (13 points):
In this question, you'll practice working with formal general constructions
for automata and translating between state diagrams and formal definitions.


Recall the definitions of operations we've talked about that produce
new languages from old: for each language $L$ over an alphabet $\Sigma$, 
we have the 
associated sets of strings (also over $\Sigma$)
\[
    L^* = \{ w_1 \cdots w_k \mid k \geq 0 \textrm{ and each } w_i \in L\}
\]
and
\[
    SUBSTRING(L) = \{ w \in \Sigma^* ~|~ \text{there exist } x,y \in \Sigma^* \text{ such that } xwy \in L\}
\]
and 
\[
    EXTEND(L) = \{ w \in \Sigma^* ~|~ w = uv \text{ for some strings } u \in L \text{ and } v \in \Sigma^* \}
\]
Also, recall the set operations union and intersection: for any sets $X$ and $Y$
\[
X \cup Y = \{ w \mid w \in X \text{ or } w \in Y \}
\]
\[
X \cap Y = \{ w \mid w \in X \text{ and } w \in Y \}
\]


Let $M_1 = (Q_1, \Sigma, \delta_1, q_1, F_1)$ and 
$M_2 = (Q_2, \Sigma, \delta_2, q_2, F_2)$ be DFA.

For simplicity, assume that $Q_1 \cap Q_2 = \emptyset$ and that 
$q_0 \notin Q_1 \cup Q_2$.

Consider the following definitions of new automata parameterized by 
these DFA:
\begin{itemize}
\item The NFA $N_\alpha = ( Q_1 \cup Q_2 \cup \{q_0\}, \Sigma, \delta_\alpha, q_0, F_1 \cup F_2)$ with the transition function given by 
\[
\delta_\alpha ( ~(q,x)~) = \begin{cases}
\{q_1, q_2\} &\text{if $q = q_0$, $x = \varepsilon$} \\
\emptyset &\text{if $q=q_0$, $x\in \Sigma$}\\
\{\delta_{1}(~(q,x)~)\} &\text{if $q\in Q_1$, $x\in \Sigma$}\\
\{\delta_{2}(~(q,x)~)\} &\text{if $q\in Q_2$, $x\in \Sigma$}\\
\emptyset &\text{if $q \in Q_1 \cup Q_2$, $x = \varepsilon$} \\
\end{cases}
\]
\item The NFA $N_\beta = ( Q_1 \times Q_2, \Sigma, \delta_\beta, (q_1,q_2), F_1 \times F_2)$  with the transition function given by 
\[
\delta_\beta(~(~(r,s)~, x~)~) = \{ (~\delta_{1}(~(r,x)~), \delta_{2}( ~(s,x)~)~)\}
\]
 and 
\[
\delta_\beta(~(~(r,s)~, \varepsilon~)~) = \emptyset
\]
for $r \in Q_1, s \in Q_2, x \in \Sigma$.
\item The NFA $N_{\gamma}=(Q_1\cup\{q_{0}\},\Sigma,\delta_{\gamma},q_{0},\{q \in Q_1 \mid \exists w \in \Sigma^* (\delta_1^* ((q,w)) \in F_1)\})$, and
        \[
            \delta_{\gamma} ((q,a)) = \begin{cases}
                \{\delta_1((q,a))\} &\text{if $q \in Q_1$, $a \in \Sigma$} \\
                \{q' \in Q_1 \mid \exists w \in \Sigma^* (\delta_1^*((q_1,w)) = q') \}&\text{if $q =q_{0}$, $a = \varepsilon$}\\
                \emptyset &\text{if $q = q_{0}$, $a \in \Sigma$} \\
                \emptyset &\text{if $q \in Q_1$, $a = \varepsilon$}
            \end{cases}
        \]

{\it Hint: the notation $\delta_1^*$ refers to the iterated transition function.}
\end{itemize}



\begin{enumerate}
\item\gradeCorrect 
Illustrate the construction of $N_{\alpha}$ by defining a specific pair 
of example DFAs $M_1$ and $M_2$ and applying the 
construction above to create the new NFA $N_\alpha$. Your example DFA should
\begin{itemize}
    \item Have the same input alphabet as each other, 
    \item Each have exactly three states (all reachable from the respective start state),
    \item Accept at least one string and reject at least one string, 
    \item Recognize different languages from one another, and
    \item Not have any states labelled $q_0$, and 
    \item Not share any state labels.
\end{itemize}
Apply the construction above to create the new NFA. A complete submission 
will include the state diagrams of your example DFA $M_1$ and $M_2$ and the state diagram of the NFA $N_\alpha$ resulting 
from this construction and a precise and clear description of $L(M_1)$ and $L(M_2)$ and $L(N_{\alpha})$, justified
by explaining the role each state plays in the machine, as well as a brief 
justification about how the strings accepted and rejected by the machine connect to the language.

\item\gradeCorrect 
Illustrate the construction of $N_{\beta}$ by defining a specific pair 
of example DFAs $M_1$ and $M_2$ and applying the 
construction above to create the new NFA $N_\beta$. Your example DFA should
\begin{itemize}
    \item Have the same input alphabet as each other, 
    \item Each have exactly two states (all reachable from the respective start state),
    \item Accept at least one string and reject at least one string, 
    \item Recognize different languages from one another, and
    \item Not have any states labelled $q_0$, and 
    \item Not share any state labels.
\end{itemize}
Apply the construction above to create the new NFA. A complete submission 
will include the state diagrams of your example DFA $M_1$ and $M_2$ and the state diagram of the NFA $N_\beta$ resulting 
from this construction and a precise and clear description of $L(M_1)$ and $L(M_2)$ and $L(N_{\beta})$, justified
by explaining the role each state plays in the machine, as well as a brief 
justification about how the strings accepted and rejected by the machine connect to the language.


\item\gradeCorrect 
Illustrate the construction of $N_{\gamma}$ by defining a specific
example DFA $M_1$  and applying the 
construction above to create the new NFA $N_\gamma$. Your example DFA should
\begin{itemize}
    \item Have exactly four states (all reachable from the respective start state),
    \item Accept at least one string and reject at least one string, 
    \item Not have any states labelled $q_0$.
\end{itemize}
Apply the construction above to create the new NFA. A complete submission 
will include the state diagram of your example DFA $M_1$ and the state diagram of the NFA $N_\gamma$ resulting 
from this construction and a precise and clear description of $L(M_1)$ and $L(N_{\gamma})$, justified
by explaining the role each state plays in the machine, as well as a brief 
justification about how the strings accepted and rejected by the machine connect to the language.

\item \gradeComplete If possible, associate each construction above with one of the operations whose definitions we recalled at the start of the question.  For example, is it the case that (for all choices of DFA $M_1$ and $M_2$) $L(N_\alpha) = L(M_1) \cup L(M_2)$? or $L(N_\alpha) = L(M_1) \cap L(M_2)$? etc.

A complete solution will consider each of the constructions $N_\alpha, N_\beta, N_\gamma$ in turn, and for each, either name the operation that's associated with the construction (and explain why) or explain why none of the operations mentioned is associated with the construction.
\end{enumerate}

\end{enumerate}
\newpage

\title{HW3CSE105W25: Homework assignment 3}
\date{Due: February 6th at 5pm, via Gradescope}


\maketitle
\thispagestyle{fancy}

{\bf In this assignment,}

You will demonstrate the richness of the class of regular languages, as well as its boundaries, and explore push-down automata and their design.


{\bf Resources}: To review the topics 
for this assignment, see the class material from Week 3 and Week 4.
We will post frequently asked questions and our answers to them in a 
pinned Piazza post. 

{\bf Reading and extra practice problems}:  
Sipser Chapter 1 and Section 2.2.
Chapter 1 exercises 1.28, 1.29, 1.30. Chapter 1 problem 1.53, 1.54, 1.55.
Chapter 2 exercises 2.7, 2.10.

\instructions

You will submit this assignment via Gradescope
(\href{https://www.gradescope.com}{https://www.gradescope.com}) 
in the assignment called ``hw3CSE105W25''.

{\bf Assigned questions}
\begin{enumerate}[wide, labelwidth=!, labelindent=0pt]



%%%%%%%%%%% PROBLEM 1 %%%%%%%%%%%
\item \textbf{Static analysis} (10 points):
In software engineering, static analysis is an approach to debugging and testing where the properties of a piece of code are inferred without actually running it.  
In the context of finite automata, we can think of static analysis as the process of inferring properties of the language recognized by 
a finite automaton from properties of the graph underlying its state diagram. The Pumping Lemma is one example of static analysis.
In this question, you'll explore other examples of how properties of the 
graph underlying the state diagram of a machine can give us information about 
the language recognized by the machine.
\begin{enumerate}
\item\gradeCompleteFirst Suppose you are given an NFA $N_0$ over an alphabet $\Sigma$ and each accepting state in $N_0$ is not reachable from the start state of $N_0$.  What can you conclude about the language of the NFA?

\item\gradeCorrectFirst Prove or disprove: For any alphabet $\Sigma$ and any  DFA $M$ over $\Sigma$, if every state in $M$ is accepting then $L(M) = \Sigma^*$. A complete answer will clearly indicate whether the statement is true or false and then will justify with a complete and correct argument.

\item\gradeCorrect Prove or disprove: For any alphabet $\Sigma$ and any  NFA $N$ over $\Sigma$, if every state in $N$ is accepting then $L(N) = \Sigma^*$. A complete answer will clearly indicate whether the statement is true or false and then will justify with a complete and correct argument.
\end{enumerate}

%%%%%%%%%%% PROBLEM 2 %%%%%%%%%%%
\item \textbf{Multiple representations} (12 points):

\begin{enumerate}
   \item Consider the language $A_1 = \{ uw \mid \text{$u$ and
   $w$ are strings over $\{0,1\}$ and have the same length} \}$
   and the following argument.

   \begin{quote}
      ``Proof" that $A_1$ is not regular using the Pumping Lemma: Let $p$ be 
      an arbitrary positive integer. We will show that $p$ is not a pumping length for $A_1$. 
      
      Choose $s$ to be the string $1^p 0^p$, which is in $A_1$ because
      we can choose $u = 1^p$ and $w = 0^p$ which each have length $p$.
      Since $s$ is in $A_1$ and has length greater than or equal to $p$, if $p$ were to be a
      pumping length for $A_1$, $s$ ought to be pump'able. 
      That is, there should be a way of dividing $s$ into parts $x,y,z$ where $s=xyz$,
      $|y| >0$, $|xy| \leq p$, and for each $i \geq 0$, $xy^iz \in A_1$.
      Suppose $x,y,z$ are such that $s = xyz$, $|y| > 0$ and $|xy| \leq p$.
      Since the first $p$ letters of $s$ are all $1$ and $|xy| \leq p$, we know
      that $x$ and $y$ are made up of all $1$s.  If we let $i=2$, we get 
      a string $xy^iz$ that is not in $A_1$ because repeating $y$ twice adds $1$s to 
      $u$ but not to $w$, and strings in $A_1$ are required to have $u$ and $w$ be the same
      length. Thus, $s$ is not pumpable (even though it should have been if $p$ were to be a pumping length)
      and so $p$ is not a pumping length for $A_1$.  Since $p$ was arbitrary, we have
      demonstrated that $A_1$ has no pumping length.  By the Pumping Lemma, this implies that 
      $A_1$ is nonregular.
      \end{quote}
      \begin{enumerate}
         \item \gradeComplete Find the (first and/or most significant) logical error in the ``proof" above 
         and describe why it's wrong.
   
         \item \gradeComplete Prove that the set $A_1$ is actually regular (by finding a regular expression that describes it or 
         a DFA/NFA that recognizes it, and justifying why) {\bf or} fix the proof so that it is logically sound.     
      \end{enumerate}

   \item Consider the language $A_2 = \{ u1w \mid \text{$u$ and
   $w$ are strings over $\{0,1\}$ and have the same length} \}$
   and the following argument.


   \begin{quote}
      ``Proof" that $A_2$ is not regular using the Pumping Lemma: Let $p$ be 
      an arbitrary positive integer. We will show that $p$ is not a pumping length for $A_2$. 
      
      Choose $s$ to be the string $1^{p+1} 0^{p}$, which is in $A_2$ because
      we can choose $u = 1^p$ and $w = 0^p$ which each have length $p$.
      Since $s$ is in $A_2$ and has length greater than or equal to $p$, if $p$ were to be a
      pumping length for $A_2$, $s$ ought to be pump'able. 
      That is, there should be a way of dividing $s$ into parts $x,y,z$ where $s=xyz$,
      $|y| >0$, $|xy| \leq p$, and for each $i \geq 0$, $xy^iz \in A_2$.
      When $x = \varepsilon$ and $y = 1^{p+1}$ and $z = 0^{p}$,
      we have satisfied that $s = xyz$, $|y| > 0$ (because $p$ is positive) and $|xy| \leq p$.
      If we let $i=0$, we get 
      the string $xy^iz = 0^{p}$ that is not in $A_2$ because its middle symbol is a $0$, not a $1$. 
      Thus, $s$ is not pumpable (even though it should have been if $p$ were to be a pumping length)
      and so $p$ is not a pumping length for $A_2$.  Since $p$ was arbitrary, we have
      demonstrated that $A_2$ has no pumping length.  By the Pumping Lemma, this implies that 
      $A_2$ is nonregular.
      \end{quote}

      \begin{enumerate}

      \item \gradeComplete Find the (first and/or most significant) logical error in the ``proof" above 
      and describe why it's wrong.

      \item \gradeComplete Prove that the set $A_2$ is actually regular (by finding a regular expression that describes it or 
      a DFA/NFA that recognizes it, and justifying why) {\bf or} fix the proof so that it is logically sound.
      
      \end{enumerate}
   
\end{enumerate}


%%%%%%%%%%% PROBLEM 3 %%%%%%%%%%%
\item \textbf{Pumping} (10 points):

\begin{enumerate}
\item\gradeCorrect Give an example of a language
over the alphabet $\{a,b\}$ that has cardinality $3$ and for which $5$ is a pumping length
and $4$ is not a pumping length. Is this language regular? A complete solution will give 
(1) a clear and precise
description of the language, (2) a justification for why $5$ is a pumping length, (3) a 
justification for why $4$ is not a pumping length, (4) a correct and justified answer to 
whether the language is regular.


\item\gradeComplete In class and in the reading so far, we've seen the following examples of nonregular sets:
\begin{multicols}{3}
\begin{center}
$\{ 0^n 1^n ~|~ n \geq 0 \}$
$$\{ 0^n 1^n ~|~ n \geq 2 \}$$
$$\{ 0^n 1^m ~|~  0 \leq n \leq m \}$$
$$\{ 0^n 1^m ~|~ 0 \leq m \leq n \}$$
$$\{ 0^i 1^{2i} ~|~ 0 \leq i \}$$
$$\{ 0^i 1^{i+1} ~|~ 0 \leq i \}$$
$$\{ 0^n 1^m 0^n ~|~n,m \geq 0\}$$
$$\{ w \in \{0,1\}^* ~|~w = w^{\mathcal{R}}\}$$
$$\{ w w^{\mathcal{R}} ~|~ w \in \{0,1\}^*\}$$
\end{center}
\end{multicols}
Modify one of these sets in some way and use the Pumping Lemma to prove that the resulting set is still nonregular.

\end{enumerate}

%%%%%%%%%%% PROBLEM 4 %%%%%%%%%%%
\item\textbf{Regular and nonregular languages} (12 points):
In Week 2's review quiz, we saw the definition that a set $X$ is said to be 
{\bf closed under an operation} if, for any elements in
$X$, applying to them gives an element in $X$. For example, the set of
integers is closed under multiplication because if we take any two
integers, their product is also an integer .

Recall the definitions of operations we've talked about that produce
new languages from old: for each language $L$ over an alphabet $\Sigma$, 
we have the 
associated sets of strings (also over $\Sigma$)
\[
    L^* = \{ w_1 \cdots w_k \mid k \geq 0 \textrm{ and each } w_i \in L\}
\]
and
\[
    SUBSTRING(L) = \{ w \in \Sigma^* ~|~ \text{there exist } x,y \in \Sigma^* \text{ such that } xwy \in L\}
\]
and 
\[
    EXTEND(L) = \{ w \in \Sigma^* ~|~ w = uv \text{ for some strings } u \in L \text{ and } v \in \Sigma^* \}
\]
Also, recall the set operations union and intersection: for any sets $X$ and $Y$
\[
X \cup Y = \{ w \mid w \in X \text{ or } w \in Y \}
\]
\[
X \cap Y = \{ w \mid w \in X \text{ and } w \in Y \}
\]

\begin{enumerate} 
   \item \gradeCorrect Use the general constructions that we developed
   to prove the closure of the class of regular languages under various operations to produce the state diagram of a NFA that recognizes the language
   \[
   (~SUBSTRING(\{0, 01, 111\})~)^*
   \]
   {\it Hint: Question 4 from Homework 2 might be helpful.}
   
    For full credit, submit (1) a state diagram of an NFA that recognizes $\{0,01,111\}$, (2) a state diagram of an NFA that recognizes $SUBSTRING(\{0, 01, 111\})$, and (3) a state diagram of an NFA that recognizes $(~SUBSTRING(\{0, 01, 111\})~)^*$, and a brief justification of each state diagram that 
    references the language being recognized or the general constructions being used.

   \item \gradeComplete Prove that the class of nonregular languages over $\{0,1\}$ is not closed under the $SUBSTRING$ operation by giving an example 
   language $A$ that is nonregular but for which $SUBSTRING(A)$ is regular.
   A complete solution will give 
   (1) a clear and precise
   description of the language, (2) a justification for why it is nonregular (either by proving this directly or by referring to specific examples from the class or textbook), (3) a description of the result of applying the 
   $SUBSTRING$ operation to the language, and  (4) a justification for
   why this resulting language is regular.

   \item  \gradeCorrect Prove that the class of nonregular languages over $\{0,1\}$ is not closed under the $EXTEND$ operation by giving an example 
   language $B$ that is nonregular but for which $EXTEND(B)$ is regular.
   A complete solution will give 
   (1) a clear and precise
   description of the language, (2) a justification for why it is nonregular (either by proving this directly or by referring to specific examples from the class or textbook), (3) a description of the result of applying the 
   $EXTEND$ operation to the language, and  (4) a justification for
   why this resulting language is regular.

   \item \gradeComplete   Prove that the class of nonregular languages over $\{0,1\}$ is not closed under the Kleene star operation by giving an example 
   language $C$ that is nonregular but for which $C^*$ is regular.
   A complete solution will give 
   (1) a clear and precise
   description of the language, (2) a justification for why it is nonregular (either by proving this directly or by referring to specific examples from the class or textbook), (3) a description of the result of applying the 
   Kleene star operation to the language, and  (4) a justification for
   why this resulting language is regular.
\end{enumerate}

%%%%%%%%%%% PROBLEM 5 %%%%%%%%%%%
\item \textbf{Regular and nonregular languages and Push-down automata (PDA)} (6 points): 
On page 7 of the week 4 notes, we have the following list of languages over the alphabet $\{a,b\}$

\begin{center}
\begin{tabular}{ccc}
    $\{a^nb^n \mid 0  \leq n  \leq 5 \}$
    &$\{b^n a^n \mid  n  \geq 2\}$
    &$\{a^m b^n \mid  0 \leq m\leq n\}$
\end{tabular}
\begin{tabular}{cc}
    $\{a^m b^n \mid  m \geq n+3,  n \geq 0\}$
    &$\{b^m a^n \mid  m \geq 1, n \geq  3\}$\\
    $\{ w  \in \{a,b\}^* \mid w = w^\mathcal{R} \}$
    &$\{ ww^\mathcal{R} \mid w\in \{a,b\}^* \}$ \\
\end{tabular}
\end{center}
\begin{enumerate}
    \item\gradeComplete Pick one of the regular languages and design a regular expression that describes it and a DFA that recognizes it. 
    Briefly justify your regular expression by connecting
    the subexpressions of it to the intended language and referencing relevant definitions. Briefly justify your DFA design by explaining the role each state plays in the machine, 
    as well as a brief 
    justification about how the strings accepted and rejected by the machine connect to the specified language.
    \item\gradeComplete Pick one of the nonregular languages and design a PDA that recognizes it. Draw the state diagram of 
    your PDA. Briefly justify your design by explaining the role each state plays in the machine, 
    as well as a brief 
    justification about how the strings accepted and rejected by the machine connect to the specified language.
\end{enumerate}
\end{enumerate}
\newpage

\title{HW4CSE105W25: Homework assignment 4}
\date{Due: February 20th at 5pm, via Gradescope}


\maketitle
\thispagestyle{fancy}

{\bf In this assignment,}

You will work with context-free languages and their representations.
You will also practice analyzing, designing, and working with Turing machines.
You will explore recognizable and decidable languages.

{\bf Resources}: To review the topics 
for this assignment, see the class material from Weeks 5 and 6.
We will post frequently asked questions and our answers to them in a 
pinned Piazza post. 

{\bf Reading and extra practice problems}:  
Sipser Chapters 2 and 3.
Chapter 2 exercises 2.1, 2.2, 2.3, 2.4, 2.5, 2.6, 2.9, 2.11, 2.12, 2.13, 2.16, 2.17.
Chapter 3 exercises 3.1, 3.2, 3.5, 3.8.

\instructions

You will submit this assignment via Gradescope
(\href{https://www.gradescope.com}{https://www.gradescope.com}) 
in the assignment called ``hw4CSE105W25''.

{\bf Assigned questions}
\begin{enumerate}[wide, labelwidth=!, labelindent=0pt]


%%%%%%%%%%% PROBLEM 1 %%%%%%%%%%%
\item \textbf{Regular and nonregular languages and context-free grammars (CFG)} (6 points): 
On page 7 of the week 4 notes, we have the following list of languages over the alphabet $\{a,b\}$

\begin{center}
\begin{tabular}{ccc}
    $\{a^nb^n \mid 0  \leq n  \leq 5 \}$
    &$\{b^n a^n \mid  n  \geq 2\}$
    &$\{a^m b^n \mid  0 \leq m\leq n\}$
\end{tabular}
\begin{tabular}{cc}
    $\{a^m b^n \mid  m \geq n+3,  n \geq 0\}$
    &$\{b^m a^n \mid  m \geq 1, n \geq  3\}$\\
    $\{ w  \in \{a,b\}^* \mid w = w^\mathcal{R} \}$
    &$\{ ww^\mathcal{R} \mid w\in \{a,b\}^* \}$ \\
\end{tabular}
\end{center}
\begin{enumerate}
    \item\gradeCompleteFirst Pick one of the regular languages and design a context-free grammar that generates it. 
    Briefly justify your grammar by describing the role of each of the rules
    and connecting it to the intended language and referencing relevant definitions.
    \item\gradeComplete Pick one of the nonregular languages and design a context-free grammar that generates it.  Briefly justify your grammar by describing the role of each of the rules
    and connecting it to the intended language and referencing relevant definitions.
\end{enumerate}

%%%%%%%%%%% PROBLEM 2 %%%%%%%%%%%
\item \textbf{General constructions for context-free languages} (15 points):

In class in week 5, we described several general constructions 
with PDAs and CFGs, leaving their details to
homework. In this question, we'll fill in these details. The first constructions
help us prove that the class of regular languages is a subset of the
class of context-free languages. The other construction allows us 
to make simplifying assumptions about PDAs recognizing languages.

\begin{enumerate}

\item\gradeComplete When we first introduced PDAs we observed 
that any NFA can be transformed to a PDA by not using the stack 
of the PDA at all. Suppose a friend gives you the following construction
to formalize this transformation:

\begin{quote}
Given a NFA $N = (Q, \Sigma, \delta_N, q_0, F)$ we define a PDA $M$
with $L(M) = L(N)$ by letting $M = ( Q, \Sigma, \Sigma, \delta, q_0, F)$ where 
$\delta(~(q,a,b)~) = \delta_N(~(q,a)~)$ for each $q \in Q$, 
$a \in \Sigma_{\varepsilon}$ and $b \in \Sigma_{\varepsilon}$.
\end{quote}

For each of the six defining parameters for the PDA, explain whether 
it's defined correctly or not. If it is not defined correctly, 
explain why not and give a new definition for this parameter that 
corrects the mistake.

\item\gradeCorrectFirst In the book on page 107, the top paragraph describes a procedure for converting DFAs to CFGs:
\begin{quote}
   You can convert any DFA into an equivalent CFG as follows. 
   Make a variable $R_i$ for each state $q_i$ of the DFA. Add the rule $R_i \to aR_j$ to the
   CFG if $\delta(q_i,a) =q_j$ is a transition in the DFA. Add the rule
   $R_i\to \varepsilon$ if $q_i$ is an accept state of the DFA. Make $R_0$ the start variable ofthe grammar, 
   where $q_0$ is the start state of the machine. Verify on your own that the resulting CFG 
   generates the same language that the DFA recognizes.
\end{quote}

Use this construction to get a context-free grammar generating the language 
\[
    \{ w \in \{a,b\}^* \mid w \text{ has at least one $a$ and does not end in  $bb$}\}
\]
by (1) designing a DFA that recognizes this language and then (2) applying the construction from the book to convert the 
DFA to an equivalent CFG. A complete and correct submission will include the state diagram of the DFA, a brief justification of why 
it recognizes the language, and then the complete and precise definition of the CFG that results from applying the construction 
from the book to this DFA. {\it Ungraded bonus: take a sample string in the language and see how the computation of 
the DFA on this string translates to a derivation in your grammar.}

\item Let $M_1 = (Q_1, \Sigma, \Gamma_1, \delta_1, q_1, F_1)$ 
be a PDA and let $q_{new}, r_{new}, s_{new}$ be three fresh state labels 
(i.e. $Q_1 \cap \{q_{new}, r_{new}, s_{new}\} = \emptyset$) 
and let $\#$ be a fresh stack symbol (i.e. $\# \notin \Gamma_1$).
We define the PDA $M_2$ as
\[
   (Q_2, \Sigma, \Gamma_2, \delta_2, q_{new}, \{s_{new}\})
\]
with $Q_2 = Q_1 \cup \{q_{new}, r_{new}, s_{new}\}$ 
and $\Gamma_2 = \Gamma_1 \cup \{\#\}$
and 
$\delta_2 : Q_2 \times \Sigma_\varepsilon \times {\Gamma_2}_\varepsilon \to 
\mathcal{P}(Q_2 \times {\Gamma_2}_\varepsilon)$ given by 
\[
\delta_2 ( ~(q,a,b)~) = 
\begin{cases}
\{(q_1, \#)\} &\text{if } q = q_{new}, a = \varepsilon, b = \varepsilon\\
\delta_1( ~(q,a,b)~) &\text{if } q\in Q_1 \setminus F_1, a \in \Sigma_{\varepsilon}, b \in {\Gamma_1}_\varepsilon \\
\delta_1( ~(q,a,b)~) &\text{if } q\in F_1, a \in \Sigma, b \in {\Gamma_1}_\varepsilon \\
\delta_1( ~(q,a,b)~) &\text{if } q\in F_1, a =\varepsilon, b \in {\Gamma_1} \\
\delta_1( ~(q,a,b)~) \cup \{(r_{new}, \varepsilon)\} &\text{if } q\in F_1, a =\varepsilon, b =\varepsilon \\
\{(r_{new}, \varepsilon)\} &\text{if } q = r_{new}, a =\varepsilon, b \in \Gamma_{1} \\
\{(s_{new}, \varepsilon)\} &\text{if } q= r_{new}, a = \varepsilon, b = \#\\
\emptyset & \text{otherwise}
\end{cases}
\]
for each $q \in Q_2$, $a \in \Sigma_{\varepsilon}$, 
and $b \in {\Gamma_2}_\varepsilon$.

In this question, we'll apply this construction for a specific PDA and 
use this example to extrapolate the effect of this construction.
\begin{enumerate}

\item\gradeCorrect Consider the PDA $M_1$ with input alphabet $\{0,1\}$ 
and stack alphabet $\{0,1, \$\}$ whose state diagram is

\begin{center}

   \begin{tikzpicture}[->,>=stealth',shorten >=1pt, auto, node distance=2cm, semithick]
      \tikzstyle{every state}=[text=black, fill=none]
    
    \node[initial,state] (s1)          {$s_1$};
    \node[state] (s2) [right of=s1, xshift=20pt]  {$s_2$};
    \node[state] (s3) [right of=s2, xshift=20pt]  {$s_3$};
    \node[state] (s4) [right of=s3, xshift=20pt]  {$s_4$};
    \node[state, accepting] (s5)  [right of=s4, xshift=20pt]         {$s_5$};

    \path (s1) edge  [bend left=0] node {$\varepsilon,\varepsilon; \$$} (s2)
    (s2) edge  [loop above, near start] node {$1,\varepsilon; 0$} (s2)
    (s2) edge  [bend left=0] node {$\varepsilon,\varepsilon; \varepsilon$} (s3)
    (s3) edge  [loop above, near start] node {$0,0; \varepsilon$} (s3)
    (s3) edge  [bend left=0] node {$0,\$; 0$} (s4)
    (s4) edge  [bend left=0] node {$0,\varepsilon; 0$} (s5)
    ;
    \end{tikzpicture}
\end{center}

Draw the state diagram for the PDA $M_2$ that results from applying 
the construction to $M_1$. Also, give an example string of length $4$ 
that is accepted by both $M_1$ and $M_2$ and justify your choice by describing 
an accepting computation for each of the PDAs on your input string.

\item\gradeComplete Compare $L(M_1)$ and $L(M_2)$. Are these sets 
equal? Does your answer depend on the specific choice of $M_1$? Why or why not?

\item\gradeComplete Consider the PDA $N$ with input alphabet $\{0,1\}$ and
stack alphabet $\{0,1\}$ whose state diagram is 

\begin{center}
   \begin{tikzpicture}[->,>=stealth',shorten >=1pt, auto, node distance=2cm, semithick]
      \tikzstyle{every state}=[text=black, fill=none]
    
    \node[initial,state, accepting] (q1)          {$q_1$};
    \node[state, accepting]         (q2) [right of=q1, xshift=20pt] {$q_2$};
    
    \path (q1) edge  [loop above, near start] node {$0,\varepsilon; 0$} (q1)
        (q1) edge [bend left=0] node {$1,0;\varepsilon$} (q2)
        (q2) edge [loop above, near start] node {$1,0;\varepsilon$} (q2)
    ;
    \end{tikzpicture}

\end{center}

   Remember that the definition of set-wise concatenation is:
    for languages $L_1, L_2$ over the alphabet $\Sigma$, we have the 
    associated set of strings
    \[
       L_1 \circ L_2 = \{ w \in \Sigma^* ~|~ w = uv \text{ for some strings } u \in L_1 \text{ and } v \in L_2 \}
    \]
    In class, we discussed how extrapolating the construction that we used to prove that the class of regular languages
    is closed under set-wise concatenation by drawing 
    spontaneous transitions from the accepting states in the first 
    machine to the start state of the second machine doesn't work.
    Use the example of $M_1$ and $N$ to prove this by showing that
    \[
    L(M_1) \circ L(N)
    \]
    is {\bf not} the language recognized by the machine 
    results from taking the two machines
    $M_1$ and $N$, setting the start state of $M_1$ to be the start state of 
    the new machine, setting the set of accepting states of $N$ to be
    the set of accepting states of the new machine, and drawing spontaneous 
    arrows from the accepting states
    of $M_1$ to the start state of $N$.
    Then, describe the language recognized by the machine 
    that results from taking the two machines
    $M_2$ and $N$, setting the start state of $M_2$ to be the start state of 
    the new machine, setting the set of accepting states of $N$ to be
    the set of accepting states of the new machine, and drawing spontaneous 
    arrows from the accepting states
    of $M_2$ to the start state of $N$. Use this description 
    to explain why we used the construction of $M_2$ from $M_1$
    and how this construction could be used in a proof of the 
    closure of the class of context-free languages under set-wise concatenation.

    A complete response will give an example string that witnesses that $L(M_1) \circ L(N)$ is not equal to the language recognized by the PDA resulting from the wrong construction (described above) {\bf and} the state diagram 
    of the PDA that results from applying that construction to $M_2$ and $N$ (instead of $M_1$), with a brief justification about why that approach works.
\end{enumerate}

\end{enumerate}


%%%%%%%%%%% PROBLEM 3 %%%%%%%%%%%
\item \textbf{Turing machines} (9 points):

Consider the Turing machine $T$ over the input alphabet $\Sigma = \{0,1\}$ with  the state
    diagram below (the tape alphabet is $\Gamma = \{ 0,1,X,\square\}$).  
    Convention: we do not include the node for the reject state $qrej$ and any missing transitions in the state diagram have value $(qrej,\square,R)$
    \begin{center}
      \begin{tikzpicture}[->,>=stealth',shorten >=1pt, auto, node distance=2cm, semithick]
         \tikzstyle{every state}=[text=black, fill=none]
        
        \node[initial,state] (q0)          {$q_0$};
        \node[state] (q1)  [above right of=q0, xshift=60pt] {$q_{1}$};
        \node[state] (q2)  [right of=q1, xshift=50pt] {$q_{2}$};
        \node[state] (q3)  [below right of=q0, xshift=60pt] {$q_{3}$};
        \node[state] (q4)  [right of=q3, xshift=50pt] {$q_{4}$};
        \node[state] (q5)  [right of=q2, xshift=50pt] {$q_5$};
        \node[state] (q6)  [right of=q4, xshift=50pt] {$q_6$};
        \node[state, accepting]         (qacc) [below right of=q5, xshift=50pt] {$q_{acc}$};
        
        \path (q1) edge  [loop above, near start] node {$0,1 \to R$} (q1)
            (q3) edge  [loop below, near start] node {$0,1 \to L$} (q3)
            (q0) edge [bend left=0, above, near start] node {$0\to X, R$} (q1)
            (q0) edge [bend left=0, below, near start] node {$1\to R$} (q3)
            (q1) edge [bend left=0] node {$\square \to L$} (q2)
            (q3) edge [bend left=0] node {$\square \to L$} (q4)
            (q2) edge [bend left=0] node {$0,1 \to X, L$} (q5)
            (q4) edge [bend left=0] node {$0,1 \to X, L$} (q6)
            (q5) edge [bend left=0, above, near end] node {$1\to X, R$} (qacc)
            (q6) edge [bend left=0, below, near end] node {$0\to X, R$} (qacc)
        ;
        \end{tikzpicture}
    \end{center}

    \begin{enumerate}

        \item\gradeCorrect Specify an example string $w_1$ of length $4$ over 
        $\Sigma$ that is {\bf accepted} by this Turing machine, or explain why there is no such 
        example. A complete solution will include either (1) a precise and clear 
        description of your example  string and a precise and clear description of the accepting computation
        of the Turing machine on this string or (2) a sufficiently
        general and correct argument why there is no such example, referring back to the relevant definitions.
        
        To describe a computation of a Turing machine, include the contents of the 
        tape, the state of the machine, and the location of the read/write head at each step in the computation.
        
        {\it Hint:} In class we've drawn pictures to represent the configuration of the machine at each step 
        in a computation.  You may do so or you may choose to describe these configurations in words.
        
        \item\gradeCorrect Specify an example string $w_2$ of length $3$ over $\Sigma$ 
        that is {\bf rejected} by this Turing machine
        or explain why there is no such 
        example. A complete solution will include either (1) a precise and clear 
        description of your example  string and a precise and clear description of the rejecting computation
        of the Turing machine on this string or (2) a sufficiently
        general and correct argument why there is no such example, referring back to the relevant definitions.

        \item\gradeCorrect Specify an example string $w_3$ of length $3$ over $\Sigma$ 
        on which  the computation of this Turing machine is {\bf never halts}
        or explain why there is no such 
        example. A complete solution will include either (1) a precise and clear 
        description of your example  string and a precise and clear description of the looping (non-halting) 
        computation
        of the Turing machine on this string or (2) a sufficiently
        general and correct argument why there is no such example, referring back to the relevant definitions.

\end{enumerate}



%%%%%%%%%%% PROBLEM 4 %%%%%%%%%%%
\item\textbf{Implementation-level descriptions of deciders and recognizers} (12 points): Consider the language $$\{a^i b^j \mid i \geq 0, j > 1\}$$
over the alphabet $\{a,b\}$.

\begin{enumerate}
\item[(a)]\gradeCorrect Give an example of a Turing machine that {\bf decides} this language.
A complete solution will include  {\bf both} a state diagram and an implementation-level description 
of this Turing machine, along with a brief explanation of why it recognizes this language, and why it is a decider.

\item[(b)]\gradeCorrect Give an example of a Turing machine that {\bf recognizes but does not decide} this language.
A complete solution will include  {\bf both} a state diagram and an implementation-level description 
of this Turing machine, along with a brief explanation of why it recognizes this language, and why it is not a decider.

\end{enumerate}

%%%%%%%%%%% PROBLEM 5 %%%%%%%%%%%
\item\textbf{Classifying languages} (8 points):
Our first example of a more complicated Turing machine was of a Turing machine 
that recognized the language $\{w \# w \mid w \in\{0,1\}^*\}$ 
(Figure 3.10 in the textbook), which we 
know is not context-free. Let's call that Turing machine $M_0$. The language
\[
    L = \{ww \mid w \in \{0,1\}^*\}
\]
is also not context-free. 

\begin{enumerate}
    \item\gradeCorrect Choose an example string of length $2$ in $L$ that is in {\bf not} in $\{w \# w \mid w \in\{0,1\}^*\}$ and describe the computation of the Turing machine $M_0$ on your example string. 
    Include the contents of the  tape, the state of the machine, and the location of the read/write head at each step in the computation.
    \item\gradeComplete Explain why the Turing machine from the textbook 
    and class that recognized $\{w \# w \mid w \in\{0,1\}^*\}$ does 
    not recognize $\{ww \mid w \in \{0,1\}^*\}$. Use your example to explain why $M_0$ doesn't recognize $L$.
    \item\gradeComplete Explain how you would change $M_0$ to get a 
    new Turing machine that does recognize $L$. Describe this new Turing machine using both an implementation-level definition and a state diagram of the Turing machine. You may use all 
    our usual conventions for state diagrams of Turing machines 
    (we do not include the node for the reject state $qrej$ and any missing transitions 
    in the state diagram have value $(qrej,\square,R)$; 
    $b \to R$ label means $b \to b, R$ ).
\end{enumerate}
\end{enumerate}
\newpage

\title{HW5CSE105W25: Homework assignment 5}
\date{Due: February 27th at 5pm, via Gradescope}


\maketitle
\thispagestyle{fancy}

{\bf In this assignment,}
You will practice analyzing, designing, and working with Turing machines.
You will use general constructions and specific machines to explore the classes 
of recognizable and decidable languages. 
You will explore various ways to encode machines as strings so that 
computational problems can be recognized and solved.

{\bf Resources}: To review the topics 
for this assignment, see the class material from Weeks 6, 7, and 8.
We will post frequently asked questions and our answers to them in a 
pinned Piazza post. 

{\bf Reading and extra practice problems}:  
Sipser Chapters 3 and 4.
Chapter 3 exercises 3.1, 3.2, 3.5, 3.8.
Chapter 4 exercises 4.1, 4.2, 4.3, 4.4, 4.5.

\instructions

You will submit this assignment via Gradescope
(\href{https://www.gradescope.com}{https://www.gradescope.com}) 
in the assignment called ``hw5CSE105W25''.

{\bf Assigned questions}
\begin{enumerate}[wide, labelwidth=!, labelindent=0pt]

%%%%%%%%%%% PROBLEM 1 %%%%%%%%%%%
\item\textbf{Equally expressive models} (10 points):
The {\bf  Church-Turing Thesis} (Sipser p.~183) says that the informal notion of algorithm is formalized completely  and correctly by the 
formal definition of a  Turing machine. In other words: all reasonably expressive models of 
computation are equally expressive with the standard Turing machine.
In this question, we will give support for this thesis by showing that 
some adaptations of the standard (Chapter 3) Turing machine model still 
gives us a new model that is equally expressive.

\begin{enumerate}
\item\gradeCompleteFirst Let's define a new machine model, and call it the {\bf May-stay}  machine.
The May-stay machine model is the same as the usual Turing machine model,  except that
on each transition, the tape head may move L, move R, or Stay. 

Formally: a May-stay machine is given by the $7$-tuple
$(Q, \Sigma, \Gamma, \delta, q_0, q_{accept}, q_{reject})$ where 
$Q$ is a finite set with $q_0 \in Q$ and $q_{accept} \in Q$
and $q_{reject} \in Q$ and $q_{accept} \neq q_{reject}$, 
$\Sigma$ and $\Gamma$ are alphabets and $\Sigma \subseteq \Gamma$ and $\square \in \Gamma$ and $\square \notin \Sigma$, and the transition function 
has signature
\[
  \delta: Q \times \Gamma \to Q \times \Gamma \times \{L, R, S\}
\]
The notions of computation and acceptance are analogous to that from Turing machines.

Prove that Turing machines and May-stay machines are equally expressive.
A complete proof will use the formal definitions of the machines.

{\it Hint: Include two directions of implications. First, let $M$ be an 
arbitrary Turing machine and prove that there's a May-stay machine 
that recognizes the language recognized by $M$. Next, let $M_S$ 
be an arbitrary May-stay 
machine and prove that there's a Turing machine that recognizes the language
recognized by $M_S$.}

\item\gradeCorrectFirst Let's define a new machine model, and call it the {\bf Double-move}  machine.
The Double-move machine model is the same as the usual Turing machine model,  except that on each transition, the tape head may move L, move R one cell, or 
move R two cells. Formally: a Double-move machine is given by the $7$-tuple 
$(Q, \Sigma, \Gamma, \delta, q_0, q_{accept}, q_{reject})$ where 
$Q$ is a finite set with $q_0 \in Q$ and $q_{accept} \in Q$
and $q_{reject} \in Q$ and $q_{accept} \neq q_{reject}$, 
$\Sigma$ and $\Gamma$ are alphabets and $\Sigma \subseteq \Gamma$ and $\square \in \Gamma$ and $\square \notin \Sigma$, and the transition function 
has signature
\[
  \delta: Q \times \Gamma \to Q \times \Gamma \times \{L, R, T\}
\]
where $L$ means that the read-write head moves to the left one cell (or stays
put if it's at the leftmost cell already), $R$ means that the read-write head 
moves one cell to the right , and $T$ means that the read-write head moves 
two cells to the right. The notion of computation and acceptance are analogous to that from Turing machines.

Prove that Turing machines and Double-move machines are equally expressive.
A complete proof will use the formal definitions of the machines.

{\it Hint: Include two directions of implications. First, let $M$ be an 
arbitrary Turing machine and prove that there's a Double-move machine 
that recognizes the language recognized by $M$. Next, let $M_D$ 
be an arbitrary Double-move
machine and prove that there's a Turing machine that recognizes the language
recognized by $M_D$.}

\item \gradeComplete In your proofs of equal expressivity in the previous 
parts of this question, you proved that a language is recognizable by some
Turing machine if and only if it is recognizable by some May-stay machine
or by some Double-move machine. Do your proofs also prove that a language is 
decidable by some
Turing machine if and only if it is decidable by some May-stay machine
or by some Double-move machine? Justify your answer.
\end{enumerate}

%%%%%%%%%%% PROBLEM 2 %%%%%%%%%%%
\item\textbf{Modifying machines} (12 points)

\begin{enumerate}
\item\gradeCorrect
Suppose a friend suggests that the following construction can be used
to prove that the class of decidable languages is closed under intersection.

Construction: given deciders $M_1$ and $M_2$ build the following machine $M$
\begin{align*}
    M &= ``\text{On input }w:\\
     &\text{1. Run $M_1$ on input $w$.}\\
     &\text{2. If $M_1$ accepts $w$, accept. } \\
     &\text{3. Run $M_2$ on input $w$.} \\
     &\text{4. If $M_2$ accepts $w$, accept.}\\
     &\text{5. If $M_2$ rejects $w$, reject."}\\
 \end{align*}

 Build a counterexample that could be used to convince your friend that this 
 construction doesn't work. A complete counterexample will include (1) a high-level description of $M_1$, (2) a high-level description of $M_2$, (3) a justification for why they provide a counterexample (that references
 the definitions of $M$ , decidable languages, and intersection).

 {\it Ungraded bonus:} Is it possible to change one line of the construction to make it work?

 \item\gradeCorrect
 Suppose a friend suggests that the following construction can be used
 to prove that the class of recognizable languages is closed under intersection.
 
 Construction: given Turing machines $M_1$ and $M_2$ build the following machine $M'$
\begin{align*}
    M' &= ``\text{On input }w:\\
     &\text{1. Run $M_1$ on input $w$.}\\
     &\text{2. If $M_1$ rejects $w$, reject.} \\
     &\text{3. Run $M_2$ on input $w$.} \\
     &\text{4. If $M_2$ rejects $w$, reject."}\\
 \end{align*}

 Build a counterexample that could be used to convince your friend that this 
 construction doesn't work. A complete counterexample will include (1) a high-level description of $M_1$, (2) a high-level description of $M_2$, (3) a justification for why they provide a counterexample (that references the definition of $M'$, recognizable languages, and intersection).


 {\it Ungraded bonus:} Is it possible to change one line of the construction to make it work?
\end{enumerate}

%%%%%%%%%%% PROBLEM 3 %%%%%%%%%%%
\item\textbf{Closure} (12 points):

For each language $L$ over an alphabet $\Sigma$, we have the 
associated sets of strings (also over $\Sigma$)
\[
    L^* = \{ w_1 \cdots w_k \mid k \geq 0 \textrm{ and each } w_i \in L\}
\]
and
\[
    SUBSTRING(L) = \{ w \in \Sigma^* ~|~ \text{there exist } x,y \in \Sigma^* \text{ such that } xwy \in L\}
\]
and 
\[
    EXTEND(L) = \{ w \in \Sigma^* ~|~ w = uv \text{ for some strings } u \in L \text{ and } v \in \Sigma^* \}
\]

\begin{enumerate}
\item[(a)]\gradeCorrect Prove whether this Turing machine construction below 
{\bf can} or {\bf cannot} be used to prove that the
class of recognizable languages over $\Sigma$ is closed under the 
Kleene star operation or the $SUBSTRING$ operation or the $EXTEND$ operation.

Suppose $M$ is a Turing machine over the alphabet $\Sigma$. 
Let $s_1, s_2, \ldots$ be a list of all strings in 
$\Sigma^*$ in string (shortlex) order.
We define a new Turing machine
by giving its high-level description as follows: 
\begin{align*}
   M_{a} &= ``\text{On input }w:\\
    &\text{1. For $n = 1, 2, \ldots$}\\
    &\text{2.~~~For $j = 1, 2, \ldots n$} \\
    &\text{3.~~~~~~For $k = 1, 2, \ldots, n$} \\
    &\text{4.~~~~~~~~~Run the computation of $M$ on $s_jws_k$ for at most $n$ steps}\\
    &\text{5.~~~~~~~~~If that computation halts and accepts within $n$ steps, accept.}\\
    &\text{6.~~~~~~~~~Otherwise, continue with the next iteration of this inner loop"}\\
\end{align*}

A complete and correct answer will either identify which operation works and give the proof of correctness why, for any Turing machine $M$, $L(M_{a})$ is equal to the result of applying this operation to $L(M$); {\bf or} give a counterexample (a recognizable set $A$ and a Turing machine $M$ recognizing $A$
and a description of why $L(M_a)$ where $M_a$ is the result of the construction 
applied to $M$ doesn't equal $A^*$ and doesn't equal $SUBSTRING(A)$ and doesn't equal $EXTEND(A)$.

\item[(b)]\gradeCorrect Prove whether this Turing machine construction below 
{\bf can} or {\bf cannot} be used to prove that the
class of recognizable languages over $\Sigma$ is closed under the 
Kleene star operation or the $SUBSTRING$ operation or the $EXTEND$ operation.

Suppose $M$ is a Turing machine over the alphabet $\Sigma$. 
Let $s_1, s_2, \ldots$ be a list of all strings in 
$\Sigma^*$ in string (shortlex) order.
We define a new Turing machine
by giving its high-level description as follows: 
\begin{align*}
    M_{b} &= ``\text{On input }w:\\
     &\text{1. For $n = 1, 2, \ldots$}\\
     &\text{2.~~~For $j = 0, \ldots, |w|$} \\
     &\text{3.~~~~~~Let $u$ be the string consisting of the first $j$ characters of $w$} \\
     &\text{4.~~~~~~Run the computation of $M$ on $u$ for at most $n$ steps}\\
     &\text{5.~~~~~~If that computation halts and accepts within $n$ steps, accept.}\\
     &\text{6.~~~~~~Otherwise, continue with the next iteration of this inner loop"}\\
 \end{align*}

A complete and correct answer will either identify which operation works and give the proof of correctness why, for any Turing machine $M$, $L(M_{b})$ is equal to the result of applying this operation to $L(M$); {\bf or} give a counterexample (a recognizable set $B$ and a Turing machine $M$ recognizing $B$
and a description of why $L(M_b)$ where $M_b$ is the result of the construction 
applied to $M$ doesn't equal equal $B^*$ and doesn't equal $SUBSTRING(B)$ and doesn't equal $EXTEND(B)$.
\end{enumerate}

%%%%%%%%%%% PROBLEM 4 %%%%%%%%%%%
\item \textbf{Computational problems} (8 points):
Recall the definitions of some example computational problems from class

\hspace{-30pt}
    \begin{tabular}{|lcl|}
    \hline
    \multicolumn{3}{|l|}{{\bf  Acceptance problem} } \\
    & & \\
    \ldots for DFA & $A_{DFA}$ & $\{ \langle B,w \rangle \mid  \text{$B$ is a  DFA that accepts input 
    string $w$}\}$ \\
    \ldots for NFA & $A_{NFA}$ & $\{ \langle B,w \rangle \mid  \text{$B$ is a  NFA that accepts input 
    string $w$}\}$ \\
    \ldots for regular expressions & $A_{REX}$ & $\{ \langle R,w \rangle \mid  \text{$R$ is a  regular
    expression that generates input string $w$}\}$ \\
    \ldots for CFG & $A_{CFG}$ & $\{ \langle G,w \rangle \mid  \text{$G$ is a context-free grammar 
    that generates input string $w$}\}$ \\
    \ldots for PDA & $A_{PDA}$ & $\{ \langle B,w \rangle \mid  \text{$B$ is a PDA that accepts input string $w$}\}$ \\
    & & \\
    \hline
    \multicolumn{3}{|l|}{{\bf Language emptiness  testing} } \\
    & & \\
    \ldots for DFA & $E_{DFA}$ & $\{ \langle A \rangle \mid  \text{$A$ is a  DFA and  $L(A) = \emptyset$\}}$ \\
    \ldots for NFA & $E_{NFA}$ & $\{ \langle A\rangle \mid  \text{$A$ is a NFA and  $L(A) = \emptyset$\}}$ \\
    \ldots for regular expressions & $E_{REX}$ & $\{ \langle R \rangle \mid  \text{$R$ is a  regular
    expression and  $L(R) = \emptyset$\}}$ \\
    \ldots for CFG & $E_{CFG}$ & $\{ \langle G \rangle \mid  \text{$G$ is a context-free grammar 
    and  $L(G) = \emptyset$\}}$ \\
    \ldots for PDA & $E_{PDA}$ & $\{ \langle A \rangle \mid  \text{$A$ is a PDA and  $L(A) = \emptyset$\}}$ \\
    & & \\
    \hline
    \multicolumn{3}{|l|}{{\bf Language equality testing} } \\
    & & \\
    \ldots for DFA & $EQ_{DFA}$ & $\{ \langle A, B \rangle \mid  \text{$A$ and $B$ are DFAs and  $L(A) =L(B)$\}}$\\
    \ldots for NFA & $EQ_{NFA}$ & $\{ \langle A, B \rangle \mid  \text{$A$ and $B$ are NFAs and  $L(A) =L(B)$\}}$\\
    \ldots for regular expressions & $EQ_{REX}$ & $\{ \langle R, R' \rangle \mid  \text{$R$ and $R'$ are regular
    expressions and  $L(R) =L(R')$\}}$\\
    \ldots for CFG & $EQ_{CFG}$ & $\{ \langle G, G' \rangle \mid  \text{$G$ and $G'$ are CFGs and  $L(G) =L(G')$\}}$ \\
    \ldots for PDA & $EQ_{PDA}$ & $\{ \langle A, B \rangle \mid  \text{$A$ and $B$ are PDAs and  $L(A) =L(B)$\}}$ \\
    \hline
    \end{tabular}

\begin{enumerate}
    \item[(a)] \gradeComplete Pick three of the computational problems above and give 
    examples (preferably different from the ones we talked about in class) of strings that are
    in each of the corresponding languages. Remember to use the 
    notation $\langle \cdots \rangle$ to denote the string encoding of relevant objects.
    {\it Extension, not for credit:} Explain why it's hard to write a specific string of 
    $0$s and $1$s and make a claim about membership in one of these sets.
    \item[(b)] \gradeComplete Computational problems can also be defined for Turing machines.
    Consider the two high-level descriptions of Turing machines below.
    Reverse-engineer them to define the computational problem that is being
    recognized, where $L(M_{DFA})$ is the language corresponding to this computational
    problem about DFA and $L(M_{TM})$ is the language corresponding to this computational
    problem about Turing machines. {\it Hint}: the computational problem is not acceptance,
    language emptiness, or language equality (but is related to one of them).

    Let $s_1, s_2, \ldots$ be a list of all strings in 
    $\{0,1\}^*$ in string (shortlex) order. Consider the following Turing machines
    \begin{align*}
        M_{DFA} &= ``\text{On input $\langle D \rangle$ where $D$ is a DFA}:\\
         &\text{1. for $i=1, 2, 3, \ldots$} \\
         &\text{2.~~~ Run $D$ on $s_i$} \\
         &\text{3.~~~~If it accepts, accept.}\\
         &\text{4.~~~~If it rejects, go to the next iteration of the loop"}\\
     \end{align*}
     and
     \begin{align*}
        M_{TM} &= ``\text{On input $\langle T \rangle$ where $T$ is a Turing machine}:\\
         &\text{1. for $i=1, 2, 3, \ldots$} \\
         &\text{2.~~~ Run $T$ for $i$ steps on each input $s_1, s_2, \ldots, s_i$ in turn} \\
         &\text{3.~~~~If $T$ has accepted any of these, accept.}\\
         &\text{4.~~~~Otherwise, go to the next iteration of the loop"}\\
     \end{align*}
\end{enumerate}


%%%%%%%%%%% PROBLEM 5 %%%%%%%%%%%
\item\textbf{Computational problems} (8 points):

\begin{enumerate}
    \item\gradeComplete Prove that the language $$\{\langle D \rangle \mid D \text{ is an NFA over $\{0,1\}$ and $D$ accepts at least $3$ strings 
    of length less than $5$ }\}$$
    is decidable.
    \item\gradeCorrect Prove that the language $$\{\langle R \rangle \mid R \text{ is a regular expression over $\{0,1\}$ and } L(R) \text{ has
    infinitely many strings starting with $0$} \}$$ is decidable.
\end{enumerate}

\end{enumerate}
\newpage

\title{HW6CSE105W25: Homework assignment 6}
\date{Due: March 13, 2025 at 5pm, via Gradescope}



\maketitle
\thispagestyle{fancy}

{\bf In this assignment,}

You will  practice analyzing, designing, and working with reductions to compare 
the difficulty level of computational problems.
You will explore various ways to encode machines as strings so that 
computational problems can be recognized.

{\bf Resources}: To review the topics 
for this assignment, see the class material from Weeks 8 and 9.
We will post frequently asked questions and our answers to them in a 
pinned Piazza post. 

{\bf Reading and extra practice problems}:  
Sipser Sections 4.2, 5.3, 5.1.
Chapter 4 exercises 4.9, 4.12.
Chapter 5 exercises 5.4, 5.5, 5.6, 5.7. 
Chapter 5 problems 5.22, 5.23, 5.24, 5.28

\instructions

You will submit this assignment via Gradescope
(\href{https://www.gradescope.com}{https://www.gradescope.com}) 
in the assignment called ``hw6CSE105W25''.

{\bf Assigned questions}
\begin{enumerate}[wide, labelwidth=!, labelindent=0pt]

%%%%%%%%%%% PROBLEM 1 %%%%%%%%%%%
\item \textbf{What's wrong with these reductions? (if anything)} (16 points):
Suppose your friends are practicing
coming up with mapping reductions $A \leq_m B$ and their witnessing
functions $f: \Sigma^* \to \Sigma^*$. For each of the following 
attempts, determine if it  has error(s) or is correct.
Do so by labelling each attempt with all and 
only the labels below that apply, and justifying
this labelling.
\begin{itemize}
\item \textit{Error Type 1:} The given function 
can't witness the claimed mapping reduction because there
exists an $x \in A$ such that $f(x) \not\in B$.
\item \textit{Error Type 2:} The given function 
can't witness the claimed mapping reduction because there 
exists an $x \not\in A$ such that $f(x) \in B$.
\item \textit{Error Type 3:} The given function 
can't witness the claimed mapping reduction because the specified
function is not computable.
\item \textit{Correct:} The 
claimed mapping reduction is true and 
is witnessed by the given function.
\end{itemize}

Clearly present your answer by
providing a brief (3-4 sentences or so) justification for 
whether {\bf each} of these labels applies to each example.


\begin{enumerate}
\item\gradeCompleteFirst $A_{\mathrm{TM}} \le_m HALT_{\mathrm{TM}}$ and 
\[
f(x) = \begin{cases}
 \scalebox{.7}{$\langle$ \hspace{-.5cm} \raisebox{-.4cm}{
\begin{tikzpicture}[->,>=stealth',shorten >=1pt, auto, node distance=2cm, semithick]
  \tikzstyle{every state}=[text=black, fill=none]
  \node[initial,state,accepting] (q0)                    {$q_{\mathrm{acc}}$};
 ;
\end{tikzpicture}}
$, \varepsilon \rangle$}  
& \text{if } x = \langle M, w \rangle \text{ for a Turing machine $M$ and string $w$}\\
& \qquad \qquad \text{ and } w \in L(M) \\

\scalebox{.7}{$\langle$ \hspace{-.5cm} \raisebox{-.4cm}{
\begin{tikzpicture}[->,>=stealth',shorten >=1pt, auto, node distance=2cm, semithick]
  \tikzstyle{every state}=[text=black, fill=none]
  \node[initial,state] (q0)                    {$q_0$};
  \node[state,accepting] (qacc) [right of = q0, xshift = 20]{$q_{acc}$};
  \path (q0) edge  [loop above] node {$0, 1, \blank \to R$} (q0)
 ;
\end{tikzpicture}}
$\rangle$} 
& \text{otherwise}
\end{cases} 
\]


\item\gradeComplete $A_{\mathrm{TM}} \le_m EQ_{\mathrm{TM}} $ with 
\[
f(x) = \begin{cases}
 \scalebox{.7}{$\langle$ \hspace{-.5cm} \raisebox{-.4cm}{
\begin{tikzpicture}[->,>=stealth',shorten >=1pt, auto, node distance=2cm, semithick]
  \tikzstyle{every state}=[text=black, fill=none]
  \node[initial,state,accepting] (q0)                    {$q_{\mathrm{acc}}$};
 ;
\end{tikzpicture}}
, $M_w \rangle$}  & \text{if } x = \langle M, w \rangle \text{ for a Turing machine $M$ and string $w$}\\\\
\scalebox{.7}{$\langle$ \hspace{-.5cm} \raisebox{-.4cm}{
    \begin{tikzpicture}[->,>=stealth',shorten >=1pt, auto, node distance=2cm, semithick]
      \tikzstyle{every state}=[text=black, fill=none]
      \node[initial,state,accepting] (q0)                    {$q_{\mathrm{acc}}$};
     ;
    \end{tikzpicture}}}
    , ~~~
    \scalebox{.7}{\hspace{-.5cm} \raisebox{-.4cm}{\begin{tikzpicture}[->,>=stealth',shorten >=1pt, auto, node distance=2cm, semithick]
        \tikzstyle{every state}=[text=black, fill=none]
        \node[initial,state] (qrej)                    {$q_{\mathrm{rej}}$};
        \node[state,accepting] (qacc) [right of=qrej]            {$q_{\mathrm{acc}}$};
       ;
      \end{tikzpicture}}$\rangle$ }  & \text{otherwise}.
\end{cases}
\]
Where for each Turing machine $M$, we  define 
\begin{align*}
    M_w = ``&\text{On input } y \\
    &1. \text{   Simulate $M$ on $w$.}\\
    &2. \text{   If it accepts, accept.}\\
    &3. \text{   If it rejects, reject."}
\end{align*}

\item\gradeCorrectFirst $HALT_{\mathrm{TM}} \le_m EQ_{\mathrm{TM}} $ with 
\[
f(x) = \begin{cases}
 \scalebox{.7}{$\langle$ \hspace{-.5cm} \raisebox{-.4cm}{
\begin{tikzpicture}[->,>=stealth',shorten >=1pt, auto, node distance=2cm, semithick]
  \tikzstyle{every state}=[text=black, fill=none]
  \node[initial,state,accepting] (q0)                    {$q_{\mathrm{acc}}$};
 ;
\end{tikzpicture}}
, $M_w \rangle$}  & \text{if } x = \langle M, w \rangle \text{ for a Turing machine $M$ and string $w$}\\\\
\scalebox{.7}{$\langle$ \hspace{-.5cm} \raisebox{-.4cm}{
    \begin{tikzpicture}[->,>=stealth',shorten >=1pt, auto, node distance=2cm, semithick]
      \tikzstyle{every state}=[text=black, fill=none]
      \node[initial,state,accepting] (q0)                    {$q_{\mathrm{acc}}$};
     ;
    \end{tikzpicture}}}
    , ~~~
    \scalebox{.7}{\hspace{-.5cm} \raisebox{-.4cm}{\begin{tikzpicture}[->,>=stealth',shorten >=1pt, auto, node distance=2cm, semithick]
        \tikzstyle{every state}=[text=black, fill=none]
        \node[initial,state] (qrej)                    {$q_{\mathrm{rej}}$};
        \node[state,accepting] (qacc) [right of=qrej]            {$q_{\mathrm{acc}}$};
       ;
      \end{tikzpicture}}$\rangle$ }  & \text{otherwise}.
\end{cases}
\]
Where for each Turing machine $M$, we  define 
\begin{align*}
    M_w = ``&\text{On input } y \\
    &1. \text{   If } y \text{ is not the empty string, accept.}\\
    &2. \text{   Else, simulate $M$ on $w$.}\\
    &3. \text{   If it accepts, accept.}\\
    &4. \text{   If it rejects, reject."}
\end{align*}

\item\gradeCorrect $\{w w \mid w \in \{0,1\}^* \} \leq \Sigma^*$ and
$f(x) = 11$ for each $x \in \{0,1\}^*$.

\item\gradeCorrect $\Sigma^* \le_m \{w w \mid w \in \{0,1\}^* \}$ and
$f(x) = 11$ for each $x \in \{0,1\}^*$.


\end{enumerate}

%%%%%%%%%%% PROBLEM 2 %%%%%%%%%%%
\item\textbf{Using mapping reductions} (14 points):
Consider the following computational problems we've discussed
\begin{align*}
A_{TM} &= \{ \langle M, w \rangle \mid M \text{ is a Turing machine, } w \text{ is a string and $M$ accepts $w$}\} \\
HALT_{TM} &= \{ \langle M, w \rangle \mid M \text{ is a Turing machine, } w \text{ is a string and $M$ halts on $w$}\} \\
E_{TM} &=  \{ \langle M \rangle \mid M \text{ is a Turing machine and } L(M) = \emptyset\} \\
EQ_{TM} &= \{ \langle M_1, M_2 \rangle \mid M_1, M_2 \text{ are both Turing machines and } L(M_1) = L(M_2) \}
\end{align*}
and the new computational problem
\begin{align*}
    IncludesEmptyString_{TM} &= \{ \langle M \rangle \mid M \text{ is a Turing machine and }\\
    &\text{$M$ accepts the empty string (and maybe other strings too)} \}
\end{align*}

\begin{enumerate}
\item[(a)] \gradeCorrect Give an example of a string that is an element of $IncludesEmptyString_{TM}$ and a string that is not an element of
$IncludesEmptyString_{TM}$ and briefly justify your choices.
\item[(b)] \gradeComplete Prove that $IncludesEmptyString_{TM}$ is not decidable by showing that $A_{TM} \leq_m IncludesEmptyString_{TM}$.
\item[(c)] \gradeCorrect Give a different proof that $IncludesEmptyString_{TM}$ is not decidable by showing that $HALT_{TM} \leq_m IncludesEmptyString_{TM}$.
\item[(d)] \gradeComplete Is $IncludesEmptyString_{TM}$ recognizable? Justify your answer.
\end{enumerate}

%%%%%%%%%%% PROBLEM 3 %%%%%%%%%%%
\item\textbf{Using mapping reductions} (14 points):
Consider the following computational problems we've discussed
\begin{align*}
A_{TM} &= \{ \langle M, w \rangle \mid M \text{ is a Turing machine, } w \text{ is a string and $M$ accepts $w$}\} \\
HALT_{TM} &= \{ \langle M, w \rangle \mid M \text{ is a Turing machine, } w \text{ is a string and $M$ halts on $w$}\} \\
E_{TM} &=  \{ \langle M \rangle \mid M \text{ is a Turing machine and } L(M) = \emptyset\} \\
EQ_{TM} &= \{ \langle M_1, M_2 \rangle \mid M_1, M_2 \text{ are both Turing machines and } L(M_1) = L(M_2) \}
\end{align*}
and the new computational problem
\begin{align*}
    NotIncludesEmptyString_{TM} &= \{ \langle M \rangle \mid M \text{ is a Turing machine and $M$ does not accept the empty string} \}
\end{align*}
\begin{enumerate}
\item[(a)] \gradeCorrect Prove that $NotIncludesEmptyString_{TM}$ is not the complement of $E_{TM}$ and is also not the complement of $IncludesEmptyString_{TM}$.
\item[(b)] \gradeComplete Prove that $NotIncludesEmptyString_{TM}$ is not decidable by showing that $\overline{HALT_{TM}} \leq_m NotIncludesEmptyString_{TM}$.
\item[(c)] \gradeCorrect Give a different proof that $NotIncludesEmptyString_{TM}$ is not decidable by showing that $\overline{A_{TM}} \leq_m NotIncludesEmptyString_{TM}$. 
\item[(d)] \gradeComplete Is $NotIncludesEmptyString_{TM}$ recognizable? Justify your answer.
\end{enumerate}


%%%%%%%%%%% PROBLEM 4 %%%%%%%%%%%
\item \textbf{Examples of languages} (6 points):

For each part of the question, use precise mathematical notation or English to define your examples
and then briefly justify why they work.


For each language $L$ over an alphabet $\Sigma$, we have the 
associated sets of strings (also over $\Sigma$)
\[
    L^* = \{ w_1 \cdots w_k \mid k \geq 0 \textrm{ and each } w_i \in L\}
\]
and
\[
    SUBSTRING(L) = \{ w \in \Sigma^* ~|~ \text{there exist } x,y \in \Sigma^* \text{ such that } xwy \in L\}
\]
and 
\[
    EXTEND(L) = \{ w \in \Sigma^* ~|~ w = uv \text{ for some strings } u \in L \text{ and } v \in \Sigma^* \}
\]

\begin{enumerate}
    \item\gradeCorrect Two undecidable languages $L_1$ and $L_2$ over the same alphabet
        whose union $L_1 \cup L_2$ is co-recognizable, or write {\bf NONE}
        if there is no such example (and explain why).
    
    \item\gradeCorrect An unrecognizable
        language $L_3$ for which $EXTEND(L_3)$ is regular
        or write {\bf NONE} if there is no such example (and explain why).
    
    
    \item\gradeComplete A co-recognizable language $L_4$ that is NP-complete,
         or write {\bf NONE} if there is no such example (and explain why).
        Recall the definition: A language $L$ over an  alphabet $\Sigma$ is called {\bf co-recognizable} if its complement,  defined
        as $\Sigma^* \setminus L  = \{ x  \in  \Sigma^* \mid x \notin  L \}$, is Turing-recognizable.

        {\it This part of the question uses definitions from Week 10 of the course.}
        
\end{enumerate}
    

\end{enumerate}
\newpage
\titleformat{\subsubsection}[runin]
   {\normalfont\bfseries}{}{}{}
   
\title{ProjectCSE105W25: Project}
\date{Due March 19, 2025 at 8am}


\maketitle

\thispagestyle{fancy}


\vspace{-20pt}

The CSE 105 project is designed for you to go deeper and 
extend your work on assignments 
and to see how some of the abstract notions we discuss can 
be implemented in concrete ways. 
The project is an individual assignment and has two tasks: 

Task 1: Checking the consistency of two regular properties, and

\vspace{-15pt}

Task 2: Illustrating a mapping reduction

In each task, you'll be implementing some of the theoretical concepts we've talked about in a programming language of your choosing, and then presenting your reasoning and demonstrating your code. Getting practice with this 
style of presentation is a good thing  for you to learn in general and a rich 
way for us to assess your skills. 

\vspace{-20pt}

\subsubsection*{What resources can you use?} This project must be completed individually, 
without any help from other people, including the course staff (other than logistics support if 
you get stuck with screencast). You should be the architect of your approach to the project.
You can refer to any of this quarter's CSE 105 offering (notes, readings, class videos, homework feedback). 
Tools for drawing state diagrams (like Flap.js and JFLAP and the PrairieLearn automata library) can be used to help draw the diagrams 
in the project too. However, do not copy / screenshot material that was not produced by you directly to your project writeup. Instead, you can refer to it in your own words and include a citation to the resource you referenced.

These resources should be more than enough.
If you are struggling to get started and want to look elsewhere online, 
you must acknowledge this by listing and citing any resources you consult 
(even if you do not explicitly quote them), including any large-language model style resources (ChatGPT, Bard, Co-Pilot, etc.). 
Link directly to them and include the name of the author / video creator, 
any and all search strings or prompts you used, and the reason you consulted this reference.  Also, a word of caution, make sure you validate and check 
any code produced by these aids. Last quarter there were a lot of examples
of project submissions that fed the prompt of the project 
directly to a LLM code  generator and got wrong implementations back.

{\bf Submitting the project} You will submit a PDF plus a video file (.mp4) for each. All file submissions will be in Gradescope. Upload the four files themselves. It is your responsibility to ensure that the files are playable within Gradescope. No Google Drive links, YouTube links, or .mov files.

\newpage
\subsubsection*{Your videos:} You may produce screencasts 
with any software you choose. 
One option is to record yourself with Zoom; a tutorial on how to use 
Zoom to record a 
screencast (courtesy of Prof. Joe Politz)  is here: 

\url{https://drive.google.com/open?id=1KROMAQuTCk40zwrEFotlYSJJQdcG_GUU}.

The video that was produced from that recording session in Zoom is here:

\url{https://drive.google.com/open?id=1MxJN6CQcXqIbOekDYMxjh7mTt1TyRVMl}

Please send an email to the instructor 
(minnes@ucsd.edu) if you have 
concerns about  the video / screencast components of this project or 
cannot complete projects in this style for some reason.

\vspace{-20pt}

\subsubsection*{Task 1: Checking the consistency of two regular properties}

When we have a list of desired properties, it's helpful to know whether there 
are any examples that satisfy *all* of them at once; 
in other words, whether the  properties are 
consistent with each other. For example, the properties of 
a string starting with $0$ and a string starting with $1$ are {\it not} consistent, 
because there isn't any example of a string that simultaneously starts with $0$ and with $1$.
In this part of the 
project, you'll use the decidability of the emptiness problem for DFA to 
build an algorithm that checks if two given regular properties are consistent.

Specifically:

\vspace{-20pt}

\begin{enumerate}
\item Write a program in Java, Python, JavaScript, C++ , or another programming language of your choosing that checks the consistency of two arbitrary regular properties.  The function input must be a {\bf pair of strings} and part of your work in this program  is to design string representations for any arbitrary DFA since those DFAs will be how you represent the properties. The function output must be a {\bf boolean}:
true (if the pair of strings represent consistent regular properties) or 
false (if the pair of strings do not represent consistent regular properties).
\begin{itemize}
   \item You might find it useful to use the algorithm for building a DFA that recognizes the {\bf intersection} of the languages of two DFA and the algorithm for testing the {\bf emptiness} of the language of a DFA.
   \item One test case for your program is the following: Consider the DFA $M_0$ and $M_1$ 

   \begin{tabular}{cc}
      $M_0$& $M_1$ \\
      \begin{tikzpicture}[->,>=stealth',shorten >=1pt, auto, node distance=2cm, semithick]
         \tikzstyle{every state}=[text=black, fill=none]
         
         \node[initial,state] (qs)          {$q_{start}$};
         \node[state,accepting]    (q0) [above right of=qs, xshift=20pt] {$q0$};
         \node[state]         (q1) [below right of=qs, xshift=20pt] {$q1$};
         
         \path (qs) edge  [bend left=0] node {$0$} (q0)
             (q0) edge [loop right] node {$0,1$} (q0)
             (qs) edge [bend left=0] node {$1$} (q1)
             (q1) edge [loop right] node {$0,1$} (q1)
         ;
      \end{tikzpicture}&
      \begin{tikzpicture}[->,>=stealth',shorten >=1pt, auto, node distance=2cm, semithick]
         \tikzstyle{every state}=[text=black, fill=none]
         
         \node[initial,state] (rs)          {$r_{start}$};
         \node[state]    (r0) [above right of=qs, xshift=20pt] {$r0$};
         \node[state,accepting]         (r1) [below right of=qs, xshift=20pt] {$r1$};
         
         \path (rs) edge  [bend left=0] node {$0$} (r0)
             (r0) edge [loop right] node {$0,1$} (r0)
             (rs) edge [bend left=0] node {$1$} (r1)
             (r1) edge [loop right] node {$0,1$} (r1)
         ;
      \end{tikzpicture}
   \end{tabular}

   and let $w_0$ be the string representing $M_0$ and let $w_1$ be the string representing $M_1$. Then the result of your program on the input $w_0, w_1$ 
   should be {\bf false} because the properties represented by $M_0$ and $M_1$ 
   are inconsistent.\footnote{To see why, notice that $L(M_0)$ is the set of strings that start with $0$ and $L(M_1)$ is the set of strings that start with $1$.}
   \item The algorithm you implement needs to work with any pair of strings given as input (you should first parse each string in the input to see if it is formatted to represent a DFA). Your explanation of the algorithm should be such that most programmers can replicate the algorithm correctly. If you would like, you may use aids such as co-pilot or ChatGPT to help you write this program. 
   However, you should test the code that is produced and be able to explain what it is doing. Your code needs to be well-organized and well-documented.
   As a header in your code file or as an appendix in your PDF submission, include a comment block describing any resources that were used to 
   help generate your code, including any and all prompts used in interactions 
   with LLM coding tools.
\end{itemize}
\item To demonstrate your program, you will show how it runs (at least) twice: once with the test input $w_0, w_1$ described above, and once with test input $x_0, x_1$ that you define to demonstrate when the program outputs {\bf true}. Your choice of $x_0, x_1$ needs to satisfy the following conditions:
\begin{itemize}
\item $x_0$ and $x_1$ each represent DFA over the same fixed alphabet, 
\item the DFA represented by $x_0$ and $x_1$ do not recognize the same language,
\item the DFA represented by $x_0$ and $x_1$ do not recognize the empty set or the set of all strings or the set of all strings starting with $0$ or the set of all strings starting with $1$.
\end{itemize}  
As part of your demonstration, describe your choice of $x_0$ and $x_1$, clearly 
specifying the DFAs they represent and the languages recognized by these DFAs.
Each demo run of the program should include: 
\begin{itemize}
\item Side-by-side view of an English / mathematical formulation of the properties and DFA corresponding to the demo run, along with their string representation as input strings to the program.
\item Talk-aloud trace of the running of your program on the input pair of strings representing the two properties for the demo. During the trace, talk about the software design choices you made
(e.g. which data structures are you using, etc.) and how they impact the program. Also, give credit to any resources you used to make these design choices or develop the code. 
\item Recording of actually running your program on the input pair of strings for this demo, including interpreting the output the program gives and connecting it to whether the properties represented by the input are consistent or not.
\end{itemize}
\end{enumerate}

\newpage 

{\bf Checklist for submission} For this task, you will submit a PDF file plus a 3-5 minute video.


\begin{itemize}
   \item[(PDF)] Submit a single PDF file that clearly describes the design choices you made, includes
   all relevant code, and includes all relevant information (definition, representation, justification) about the examples used to demonstrate your program and screenshots from running your program on the examples. The writeup should use precise language and notation for all terms and clearly communicate the goal and approach of your program.
   \item[(PDF)] The documentation for your program should include a description of how input strings are parsed to represent DFA, in general and for the specific examples of $w_0, w_1, x_0, x_1$.
   \item[(Video)] The video should start with your face and your student ID visible for a few seconds at the beginning, and introduce yourself audibly while on screen. You don't have to be on camera for the  rest of the video, though it's fine if you are. We are looking for a brief confirmation that  it's you creating the video and doing the work you submitted.
\item[(Video)] The video includes the full program demo twice (once with the input $w_0,w_1$ and once with the input $x_0,x_1$), and includes the 
a mathematical and/or English definition of the regular properties you are using, connected to their
representations as strings when input to the program, and the talk-aloud trace of running the program on these inputs and its output connecting back to the notion of consistency of regular properties.
\end{itemize}

{\bf Note}: Clarity and brevity are both important aspects of your video.  In previous years, we've seen 
students speed up their videos to get below the 5 minute upper bound. This is ok so long as it doesn't 
compromise clarity. If the graders need to slow your video down to understand
it, it may not earn full credit.



{\bf Possible extensions}: If you're enjoying working on this and want to go deeper, here are a few additions you can consider. You will not be graded on any of these, and you should still make sure your project has the core functionality described above, but these extensions give you an opportunity to explore further.
\begin{itemize}
   \item Build a preprocesing step so that the regular properties can be expressed using NFA or regular expressions (in addition to DFA). You'll need to think about how to represent NFA and/or regular expressions as strings, and then how to convert them to DFA.
   \item Extend your work so that your program can test for consistency of more than two regular properties.
\end{itemize}

\newpage
\subsubsection*{Task 2: Illustrating a mapping reduction}

We can use mapping reductions to prove that interesting computational 
problems are undecidable, building on 
the undecidability of other computational problems.
In this part of the project, you'll choose a specific {\bf mapping reduction}
from an undecidable language of your choice to 
$$EQ_{TM} = \{ \langle M_1, M_2 \rangle \mid M_1, M_2 \text{ are Turing machines and } L(M_1) = L(M_2) \}$$
and implement a computable function that witnesses it
using a  programming language of your choice (aka a high-level description of a Turing machine that computes it).
You will then demonstrate  how your construction works for some test examples.

Specifically:

\vspace{-20pt}

\begin{enumerate}
\item Choose an undecidable language (other than $EQ_{TM}$) that 
we discussed in class or in the homework
or in review quizzes or in the textbook . {\it Note:
if you'd like to consider an undecidable language we have not discussed instead, 
please check with Prof. Minnes first. 
You must do so no later than the start of Week 10.}
\item Write a program in Java, Python, JavaScript, C++ , or another programming language of your choosing that implements a computable function witnessing this mapping reduction.  The function input must be a {\bf string}  and the function 
output must be a {\bf string}. Part of your work in this program 
is to design string representations for arbitrary instances of the model of 
computation the computational problems being compared in the mapping reduction.
\begin{itemize}
   \item You may use our class material for ideas on the algorithm that your program will implement. The algorithm you implement needs to be general enough to decide whether each input string is in the language or not. Your explanation of the algorithm should be such that most programmers can replicate the algorithm correctly.
   \item If you would like, you may use aids such as co-pilot or ChatGPT to help you write this program. 
   However, you should test the code that is produced and be able to explain what it is doing. Your code needs to be well-organized and well-documented.
   As a header in your code file or as an appendix in your PDF submission, include a comment block describing any resources that were used to 
   help generate your code, including any and all prompts used in interactions 
   with LLM coding tools.
\end{itemize}

\item To demonstrate your program, you will need to run it for an example positive and negative instance. That is to say, if you are implementing 
a computable function witnessing $X \leq_m EQ_{TM}$, you will select one string that is in $X$ and one string that is not in $X$, and you will 
 demonstrate running your program on each of these strings and explain why 
 the output of the function is good.
\end{enumerate}

\newpage

{\bf Checklist for submission} For this task, you will submit a PDF file plus a 3-5 minute video.

\begin{itemize}
   \item[(PDF)] Submit a single PDF file that clearly describes the mapping reduction, including defining the undecidable language you 
   chose and the computable function that you will implement to witness its mapping reduction to $EQ_{TM}$, and includes all relevant code and documentation documentation for the program computing the function witnessing this mapping reduction. In particular, include a description of how input strings are parsed and how output strings correspond to input strings and a clear specification of your two example strings, explaining which is is a positive instance (and why) and which is a negative instance (and why not). The writeup should use precise language and notation for all terms and clearly communicate the goal and approach of your program.
   \item[(Video)] The video should start with your face and your student ID visible for a few seconds at the beginning, and introduce yourself audibly while on screen. You don't have to be on camera for the  rest of the video, though it's fine if you are. We are looking for a brief confirmation that  it's you creating the video and doing the work you submitted.
\item[(Video)] The video includes the mapping reduction you will be working with, and the 
example strings that you will be using, including explanations of why you chose this reduction and these strings (and why one of the strings is a positive instance and the other is a negative instance). The full program demo should be part of the video, with an explanation of the code and the software design choices you made and any resources you used, and then live screencasts running your code on each of your example inputs. 
Explain why the output of your program is what you would expect, by connecting the output of the 
program to the definition of the mapping reduction and your chosen parsing of input strings.
\end{itemize}

{\bf Note}: Clarity and brevity are both important aspects of your video.  In previous years, we've seen 
students speed up their videos to get below the 5 minute upper bound. This is ok so long as it doesn't 
compromise clarity. If the graders need to slow your video down to understand
it, it may not earn full credit.

\newpage

\end{document}