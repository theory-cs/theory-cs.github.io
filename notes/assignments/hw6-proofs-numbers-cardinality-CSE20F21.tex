\documentclass[12pt, oneside]{article}

\usepackage[letterpaper, scale=0.8, centering]{geometry}
\usepackage{fancyhdr}
\setlength{\parindent}{0em}
\setlength{\parskip}{1em}

\pagestyle{fancy}
\fancyhf{}
\renewcommand{\headrulewidth}{0pt}
\rfoot{{\footnotesize Copyright Mia Minnes, 2021, Version \today~(\thepage)}}

\author{CSE20F21}

\newcommand{\instructions}{{\bf For all HW assignments:}

Weekly homework may be done individually or in groups of up to 3 students. 
You may switch HW partners for different HW assignments. 
The lowest HW score will not be included in your overall HW average. 
Please ensure your name(s) and PID(s) are clearly visible on the first page of your homework submission.

All submitted homework for this class must be typed. 
Diagrams may be hand-drawn and scanned and included in the typed document. 
You can use a word processing editor if you like (Microsoft Word, Open Office, Notepad, Vim, Google Docs, etc.) 
but you might find it useful to take this opportunity to learn LaTeX. 
LaTeX is a markup language used widely in computer science and mathematics. 
The homework assignments are typed using LaTeX and you can use the source files 
as templates for typesetting your solutions\footnote{To use this template, copy the source file (extension \texttt{.tex}) 
to your working directory or upload to Overleaf.}.


{\bf Integrity reminders}
\begin{itemize}
\item Problems should be solved together, not divided up between the partners. The homework is
designed to give you practice with the main concepts and techniques of the course, 
while getting to know and learn from your classmates.
\item You may not collaborate on homework with anyone other than your group members.
You may ask questions about the homework in office hours (of the instructor, TAs, and/or tutors) and 
on Piazza (as private notes viewable only to the Instructors).  
You \emph{cannot} use any online resources about the course content other than the class material 
from this quarter -- this is primarily to ensure that we all use consistent notation and
definitions we will use this quarter.
\item Do not share written solutions or partial solutions for homework with 
other students in the class who are not in your group. Doing so would dilute their learning 
experience and detract from their success in the class.
\end{itemize}

}
\usepackage{amssymb,amsmath,pifont,amsfonts,comment,enumerate,enumitem}
\usepackage{currfile,xstring,hyperref,tabularx,graphicx,wasysym}
\usepackage[labelformat=empty]{caption}
\usepackage{xcolor}
\usepackage{multicol,multirow,array,listings,tabularx,lastpage,textcomp,booktabs}

% NOTE(joe): This environment is credit @pnpo (https://tex.stackexchange.com/a/218450)
\lstnewenvironment{algorithm}[1][] %defines the algorithm listing environment
{   
    \lstset{ %this is the stype
        mathescape=true,
        frame=tB,
        numbers=left, 
        numberstyle=\tiny,
        basicstyle=\rmfamily\scriptsize, 
        keywordstyle=\color{black}\bfseries,
        keywords={,procedure, div, for, to, input, output, return, datatype, function, in, if, else, foreach, while, begin, end, }
        numbers=left,
        xleftmargin=.04\textwidth,
        #1
    }
}
{}

\newcommand\abs[1]{\lvert~#1~\rvert}
\newcommand{\st}{\mid}

\newcommand{\cmark}{\ding{51}}
\newcommand{\xmark}{\ding{55}}




\title{HW6 Proofs, Numbers, and Cardinality}
\date{Due: Tuesday, November 23, 2021 at 11:00PM on Gradescope}

\begin{document}
\maketitle
\thispagestyle{fancy}

{\bf In this assignment,}

You will practice determining and justifying whether 
statements are true in multiple contexts.

Instructions and academic integrity reminders for all homework assignments in 
CSE20 this quarter are on the class website and on the hw1-definitions-and-notations
assignment.

You will submit this assignment via Gradescope
(\href{https://www.gradescope.com}{https://www.gradescope.com}) 
in the assignment called ``hw6-proofs-numbers-cardinality''.

{\bf Resources}: To review the topics you are working with 
for this assignment, see the class material from Weeks 6 through 8.
We will post frequently asked questions and our answers to them in a 
pinned Piazza post.


In your proofs and disproofs of statements below, justify each  step
by reference to  a component of the  following proof  strategies
we  have discussed so far, and/or to relevant definitions and calculations.
\begin{itemize}
    \item A counterexample can be used to prove that  $\forall x P(x)$ is {\bf false}.
    \item  A witness can be used to prove that  $\exists x P(x)$ is {\bf true}.
    \item {\bf Proof of universal by exhaustion}: To prove that $\forall x \, P(x)$
is true when $P$ has a finite domain, evaluate the predicate at {\bf each} domain element to confirm that it is always T.
    \item  {\bf Proof by universal generalization}: To prove that $\forall x \, P(x)$
is true, we can take an arbitrary element $e$ from the domain and show that $P(e)$ is true, without making any assumptions 
about $e$ other than that it comes from the domain.
    \item To  prove  that $\exists x P(x)$ is {\bf false}, write the universal statement that is 
    logically equivalent to its negation and then prove it true using universal generalization.
    \item {\bf Strategies for conjunction}: To prove that $p \land q$ is true, have two subgoals: 
    subgoal (1) prove $p$ 
is  true; and, subgoal (2) prove $q$ is true. To prove that $p \land q$ is false, it's enough to prove that $p$ is false.
 To prove that $p \land q$ is false, it's enough to prove that $q$ is false.
    \item {\bf Proof of Conditional by Direct Proof}: To prove that the implication $p \to q$ is true, 
    we can assume $p$ is true and use that assumption to show $q$ is true.
    \item {\bf Proof of Conditional by Contrapositive Proof}: To prove that the implication $p \to q$ is true, 
    we can assume $\neg q$ is true and use that assumption to show $\neg p$ is true.
    \item {\bf Proof of disjuction using equivalent conditional}: To prove that the 
    disjunction $p \lor q$ is true, we can rewrite it equivalently as $\lnot p \to q$ and
    then use direct proof or contrapositive proof.
    \item {\bf Proof by Cases}: To prove $q$ when we know $p_1 \lor p_2$, show that $p_1 \to q$ and $p_2 \to q$.
    \item
    {\bf Proof by Structural Induction}: To prove that $\forall x \in X \, P(x)$ where $X$ is a recursively defined set, prove two cases:
        
        \begin{tabularx}{\textwidth}{l X}
        Basis Step: & Show the statement holds for elements specified in the basis step of the definition. \\
        Recursive Step: & Show that if the statement is true for each of the elements used to construct
    new elements in the recursive step of the definition, the result holds for these new elements.
    \end{tabularx}
    
    \item {\bf Proof by Mathematical Induction}: To prove a universal quantification over the set of  all integers greater than  or  equal to some base integer $b$:
    
    \begin{tabularx}{\textwidth}{l X}
        Basis Step: & Show the statement holds for $b$. \\
        Recursive Step: & Consider an arbitrary integer $n$ greater than or  equal to  $b$, assume
        (as the {\bf induction hypothesis})  that the property holds  for $n$, and use  this and
        other facts to  prove that  the property holds for $n+1$.
    \end{tabularx}
    
    \item {\bf Proof by Strong Induction} To prove that a universal quantification over the set of all integers greater than or equal to some  base integer $b$ holds,  pick a  fixed nonnegative integer  $j$ and then: \hfill 
    
    \begin{tabularx}{\textwidth}{l X}
        Basis Step: & Show the statement holds for $b$, $b+1$, \ldots, $b+j$. \\
        Recursive Step: & Consider an arbitrary integer $n$ greater than or  equal to  $b+j$, assume
        (as the {\bf strong  induction hypothesis})  that the property holds  for {\bf each of} $b$, $b+1$, \ldots, $n$, 	
        and use  this and
        other facts to  prove that  the property holds for $n+1$.
    \end{tabularx}

    \item {\bf Proof by Contradiction} 

    To prove that a statement $p$ is true, pick another statement $r$ and once we show
    that $\neg p  \to (r \wedge  \neg r)$ then  we can conclude that  $p$ is  true.
    
    {\it Informally} The statement we care about can't possibly be false, so it must be true.
\end{itemize}

\newpage
{\bf Assigned questions}

\begin{enumerate}
    \item Recall the definition of the set of linked lists from class, and some associated functions.
    \[
        \begin{array}{ll}
        \textrm{Basis Step: } & [] \in L \\
        \textrm{Recursive Step: } & \textrm{If } l \in L\textrm{ and }n \in \mathbb{N} \textrm{, then } (n, l) \in L
        \end{array}
    \]
    The length function $length: L \to \mathbb{N}$ is defined by
    \[
        \begin{array}{llll}
        \textrm{Basis Step:} &  & length(~[]~) &= 0 \\
        \textrm{Recursive Step:} & \textrm{If } l \in L\textrm{ and }n \in \mathbb{N}\textrm{, then  } & length(~(n, l)~)  &= 1+ length(l)
        \end{array}
    \]
    The function $prepend : L \times \mathbb{N} \to L$ is defined by
    \[
        prepend(~(l, n)~) = (n, l)
    \]
    The function $append : L \times \mathbb{N} \to L$ is defined by
    \[
    \hspace{-40pt}\begin{array}{llll}
    \textrm{Basis Step:} & \textrm{If } m \in \mathbb{N}\textrm{ then } & append(~([], m)~) & = (m, []) \\
    \textrm{Recursive Step:} & \textrm{If } l \in L\textrm{ and }n \in \mathbb{N}\textrm{ and }m \in \mathbb{N}\textrm{, then  } & append(~(~(n, l), m~)~) &= (n, append(~(l, m)~)~)
    \end{array}
    \]
    \begin{enumerate}
        \item ({\it Graded for fair effort completeness}\footnote{Graded for fair effort completeness means 
        you will get full credit so long as your submission demonstrates honest 
        effort to answer the question. You will not be penalized for incorrect answers.}) Fill in the blanks in the following proof of the statement
        \[
            \forall l\in L~ \forall m \in \mathbb{N} ~(~length(prepend(~(l,m)~)) = length(append(~(l,m)~))~)
        \]
        {\bf Proof}: We proceed by structural induction on $L$.
        
        {\bf Basis step}: We need to show that 
        \[
            \forall m \in \mathbb{N}  ~(~length(prepend(~([],m)~)) = length(append(~([],m)~))~)
        \]
        Towards universal generalization, let $m$ be an arbitrary natural number. Calculating:
        \[
            LHS = \underline{\text{BLANK~1}}
        \]
        \[
            RHS = \underline{\text{BLANK~2}}
        \]
        Since $LHS = RHS$, the basis step is complete.

        {\bf Recursive step}: Consider an arbitrary: $l' \in L$, $n \in \mathbb{N}$, and we  assume
        as the {\bf induction hypothesis} that:
        \[
            \underline{\text{BLANK~3}}
        \]
        Our goal is to show that 
        \[
            \forall m \in \mathbb{N} ~(~length(prepend(~(~(n,l')~,m)~)) = length(append(~(~(n,l'),m)~))~)
        \]
        is also true. Let $m$ be an arbitrary natural number. Calculating: 
        \[
            LHS = \underline{\text{BLANK~4}}
        \]
        \[
            RHS = \underline{\text{BLANK~5}}
        \]
        Since $LHS = RHS$, the recursive step is complete.

        \item ({\it Graded for correctness}\footnote{Graded for correctness means your solution will be
        evaluated not only on the correctness of your answers, but on your ability to 
        present your ideas clearly and logically. You should explain how you arrived at your conclusions, using 
        mathematically sound reasoning. Whether you use formal proof techniques or write a more informal argument for why 
        something is true, your answers should always be well-supported. Your goal should be to convince the reader that 
        your results and methods are sound.}) Disprove the statement 
        \[
            \forall l\in L~ \forall m \in \mathbb{N} ~(~prepend(~(l,m)~) = append(~(l,m)~)~)
        \]

        \item ({\it Graded for correctness}) Determine whether the statement
        \[
            \exists l\in L~ \exists m \in \mathbb{N} ~(~prepend(~(l,m)~) = append(~(l,m)~)~)
        \]
        is true or false, and justify your conclusion using valid proof strategies.
    \end{enumerate}

    \item Recall the definition of the set of rational numbers,
    $$Q = \left\{ \frac{p}{q} \mid p \in \mathbb{Z}  \text{ and  } q  \in \mathbb{Z} \text{ and } q \neq  0 \right\}$$
    We define the set of {\bf irrational} numbers $\overline{\mathbb{Q}} = \mathbb{R} - \mathbb{Q}
    = \{ x \in \mathbb{R} \mid x \notin \mathbb{Q} \}$. 

    \begin{enumerate}
        \item ({\it Translation graded for fair effort completeness; Witness graded 
        for correctness}) Translate the statement to English and then give a witness 
        that could be used to prove the statement 
        \[
        \exists x \in \mathbb{Q}~ \forall y \in \overline{\mathbb{Q}} ~(x\cdot y \in \mathbb{Q})
        \]
        You do not need to justify your answer.  However, if you include clear explanations, 
        we may be able to give partial credit for an answer with some imprecision.
   
        \item ({\it Translation graded for fair effort completeness; Counterexample graded 
        for correctness}) Translate the statement to English and then give a counterexample 
        that could be used to disprove the statement
        \[
        \forall x \in \overline{\mathbb{Q}} ~\left(  x > 0 ~\to~ x \geq 1 \right)
        \]
        You do not need to justify your answer.  However, if you include clear explanations, 
        we may be able to give partial credit for an answer with some imprecision.
   
        \item ({\it Translation graded for fair effort completeness; Witness graded 
        for correctness}) Translate the statement to English and then give a witness that 
        could be used to prove the statement 
        \[
        \exists (x,y,z) \in \overline{\mathbb{Q}}\times\mathbb{Q}\times \mathbb{Q} ~(y \neq z \land x^ y= z)
        \]
        You do not need to justify your answer.  However, if you include clear explanations, 
        we may be able to give partial credit for an answer with some imprecision.
   
        \item ({\it Graded for fair effort completeness}\footnote{Fair effort completeness
        for this question means either attempting to correctly answer each part
        or to write a sentence or two on where you get stuck in your attempt to correctly answer
        the question.}) Fill in the blanks in the following argument.

        {\bf Claimed statement}: 
        $\exists x \in \overline{\mathbb{Q}}~ \exists y \in \overline{\mathbb{Q}} 
        ~\exists z\in \mathbb{Q} ~(x^y = z)$.
        
        \begin{quote}
        {\bf Proof}: We need to give a witness to prove this existential claim. 
        We proceed in a proof by cases, since the disjunction
        \textbf{(i)}$\underline{\phantom{\hspace{1.3in}}}$ is true.
        \begin{itemize}
        \item {\bf Case 1}: We need to show that 
        \[
            (\sqrt{2}^{\sqrt{2}} \in \mathbb{Q}) \to 
            \exists x \in \overline{\mathbb{Q}}~ \exists y \in \overline{\mathbb{Q}} 
            ~\exists z\in \mathbb{Q} ~(x^y = z)
        \]
        Assume towards a direct proof that \textbf{(ii)}$\underline{\phantom{\hspace{1.3in}}}$. We
        choose the witnesses $x = \sqrt{2}$, $y = \sqrt{2}$, $z = \sqrt{2}^{\sqrt{2}}$.  By the 
        theorem we proved in class, $\sqrt{2} \notin \mathbb{Q}$.
        Since $x = y = \sqrt{2}$, $x \in \overline{\mathbb{Q}}$ and  $y \in \overline{\mathbb{Q}}$.
        By the assumption of this direct proof, $z \in \mathbb{Q}$. Thus, the witnesses we picked
        are in the required domains.  Moreover, by definition, $z = x^y$, as required. Thus, the existential claim
        is proved and we have completed the direct proof required for this case.
   
        \item {\bf Case 2}: We need to show that 
        \[
            (\sqrt{2}^{\sqrt{2}} \in \overline{\mathbb{Q}}) \to 
            \exists x \in \overline{\mathbb{Q}}~ \exists y \in \overline{\mathbb{Q}} 
            ~\exists z\in \mathbb{Q} ~(x^y = z)
        \]
        Assume towards a direct proof that $(\sqrt{2}^{\sqrt{2}} \in \overline{\mathbb{Q}})$. We
        choose the witnesses 
        \[
        x = \textbf{(iii)}\underline{\phantom{\hspace{1.3in}}}, ~y = 
        \textbf{(iv)}\underline{\phantom{\hspace{1.3in}}}, ~z = \textbf{(v)}\underline{\phantom{\hspace{1.3in}}}
        \]  
        By the assumption of this direct proof, $x \in \overline{\mathbb{Q}}$. As we mentioned above, $\sqrt{2} \notin \mathbb{Q}$ so $y \in \overline{\mathbb{Q}}$. Picking $p = 2, q = 1$, we observe that $z = \frac{2}{1}$
        and since \textbf{(vi)}\underline{\phantom{\hspace{1.3in}}}, $z \in \mathbb{Q}$.  Thus, the 
        three witnesses we picked are in the required domains. 
        Calculating: 
        \[
        \left( \sqrt{2}^{\sqrt{2}}\right)^{\sqrt{2}} = \left( \sqrt{2} \right)^{\sqrt{2} \cdot \sqrt{2} }
        = \left(\sqrt{2} \right)^2 = \sqrt{2} \cdot \sqrt{2} = 2
        \]
        which proves that $\textbf{(vii)}\underline{\phantom{\hspace{1.3in}}}$, and hence $x,y,z$ are 
        the required witnesses. Thus, the existential claim
        is proved and we have completed the direct proof required for this case.
        \end{itemize}
        The proof by cases is now complete and the statement has been proved. QED
        \end{quote}

   \end{enumerate}
   
   \item Recall that  A {\bf hex color} is a nonnegative
   integer, $n$, that has a base $16$ fixed-width $6$ expansion
   $$n = (r_1r_2g_1g_2b_1b_2)_{16,6}$$ 
   where $(r_1r_2)_{16,2}$ is the red
   component, $(g_1g_2)_{16,2}$ is the green component, and $(b_1b_2)_{16,2}$ is the
   blue component.   For notational convenience, we define the set 
   $C = \{ x \in \mathbb{N} ~\mid~x  < 16^6 \}$.  This is the set of possible hex colors because these
   are all numbers that have hexadecimal fixed-width $6$ expansions. 
   \begin{enumerate}
    \item ({\it Graded for correctness}) Determine and briefly justify whether $C$ is finite, countably infinite, or uncountable.
    \item Consider the function $red: C \to C$ given by $red(~(r_1r_2g_1g_2b_1b_2)_{16,6}~) = (r_1r_20000)_{16,6}$.
        \begin{enumerate}
            \item ({\it Graded for correctness}) Determine whether $red$ is one-to-one, 
            and justify your conclusion using valid proof strategies.
            \item ({\it Graded for correctness}) Determine whether $red$ is onto, 
            and justify your conclusion using valid proof strategies.
        \end{enumerate}
   \end{enumerate}

    \item Consider the set of ratings in a 3-movie database 
    $R = \{ -1,0,1\} \times \{-1,0,1\} \times \{-1,0,1\}$ and the set of 
    bases of RNA strands $B = \{\A, \C, \U, \G\}$.
    \begin{enumerate}
        \item ({\it Graded for fair effort completeness}) Give a (well-defined) one-to-one function with domain $R$ and codomain $B$ or explain why there is no such function.
        \item ({\it Graded for fair effort completeness}) Give a (well-defined) one-to-one function with domain $B$ and codomain $R$ or explain why there is no such function.
        \item ({\it Graded for fair effort completeness}) Give a (well-defined) onto function with domain $R$ and codomain $B$ or explain why there is no such function.
        \item ({\it Graded for fair effort completeness}) Give a (well-defined) onto function with domain $B$ and codomain $R$ or explain why there is no such function.
    \end{enumerate}

    \rule{0.5\textwidth}{.4pt}

    {\it Sample calculation that can be used as reference for the detail expected 
    in your answer when specifying functions and reasoning about their properties:} 
    
    We give a (well-defined) function with domain $R$ and codomain $B$ that is neither one-to-one nor onto.

    Define $g: R \to B$ by, for $(x_1, x_2,x_3) \in R$,
    \[
        g(~(x_1, x_2, x_3)~) = \begin{cases} 
            \A \qquad &\text{if $x_1 = 1$} \\
            \C \qquad &\text{if $x_1 = 0$} \\
            \G \qquad &\text{if $x_1 = -1$}
        \end{cases}
    \]
    This function is well-defined because each ratings $3$-tuples is mapped to a unique base.
    However, this function is not one-to-one, as we can see from the counterexample: 
    $a = (1,1,1)$, $b = (1, 0,0)$. These are ratings $3$-tuples (in the domain) which are 
    distinct (they disagree about the second and third movies) but
    \[
        g(a) = g(~(1,1,1)~) = \A = g(~(1,0,0)~) = g(b)
    \]
    because the two ratings agree on the first movie. Distinct domain 
    elements getting mapped to the same codomain elements is a counterexample to injectivity.

    The function $g$ is also not onto, as we can see from the counterexample $\U$. This is an 
    element of the codomain which is not $f(x)$ for any $x$ in the domain, as we can see 
    from the piecewise definition of $g$, where in no case do we have the output value $\U$.
    \rule{0.5\textwidth}{.4pt}


\end{enumerate}


\end{document}    
