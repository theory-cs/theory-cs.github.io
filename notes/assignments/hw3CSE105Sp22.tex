\documentclass[12pt, oneside]{article}

\usepackage[letterpaper, scale=0.8, centering]{geometry}
\usepackage{fancyhdr}
\setlength{\parindent}{0em}
\setlength{\parskip}{1em}

\pagestyle{fancy}
\fancyhf{}
\renewcommand{\headrulewidth}{0pt}
\rfoot{{\footnotesize Copyright Mia Minnes, 2021, Version \today~(\thepage)}}

\author{CSE20F21}

\newcommand{\instructions}{{\bf For all HW assignments:}

Weekly homework may be done individually or in groups of up to 3 students. 
You may switch HW partners for different HW assignments. 
The lowest HW score will not be included in your overall HW average. 
Please ensure your name(s) and PID(s) are clearly visible on the first page of your homework submission.

All submitted homework for this class must be typed. 
Diagrams may be hand-drawn and scanned and included in the typed document. 
You can use a word processing editor if you like (Microsoft Word, Open Office, Notepad, Vim, Google Docs, etc.) 
but you might find it useful to take this opportunity to learn LaTeX. 
LaTeX is a markup language used widely in computer science and mathematics. 
The homework assignments are typed using LaTeX and you can use the source files 
as templates for typesetting your solutions\footnote{To use this template, copy the source file (extension \texttt{.tex}) 
to your working directory or upload to Overleaf.}.


{\bf Integrity reminders}
\begin{itemize}
\item Problems should be solved together, not divided up between the partners. The homework is
designed to give you practice with the main concepts and techniques of the course, 
while getting to know and learn from your classmates.
\item You may not collaborate on homework with anyone other than your group members.
You may ask questions about the homework in office hours (of the instructor, TAs, and/or tutors) and 
on Piazza (as private notes viewable only to the Instructors).  
You \emph{cannot} use any online resources about the course content other than the class material 
from this quarter -- this is primarily to ensure that we all use consistent notation and
definitions we will use this quarter.
\item Do not share written solutions or partial solutions for homework with 
other students in the class who are not in your group. Doing so would dilute their learning 
experience and detract from their success in the class.
\end{itemize}

}
\usepackage{amssymb,amsmath,pifont,amsfonts,comment,enumerate,enumitem}
\usepackage{currfile,xstring,hyperref,tabularx,graphicx,wasysym}
\usepackage[labelformat=empty]{caption}
\usepackage{xcolor}
\usepackage{multicol,multirow,array,listings,tabularx,lastpage,textcomp,booktabs}

% NOTE(joe): This environment is credit @pnpo (https://tex.stackexchange.com/a/218450)
\lstnewenvironment{algorithm}[1][] %defines the algorithm listing environment
{   
    \lstset{ %this is the stype
        mathescape=true,
        frame=tB,
        numbers=left, 
        numberstyle=\tiny,
        basicstyle=\rmfamily\scriptsize, 
        keywordstyle=\color{black}\bfseries,
        keywords={,procedure, div, for, to, input, output, return, datatype, function, in, if, else, foreach, while, begin, end, }
        numbers=left,
        xleftmargin=.04\textwidth,
        #1
    }
}
{}

\newcommand\abs[1]{\lvert~#1~\rvert}
\newcommand{\st}{\mid}

\newcommand{\cmark}{\ding{51}}
\newcommand{\xmark}{\ding{55}}




\title{HW3 : Nonregular Languages and Pushdown Automata}
\date{Due: 4/28/22 at 5pm (no penalty late submission until 8am next morning), via Gradescope}

\begin{document}
\maketitle
\thispagestyle{fancy}

{\bf In this assignment,}

You will practice distinguishing between regular and nonregular languages using both closure
arguments and the pumping lemma. You will also practice with the definition of pushdown automata.

{\bf Resources}: To review the topics you are working with 
for this assignment, see the class material from  Week 2 through Week 4.
We will post frequently asked questions and our answers to them in a 
pinned Piazza post.

{\bf Reading and extra practice problems}: Sipser Section 1.4, 2.2.
Chapter 1 exercises 1.29, 1.30. Chapter 1 problems 1.49, 1.50, 1.51. Chapter 2 exercises 2.5, 2.7.

{\bf Key Concepts}: Pumping lemma, pumping length, regular languages, 
nonregular languages, pushdown automata, stack.

\instructions

You will submit this assignment via Gradescope
(\href{https://www.gradescope.com}{https://www.gradescope.com}) 
in the assignment called ``HW3CSE105Sp22''.

\newpage
{\bf Assigned questions}
\begin{enumerate}
\item ({\it Graded for fair effort completeness}\footnote{This means 
you will get full credit so long as your submission demonstrates honest 
effort to answer the question. You will not be penalized for incorrect answers. 
To demonstrate your honest effort in answering the question, we ask that you 
include your attempt to answer *each* part of the question. If you get stuck 
with your attempt, you can still demonstrate your effort by explaining where 
you got stuck and what you did to try to get unstuck.})
Do the following for each of the following attempted ``proofs" that  a set is nonregular:
\begin{itemize}
\item[i] Find the (first and/or most significant) logical error in the ``proof" and describe why it's wrong.
\item[ii] Either prove that the set is actually regular (by finding a regular expression that describes it or 
a DFA/NFA that recognizes it, and justifying why) {\bf or} fix the proof so that it is logically sound.
\end{itemize}

\begin{enumerate}
\item The language $X_1 = \{ uw \mid \text{$u$ and
$w$ are strings over $\{0,1\}$ and have the same length} \}$.

\begin{quote}
``Proof" that $X_1$ is not regular using the Pumping Lemma: Let $p$ be 
an arbitrary positive integer. We will show that $p$ is not a pumping length for $X_1$. 

Choose $s$ to be the string $1^p 0^p$, which is in $X_1$ because
we can choose $u = 1^p$ and $w = 0^p$ which each have length $p$.
Since $s$ is in $X_1$ and has length greater than or equal to $p$, if $p$ were to be a
pumping length for $X_1$, $s$ ought to be pump'able. 
That is, there should be a way of dividing $s$ into parts $x,y,z$ where $s=xyz$,
$|y| >0$, $|xy| \leq p$, and for each $i \geq 0$, $xy^iz \in X_1$.
Suppose $x,y,z$ are such that $s = xyz$, $|y| > 0$ and $|xy| \leq p$.
Since the first $p$ letters of $s$ are all $1$ and $|xy| \leq p$, we know
that $x$ and $y$ are made up of all $1$s.  If we let $i=2$, we get 
a string $xy^iz$ that is not in $X_1$ because repeating $y$ twice adds $1$s to 
$u$ but not to $w$, and strings in $X_1$ are required to have $u$ and $w$ be the same
length. Thus, $s$ is not pumpable (even though it should have been if $p$ were to be a pumping length)
and so $p$ is not a pumping length for $X_1$.  Since $p$ was arbitrary, we have
demonstrated that $X_1$ has no pumping length.  By the Pumping Lemma, this implies that 
$X_1$ is nonregular.
\end{quote}


\item The language $X_2 = \{ u0w \mid \text{$u$ and
$w$ are strings over $\{0,1\}$ and have the same length} \}$.

\begin{quote}
``Proof" that $X_2$ is not regular using the Pumping Lemma: Let $p$ be 
an arbitrary positive integer. We will show that $p$ is not a pumping length for $X_2$. 

Choose $s$ to be the string $1^{p} 0^{p+1}$, which is in $X_2$ because
we can choose $u = 1^p$ and $w = 0^p$ which each have length $p$.
Since $s$ is in $X_2$ and has length greater than or equal to $p$, if $p$ were to be a
pumping length for $X_2$, $s$ ought to be pump'able. 
That is, there should be a way of dividing $s$ into parts $x,y,z$ where $s=xyz$,
$|y| >0$, $|xy| \leq p$, and for each $i \geq 0$, $xy^iz \in X_2$.
When $x = \varepsilon$ and $y = 1^{p}$ and $z = 0^{p+1}$,
we have satisfied that $s = xyz$, $|y| > 0$ (because $p$ is positive) and $|xy| \leq p$.
If we let $i=2$, we get 
the string $xy^iz = 1^{2p}0^{p+1}$ that is not in $X_2$ because its middle symbol is a $1$, not a $0$. 
Thus, $s$ is not pumpable (even though it should have been if $p$ were to be a pumping length)
and so $p$ is not a pumping length for $X_2$.  Since $p$ was arbitrary, we have
demonstrated that $X_2$ has no pumping length.  By the Pumping Lemma, this implies that 
$X_2$ is nonregular.
\end{quote}
\end{enumerate}

\item ({\it Graded for correctness}\footnote{This means your solution will be
evaluated not only on the correctness of your answers, but on your ability to 
present your ideas clearly and logically. You should explain how you arrived at 
your conclusions, using  mathematically sound reasoning. Whether you use formal proof techniques or 
write a more informal argument for why 
something is true, your answers should always be well-supported. Your goal 
should be to convince the reader that 
your results and methods are sound.}) Give an example of a language
over the alphabet $\{a,b,c\}$ that has cardinality $2$ and for which $4$ is a pumping length
and $3$ is not a pumping length.  A complete solution will give a clear and precise
description of the language, a justification for why $4$ is a pumping length, and a 
justification for why $3$ is not a pumping length.

\item  ({\it Graded for fair effort completeness})
Prove or disprove each of the following statements. (In other words, 
decide whether each statement is true or false and justify your decision.)
Fix $\Sigma$ an arbitrary (but unknown) alphabet.

\begin{enumerate}
\item If a language $L$ over $\Sigma$ is nonregular then its complement $\overline{L}$ is regular.
\item Each nonregular language over $\Sigma$ is infinite.
\item For each $w \in \Sigma^*$, there is a regular language $L_{w}$ such that $w \in L_{w}$.
\item For each $w \in \Sigma^*$, there is a nonregular language $L_{w}$ such that $w \in L_{w}$.
\item If a language over $\Sigma$ is recognized by a PDA then it is nonregular.
\end{enumerate}

\item ({\it Graded for correctness}) 
In the first week's homework, 
we saw the definitions of two functions on the set of languages over $\{0,1\}$:
for $L$ a set of strings over the alphabet $\{0,1\}$, we can define the following associated sets
\[
LZ(L) = \{ 0^k w \mid w \in L, k \in \mathbb{Z}, k \geq 0 \}
\]
\[
TZ(L) = \{ w 0^k \mid w \in L, k \in \mathbb{Z}, k \geq 0 \}
\]
This week we'll just focus on $LZ(L)$. 
In class and in the reading so far, we've seen the following examples of nonregular languages:
\begin{multicols}{3}
\begin{center}
$\{ 0^n 1^n ~|~ n \geq 0 \}$
$$\{ 0^n 1^n ~|~ n \geq 2 \}$$
$$\{ 0^n 1^m ~|~  0 \leq n \leq m \}$$
$$\{ 0^n 1^m ~|~ 0 \leq m \leq n \}$$
$$\{ 0^n 1^{2n} ~|~ 0 \leq n \}$$
$$\{ 0^n 1^{n+1} ~|~ 0 \leq n \}$$
$$\{ 1^{n^2} ~|~ 0 \leq n \}$$
$$\{ 0^n 1^m 0^n ~|~n,m \geq 0\}$$
$$\{ w \in \{0,1\}^* ~|~w = w^R\}$$
$$\{ w w^R ~|~ w \in \{0,1\}^*\}$$
\end{center}
\end{multicols}

Use (some of) the sets above, along with any regular sets you would like, to 
prove or disprove the statement: ``The class of nonregular languages
is closed under the function $LZ$.''

A complete solution will include a precise description of whether
the statement is true or false, referring back to the definition of closure, 
the definition of the function $LZ$, and the definition of nonregularity.
You may use any claims we proved in class or that are proved in the textbook reading,
so long as you reference them clearly in your argument by referring to a specific page 
in the notes, timestamp of a video, or page in the book.

{\it Bonus; not for credit: extend this homework problem for $TZ(L)$ as well.}

\item Consider the PDA with input
alphabet $\Sigma = \{ 0, 1\}$ and stack alphabet $\Gamma = \{\$, X\}$ 
and the following state diagram

\begin{center}
    \includegraphics[width=3in]{../../resources/machines/hw3PDA.png}
\end{center}

\begin{enumerate}
\item ({\it Graded for correctness}) Specify an example string $w_1$ over $\Sigma$ that is accepted by this PDA, or explain why there is no such 
example. A complete solution will include either (1) a precise and clear 
description of your example  string and a precise and clear description of the accepting computation
of the PDA on this string (potentially using diagrams like those we used in class when tracing PDA
computations) or (2) a sufficiently
general and correct argument why there is no such example, referring back to the relevant definitions.
\item ({\it Graded for correctness}) Specify an example string $w_2$ over $\Sigma$ that is {\bf not} accepted by this PDA, 
or explain why there is no such 
example. A complete solution will include either (1) a precise and clear 
description of your example  string and a precise and clear description of all possible computations
of the PDA on this string (potentially using diagrams like those we used in class when tracing PDA
computations) to show that none of them are accepting or (2) a sufficiently
general and correct argument why there is no such example, referring back to the relevant definitions.
\item ({\it Graded for completeness}) Is the language recognized by this PDA regular or nonregular? You might 
find it useful to first write out this language in set notation.
\item ({\it Graded for completeness}) Modify the set of accept states of this state diagram to get a different PDA
(with the same set of states, input alphabet, stack alphabet, start state, and transition function) 
that recognizes an {\bf infinite  regular language}, if possible. A complete solution will include either (1) the 
diagram of this new PDA and an explanation of why the language it recognizes
is both infinite and regular, or (2) a sufficiently general and correct argument for why there is no way to choose 
the set of accept states to satisfy this requirement.
\end{enumerate}
\end{enumerate}
\end{document}