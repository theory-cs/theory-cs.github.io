\documentclass[12pt, oneside]{article}

\usepackage[letterpaper, scale=0.8, centering]{geometry}
\usepackage{fancyhdr}
\setlength{\parindent}{0em}
\setlength{\parskip}{1em}

\pagestyle{fancy}
\fancyhf{}
\renewcommand{\headrulewidth}{0pt}
\rfoot{{\footnotesize Copyright Mia Minnes, 2021, Version \today~(\thepage)}}

\author{CSE20F21}

\newcommand{\instructions}{{\bf For all HW assignments:}

Weekly homework may be done individually or in groups of up to 3 students. 
You may switch HW partners for different HW assignments. 
The lowest HW score will not be included in your overall HW average. 
Please ensure your name(s) and PID(s) are clearly visible on the first page of your homework submission.

All submitted homework for this class must be typed. 
Diagrams may be hand-drawn and scanned and included in the typed document. 
You can use a word processing editor if you like (Microsoft Word, Open Office, Notepad, Vim, Google Docs, etc.) 
but you might find it useful to take this opportunity to learn LaTeX. 
LaTeX is a markup language used widely in computer science and mathematics. 
The homework assignments are typed using LaTeX and you can use the source files 
as templates for typesetting your solutions\footnote{To use this template, copy the source file (extension \texttt{.tex}) 
to your working directory or upload to Overleaf.}.


{\bf Integrity reminders}
\begin{itemize}
\item Problems should be solved together, not divided up between the partners. The homework is
designed to give you practice with the main concepts and techniques of the course, 
while getting to know and learn from your classmates.
\item You may not collaborate on homework with anyone other than your group members.
You may ask questions about the homework in office hours (of the instructor, TAs, and/or tutors) and 
on Piazza (as private notes viewable only to the Instructors).  
You \emph{cannot} use any online resources about the course content other than the class material 
from this quarter -- this is primarily to ensure that we all use consistent notation and
definitions we will use this quarter.
\item Do not share written solutions or partial solutions for homework with 
other students in the class who are not in your group. Doing so would dilute their learning 
experience and detract from their success in the class.
\end{itemize}

}
\usepackage{amssymb,amsmath,pifont,amsfonts,comment,enumerate,enumitem}
\usepackage{currfile,xstring,hyperref,tabularx,graphicx,wasysym}
\usepackage[labelformat=empty]{caption}
\usepackage{xcolor}
\usepackage{multicol,multirow,array,listings,tabularx,lastpage,textcomp,booktabs}

% NOTE(joe): This environment is credit @pnpo (https://tex.stackexchange.com/a/218450)
\lstnewenvironment{algorithm}[1][] %defines the algorithm listing environment
{   
    \lstset{ %this is the stype
        mathescape=true,
        frame=tB,
        numbers=left, 
        numberstyle=\tiny,
        basicstyle=\rmfamily\scriptsize, 
        keywordstyle=\color{black}\bfseries,
        keywords={,procedure, div, for, to, input, output, return, datatype, function, in, if, else, foreach, while, begin, end, }
        numbers=left,
        xleftmargin=.04\textwidth,
        #1
    }
}
{}

\newcommand\abs[1]{\lvert~#1~\rvert}
\newcommand{\st}{\mid}

\newcommand{\cmark}{\ding{51}}
\newcommand{\xmark}{\ding{55}}




\title{HW4 : Pushdown Automata and Context-free grammars}
\date{Due: May 2nd at 5pm (no penalty late submission until 8am next morning), via Gradescope}

\begin{document}
\maketitle
\thispagestyle{fancy}

\textbf{In this assignment:}

You will  practice with the definition of pushdown automata and context-free grammars and reason
about regular and context-free languages.

\textit{Resources}: To review the topics you are working with for this assignment, 
see the class material from Week 3 through Week 4. We will post frequently asked questions and 
our answers to them in a pinned Piazza post.

\textit{Reading and extra practice problems}: Sipser Sections 2.1, 2.2. 
Chapter 2 exercises 2.1, 2.2, 2.3, 2.4, 2.5, 2.6, 2.7, 2.9, 2.10, 2.11, 2.12, 2.13, 2.16, 2.17.

\textit{Key Concepts:} Pushdown automata, stack, context-free grammars, derivations, 
context-free languages.

\instructions

You will submit this assignment via Gradescope
(\href{https://www.gradescope.com}{https://www.gradescope.com}) 
in the assignment called ``hw4CSE105Sp23''.

\textbf{Requests from your TAs and tutors}
To help us with grading please 
\begin{itemize}
    \item Start each question on a new page.
    \item Label the start of each solution with {\bf Answer}.
\end{itemize}

\textbf{Assigned questions}


\begin{enumerate} 

%%%%%%%%%%% PROBLEM 1 %%%%%%%%%%%
\item \textbf{A PDA with multiple possibilities} (22 points): \\
Consider the PDA with input and stack alphabet $\Gamma = \{0,1,2\}$ whose ``unfinished" 
state diagram is given below:

\begin{center}
\begin{tikzpicture}[->,>=stealth',shorten >=1pt, auto, node distance=2.5cm, semithick]
  \tikzstyle{every state}=[text=black, fill=yellow!40]

  \node[initial,state, accepting] (q0)      {$q_0$};
  \node[state]         (q1) [right of=q0,  xshift=0cm] {$q_1$};
  \node[state]         (q2) [right of=q1,  xshift=1cm] {$q_2$};
  \node[state]         (q3) [above of=q2] {$q_3$};
  \node[state, accepting]         (q4) [below of=q2] {$q_4$};

  \path (q0) edge [loop above] node[align=left] {$0, \varepsilon; \varepsilon$ \\ $1, \varepsilon; \varepsilon$}  (q0)
  		edge node {$2, \varepsilon; 2$} (q1)
	(q1) 	edge node {$\varepsilon, \varepsilon; \varepsilon$} (q2)
		edge node[below, sloped] {$\varepsilon, \varepsilon; \varepsilon$} (q4)
	(q2) edge [loop right] node {$E_1$} (q2)
		edge [right] node {$\varepsilon, \varepsilon; \varepsilon$} (q3)
	(q3) edge [loop right] node {$E_2$} (q3)
		edge node[sloped] {$E_3$} (q1)
	(q4) edge [loop right] node[align=left] {$0, \varepsilon; \varepsilon$ \\ $1, \varepsilon; \varepsilon$} (q4)
 ;
\end{tikzpicture}
\end{center}

There are three labels ($E_1$, $E_2$, and $E_3$) on the edges that are unspecified. 
To be precise, each $E_i$ is of the form ``$x,y; z$'' where $x, y, z \in \Gamma_{\varepsilon}$ 
(recall $\Gamma_{\varepsilon} = \Gamma \cup \{\varepsilon\}$).

\begin{enumerate}
    \item\gradeCorrectFirst Prove that (no matter how the labels $E_1, E_2, E_3$ are specified), 
    the language recognized by this 
    PDA is infinite. A complete solution will include a precise
    description of an infinite collection of strings each 
    of which is accepted by the PDA, with 
    a precise and
    clear description of the accepting computation of the PDA on 
    each of these strings.

    \item\gradeCompleteFirst Prove/Disprove: Over all the possible choices for the labels $E_1, E_2, E_3$, 
    this PDA can only recognize finitely many languages. Justify your solution by referring back to the 
    relevant definitions.

    \item\gradeCorrect Recall that for $L \subseteq \Sigma^*$ with $\Sigma = \{0,1\}$, we define
    \begin{align*}
    \REP(L) &:= \{ w \in \Gamma^* \mid \text{between every pair of successive $2$'s in $w$ is a string in $L$}\}\\
    &\phantom{:}=\{w \in \Gamma^* \mid \text{for all } v \in \Sigma^* \text{ if } 2v2 \in \SUBSTRING(\{w\})  \text{, then } v \in L\} 
    \end{align*}
    where for all languages $K \subseteq \Gamma^*$ we let
    \[
    \SUBSTRING(K) := \{ w \in \Gamma^* \mid \text{there exist } a,b \in \Gamma^* \text{ such that } awb \in K\}.
    \]

    Determine how to set the labels $E_1, E_2, E_3$ so that the language of the PDA is 
    \[
    \REP(\{0^n1^m \mid n \ge 0, m \ge 0\})
    \]
    In addition to specifying each $E_i$, a complete justification 
    will include a precise description of why this choice of the $E_i$'s
    means that the PDA recognizes the language indicated.

    \item\gradeCorrect Determine how to set the labels $E_1, E_2, E_3$ so that the language of the PDA is 
    \[
    \REP(\{0^n1^n \mid n \ge 0\})
    \]
    In addition to specifying each $E_i$, a complete justification 
    will include a precise description of why this choice of the $E_i$'s
    means that the PDA recognizes the language indicated.
\end{enumerate}

%%%%%%%%%%% PROBLEM 2 %%%%%%%%%%%
\item \textbf{Grammar practice} (12 points): \\
For each of the languages listed below, 
define a context-free grammar 
$G = (V, \Sigma, R, S)$ that generates the 
language. Instead of formally justifying
your grammar, illustrate it by giving 
{\bf two examples} 
of strings in the language 
and their derivations using your grammar
and {\bf one example} of a string not
in the language with an explanation
of why it cannot appear 
on the right side of any derivation 
in your grammar. Choose your
examples so they are different enough 
to illustrate the role of 
as many of the variables 
in your grammar as possible.

\begin{enumerate}
    \item\gradeCorrect $\REP(\{0^n1^n \mid n \ge 0\})$

    \item\gradeCorrect $\{1^n = 1^a + 1^b \in \{1,=,+\}^* \mid a,b,n \ge 1 \text{ such that } a + b = n \}$
\end{enumerate}

%%%%%%%%%%% PROBLEM 3 %%%%%%%%%%%
\item \textbf{Substrings and regularity} (16 points): \\
For this problem, we fix the 
alphabet $\Gamma = \{0,1,2\}$. Recall the 
definition of the function $\SUBSTRING$ from 
Problem 1.
\begin{enumerate}
\item\gradeCorrect Prove that $\SUBSTRING(\{ 0^n 1^n \mid n \ge 0\})$ is regular.
A complete 
solution will include a 
precise definition of a DFA, NFA, or regular 
expression that recognizes or describes it, along 
with a brief justification
of your construction by explaining the role each 
state plays in the machine
and referring back to relevant definitions.
\item\gradeCorrect Prove that 
$\SUBSTRING(\{ 0^n 1^n 2^n \mid n \ge 0\})$
is not regular.
\item\gradeComplete Is  $\SUBSTRING(\{ 0^n 1^n 2^n \mid n \ge 0\})$
context-free?
Informally justify your answer, referring 
to class discussions and/or the textbook.

\end{enumerate}
%%%%%%%%%%% END PROBLEMS  %%%%%%%%%%%
\end{enumerate}

\end{document}