\documentclass[12pt, oneside]{article}

\usepackage[letterpaper, scale=0.89, centering]{geometry}
\usepackage{fancyhdr}
\setlength{\parindent}{0em}
\setlength{\parskip}{1em}

\usepackage{tikz}
\usetikzlibrary{automata,positioning,arrows}

\pagestyle{fancy}
\fancyhf{}
\renewcommand{\headrulewidth}{0pt}
\rfoot{\href{https://creativecommons.org/licenses/by-nc-sa/2.0/}{CC BY-NC-SA 2.0} Version \today~(\thepage)}

\usepackage{amssymb,amsmath,pifont,amsfonts,comment,enumerate,enumitem}
\usepackage{currfile,xstring,hyperref,tabularx,graphicx,wasysym}
\usepackage[labelformat=empty]{caption}
\usepackage{xcolor}
\usepackage{multicol,multirow,array,listings,tabularx,lastpage,textcomp,booktabs}

\lstnewenvironment{algorithm}[1][] {   
    \lstset{ mathescape=true,
        frame=tB,
        numbers=left, 
        numberstyle=\tiny,
        basicstyle=\rmfamily\scriptsize, 
        keywordstyle=\color{black}\bfseries,
        keywords={,procedure, div, for, to, input, output, return, datatype, function, in, if, else, foreach, while, begin, end, }
        numbers=left,
        xleftmargin=.04\textwidth,
        #1
    }
}
{}

\newcommand\abs[1]{\lvert~#1~\rvert}
\newcommand{\st}{\mid}

\newcommand{\cmark}{\ding{51}}
\newcommand{\xmark}{\ding{55}}
 
\begin{document}
\begin{flushright}
    \StrBefore{\currfilename}{.}
\end{flushright} \section*{Week5 monday}


These definitions are on pages 101-102.

\vspace{-20pt}

\begin{center}
    \begin{tabular}{|p{2.4in}cp{3.6in}|}
    \hline 
    {\bf Term} & {\bf Typical symbol} & {\bf Meaning} \\
     & or {\bf Notation} & \\
    \hline
    \hline
    {\bf Context-free grammar} (CFG) & $G$ & $G = (V, \Sigma, R, S)$ \\
    The set of {\bf variables}& $V$ & Finite  set of symbols that represent phases in production pattern\\
    The set of {\bf terminals} & $\Sigma$ & Alphabet of symbols of strings generated  by CFG \\
    & & $V \cap \Sigma = \emptyset$ \\
    The set of {\bf rules}& $R$ & Each rule is  $A \to u$ with $A \in V$ and $u  \in (V  \cup \Sigma)^*$\\
    The {\bf start} variable&  $S$  & Usually  on left-hand-side of first/ topmost rule \\
    & &\\
    {\bf Derivation} & $S \Rightarrow \cdots \Rightarrow w$& 
    Sequence  of substitutions in a  CFG (also written $S \Rightarrow^* w$). At each step, we can apply one rule 
    to one occurrence of a variable in the current string by substituting that occurrence of the variable with the 
    right-hand-side of the rule. The derivation must end when the current string has only terminals (no variables)
    because then there are no instances of variables to apply a rule to.\\
    Language {\bf generated} by the context-free grammar $G$ & $L(G)$ &The set of strings for which there is a derivation in $G$. 
    Symbolically: $\{  w \in \Sigma^* \mid S \Rightarrow^* w \}$ i.e. $$\{  w \in \Sigma^* \mid \text{there is  derivation in $G$ that ends
    in $w$} \}$$\\
    {\bf Context-free language} & & A language that is the language generated by some context-free grammar\\
    \hline
    \end{tabular}
\end{center}

\vfill

  
{\bf Examples of context-free grammars, derivations in those grammars, and the languages generated by those grammars}
  
$G_1 =  (\{S\}, \{0\}, R, S)$ with rules
  \begin{align*}
    &S \to 0S\\
    &S \to 0\\
  \end{align*}
  In  $L(G_1)$ \ldots 
  
\vfill

  Not in $L(G_1)$ \ldots 


  \vfill

\newpage
  $G_2 =  (\{S\}, \{0,1\}, R, S)$
  \[
  S \to 0S \mid 1S \mid \varepsilon
  \]
  In  $L(G_2)$ \ldots 
  
  \vspace{110pt}
  
  Not in $L(G_2)$ \ldots 

  \vspace{110pt}

  $(\{S, T\}, \{0, 1\}, R, S)$ with  rules
  \begin{align*}
  &S \to T1T1T1T \\
  &T \to  0T \mid 1T \mid \varepsilon
  \end{align*}

  In  $L(G_3)$ \ldots 
  
  \vspace{110pt}
  
  Not in $L(G_3)$ \ldots 

  \vspace{110pt}

\newpage
  $G_4 =  (\{A, B\}, \{0, 1\}, R, A)$ with rules
  \[
    A \to 0A0 \mid  0A1 \mid 1A0  \mid 1A1 \mid  1
  \]
  In  $L(G_4)$ \ldots 
  
  \vspace{110pt}
  
  Not in $L(G_4)$ \ldots 

  \vspace{110pt}


  \begin{comment}
    I moved the following to the review quiz.

    {\it Extra practice}: Is there a CFG $G$ with $L(G) = \emptyset$?


  Three different CFGs that each generate the  language $\{abba\}$
  
  \begin{align*}
  & ( \{ S, T, V, W\}, \{a,b\}, \{ S \to aT, T \to bV, V \to bW, W \to a\}, S)\\
  & \\ 
  & \\ 
  & \\ 
  & ( \{ Q \}, \{a,b\}, \{Q \to abba\}, Q) \\
  & \\ 
  & \\ 
  & \\
  & ( \{ X,Y \}, \{a,b\}, \{X \to aYa, Y \to bb\}, X) 
  & \\ 
  & \\ 
  \end{align*} 
\end{comment}
  \newpage
  Design a CFG to generate the  language $\{a^n b^n \mid  n  \geq  0\}$
  
  \vspace{100pt}
  
  {\it Sample derivation:} 
  
  \vspace{100pt}
  
  
  \vfill 


\newpage
 \vfill
\section*{Week5 wednesday}


Warmup:   Design a CFG to generate the  language $\{a^i b^j \mid j \geq i  \geq 0\}$
  
\vfill
{\it Sample derivation:} 

\vspace{100pt}



Design a PDA to recognize the  language $\{a^i b^j \mid j \geq i  \geq 0\}$
  
\vspace{100pt}


\vfill
\newpage

{\bf Theorem  2.20}: A language is  generated by some context-free  grammar
if  and only if it is recognized by some push-down automaton.

Definition: a language is called {\bf context-free} if it is the language generated by a context-free grammar.
The class of all context-free language over a given alphabet $\Sigma$ is called {\bf CFL}.

Consequences:
\begin{itemize}
    \item Quick proof that every regular language is context free 
    \item To prove closure of the class of context-free languages under a given operation, we can choose 
    either of two modes 
    of proof (via CFGs or PDAs) depending on which is easier
    \item To fully specify a PDA we could give its $6$-tuple formal definition or we could give its input 
alphabet, stack alphabet, and state diagram.
An informal description of a PDA is a step-by-step description of how its computations 
would process input strings; the reader should be able to reconstruct the state diagram or formal 
definition precisely from such a descripton. The informal description of a PDA can refer to some 
common modules or subroutines that are computable by PDAs:
\begin{itemize}
  \item PDAs can ``test for emptiness of stack'' without providing details. 
  {\it How?} We can always push a special end-of-stack symbol, $\$$, at the start, before processing
  any input, and then use this symbol as a flag.
  \item PDAs can ``test for end of input'' without providing details.
  {\it How?} We can transform a PDA to one where accepting states are only those reachable 
  when there are no more input symbols.
\end{itemize}

\end{itemize}




\vfill

\newpage
Suppose $L_1$ and $L_2$ are context-free languages over $\Sigma$.  {\bf Goal}:  $L_1 \cup L_2$  is  also context-free.

{\it Approach 1: with  PDAs}

Let $M_1 = ( Q_1, \Sigma, \Gamma_1, \delta_1, q_1, F_1)$ and
$M_2 = ( Q_2, \Sigma, \Gamma_2, \delta_2, q_2, F_2)$ be PDAs with 
$L(M_1) =  L_1$  and  $L(M_2) = L_2$.

Define $M = $

\vfill

{\it Approach  2: with CFGs}

Let $G_1 = (V_1, \Sigma, R_1, S_1)$  and   $G_2 = (V_2, \Sigma, R_2, S_2)$  be CFGs  with
$L(G_1) =  L_1$  and  $L(G_2) = L_2$.

Define $G = $

\vfill

\newpage
Suppose $L_1$ and $L_2$ are context-free languages over $\Sigma$.  {\bf Goal}:  $L_1 \circ L_2$  is  also context-free.


{\it Approach 1: with  PDAs}

Let $M_1 = ( Q_1, \Sigma, \Gamma_1, \delta_1, q_1, F_1)$ and
$M_2 = ( Q_2, \Sigma, \Gamma_2, \delta_2, q_2, F_2)$ be PDAs with 
$L(M_1) =  L_1$  and  $L(M_2) = L_2$.

Define $M = $

\vfill

{\it Approach  2: with CFGs}

Let $G_1 = (V_1, \Sigma, R_1, S_1)$  and   $G_2 = (V_2, \Sigma, R_2, S_2)$  be CFGs  with
$L(G_1) =  L_1$  and  $L(G_2) = L_2$.

Define $G = $

\vfill
\newpage
{\it Summary}

Over a fixed alphabet $\Sigma$, a language $L$ is {\bf regular}

\vspace{-20pt}
\begin{center}
    iff it is described by some regular expression \\
    iff it is recognized by some DFA\\
    iff it is recognized by some NFA
\end{center}

Over a fixed alphabet $\Sigma$, a language $L$ is {\bf context-free}

\vspace{-20pt}
\begin{center}
    iff it is generated by some CFG\\
    iff it is recognized by some PDA
\end{center}

{\bf Fact}: Every regular language is a context-free language.

{\bf Fact}: There are context-free languages that are not nonregular.

{\bf Fact}: There are countably many regular languages.

{\bf Fact}: There are countably inifnitely many context-free languages.

{\it Consequence}: Most languages are {\bf not} context-free!

{\bf Examples  of non-context-free languages}

\begin{align*}
    &\{ a^n b^n c^n \mid 0 \leq n , n \in \mathbb{Z}\}\\
    &\{ a^i b^j c^k \mid 0 \leq i \leq j \leq k , i \in \mathbb{Z}, j \in \mathbb{Z}, k \in \mathbb{Z}\}\\
    &\{ ww \mid w \in \{0,1\}^* \}
\end{align*}
(Sipser Ex 2.36, Ex 2.37, 2.38)

There is a Pumping Lemma for CFL that can be used to prove a specific language is non-context-free: 
If $A$ is a context-free language, there there
is a number $p$ where, if $s$ is any string in $A$ of length at least $p$, then $s$ may be divided 
into five pieces $s = uvxyz$ where (1) for each $i \geq 0$, $uv^ixy^iz \in A$, (2) $|uv|>0$, (3) $|vxy| \leq p$.
{\it We will not go into the details of the proof or application of Pumping Lemma for CFLs this quarter.}

\begin{comment}
##Moved to review quiz
A set $X$ is said to be {\bf closed} under an operation $OP$ if, for any elements in $X$, applying 
$OP$ to them gives an element in $X$.  


\begin{center}
\begin{tabular}{|c|l|}
\hline
True/False & Closure claim \\
\hline
True &  The set of integers is closed under multiplication. \\
& $\forall x \forall y \left( ~(x \in \mathbb{Z} \wedge y \in \mathbb{Z})\to xy \in \mathbb{Z}~\right)$ \\
\hline
True & For each set $A$, the power set of $A$ is closed under intersection. \\
& $\forall A_1 \forall A_2 \left( ~(A_1 \in \mathcal{P}(A) \wedge A_2 \in \mathcal{P}(A) \in \mathbb{Z}) \to A_1 \cap A_2 \in \mathcal{P}(A)~\right)$ \\
\hline
  & The class of regular languages over $\Sigma$ is closed under complementation. \\
  & \\
 \hline
  & The class of regular languages over $\Sigma$ is closed under union. \\
  & \\
 \hline
  & The class of regular languages over $\Sigma$ is closed under intersection. \\
  & \\
  \hline
  & The class of regular languages over $\Sigma$ is closed under concatenation. \\
  & \\
 \hline
  & The class of regular languages over $\Sigma$ is closed under Kleene star. \\
  & \\
\hline
    & The class of context-free languages over $\Sigma$ is closed under complementation. \\
  & \\
\hline
    & The class of context-free languages over $\Sigma$ is closed under union. \\
  & \\
\hline
    & The class of context-free languages over $\Sigma$ is closed under intersection. \\
  & \\
\hline
    & The class of context-free languages over $\Sigma$ is closed under concatenation. \\
  & \\
\hline
    & The class of context-free languages over $\Sigma$ is closed under Kleene star. \\
  & \\
\hline
\end{tabular}
\end{center}
\end{comment}

 \vfill
\section*{Week5 friday}


 \vfill
\section*{Week4 wednesday}


Regular sets are not the end of the story
\begin{itemize}
    \item Many nice / simple / important sets are not regular
    \item Limitation of the finite-state automaton model: Can't ``count", Can only remember finitely far into the past,
    Can't backtrack, Must make decisions in ``real-time"
    \item We know actual computers are more powerful than this model...
\end{itemize}

The {\bf next} model of computation. Idea: allow some memory of unbounded size. How? 
\begin{itemize}
    \item To generalize regular expressions: {\bf context-free grammars}\\
    \item To generalize NFA: {\bf Pushdown automata}, which is like an NFA with access to a stack: 
    Number of states is fixed, number of entries in stack is unbounded. At each step
    (1) Transition to new state based on current state, letter read, and top letter of stack, then
    (2) (Possibly) push or pop a letter to (or from) top of stack. Accept a string iff
    there is some sequence of states and some sequence of stack contents 
    which helps the PDA processes the entire input string and ends in an accepting state.
\end{itemize}

\vfill

\vfill

Is there a PDA that recognizes the nonregular language $\{0^n1^n \mid n \geq 0 \}$?

\vfill

\newpage


\includegraphics[width=4in]{../../resources/machines/Lect9PDA.png}

The PDA with state diagram above can be informally described as:
\begin{quote}
    Read symbols from the input. As each 0 is read, push it onto the stack. 
    As soon as 1s are seen, pop a 0 off the stack for each 1 read. 
    If the stack becomes empty and we are at the end of the input string, accept the input. 
    If the stack becomes empty and there are 1s left to read, 
    or if 1s are finished while the stack still contains 0s, or if any 0s
    appear in the string following 1s, 
    reject the input.
\end{quote}
    

Trace the computation of this PDA on the input string $01$.

\vfill
    
Trace the computation of this PDA on the input string $011$.

\vfill

\newpage
A PDA recognizing the set $\{ \hspace{1.5 in} \}$ can be informally described as:
\begin{quote}
    Read symbols from the input. As each 0 is read, push it onto the stack. 
    As soon as 1s are seen, pop a 0 off the stack for each 1 read. 
    If the stack becomes empty and there is exactly one 1 left to read, read that 1 and accept the input. 
    If the stack becomes empty and there are either zero or more than one 1s left to read, 
    or if the 1s are finished while the stack still contains 0s, or if any 0s appear in the input following 1s, 
    reject the input.
\end{quote}
Modify the state diagram below to get a PDA that implements this description:

\includegraphics[width=4in]{../../resources/machines/Lect9PDA.png}


\vfill
{\bf Definition} A {\bf pushdown automaton} (PDA) is  specified by a  $6$-tuple $(Q, \Sigma, \Gamma, \delta, q_0, F)$
where $Q$ is the finite set of states, $\Sigma$ is the input alphabet,  $\Gamma$ is the stack alphabet,
\[
    \delta: Q \times \Sigma_\varepsilon  \times  \Gamma_\varepsilon \to \mathcal{P}( Q \times \Gamma_\varepsilon)
\]
is the transition function,  $q_0 \in Q$ is the start state, $F \subseteq  Q$ is the set of accept states.
    
 \vfill
\section*{Week4 friday}




Draw the state diagram and give the formal definition of a PDA with $\Sigma = \Gamma$.

\vfill

Draw the state diagram and give the formal definition of a PDA with $\Sigma \cap \Gamma = \emptyset$.
    
\vfill

\newpage
For the PDA state diagrams below, $\Sigma = \{0,1\}$.


\begin{center}
\begin{tabular}{c c}
Mathematical description of language & State diagram of PDA recognizing language\\
\hline
& $\Gamma = \{ \$, \#\}$ \hspace{2.3in} \\
& \\
& \includegraphics[width=3.5in]{../../resources/machines/Lect10PDA1.png}\\
& \\
& \\
\hline
& $\Gamma = \{ {@}, 1\}$ \hspace{2.3in} \\
& \\
& \includegraphics[width=3.5in]{../../resources/machines/Lect10PDA2.png}\\
& \\
& \\
\hline
& \\
& \\
& \\
$\{ 0^i 1^j 0^k \mid i,j,k \geq 0 \}$ & \\
& \\
& \\
\end{tabular}
\end{center}

 \vfill
 {\it Note: alternate notation is to replace $;$ with $\to$}

\begin{comment}
{\it Extra practice}: Consider the state diagram of a PDA with input alphabet 
$\Sigma$ and stack alphabet $\Gamma$.

\begin{center}
\begin{tabular}{|c|c|}
\hline
Label & means \\
\hline
$a, b ; c$ when $a \in \Sigma$, $b\in \Gamma$, $c \in \Gamma$ 
& \hspace{3in} \\
& \\
& \\
& \\
& \\
&\\
\hline
$a, \varepsilon ; c$ when $a \in \Sigma$, $c \in \Gamma$ 
& \hspace{3in} \\
& \\
& \\
& \\
& \\
&\\
\hline
$a, b ; \varepsilon$ when $a \in \Sigma$, $b\in \Gamma$
& \hspace{3in} \\
& \\
& \\
& \\
& \\
&\\
\hline
$a, \varepsilon ; \varepsilon$ when $a \in \Sigma$
& \hspace{3in} \\
& \\
& \\
& \\
& \\
&\\
\hline
\end{tabular}
\end{center}


How does the meaning change if $a$ is replaced by $\varepsilon$?
\end{comment}


{\it Big picture}: PDAs were motivated by wanting to add some memory of unbounded size to NFA. How 
do we accomplish a similar enhancement of regular expressions to get a syntactic model that is 
more expressive?

DFA, NFA, PDA: Machines process one input string at a time; the computation of a machine on its input string 
reads the input from left to right.

Regular expressions: Syntactic descriptions of all strings that match a particular pattern; the language 
described by a regular expression is built up recursively according to the expression's syntax

{\bf Context-free grammars}: Rules to produce one string at a time, adding characters from the middle, beginning, 
or end of the final string as the derivation proceeds.

 \vfill
\section*{Week10 friday}


\begin{center}
    \begin{tabular}{|p{4in}|p{3.5in}|}
        \hline
        & \\
        {\bf Model of Computation} & {\bf Class of Languages}\\
        &\\
        \hline
        & \\
        {\bf Deterministic finite automata}:
        formal definition, how to design for a given language, 
        how to describe language of a machine?
        {\bf Nondeterministic finite automata}:
        formal definition, how to design for a given language, 
        how to describe language of a machine?
        {\bf Regular expressions}: formal definition, how to design for a given language, 
        how to describe language of expression?
        {\it Also}: converting between different models. &
        {\bf Class of regular languages}: what are the closure 
        properties of this class? which languages are not in the class?
        using {\bf pumping lemma} to prove nonregularity.\\
        & \\
        \hline
        & \\
        {\bf Push-down automata}:
        formal definition, how to design for a given language, 
        how to describe language of a machine?
        {\bf Context-free grammars}:
        formal definition, how to design for a given language, 
        how to describe language of a grammar? &
        {\bf Class of context-free languages}: what are the closure 
        properties of this class? which languages are not in the class?\\
        & \\
        \hline
        & \\
        Turing machines that always halt in polynomial time
        & $P$ \\
        & \\
        Nondeterministic Turing machines that always halt in polynomial time 
        & $NP$ \\
        & \\
        \hline
        & \\
        {\bf Deciders} (Turing machines that always halt): 
        formal definition, how to design for a given language, 
        how to describe language of a machine? &
        {\bf Class of decidable languages}: what are the closure properties 
        of this class? which languages are not in the class? using diagonalization
        and mapping reduction to show undecidability \\
        & \\
        \hline
        & \\
        {\bf Turing machines}
        formal definition, how to design for a given language, 
        how to describe language of a machine? &
        {\bf Class of recognizable languages}: what are the closure properties 
        of this class? which languages are not in the class? using closure
        and mapping reduction to show unrecognizability \\
        & \\
        \hline
    \end{tabular}
\end{center}

\newpage

{\bf Given a language, prove it is regular}

{\it Strategy 1}: construct DFA recognizing the language and prove it works.

{\it Strategy 2}: construct NFA recognizing the language and prove it works.

{\it Strategy 3}: construct regular expression recognizing the language and prove it works.

{\it ``Prove it works'' means \ldots}

\vspace{100pt}

{\bf Example}: $L  = \{ w \in \{0,1\}^* \mid \textrm{$w$ has odd number of $1$s or starts with $0$}\}$

Using NFA

\vfill

Using regular expressions

\vfill


\newpage

{\bf Example}: Select all and only the options that result in a true statement: ``To show 
a language $A$ is not regular, we can\ldots'' 

\begin{enumerate}
    \item[a.] Show $A$ is finite
    \item[b.] Show there is a CFG generating $A$
    \item[c.] Show $A$ has no pumping length
    \item[d.] Show $A$ is undecidable
\end{enumerate}

\newpage

{\bf Example}: What is the language generated by the CFG with rules
\begin{align*}
    S &\to aSb \mid bY \mid Ya \\
    Y &\to bY \mid Ya \mid \varepsilon 
\end{align*}

\newpage

{\bf Example}: Prove that the language 
$T = \{ \langle M \rangle \mid \textrm{$M$ is a Turing machine and $L(M)$ is infinite}\}$ 
is undecidable.

\newpage

{\bf Example}: Prove that the class of decidable languages is closed under concatenation.
 \vfill
\end{document}