\documentclass[12pt, oneside]{article}

\usepackage[letterpaper, scale=0.89, centering]{geometry}
\usepackage{fancyhdr}
\setlength{\parindent}{0em}
\setlength{\parskip}{1em}

\pagestyle{fancy}
\fancyhf{}
\renewcommand{\headrulewidth}{0pt}
\rfoot{\href{https://creativecommons.org/licenses/by-nc-sa/2.0/}{CC BY-NC-SA 2.0} Version \today~(\thepage)}

\usepackage{amssymb,amsmath,pifont,amsfonts,comment,enumerate,enumitem}
\usepackage{currfile,xstring,hyperref,tabularx,graphicx,wasysym}
\usepackage[labelformat=empty]{caption}
\usepackage[dvipsnames,table]{xcolor}
\usepackage{multicol,multirow,array,listings,tabularx,lastpage,textcomp,booktabs}

\lstnewenvironment{algorithm}[1][] {   
    \lstset{ mathescape=true,
        frame=tB,
        numbers=left, 
        numberstyle=\tiny,
        basicstyle=\rmfamily\scriptsize, 
        keywordstyle=\color{black}\bfseries,
        keywords={,procedure, div, for, to, input, output, return, datatype, function, in, if, else, foreach, while, begin, end, }
        numbers=left,
        xleftmargin=.04\textwidth,
        #1
    }
}
{}
\lstnewenvironment{java}[1][]
{   
    \lstset{
        language=java,
        mathescape=true,
        frame=tB,
        numbers=left, 
        numberstyle=\tiny,
        basicstyle=\ttfamily\scriptsize, 
        keywordstyle=\color{black}\bfseries,
        keywords={, int, double, for, return, if, else, while, }
        numbers=left,
        xleftmargin=.04\textwidth,
        #1
    }
}
{}

\newcommand\abs[1]{\lvert~#1~\rvert}
\newcommand{\st}{\mid}

\newcommand{\A}[0]{\texttt{A}}
\newcommand{\C}[0]{\texttt{C}}
\newcommand{\G}[0]{\texttt{G}}
\newcommand{\U}[0]{\texttt{U}}

\newcommand{\cmark}{\ding{51}}
\newcommand{\xmark}{\ding{55}}

 
\begin{document}
\begin{flushright}
    \StrBefore{\currfilename}{.}
\end{flushright} \section*{Netflix intro}


What data should we encode about each Netflix account holder to help us make effective recommendations?

\vfill

In machine learning, clustering can be used to group similar data for prediction and recommendation.  For example,
each Netflix user's viewing history can be represented as a $n$-tuple indicating their preferences about
movies in the database, where $n$ is the number of movies in the database.  People with similar tastes in movies can then be clustered to provide recommendations
of movies for one another.  Mathematically, clustering is based on a notion of distance between pairs of $n$-tuples.

\vfill
 \vfill
\section*{Data types}


\begin{center}
    \begin{tabular}{p{4.4in}p{2.8in}}
    {\bf  Term} & {\bf Examples}:\\
    &  (add additional examples from class)\\
    \hline 
    {\bf set} \newline
    unordered collection of elements & $7 \in \{43, 7, 9 \}$ \qquad $2 \notin \{43, 7, 9 \}$ \\
    {\it repetition doesn't matter} & \\
    {\it Equal means agree on membership of all elements}& \\
    \hline
    {\bf $n$-tuple} \newline
    ordered sequence of elements with $n$ ``slots" & \\
    {\it repetition matters, fixed length} &\\
    {\it Equal means corresponding components equal}& \\
    \hline
    {\bf string} \newline
    ordered finite sequence of elements each from specified
    set & \\
    {\it repetition matters, arbitrary finite length} &\\
    {\it Equal means same length and corresponding characters equal}
    \end{tabular}
    \end{center}

\vfill \vfill
\section*{Infinite finite sets def}


\begin{tabular}{clc}
    $\mathbb{Z}$ &  The  set of integers  & $\{ \ldots, -2, -1, 0,  1, 2, \ldots\}$ \\
    $\mathbb{Z}^+$ &  The  set of positive integers  & $\{1, 2, \ldots\}$ \\
    $\mathbb{N}$ &  The  set of nonnegative integers  & $\{0, 1, 2, \ldots\}$ \\
    $\mathbb{Q}$ &  The  set of rational numbers  & $\left\{ \frac{p}{q} \mid p \in \mathbb{Z}  \text{ and  } q  \in \mathbb{Z} \text{ and } q \neq  0 \right\}$\\
    $\mathbb{R}$ & The  set  of  real numbers &  \\
    \end{tabular}
    \[
    \underline{\phantom{\mathbb{Z}^+}} ~~\subsetneq~~ \underline{\phantom{\mathbb{N}~}} ~~\subsetneq ~~\underline{\phantom{\mathbb{Z}~}}~~ \subsetneq~~ \underline{\phantom{\mathbb{Q}~}} 
    ~~\subsetneq~~ \underline{\phantom{\mathbb{R}~}} 
    \]
    
    
    The above sets are all {\bf infinite}.
    
    
    A {\bf finite} set is one whose distinct elements can be counted by a natural number.
    
    {\it Examples of finite sets}: $\emptyset$ , $\{ \sqrt{2} \}$
    
    
    {\bf Motivating question}: Are some of the above sets {\it bigger than} others? \vfill
\section*{Musical chairs analogy}


{\it Analogy}: Musical chairs

\begin{multicols}{2}
\includegraphics[width=2in]{../../resources/images/musicalchairs.png}
\columnbreak

People try to sit down when the music stops

Person\sun~ sits in Chair 1,
Person\smiley~ sits in Chair 2,

Person\frownie~  is left standing!
\end{multicols}
What does this say about the number of chairs and the number of people? \vfill
\section*{Defined functions}


\fbox{\parbox{\textwidth}{{\bf Defining functions} A function is defined by its (1) domain, (2) codomain, and (3) rule assigning each 
element in the domain exactly one element in the codomain. The domain and codomain are nonempty sets.
The rule can be depicted as a table, formula, English description, etc.\\

(Rosen p139)
}}


{\it Example}: $f_A: \mathbb{R}^+ \to \mathbb{Q}$ with $f_A(x) = x$ is {\bf not} a well-defined function because

\vfill


{\it Example}: $f_B: \mathbb{Q} \to \mathbb{Z}$ with $f_B\left(\frac{p}{q}\right) = p+q$ is {\bf not} a well-defined function because

\vfill


{\it Example}: $f_C: \mathbb{Z} \to \mathbb{R}$ with $f_C(x) = \frac{x}{|x|}$ is {\bf not} a well-defined function because \vfill
\section*{Injective cardinality definition}


{\bf Definition}  (Rosen p141): A function $f: D  \to C$ is {\bf one-to-one} (or  injective) means for every $a,b$ in the domain $D$, 
if $f(a) = f(b)$ then  $a=b$.

{\bf Definition}:  For sets $A, B$, we say that  {\bf the  cardinality of $A$ is  no  bigger than the cardinality of  $B$}, and 
write $|A| \leq |B|$, to mean there is a  one-to-one function  with domain $A$  and codomain $B$. \vfill
\section*{Injective cardinality musical chairs}


{\it In the analogy}: The function $sitter: \{ Chair1, Chair2\} \to \{ Person\sun, Person\smiley, Person\frownie \}$ given
by $sitter(Chair1) = Person\sun$,  $sitter(Chair2) = Person\smiley$, is one-to-one and witnesses that 
\[
| \{ Chair1, Chair2\} | \leq |\{ Person\sun, Person\smiley, Person\frownie \}|
\] \vfill
\section*{Rna injective cardinality}


Let $S_2$ be the set of RNA strands of length 2.

\vspace{-20pt}

\begin{center}
\begin{tabular}{|c|p{5in}|}
\hline
Statement  &  True/False , justification \\
\hline
$| \{\A,\U,\G,\C\} | \leq |S_2 |$ &  \\
&\\&\\&\\
\hline
$| \{\A,\U,\G,\C\} \times \{\A, \U, \G,\C\} | \leq |S_2 |$ &  \\
&\\&\\&\\
\hline
\end{tabular}
\end{center} \vfill
\section*{Surjective cardinality definition}


{\bf Definition}  (Rosen p143): A function $f: D  \to C$ is {\bf onto} (or  surjective) means for every $b$ in the codomain, 
there  is an element $a$ in the domain with  $f(a) = b$.


Formally, $f: D  \to  C$ is  onto  means $\underline{\phantom{\forall b \in C  \exists a \in D ( f(a) = b)}}$.


{\bf Definition}:  For sets $A, B$, we say that  {\bf the  cardinality of $A$ is  no  smaller than the cardinality of  $B$}, and 
write $|A| \geq |B|$, to mean there is an onto function  with domain $A$  and codomain $B$.
 \vfill
\section*{Surjective cardinality musical chairs}


{\it In the analogy}: The function $triedToSit: \{ Person\sun, Person\smiley, Person\frownie \} \to  \{ Chair1, Chair2\} $ given
by $triedToSit(Person\sun) = Chair1$,  $triedToSit(Person\smiley) = Chair2$, 
$triedToSit(Person\frownie) = Chair2$, is onto and witnesses that 
\[
 |\{ Person\sun, Person\smiley, Person\frownie \}| \geq | \{ Chair1, Chair2\} |
\] \vfill
\section*{Rna surjective cardinality}


Let $S_2$ be the set of RNA strands of length 2.

\vspace{-20pt}

\begin{center}
\begin{tabular}{|c|p{5in}|}
\hline
Statement  &  True/False , justification \\
\hline
$ |S_2 | \geq | \{\A,\U,\G,\C\} |$ &  \\
&\\&\\&\\
\hline
$ |S_2 | \geq | \{\A,\U,\G,\C\} \times \{\A, \U, \G,\C\} |$ &  \\
&\\&\\&\\
\hline
\end{tabular}
\end{center} \vfill
\section*{Bijection definition}


{\bf Definition}  (Rosen p144): A function $f: D  \to C$ is a {\bf bijection} means that it is both  one-to-one  and onto.
The {\bf inverse} of a  bijection $f: D  \to  C$ is  the function $g: C  \to  D$  such that $g(b) = a$ iff  $f(a) =  b$.


\fbox{\parbox{\textwidth}{For nonempty sets $A, B$ we say
\begin{align*}
|A| \leq |B| &\text{ means there is a one-to-one function with domain $A$, codomain $B$} \\
|A| \geq |B| &\text{ means there is an onto function with domain $A$, codomain $B$} \\
|A| = |B| &\text{ means there is a bijection with domain $A$, codomain $B$}
\end{align*}
}} \vfill
\section*{Cardinality properties}


{\bf Properties of cardinality}
\begin{align*}
&\forall A ~ (~  |A| = |A| ~)\\
&\forall A ~ \forall B ~(~ |A| = |B|  ~\to ~ |B| = |A|~)\\
&\forall A ~ \forall B ~ \forall C~ (~ (|A| = |B| ~\wedge~ |B| = |C|) ~\to ~ |A| = |C|~)
\end{align*}

{\it Extra practice with proofs:} Use the definitions of bijections to prove these properties. \vfill
\section*{Cantor schroder bernstein theorem}


{\bf Cantor-Schroder-Bernstein Theorem}: For all nonempty sets,
\[
|A| = |B| \qquad\text{if and only if} \qquad (|A| \leq |B| ~\text{and}~ |B| \leq |A|)
\qquad\text{if and only if} \qquad (|A| \geq |B| ~\text{and}~ |B| \geq |A|)
\]

To prove $|A| = |B|$,  we can do any {\bf one} of the following

\begin{itemize}\setlength{\itemsep}{-5pt}
\item Prove there exists  a bijection $f:  A \to B$;
\item Prove there exists a  bijection  $f: B  \to  A$;
\item Prove there exists two functions $f_1: A \to B$, $f_2: B \to  A$ where each of $f_1, f_2$ is one-to-one.
\item Prove there exists two functions $f_1: A \to B$, $f_2: B \to  A$ where each of $f_1, f_2$ is onto.
\end{itemize} \vfill
\section*{Logical equivalence}


\begin{tabular}{lp{4in}p{2in}}
{\bf Logical equivalence } &Two compound  propositions are {\bf logically  equivalent} means that  they 
have the  same  truth  values for all settings of truth  values to their propositional  variables.\\
{\bf Tautology} & A compound proposition that evaluates to true
for all settings of truth  values to its propositional  variables; it is  abbreviated $T$.\\
{\bf Contradiction} & A compound proposition that  evaluates  to  false 
for  all settings of truth  values to its propositional  variables; it  is abbreviated $F$.\\
{\bf Contingency} & A compound proposition that is neither a tautology nor a contradiction.\\
\end{tabular}
 \vfill
\section*{Logical equivalence extra note}


{\it Can replace $p$ and $q$ with any compound proposition}
 \vfill
\section*{Consistency def}


{\bf Definition}: A collection of  compound  propositions
is called {\bf consistent} if  there
is  an assignment  of  truth values
to  the  propositional variables that makes
each of the compound propositions  true.
 \vfill
\section*{Consistency example}


{\bf Consistency}: 
\begin{quote}
Whenever the system software is being upgraded, users cannot access the file system. 
If users can access the file system, then they can save new files. 
If users cannot save new files, then the system software is not being upgraded.
\end{quote}

\begin{enumerate}
\item Translate to symbolic compound propositions
\vfill
\item Look for some truth assignment to the propositional variables for which all the compound propositions output $T$
\vfill
\end{enumerate} \vfill
\section*{Redundancy algorithm}


Consider the following algorithm to introduce redundancy in a string of $0$s and $1$s.
\begin{algorithm}[caption={Create redundancy by repeating each bit three times}]
procedure $\textit{redun3}$($a_{k-1} \cdots a_0$: a binary string)
for $i$ := $0$ to $k-1$
  $c_{3i}$ := $a_i$
  $c_{3i+1}$ := $a_i$
  $c_{3i+2}$ := $a_i$
return $c_{3k-1} \cdots c_0$
\end{algorithm}

\begin{algorithm}[caption={Decode sequence of bits using majority rule on consecutive three bit sequences}]
procedure $\textit{decode3}$($c_{3k-1} \cdots c_0$: a binary string whose length is an integer multiple of $3$)
for $i$ := $0$ to $k-1$
  if exactly two or three of $c_{3i}, c_{3i+1}, c_{3i+2}$ are set to $1$
    $a_i$ := 1
  else 
    $a_i$ := 0
return $a_{k-1} \cdots a_0$
\end{algorithm}

Give a recursive definition of the set of outputs of the $redun3$ procedure, $Out$,

\vspace{-20pt}

\begin{itemize}
\item[] {\bf Basis step}: \underline{ \phantom{$\lambda \in Out$ \qquad}}
\item[] {\bf Recursive step}: \underline{\phantom{If $x \in Out$ then $x000 \in Out$ and $x111 \in Out$} }
\end{itemize}


Consider the message $m = 0001$ so that the sender calculates $redun3(m) = redun3(0001) = 000000000111$.

Introduce $\underline{\phantom{~~4~~}} $
errors into the message so that the signal received by the 
receiver is $\underline{\phantom{010100010101}}$
but the receiver is still able to decode the original message.

\vspace{-10pt}

{\it Challenge: what is the biggest number of errors you can introduce?} 

Building a circuit for line 3 in $decode$ procedure: given three input bits, we need to determine whether the
majority is a $0$ or a $1$.

\begin{center}
\begin{multicols}{2}\begin{tabular}{ccc|c}
$c_{3i}$ & $c_{3i+1}$ & $c_{3i+2}$ & $a_i$ \\
\hline
$1$ & $1$ & $1$ & $\phantom{1}$ \\
$1$ & $1$ & $0$ & $\phantom{1}$ \\
$1$ & $0$ & $1$ & $\phantom{1}$ \\
$1$ & $0$ & $0$ & $\phantom{0}$ \\
$0$ & $1$ & $1$ & $\phantom{1}$ \\
$0$ & $1$ & $0$ & $\phantom{0}$ \\
$0$ & $0$ & $1$ & $\phantom{0}$ \\
$0$ & $0$ & $0$ & $\phantom{0}$ \\\\
\end{tabular}
\columnbreak

Circuit 
\end{multicols}
\end{center} \vfill
\section*{Predicate definition}


{\bf  Definition}: A  {\bf predicate}  is  a function from a given set (domain) to $\{T,F\}$.

A predicate can be applied, or {\bf evaluated} at, an element of the domain. \vfill
\section*{Predicate equivalent definition}


Two predicates over the same domain are {\bf equivalent} means they evaluate to
the same truth values for all possible assignments of domain elements to the
input. \vfill
\section*{Predicate truth tables}


\begin{center}
    \begin{tabular}{c||c|c|c}
    Input & \multicolumn{3}{c}{Output} \\
    &$P(x)$ & $N(x)$ & $Mystery(x)$\\
    $x$ & $[x]_{2c,3} > 0$ & $[x]_{2c,3} < 0$& \\
    \hline
    $000$  & $F$ & & $T$\\
    $001$  & $T$ & & $T$\\
    $010$  & $T$ & & $T$\\
    $011$  & $T$ & & $F$\\
    $100$  & $F$ & & $F$\\
    $101$  & $F$ & & $T$\\
    $110$  & $F$ & & $F$\\
    $111$  & $F$ & & $T$\\
    \end{tabular}
    \end{center}
    
    The domain for each of the predicates $P(x)$, $N(x)$, $Mystery(x)$ is
    \underline{\phantom{$\{ b_1b_2b_3 ~\mid~ b_i \in \{0,1\} \textrm{ for each } i, 1 \leq i \leq 3 \}$}}. \vfill
\section*{Truth set definition}


{\bf Definition}: The {\bf truth  set} of a  predicate is the collection of all elements in its
domain where the predicate evaluates to $T$. \vfill
\section*{Truth set exercise}


The truth set for the predicate $P(x)$ is $\underline{\phantom{\{ x ~\mid~ P(x) = T\} = \{ 001, 010, 011 \}}}$.

The truth set for the predicate $N(x)$ is $\underline{\phantom{\{ x ~\mid~ N(x) = T\} = \{ 101, 111 \}}}$.

The truth set for the predicate $Mystery(x)$ is $\underline{\phantom{\{ x ~\mid~ Mystery(x) = T\} = \{ 000, 001, 010, 101, 111 \}}}$.


\vfill \vfill
\section*{Quantification definition}


{\bf Definitions} (Rosen 40-45): 

\vspace{-15pt}

The {\bf universal quantification} of $P(x)$ is the statement ``$P(x)$ for all values of $x$ in the domain''
and is written $\forall x P(x)$. An element for which $P(x) = F$ is called a {\bf counterexample} of $\forall x P(x)$.

The {\bf existential quantification} of $P(x)$ is the statement ``There exists an element $x$ in the domain such that $P(x)$'' and is written $\exists x P(x)$. An element for which $P(x) = T$ is called a {\bf witness} of $\exists x P(x)$.

{\bf Example}: 
\underline{\phantom{$\exists x ~(P(x) \to N(x))$}} is a true existential quantification. \vfill
\section*{Quantification logical equivalence}


Statements involving predicates and quantifiers are {\bf logically equivalent} 
means they have the same truth value no matter which predicates (domains and functions) are substituted in. 

{\bf Quantifier version of De Morgan's laws}: 
$\boxed{\neg \forall x P(x) ~\equiv~ \exists x \left( \neg P(x) \right)}$
\qquad
\qquad
$\boxed{\neg \exists x Q(x) ~\equiv~ \forall x \left( \neg Q(x) \right)}$


{\bf Example}: 
\underline{\phantom{$\forall x ~(P(x) \lor N(x))$}} is a false universal quantification. It is logically equivalent to \underline{\phantom{$\exists x ~\not (P(x) \lor N(x))$}}





\vfill \vfill
\section*{Rna strand recall}


Recall: Each RNA strand is a string whose symbols are elements of the set $B  = \{\A, \C, \G, \U \}$.
The {\bf set of all RNA strands} is called $S$.
The function \textit{rnalen} that computes the length of RNA strands in $S$ is:
\[
\begin{array}{llll}
& & \textit{rnalen} : S & \to \mathbb{Z}^+ \\
\textrm{Basis Step:} & \textrm{If } b \in B\textrm{ then } & \textit{rnalen}(b) & = 1 \\
\textrm{Recursive Step:} & \textrm{If } s \in S\textrm{ and }b \in B\textrm{, then  } & \textit{rnalen}(sb) & = 1 + \textit{rnalen}(s)
\end{array}
\] \vfill
\section*{Predicate rna example}


{\bf Example predicates on $S$}

\vspace{-20pt}


\begin{center}
\begin{tabular}{|p{4in}p{3.5in}|}
\hline
& \\
$H(s) = T$ & Truth set of $H$ is \underline{\phantom{$S$\hspace{1in}}}\\
\hline
$L_3(s) = \begin{cases}
T &\qquad\text{if $rnalen(s) = 3$} \\
F & \qquad\text{otherwise}
\end{cases}$ & 
Strand where $L_3$ evaluates to $T$ is e.g.\underline{\phantom{$\A\A\A$~\hspace{0.3in}}}

\vspace{10pt}

Strand where $L_3$ evaluates to $F$ is e.g. \underline{\phantom{$\A\A\U\A$\hspace{0.3in}}}\\
\hline
& \\
$F_{\A}$ is defined recursively by: 

~~Basis step: $F_{\A}(\A) = T$, $F_{\A}(\C) = F_{\A}(\G) = F_{\A}(\U) = F$

~~Recursive step: If $s \in S$ and $b \in B$, then $F_{\A}(sb) = F_{\A}(s)$& 
Strand where $F_{\A}$ evaluates to $T$ is e.g.\underline{\phantom{$\A\C\G$~\hspace{0.3in}}}

\vspace{10pt}

Strand where $F_{\A}$ evaluates to $F$ is e.g. \underline{\phantom{$\U\A\C\U$\hspace{0.3in}}}\\
\hline
& \\
$P_{\A\U\C}$ is defined as the predicate whose truth set
is the collection of RNA strands where the string $\A\U\C$
is a substring (appears inside $s$, in order and consecutively)& 
Strand where $P_{\A\U\C}$ evaluates to $T$ is e.g.\underline{\phantom{$\A\A\A$~\hspace{0.3in}}}

\vspace{10pt}

Strand where $P_{\A\U\C}$ evaluates to $F$ is e.g. \underline{\phantom{$\A\A\U\A$\hspace{0.3in}}}\\
\hline
\end{tabular}
\end{center}

\vfill \vfill
\section*{Cartesian product definition}


{\bf Definition} (Rosen p123): The {\bf Cartesian product} of the sets $A$ and $B$, $A \times B$, is the set of all ordered pairs $(a, b)$, where $a \in A$ and $b \in B$. 
That is: $A \times B = \{(a, b) \mid (a \in A) \land (b \in B)\}.$ The Cartesian product of the sets $A_1, A_2, \ldots ,A_n$, denoted by $A_1 \times A_2 \times \cdots \times A_n$, is the
set of ordered n-tuples $(a_1, a_2,...,a_n)$, where $a_i$ belongs to $A_i$ for $i = 1, 2,\ldots,n$. That is,
$A_1 \times A_2 \times \cdots \times A_n = \{(a_1, a_2,\ldots,a_n) \mid a_i \in A_i \textrm{ for } i = 1, 2,\ldots,n\}$
 \vfill
\section*{Rna strand recall}


Recall: Each RNA strand is a string whose symbols are elements of the set $B  = \{\A, \C, \G, \U \}$.
The {\bf set of all RNA strands} is called $S$.
The function \textit{rnalen} that computes the length of RNA strands in $S$ is:
\[
\begin{array}{llll}
& & \textit{rnalen} : S & \to \mathbb{Z}^+ \\
\textrm{Basis Step:} & \textrm{If } b \in B\textrm{ then } & \textit{rnalen}(b) & = 1 \\
\textrm{Recursive Step:} & \textrm{If } s \in S\textrm{ and }b \in B\textrm{, then  } & \textit{rnalen}(sb) & = 1 + \textit{rnalen}(s)
\end{array}
\] \vfill
\section*{Rna basecount example}


A function \textit{basecount} that computes the number of a given base $b$ appearing in a RNA strand $s$ is:
\[
\begin{array}{llll}
& & \textit{basecount} : S \times B & \to \mathbb{N} \\
\textrm{Basis Step:} &  \textrm{If } b_1 \in B, b_2 \in B & \textit{basecount}(b_1, b_2) & =
        \begin{cases}
            1 & \textrm{when } b_1 = b_2 \\
            0 & \textrm{when } b_1 \neq b_2 \\
        \end{cases} \\
\textrm{Recursive Step:} & \textrm{If } s \in S, b_1 \in B, b_2 \in B &\textit{basecount}(s b_1, b_2) & =
        \begin{cases}
            1 + \textit{basecount}(s, b_2) & \textrm{when } b_1 = b_2 \\
            \textit{basecount}(s, b_2) & \textrm{when } b_1 \neq b_2 \\
        \end{cases}
\end{array}
\]

\vfill

\begin{tabular}{p{3.5in}p{0.1in}p{3.5in}}
$L$ with domain $S \times \mathbb{Z}^+$ is defined by, for $s \in S$ and $n \in \mathbb{Z}^+$,
\[
L( s, n) = \begin{cases}
T &\qquad\text{if $rnalen(s) = n$}\\
F &\qquad\text{otherwise}\\
\end{cases}
\]
&&
$BC$ with domain $\underline{\phantom{S \times B \times \mathbb{N}}}$ is defined by, for $s \in S$ and $b \in B$ and $n \in \mathbb{N}$,
\[
BC( s, b, n) = \begin{cases}
T &\qquad\text{if $basecount(s,b) = n$}\\
F &\qquad\text{otherwise}\\
\end{cases}
\]\\
Element where $L$ evaluates to $T$: $\underline{\phantom{(\A, 1)\hspace{0.2in}}}$ & & Element where $BC$ evaluates to $T$: 
$\underline{\phantom{(\A, \A1)\hspace{0.2in}}}$ \\
& & \\
& & \\
Element where $L$ evaluates to $F$: $\underline{\phantom{(\A, 2)\hspace{0.2in}}}$  & & Element where $BC$ evaluates to $F$: 
$\underline{\phantom{(\A, \C, 1)\hspace{0.2in}}}$ \\
& & \\
& & \\
\end{tabular}

\vfill \vfill
\section*{Predicate notation}


{\bf Notation}: for a predicate $P$ with domain $X_1 \times \cdots \times X_n$ and a $n$-tuple $(x_1, \ldots, x_n)$ 
with each $x_i \in X$, we write $P(x_1, \ldots, x_n)$ to mean $P( ~(x_1, \ldots, x_n)~)$.
 \vfill
\section*{Rna basecount example two}


$\exists t ~BC(t)$ \qquad In English: \underline{\phantom{There exists an ordered $3$-tuple 
at which the predicate $BC$ evaluates to $T$.}}

Witness that proves this existential quantification is true: \underline{\phantom{$(\G\G, \G, 2)$ or $(\G\A\U\G, \G, 2)$)}}

$\forall (s,b,n) ~(~BC(s,b,n)~)$  \qquad In English: \underline{\phantom{For all ordered $3$-tuples
the predicate $BC$ evaluates to $T$.}}

Counterexample that proves this universal quantification is false: \underline{\phantom{$(\G\G, \A, 2)$ or $(\G\A\U\G, \G, 3)$)}}


{\bf New predicates from old} \qquad $BC(s,b,n)$ means $basecount(s,b) = n$.
\begin{center}
\begin{tabular}{|p{2.5in}p{1.5in}p{2in}|}
\hline
{\bf Predicate} & Domain &  Example domain element \\
& & where predicate is $T$\\
\hline
&& \\
$basecount(s,b) = 3$ & \phantom{$S \times B$}& 
\phantom{$(\A\U\A\A, \A)$}\\
\hline
&& \\
$basecount(s,\A) = n $&\phantom{$S \times \mathbb{N}$} & 
\phantom{$(\A\U\A,2)$}\\
\hline
&& \\
$\exists n \in \mathbb{N} ~(basecount(s,b) = n) $&\phantom{$S \times \mathbb{N}$} & 
\phantom{$(\A\U\A,2)$}\\
\hline
&& \\
$\forall b \in B ~(basecount(s,b) = 1) $&\phantom{$S$} & 
\phantom{$\A\C\G\U$}\\
\hline
\end{tabular}
\end{center} \vfill
\section*{Alternating quantifiers}


{\bf Alternating quantifiers}
$$\forall s \exists n ~BC(s,\A,n)$$

In English: \underline{\phantom{\hspace{3in}}}

\vfill

$$\exists n \forall s ~BC(s,\U,n)$$

In English: \underline{\phantom{\hspace{3in}}}


\vfill

Evaluate each quantified statement as $T$ or $F$.

\vspace{-20pt}

\begin{center}
\begin{tabular}{|c|c|c|}
\hline
&&\\
$\forall s ~\forall b ~\exists n~BC(s,b,n)$ & $\forall s ~\forall n ~ \exists b~BC(s,b,n)$ & $\forall b ~\forall n ~\exists s~BC(s,b,n)$ \\
&&\\
&&\\
\hline
&&\\
$\exists s ~\forall b ~\exists n~BC(s,b,n)$ & $\forall s~ \exists n ~\forall b~BC(s,b,n)$ & $\exists b ~\exists n ~\forall s~BC(s,b,n)$ \\
&&\\
&&\\
\hline
\end{tabular}
\end{center}

{\it Extra example}: Write the negation of each of the statements above, and use De Morgan's law to find a 
logically equivalent version where the negation is applied only to the $BC$ predicate (not next to a quantifier).

 \vfill
\section*{Making change example}


For which nonnegative integers  $n$ can we make change for $n$ with coins of 
value $5$ cents and $3$ cents?


Restating: We can make change for \underline{\phantom{$3$~~,~~ $5$~~,~~ $6$}}, we
cannot make change for \underline{\phantom{$1$~~,~~ $2$~~,~~ $4$~~,~~ $7$}}, and 
\[
\underline{\phantom{\forall n  \in  \mathbb{Z}^{\geq 8}  \exists x \in \mathbb{N}  \exists y \in \mathbb{N}  (  5x+3y =  n)\qquad \qquad}} \star
\]

\vfill \vfill
\section*{Strong induction def}


\fbox{\parbox{\textwidth}{

{\bf New! Proof by Strong Induction} (Rosen 5.2 p337)

To prove that a universal quantification over the set of all integers greater than or equal to some  base integer $b$ holds,  pick a  fixed nonnegative integer  $j$ and then: \hfill 

\begin{tabularx}{\textwidth}{lX}
    Basis Step: & Show the statement holds for $b$, $b+1$, \ldots, $b+j$. \\
    Recursive Step: & Consider an arbitrary integer $n$ greater than or  equal to  $b+j$, assume
    (as the {\bf strong  induction hypothesis})  that the property holds  for {\bf each of} $b$, $b+1$, \ldots, $n$, 	
    and use  this and
    other facts to  prove that  the property holds for $n+1$.
\end{tabularx}
}} 

\vfill

\begin{center}
    \begin{tabular}{cll}
    $\mathbb{N}$  &  The set of  natural numbers & $\{ 0, 1, 2, 3, \ldots \}$ \\
$\mathbb{Z}^{\geq b}$ & The set of integers greater than  or equal  a  basis element $b$ & $\{ b, b+1, b+2, b+3,  \ldots  \}$ \\
    \end{tabular}
    \end{center}
    
    \vfill \vfill
\section*{Making change proof two ways}


\setlength{\columnseprule}{0.4pt}
\begin{multicols}{2}
{\bf  Proof of $\star$ by mathematical induction} ($b=8$)

{\bf Basis step}:  WTS property is true about  $8$
\vspace{50pt}

{\bf Recursive step}: Consider an  arbitrary  $n \geq 8$.
Assume (as the  IH) that  there are nonnegative integers
$x, y$ such that $n =  5x +  3y$.  WTS
that there are nonnegative integers $x', y'$ such
that  $n+1 = 5x' +  3y'$.  We consider two cases, 
depending on  whether  any  $5$ cent coins
are used for $n$.

\vspace{1in}

{\it Case 1}:  Assume $x \geq  1$.
Define $x' = x-1$ and $y'=y+2$ (both in  $\mathbb{N}$ by case assumption).

\vspace{-15pt}

Calculating:
\begin{align*}
5x' +  3y' &\overset{\text{by def}}{=}  5(x-1) +  3(y+2)  = 5x -  5 +3y +   6  \\
&\overset{\text{rearranging}} = (5x+3y) -5  + 6\\
& \overset{\text{IH}}{=} n-5+6 =  n+1
\end{align*}

\vspace{1in}

{\it  Case 2}: Assume $x = 0$.  Therefore  $n  = 3y$,  so 
since  $n \geq 8$, $y \geq 3$. Define $x' = 2$ and $y'=y-3$
(both in $\mathbb{N}$ by case assumption).
Calculating:
\begin{align*}
5x' +  3y' &\overset{\text{by def}}{=}  5(2) +  3(y-3)  = 10  +3y -9  \\
&\overset{\text{rearranging}} = 3y +10 -9 \\
&\overset{\text{IH and case}}{=} n+10-9 =  n+1
\end{align*}

\vspace{1in}

\columnbreak


{\bf Proof of $\star$ by strong induction} ($b=8$ and $j=2$)

{\bf Basis step}:  WTS property is true about  $8, 9, 10$
\vspace{50pt}

{\bf Recursive step}: Consider an  arbitrary  $n \geq 10$.
Assume (as the  IH) that the property is true about  each of $8, 9, 10, \ldots, n$.  
WTS
that there are nonnegative integers $x', y'$ such
that  $n+1 = 5x' +  3y'$.

\vspace{200pt}
\end{multicols}


\vfill \vfill
\section*{Binary expansions exist proof}


{\bf  Representing positive integers}


{\bf Theorem}: Every positive integer is a sum of (one or more) distinct powers of $2$. {\it  
binary expansions exist!}



{\bf Proof by strong induction}, with $b=1$ and $j=0$.


{\bf Basis step}:  WTS property is true about  $1$.


{\bf Recursive step}: Consider an arbitrary integer $n \geq 1$.
Assume (as the IH) that the property is true about  each of $1, \ldots, n$.  
WTS that the property is true about  $n+1$.


\vfill
 \vfill
\section*{Strong induction def}


\fbox{\parbox{\textwidth}{

{\bf New! Proof by Strong Induction} (Rosen 5.2 p337)

To prove that a universal quantification over the set of all integers greater than or equal to some  base integer $b$ holds,  pick a  fixed nonnegative integer  $j$ and then: \hfill 

\begin{tabularx}{\textwidth}{lX}
    Basis Step: & Show the statement holds for $b$, $b+1$, \ldots, $b+j$. \\
    Recursive Step: & Consider an arbitrary integer $n$ greater than or  equal to  $b+j$, assume
    (as the {\bf strong  induction hypothesis})  that the property holds  for {\bf each of} $b$, $b+1$, \ldots, $n$, 	
    and use  this and
    other facts to  prove that  the property holds for $n+1$.
\end{tabularx}
}} 

\vfill

\begin{center}
    \begin{tabular}{cll}
    $\mathbb{N}$  &  The set of  natural numbers & $\{ 0, 1, 2, 3, \ldots \}$ \\
$\mathbb{Z}^{\geq b}$ & The set of integers greater than  or equal  a  basis element $b$ & $\{ b, b+1, b+2, b+3,  \ldots  \}$ \\
    \end{tabular}
    \end{center}
    
    \vfill \vfill
\section*{Prime number def}


{\bf Definition}:  An integer $p$ greater than $1$ is called {\bf prime} if the only positive factors of 
$p$ are $1$ and $p$. A positive integer that is greater than $1$ and is not prime is called composite. 
 \vfill
\section*{Fundamental theorem proof}


{\bf Theorem}: Every positive integer {\it greater than 1} is a product of (one or more) primes.

{\bf Proof by strong induction}, with $b=2$ and $j=0$.

{\bf Basis step}:  WTS property is true about  $2$.
\vspace{20pt}

{\bf Recursive step}: Consider an arbitrary integer $n \geq 2$.
Assume (as the IH) that the property is true about  each of $2, \ldots, n$.  
WTS that the property is true about  $n+1$.


{\bf Case 1}: 


{\bf Case 2}: 

\vfill \vfill
\section*{Prime number def}


{\bf Definition}:  An integer $p$ greater than $1$ is called {\bf prime} if the only positive factors of 
$p$ are $1$ and $p$. A positive integer that is greater than $1$ and is not prime is called composite. 
 \vfill
\section*{Proof by contradiction def}


\fbox{\parbox{\textwidth}{

{\bf New! Proof by Contradiction} 

To prove that a statement $p$ is true, pick another statement $r$ and once we show
that $\neg p  \to (r \wedge  \neg r)$ then  we can conclude that  $p$ is  true.

{\it Informally} The statement we care about can't possible be false, so it must be true.
}} 

 \vfill
\section*{Least greatest proofs}


{\bf Prove} or {\bf  disprove}:  There is a least prime number.


\vfill

{\bf Prove} or {\bf  disprove}: There is a greatest integer. 

{\it Approach 1, De Morgan's and universal generalization}: 

\vfill

{\it Approach 2, proof by contradiction}: 

\vfill

\vfill

{\it Extra examples}: Prove or disprove that $\mathbb{N}$,  $\mathbb{Q}$ each have a
least and a greatest element. Prove that there is no greatest prime number.
 \vfill
\section*{Rational number def}


The {\bf set  of rational numbers}, $\mathbb{Q}$  is defined as 
\[
\left\{ \frac{p}{q} \mid p \in \mathbb{Z}  \text{ and  } q  \in \mathbb{Z} \text{ and } q \neq  0 \right\}
\text{~~~~or, equivalently,~~~~}
\{ x  \in  \mathbb{R} \mid \exists p \in \mathbb{Z}  \exists q \in \mathbb{Z}^+ ( p =  x \cdot q) \}
\]

{\it Extra practice}: Use the definition of set equality to prove that the definitions above  give the same set.

 \vfill
\section*{Proof square root2}


{\bf Goal}:  The square root of $2$ is not a rational number.  In other words: $\neg \exists x \in \mathbb{Q} ( x^2 -  2 = 0)$

{\bf Attempted proof}: The definition of the set of rational numbers is the collection of fractions $p/q$ where $p$ is an integer and $q$ is a nonzero integer. Looking for a {\bf witness} $p$ and $q$, we can write the square root of $2$ as the fraction 
$\sqrt{2 }/1$, where $1$ is a nonzero integer. Since the numerator is not in the domain, this witness is not allowed, and we have shown that the square root of $2$ is not a fraction of integers (with nonzero denominator). Thus, the square root of $2$ is not rational.


{\it The problem in the above attempted proof is that} \underline{\phantom{it only considers one candidate witness
and does not prove that no witnesses exist.}}


{\bf Proof}: 

\vfill

\vfill



{\bf Lemma 1:} For every two integers $p$  and  $q$, not both zero, $gcd\left( \frac{p}{gcd(p,q) },  \frac{q}{gcd(p,q)} \right) =  1$.


{\bf Lemma 2:} For every two integers $a$ and  $b$, not both zero, with  $gcd(a,b) = 1$, it is not the case that both $a$
is  even and $b$ is even.


{\bf Lemma 3:} For every integer  $x$, $x$ is  even if and only if $x^2$  is even.

\vfill \vfill
\section*{Gcd def}


\fbox{\parbox{\textwidth}{

{\bf Greatest common divisor} Let $a$ and $b$ be integers, not both zero. The largest integer $d$ such that 
$d$ is a  factor of $a$ and $d$ is a factor of  $b$ is called the greatest common divisor of $a$ and $b$ 
and is denoted by $gcd(a, b)$.
}
}
 \vfill
\section*{Proposition def}


\begin{tabular}{lp{5in}}
    {\bf Proposition} & Declarative sentence that is true or false (not both).\\
    {\bf Propositional variable} & Variable that represents a proposition.\\
    {\bf Compound proposition}& New propositions formed from existing propositions (potentially) using logical operators.\\
    {\bf Truth table}& Table with 1 row for each of the possible combinations of truth values of the input and 
    an additional column that shows the truth value of the result of the operation corresponding to a particular row.
    \end{tabular}
    
    {\it Note}: A propositional variable is one example of a compound proposition. \vfill
\section*{Logical operators}


{\bf Logical operators} aka propositional connectives

\begin{tabular}{lccccp{4in}}
{\bf Conjunction} & AND & $\land$ &\verb|\land| & 2 inputs & Evaluates to $T$ when {\bf both} inputs are $T$\\
{\bf Exclusive or} & XOR & $\oplus$ &\verb|\oplus| & 2 inputs & Evaluates to $T$ when {\bf exactly one} of inputs is $T$\\
{\bf Disjunction} & OR & $\lor$ &\verb|\lor| & 2 inputs & Evaluates to $T$ when {\bf at least one} of inputs is $T$\\
{\bf Negation} & NOT & $\lnot$ &\verb|\lnot| & 1 input & Evaluates to $T$ when its input is $F$\\
\end{tabular} \vfill
\section*{Logical operators truth tables}


\begin{center}
\begin{tabular}{cc||c|c|c}
\multicolumn{2}{c||}{Input}  & \multicolumn{3}{c}{Output} \\
& & {\bf Conjunction} &  {\bf Exclusive or} & {\bf Disjunction} \\
$p$ & $q$ & $p \land q$ &  $p  \oplus  q$ & $p \lor  q$ \\
\hline
$T$ & $T$ & $T$ & $F$ & $T$\\
$T$ & $F$ & $F$ & $T$ & $T$\\
$F$ & $T$ & $F$ & $T$ & $T$\\
$F$ & $F$ & $F$ & $F$ & $F$\\
\end{tabular}
\qquad \qquad\qquad
\begin{tabular}{c||c}
Input & Output \\
& {\bf Negation} \\
$p$ & $\lnot p$ \\
\hline
$T$ & $F$ \\
$F$ & $T$\\
\end{tabular}
\end{center}

\vfill \vfill
\section*{Logical operators exercise}


\begin{center}
    \begin{tabular}{ccc||p{3in}|c|c}
    \multicolumn{3}{c||}{Input}  & \multicolumn{3}{c}{Output} \\
    $p$ & $q$ & $r$  &  &  $(p \land q) \oplus (~ ( p \oplus q) \land r~)$ & $(p \land q) \vee (~ ( p \oplus q) \land r~)$ \\
    \hline
    $T$ & $T$  & $T$ &   && \\
    $T$ & $T$  & $F$ &   && \\
    $T$ & $F$  & $T$ &   && \\
    $T$ & $F$  & $F$ &   && \\
    $F$ & $T$  & $T$ &   && \\
    $F$ & $T$  & $F$ &   && \\
    $F$ & $F$  & $T$ &   && \\
    $F$ & $F$  & $F$ &   && \\
    \end{tabular}
\end{center}
    \vfill \vfill
\section*{Logical equivalence}


\begin{tabular}{lp{4in}p{2in}}
{\bf Logical equivalence } &Two compound  propositions are {\bf logically  equivalent} means that  they 
have the  same  truth  values for all settings of truth  values to their propositional  variables.\\
{\bf Tautology} & A compound proposition that evaluates to true
for all settings of truth  values to its propositional  variables; it is  abbreviated $T$.\\
{\bf Contradiction} & A compound proposition that  evaluates  to  false 
for  all settings of truth  values to its propositional  variables; it  is abbreviated $F$.\\
{\bf Contingency} & A compound proposition that is neither a tautology nor a contradiction.\\
\end{tabular}
 \vfill
\section*{Logical equivalence exercise}


{\it Extra Example}: Which of the  compound propositions in the table below are logically equivalent?
\begin{center}
\begin{tabular}{cc||c|c|c|c|c}
\multicolumn{2}{c||}{Input}  & \multicolumn{5}{c}{Output} \\
$p$ & $q$ & $\lnot (p \land \lnot q)$ & $\lnot (\lnot p  \lor \lnot q)$ &  $(\lnot p \lor  q)$
& $(\lnot q \lor \lnot p)$ & $(p \land q)$  \\
\hline
$T$ & $T$ & &&&&\\
$T$ & $F$ & &&&&\\
$F$ & $T$ & &&&&\\
$F$ & $F$ & &&&&\\
\end{tabular}
\end{center} \vfill
\section*{Logical equivalence example}


{\bf (Some)} logical equivalences) cf. Rosen pp. 26-28

\begin{tabular}{llp{3in}}
$p \lor q \equiv q \lor p$ & $p \land q \equiv q \land p$ & {\bf Commutativity} Ordering of terms\\
$(p \lor q) \lor r  \equiv p \lor (q \lor r)$ & $(p \land q) \land r  \equiv p \land (q \land r)$ & {\bf Associativity} Grouping of terms\\
$p \land F \equiv F$ \qquad $p \lor T \equiv T$ & $p \land T \equiv p$ \qquad $p \lor F \equiv p$ & {\bf Absorption} aka 
short circuit evaluation\\
$\lnot (p \land q) \equiv \lnot p \lor \lnot q$ & $\lnot (p \lor q) \equiv \lnot p \land\lnot q$  & {\bf DeMorgan's Laws}\end{tabular}


{\it Can replace $p$ and $q$ with any compound proposition}

\vfill \vfill
\section*{Truth table to compound proposition application}


{\it Application}: design a circuit given a desired input-output relationship.

\begin{center}
\begin{tabular}{cc||cc}
\multicolumn{2}{c||}{Input}  &\multicolumn{2}{c}{Output}\\
$p$ & $q$& $mystery_1$ & $mystery_2$\\
\hline
$T$ & $T$  & $T$ & $F$\\
$T$ & $F$  & $T$ & $F$\\
$F$ & $T$  & $F$ & $F$\\
$F$ & $F$  & $T$ & $T$\\
\end{tabular}
\qquad \qquad
\begin{tabular}{ccc||c}
\multicolumn{3}{c||}{Input}  & Output\\
$p$ & $q$ & $r$  &  ?\\
\hline
$T$ & $T$  & $T$ & $T$ \\
$T$ & $T$  & $F$ & $T$ \\
$T$ & $F$  & $T$ & $F$ \\
$T$ & $F$  & $F$ & $T$ \\
$F$ & $T$  & $T$ & $F$ \\
$F$ & $T$  & $F$ & $F$ \\
$F$ & $F$  & $T$ & $T$ \\
$F$ & $F$  & $F$ & $F$ \\
\end{tabular}

\end{center}


A compound proposition that  gives output $mystery_1$ is: \underline{\phantom{\hspace{3in}}}


\vfill


A compound proposition that  gives output $mystery_2$ is: \underline{\phantom{\hspace{3in}}}


\vfill \vfill
\section*{Dnf cnf definition}


{\bf  Definition} A compound proposition is in {\bf disjunctive normal form}  (DNF) means
that  it is an OR of ANDs of variables and their negations.

{\bf  Definition} A compound proposition is in {\bf conjunctive normal form}  (CNF) means
that  it is an AND of ORs of variables and their negations.


{\it Extra example}: A compound proposition that  gives output ? is: 


\vfill
\vfill \vfill
\section*{Hypothesis conclusion}


The only way to make  the conditional statement $p \to q$ false is to \underline{\phantom{\hspace{3in}}}

\begin{tabular}{llll}
The {\bf  hypothesis}  of $p \to q$ is  &\underline{\phantom{\hspace{1in}}} &
The {\bf  antecedent}  of $p \to q$ is  &\underline{\phantom{\hspace{1in}}} \\
&&&  \\
The {\bf  conclusion}  of $p \to q$ is & \underline{\phantom{\hspace{1in}}}&
The {\bf  consequent}  of $p \to q$ is & \underline{\phantom{\hspace{1in}}}\\
&&&  \\
\end{tabular}

\vfill \vfill
\section*{Logical equivalence example two}


\begin{center}
    \begin{tabular}{cc||c|c|c|c|c}
    \multicolumn{2}{c||}{Input}  & \multicolumn{5}{c}{Output} \\
     & & Conjunction &  Exclusive or & Disjunction  &  Conditional & Biconditional  \\
    $p$ & $q$ & $p \wedge q$ &  $p  \oplus  q$ & $p \vee  q$ & $p \to q$ & $p \leftrightarrow q$\\
    \hline
    $T$ & $T$ & $T$ & $F$ & $T$ & $T$& $T$\\
    $T$ & $F$ & $F$ & $T$ & $T$ & $F$& $F$\\
    $F$ & $T$ & $F$ & $T$ & $T$ & $T$& $F$\\
    $F$ & $F$ & $F$ & $F$ & $F$ & $T$& $T$\\
    \end{tabular}
    \end{center}
    
    {\it Examples} 
    
    $p \to q \equiv \lnot p \lor q$ because \underline{\phantom{\hspace{4in}}} 
    
    \vfill
    
    $p \leftrightarrow q$ is not logically equivalent to $p \land q$ because \underline{\phantom{\hspace{4in}}} 
    
    \vfill
    
    $\lnot( p \leftrightarrow q) \equiv p \oplus q$ because \underline{\phantom{\hspace{4in}}} 
    
    \vfill
    
    
    $p \to q$ is not logically equivalent to $q \to p$ because \underline{\phantom{\hspace{4in}}} 
    
    \vfill
    
    $p \leftrightarrow q \equiv q \leftrightarrow p$ because \underline{\phantom{\hspace{4in}}} 
    
    \vfill \vfill
\section*{Converse inverse contrapositive}


\begin{tabular}{ll}
    The {\bf  converse}  of $p \to q$ is & \underline{\phantom{\hspace{1.6in}}}\\
    &  \\
    The {\bf  inverse}  of $p \to q$ is  &\underline{\phantom{\hspace{1.6in}}}\\
    &  \\
    The {\bf  contrapositive}  of $p \to q$ is & \underline{\phantom{\hspace{1.6in}}}
    \end{tabular}
    Which of these is logically equivalent to $p \to q$?
    
    \vfill
    \vfill \vfill
\section*{Compound propositions translation}


{\bf Translation}: Express each of the following sentences as compound propositions, using
the given propositions.

\begin{multicols}{2}
``A sufficient condition for the warranty to be good is that you bought the computer less than a year ago"
\columnbreak
\begin{align*}
w &\text{ is  ``the warranty is good"} \\
b &\text{ is  ``you bought the computer less than a year ago"} \\
\end{align*}
\end{multicols}
\vfill

\begin{multicols}{2}
``Whenever the message was sent from an unknown system, it is scanned for viruses."
\columnbreak
\begin{align*}
s &\text{ is  ``The message is scanned for viruses"} \\
u &\text{ is  ``The message was sent from an unknown system"} \\
\end{align*}
\end{multicols}
\vfill

\begin{multicols}{2}
``I will complete my to-do list only if I put a reminder in my calendar"
\columnbreak
\begin{align*}
r &\text{ is  ``I will complete my to-do list"} \\
c &\text{ is  ``I put a reminder in my calendar"} \\
\end{align*}
\end{multicols}
\vfill \vfill
\end{document}