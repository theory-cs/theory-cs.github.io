\documentclass[12pt, oneside]{article}

\usepackage[letterpaper, scale=0.89, centering]{geometry}
\usepackage{fancyhdr}
\setlength{\parindent}{0em}
\setlength{\parskip}{1em}

\usepackage{tikz}
\usetikzlibrary{automata,positioning,arrows}

\pagestyle{fancy}
\fancyhf{}
\renewcommand{\headrulewidth}{0pt}
\rfoot{\href{https://creativecommons.org/licenses/by-nc-sa/2.0/}{CC BY-NC-SA 2.0} Version \today~(\thepage)}

\usepackage{amssymb,amsmath,pifont,amsfonts,comment,enumerate,enumitem}
\usepackage{currfile,xstring,hyperref,tabularx,graphicx,wasysym}
\usepackage[labelformat=empty]{caption}
\usepackage{xcolor}
\usepackage{multicol,multirow,array,listings,tabularx,lastpage,textcomp,booktabs}

\lstnewenvironment{algorithm}[1][] {   
    \lstset{ mathescape=true,
        frame=tB,
        numbers=left, 
        numberstyle=\tiny,
        basicstyle=\rmfamily\scriptsize, 
        keywordstyle=\color{black}\bfseries,
        keywords={,procedure, div, for, to, input, output, return, datatype, function, in, if, else, foreach, while, begin, end, }
        numbers=left,
        xleftmargin=.04\textwidth,
        #1
    }
}
{}

\newcommand\abs[1]{\lvert~#1~\rvert}
\newcommand{\st}{\mid}

\newcommand{\cmark}{\ding{51}}
\newcommand{\xmark}{\ding{55}}
 
\begin{document}
\begin{flushright}
    \StrBefore{\currfilename}{.}
\end{flushright} \section*{Week10 friday}


{\bf NP-Complete Problems}

{\bf 3SAT}: A literal is a Boolean variable (e.g.  $x$) or a negated Boolean variable (e.g.  $\bar{x}$).  
A Boolean formula is a {\bf  3cnf-formula} if it is a Boolean formula in conjunctive normal form (a conjunction  
of  disjunctive clauses of literals) and each clause  has  three literals.
\[
3SAT  = \{  \langle  \phi \rangle \mid \text{$\phi$ is  a  satisfiable 3cnf-formula} \}
\]


Example string  in $3SAT$
\[
   \langle (x \vee \bar{y} \vee {\bar z}) \wedge (\bar{x}  \vee y  \vee  z) \wedge (x \vee y  \vee z) \rangle
\]



Example  string not  in $3SAT$
\[
   \langle (x \vee y \vee z) \wedge 
    (x \vee y \vee{\bar z}) \wedge
    (x \vee \bar{y} \vee z) \wedge
    (x \vee \bar{y} \vee \bar{z}) \wedge
    (\bar{x} \vee y \vee z) \wedge
    (\bar{x} \vee y \vee{\bar z}) \wedge
    (\bar{x} \vee \bar{y} \vee z) \wedge
    (\bar{x} \vee \bar{y} \vee \bar{z}) \rangle
\]



{\bf Cook-Levin Theorem}: $3SAT$ is $NP$-complete.


{\it Are there other $NP$-complete problems?} To prove that $X$ is $NP$-complete
\begin{itemize}
\item {\it From scratch}: prove $X$ is in $NP$ and that all $NP$ problems are polynomial-time
reducible to $X$.
\item {\it Using reduction}: prove $X$ is in $NP$ and that a known-to-be $NP$-complete problem 
is polynomial-time reducible to $X$.
\end{itemize}

\vfill
\vfill


\newpage

{\bf CLIQUE}: A {\bf $k$-clique} in an undirected graph is a maximally connected subgraph with $k$  nodes.
\[
CLIQUE  = \{  \langle G, k \rangle \mid \text{$G$ is an  undirected graph with  a $k$-clique} \}
\]


Example string  in $CLIQUE$

\vfill

Example  string not  in $CLIQUE$

\vfill

Theorem (Sipser 7.32):
\[
3SAT  \leq_P CLIQUE
\]

Given a Boolean formula in conjunctive normal form with $k$ clauses and three literals per clause, we will 
map it to a graph so that the graph has a clique if the original formula is satisfiable and the 
graph does not have a clique if the original formula is not satisfiable.

The graph has $3k$ vertices (one for each literal in each clause) and an edge between all vertices except
\begin{itemize}
    \item vertices for two literals in the same clause
    \item vertices for literals that are negations of one another
\end{itemize}

Example: $(x \vee \bar{y} \vee {\bar z}) \wedge (\bar{x}  \vee y  \vee  z) \wedge (x \vee y  \vee z)$

\vfill

\vfill
\vfill
\newpage

\begin{center}
    \begin{tabular}{|p{4in}|p{3.5in}|}
        \hline
        & \\
        {\bf Model of Computation} & {\bf Class of Languages}\\
        &\\
        \hline
        & \\
        {\bf Deterministic finite automata}:
        formal definition, how to design for a given language, 
        how to describe language of a machine?
        {\bf Nondeterministic finite automata}:
        formal definition, how to design for a given language, 
        how to describe language of a machine?
        {\bf Regular expressions}: formal definition, how to design for a given language, 
        how to describe language of expression?
        {\it Also}: converting between different models. &
        {\bf Class of regular languages}: what are the closure 
        properties of this class? which languages are not in the class?
        using {\bf pumping lemma} to prove nonregularity.\\
        & \\
        \hline
        & \\
        {\bf Push-down automata}:
        formal definition, how to design for a given language, 
        how to describe language of a machine?
        {\bf Context-free grammars}:
        formal definition, how to design for a given language, 
        how to describe language of a grammar? &
        {\bf Class of context-free languages}: what are the closure 
        properties of this class? which languages are not in the class?\\
        & \\
        \hline
        & \\
        Turing machines that always halt in polynomial time
        & $P$ \\
        & \\
        Nondeterministic Turing machines that always halt in polynomial time 
        & $NP$ \\
        & \\
        \hline
        & \\
        {\bf Deciders} (Turing machines that always halt): 
        formal definition, how to design for a given language, 
        how to describe language of a machine? &
        {\bf Class of decidable languages}: what are the closure properties 
        of this class? which languages are not in the class? using diagonalization
        and mapping reduction to show undecidability \\
        & \\
        \hline
        & \\
        {\bf Turing machines}
        formal definition, how to design for a given language, 
        how to describe language of a machine? &
        {\bf Class of recognizable languages}: what are the closure properties 
        of this class? which languages are not in the class? using closure
        and mapping reduction to show unrecognizability \\
        & \\
        \hline
    \end{tabular}
\end{center}

\newpage

{\bf Given a language, prove it is regular}

{\it Strategy 1}: construct DFA recognizing the language and prove it works.

{\it Strategy 2}: construct NFA recognizing the language and prove it works.

{\it Strategy 3}: construct regular expression recognizing the language and prove it works.

{\it ``Prove it works'' means \ldots}

\vspace{100pt}

{\bf Example}: $L  = \{ w \in \{0,1\}^* \mid \textrm{$w$ has odd number of $1$s or starts with $0$}\}$

Using NFA

\vfill

Using regular expressions

\vfill


\newpage

{\bf Example}: Select all and only the options that result in a true statement: ``To show 
a language $A$ is not regular, we can\ldots'' 

\begin{enumerate}
    \item[a.] Show $A$ is finite
    \item[b.] Show there is a CFG generating $A$
    \item[c.] Show $A$ has no pumping length
    \item[d.] Show $A$ is undecidable
\end{enumerate}

\newpage

{\bf Example}: What is the language generated by the CFG with rules
\begin{align*}
    S &\to aSb \mid bY \mid Ya \\
    Y &\to bY \mid Ya \mid \varepsilon 
\end{align*}

\newpage

{\bf Example}: Prove that the language 
$T = \{ \langle M \rangle \mid \textrm{$M$ is a Turing machine and $L(M)$ is infinite}\}$ 
is undecidable.

\newpage

{\bf Example}: Prove that the class of decidable languages is closed under concatenation.
 \vfill
\end{document}