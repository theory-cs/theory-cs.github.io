%! app: TODOapp
%! outcome: TODOoutcome

The diagonalization argument constructs, 
for each function $f:  \mathbb{N} \to  \mathcal{P}(\mathbb{N})$, a set  $D_f$ defined
as
\[
D_f = \{ x \in \mathbb{N} \mid x \notin  f(x) \}
\]
which has the property  that,  for all  $n \in \mathbb{N}$, $f(n) \neq  D_f$.
Consider the following two functions with  domain $\mathbb{N}$ and codomain $\mathcal{P}(\mathbb{N})$
\[
f_1(x) =  \{  y \in  \mathbb{N} \mid y~\text{\bf mod}~3 = x~\text{\bf mod}~3  \}
\]
\[
f_2(x) =  \{  y \in  \mathbb{N} \mid (y > 0) \land
(x ~\text{\bf mod}~y \neq  0)\}
\]

Select all and only the true statements below.
\begin{enumerate}
    \item $0 \in D_{f_1}$
    \item $D_{f_1}$ is infinite
    \item $D_{f_1}$ is uncountable
    \item $1 \in D_{f_2}$
    \item $D_{f_2}$ is empty
    \item $D_{f_2}$ is countably infinite
\end{enumerate}