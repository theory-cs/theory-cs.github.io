%! app: Regular Languages
%! outcome: Formal definition of automata, Informal definition of automata, Nondeterminism

{\bf Review}: The language recognized by the NFA over $\{a,b\}$ with state diagram


    \begin{tikzpicture}[->,>=stealth',shorten >=1pt, auto, node distance=2cm, semithick]
    \tikzstyle{every state}=[text=black, fill=yellow!40]
    
    \node[initial,state] (q0)          {$q_0$};
    \node[state]         (q) [above right of=q0, xshift=20pt] {$q$};
    \node[state]         (r) [right of=q, xshift=20pt] {$r$};
    \node[state, accepting]         (s) [right of=r, xshift=20pt] {$s$};
    \node[state, accepting]         (n) [below right of=q0, xshift=20pt] {$n$};
    \node[state]         (d) [right of=n, xshift=20pt] {$d$};
    
    \path (q0) edge  [bend left=0, near start] node {$\varepsilon$} (q)
            edge [bend right=0, near start] node {$\varepsilon$} (n)
        (q) edge [bend left=0] node {$a$} (r)
            edge [loop above, near start] node {$b$} (q)
        (r) edge [bend left=0] node {$b$} (s)
            edge [loop above, near start] node {$a,b$} (r)
        (n) edge [bend left=20] node {$a,b$} (d)
        (d) edge [bend left=20] node {$a,b$} (n)
    ;
    \end{tikzpicture}
is:


\vfill

So far, we know: 
\begin{itemize}
\item The collection of languages that are each recognizable by a 
DFA is {\bf closed} under complementation.

{\it Could we do the same construction with NFA?}

\vspace{100pt}

\item The collection of languages that are each recognizable by a 
NFA is {\bf closed} under complementation.

{\it Could we do the same construction with DFA?}

\vspace{100pt}

\end{itemize}


\newpage

\vspace{50pt}

Happily, though, an analogous claim is true!

Suppose $A_1, A_2$ are languages over an alphabet $\Sigma$.
{\bf Claim:} if there is a DFA $M_1$ such that $L(M_1) = A_1$ and 
DFA $M_2$ such that $L(M_2) = A_2$, then there is another DFA, let's call it $M$, such that 
$L(M) = A_1 \cup A_2$. {\it Theorem 1.25 in Sipser, page 45}
    
    {\bf Proof idea}:
    
    
    {\bf Formal construction}: 
    
    \vfill

    
    {\bf Example}:  When $A_1 = \{w \mid w~\text{has an $a$ and ends in $b$} \}$ and 
    $A_2 = \{ w \mid w~\text{is of even length} \}$.
    
    \begin{tikzpicture}[->,>=stealth',shorten >=1pt, auto, node distance=2cm, semithick]
        \tikzstyle{every state}=[text=black, fill=yellow!40]
        
        \node[initial,state,accepting] (qn)          {$(q,n)$};
        \node[state]         (qd) [below of=qn, yshift=-40pt] {$(q,d)$};
        \node[state]         (rd) [right of=qn, xshift=20pt] {$(r,d)$};
        \node[state,accepting]         (rn) [right of=qd, xshift=20pt] {$(r,n)$};
        \node[state,accepting]         (sn) [right of=rd, xshift=20pt] {$(s,n)$};
        \node[state,accepting]         (sd) [right of=rn, xshift=20pt] {$(s,d)$};
        
        \path (qn) edge  [bend left=20, near start] node {$b$} (qd)
                edge [bend left=20, near start] node {$a$} (rd)
            (qd) edge [bend left=20, near start] node {$b$} (qn)
                edge [bend right=20, near start] node {$a$} (rn)
            (rn) edge [bend left=20, near start] node {$a$} (rd)
                edge [bend left=20, near start] node {$b$} (sd)
            (rd) edge [bend left=20, near start] node {$a$} (rn)
                edge [bend left=20, near start] node {$b$} (sn)
            (sn) edge [bend left=20, near start] node {$a$} (rd)
                edge [bend left=20, near start] node {$b$} (sd)
            (sd) edge [bend left=20, near start] node {$a$} (rn)
                edge [bend left=20, near start] node {$b$} (sn)
        ;
        \end{tikzpicture}
    
    \newpage
    
    Suppose $A_1, A_2$ are languages over an alphabet $\Sigma$.
    {\bf Claim:} if there is a DFA $M_1$ such that $L(M_1) = A_1$ and 
    DFA $M_2$ such that $L(M_2) = A_2$, then there is another DFA, let's call it $M$, such that 
    $L(M) = A_1 \cap A_2$.  {\it Footnote to Sipser Theorem 1.25, page 46}
    
    {\bf Proof idea}:
    
    
    {\bf Formal construction}: 
    
    \vspace{70pt}


    
