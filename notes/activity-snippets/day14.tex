%! app: Regular Languages, context-free languages
%! outcome: Classify language, Find example languages, context-free grammars, Formal definition of automata, Informal definition of automata


{\bf Theorem  2.20}: A language is  generated by some context-free  grammar
if  and only if it is recognized by some push-down automaton.

Definition: a language is called {\bf context-free} if it is the language generated by a context-free grammar.
The class of all context-free language over a given alphabet $\Sigma$ is called {\bf CFL}.

Consequences:
\begin{itemize}
    \item Quick proof that every regular language is context free 
    \item To prove closure of the class of context-free languages under a given operation, we can choose 
    either of two modes 
    of proof (via CFGs or PDAs) depending on which is easier
    \item To fully specify a PDA we could give its $6$-tuple formal definition or we could give its input 
alphabet, stack alphabet, and state diagram.
An informal description of a PDA is a step-by-step description of how its computations 
would process input strings; the reader should be able to reconstruct the state diagram or formal 
definition precisely from such a descripton. The informal description of a PDA can refer to some 
common modules or subroutines that are computable by PDAs:
\begin{itemize}
  \item PDAs can ``test for emptiness of stack'' without providing details. 
  {\it How?} We can always push a special end-of-stack symbol, $\$$, at the start, before processing
  any input, and then use this symbol as a flag.
  \item PDAs can ``test for end of input'' without providing details.
  {\it How?} We can transform a PDA to one where accepting states are only those reachable 
  when there are no more input symbols.
\end{itemize}

\end{itemize}




\vfill

\newpage
Suppose $L_1$ and $L_2$ are context-free languages over $\Sigma$.  {\bf Goal}:  $L_1 \cup L_2$  is  also context-free.

{\it Approach 1: with  PDAs}

Let $M_1 = ( Q_1, \Sigma, \Gamma_1, \delta_1, q_1, F_1)$ and
$M_2 = ( Q_2, \Sigma, \Gamma_2, \delta_2, q_2, F_2)$ be PDAs with 
$L(M_1) =  L_1$  and  $L(M_2) = L_2$.

Define $M = $

\vfill

{\it Approach  2: with CFGs}

Let $G_1 = (V_1, \Sigma, R_1, S_1)$  and   $G_2 = (V_2, \Sigma, R_2, S_2)$  be CFGs  with
$L(G_1) =  L_1$  and  $L(G_2) = L_2$.

Define $G = $

\vfill

\newpage
Suppose $L_1$ and $L_2$ are context-free languages over $\Sigma$.  {\bf Goal}:  $L_1 \circ L_2$  is  also context-free.


{\it Approach 1: with  PDAs}

Let $M_1 = ( Q_1, \Sigma, \Gamma_1, \delta_1, q_1, F_1)$ and
$M_2 = ( Q_2, \Sigma, \Gamma_2, \delta_2, q_2, F_2)$ be PDAs with 
$L(M_1) =  L_1$  and  $L(M_2) = L_2$.

Define $M = $

\vfill

{\it Approach  2: with CFGs}

Let $G_1 = (V_1, \Sigma, R_1, S_1)$  and   $G_2 = (V_2, \Sigma, R_2, S_2)$  be CFGs  with
$L(G_1) =  L_1$  and  $L(G_2) = L_2$.

Define $G = $

\vfill
\newpage

{\it Summary}

Over a fixed alphabet $\Sigma$, a language $L$ is {\bf regular}

\vspace{-20pt}
\begin{center}
    iff it is described by some regular expression \\
    iff it is recognized by some DFA\\
    iff it is recognized by some NFA
\end{center}

Over a fixed alphabet $\Sigma$, a language $L$ is {\bf context-free}

\vspace{-20pt}
\begin{center}
    iff it is generated by some CFG\\
    iff it is recognized by some PDA
\end{center}

\vfill

{\bf Fact}: Every regular language is a context-free language.

\vfill

{\bf Fact}: There are context-free languages that are nonregular.

\vfill

{\bf Fact}: There are countably many regular languages.

\vfill

{\bf Fact}: There are countably infinitely many context-free languages.

\vfill

{\it Consequence}: Most languages are {\bf not} context-free!

\vfill

\newpage
{\bf Examples  of non-context-free languages}

\begin{align*}
    &\{ a^n b^n c^n \mid 0 \leq n , n \in \mathbb{Z}\}\\
    &\{ a^i b^j c^k \mid 0 \leq i \leq j \leq k , i \in \mathbb{Z}, j \in \mathbb{Z}, k \in \mathbb{Z}\}\\
    &\{ ww \mid w \in \{0,1\}^* \}
\end{align*}
(Sipser Ex 2.36, Ex 2.37, 2.38)

There is a Pumping Lemma for CFL that can be used to prove a specific language is non-context-free: 
If $A$ is a context-free language, there
is a number $p$ where, if $s$ is any string in $A$ of length at least $p$, then $s$ may be divided 
into five pieces $s = uvxyz$ where (1) for each $i \geq 0$, $uv^ixy^iz \in A$, (2) $|uv|>0$, (3) $|vxy| \leq p$.
{\it We will not go into the details of the proof or application of Pumping Lemma for CFLs this quarter.}


Recall: A set $X$ is said to be {\bf closed} under an operation $OP$ if, for any elements in $X$, applying 
$OP$ to them gives an element in $X$.  


\begin{center}
\begin{tabular}{|c|l|}
\hline
True/False & Closure claim \\
\hline
True &  The set of integers is closed under multiplication. \\
& $\forall x \forall y \left( ~(x \in \mathbb{Z} \wedge y \in \mathbb{Z})\to xy \in \mathbb{Z}~\right)$ \\
\hline
True & For each set $A$, the power set of $A$ is closed under intersection. \\
& $\forall A_1 \forall A_2 \left( ~(A_1 \in \mathcal{P}(A) \wedge A_2 \in \mathcal{P}(A) \in \mathbb{Z}) \to A_1 \cap A_2 \in \mathcal{P}(A)~\right)$ \\
\hline
  & The class of regular languages over $\Sigma$ is closed under complementation. \\
  & \\
 \hline
  & The class of regular languages over $\Sigma$ is closed under union. \\
  & \\
 \hline
  & The class of regular languages over $\Sigma$ is closed under intersection. \\
  & \\
  \hline
  & The class of regular languages over $\Sigma$ is closed under concatenation. \\
  & \\
 \hline
  & The class of regular languages over $\Sigma$ is closed under Kleene star. \\
  & \\
\hline
    & The class of context-free languages over $\Sigma$ is closed under complementation. \\
  & \\
\hline
    & The class of context-free languages over $\Sigma$ is closed under union. \\
  & \\
\hline
    & The class of context-free languages over $\Sigma$ is closed under intersection. \\
  & \\
\hline
    & The class of context-free languages over $\Sigma$ is closed under concatenation. \\
  & \\
\hline
    & The class of context-free languages over $\Sigma$ is closed under Kleene star. \\
  & \\
\hline
\end{tabular}
\end{center}
