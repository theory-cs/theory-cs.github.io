%! app: Regular Languages, Context-free Languages
%! outcome: Formal definition of automata, Informal definition of automata, Nondeterminism, Classify language, Find example languages
    

{\bf Definition} A {\bf pushdown automaton} (PDA) is  specified by a  $6$-tuple $(Q, \Sigma, \Gamma, \delta, q_0, F)$
where $Q$ is the finite set of states, $\Sigma$ is the input alphabet,  $\Gamma$ is the stack alphabet,
\[
    \delta: Q \times \Sigma_\varepsilon  \times  \Gamma_\varepsilon \to \mathcal{P}( Q \times \Gamma_\varepsilon)
\]
is the transition function,  $q_0 \in Q$ is the start state, $F \subseteq  Q$ is the set of accept states.


Draw the state diagram and give the formal definition of a PDA with $\Sigma = \Gamma$.

\vfill

Draw the state diagram and give the formal definition of a PDA with $\Sigma \cap \Gamma = \emptyset$.
    
\vfill

\newpage
For the PDA state diagrams below, $\Sigma = \{0,1\}$.


\begin{center}
\begin{tabular}{c c}
Mathematical description of language & State diagram of PDA recognizing language\\
\hline
& $\Gamma = \{ \$, \#\}$ \hspace{2.3in} \\
& \\
& 
\begin{tikzpicture}[->,>=stealth',shorten >=1pt, auto, node distance=2cm, semithick]
    \tikzstyle{every state}=[text=black, fill=none]
    
    \node[initial,state] (q0)          {$q0$};
    \node[state]         (q1) [right of=q0, xshift=20pt] {$q1$};
    \node[state]         (q2) [below right of=q1, xshift=20pt] {$q2$};
    \node[state]         (q3) [right of=q2, xshift=20pt] {$q3$};
    \node[state,accepting]         (q4) [left of=q2, xshift=-20pt] {$q4$};
    
    \path (q0) edge [bend left=0] node {$\varepsilon, \varepsilon; \$$} (q1)
        (q1) edge  [loop above] node {$0, \varepsilon; \#$} (q1)
        (q1) edge [bend left=0] node {$\varepsilon, \varepsilon; \varepsilon$} (q2)
        (q2) edge  [bend left=20] node [midway, above] {$1, \#; \varepsilon$} (q3)
        (q3) edge  [bend left=20] node [midway, below] {$1, \varepsilon; \varepsilon$} (q2)
        (q2) edge  [bend left=0] node {$\varepsilon, \$; \varepsilon$} (q4)
    ;
\end{tikzpicture}
\\
& \\
& \\
\hline
& $\Gamma = \{ \sun, 1\}$ \hspace{2.3in} \\
& \\
& 
\begin{tikzpicture}[->,>=stealth',shorten >=1pt, auto, node distance=2cm, semithick]
    \tikzstyle{every state}=[text=black, fill=none]
    
    \node[initial,state] (q0)          {$q0$};
    \node[state]         (q1) [right of=q0, xshift=20pt] {$q1$};
    \node[state]         (q2) [below right of=q1, xshift=20pt] {$q2$};
    \node[state]         (q3) [right of=q2, xshift=20pt] {$q3$};
    \node[state,accepting]  (q4) [right of=q3, xshift=20pt] {$q4$};
    \node[state]         (q5) [above right of=q1, xshift=20pt] {$q5$};
    \node[state,accepting]  (q6) [right of=q5, xshift=20pt] {$q6$};
    
    \path (q0) edge [bend left=0] node {$\varepsilon, \varepsilon; \sun$} (q1)
        (q1) edge  [loop above] node {$1, \varepsilon; 1$} (q1)
        (q1) edge [bend left=0] node [below] {$\varepsilon, \varepsilon; \varepsilon$} (q5)
        (q1) edge [bend left=0] node [above]{$\varepsilon, \varepsilon; \varepsilon$} (q2)
        (q5) edge [loop above] node {$0, 1, ; \varepsilon$} (q5)
        (q5) edge [bend left=0] node {$\varepsilon, \sun; \varepsilon$} (q6)
        (q6) edge [loop above] node {$1, \varepsilon; \varepsilon$} (q6)
        (q2) edge [loop below] node {$0, \varepsilon; \varepsilon$} (q2)
        (q2) edge  [bend left=0] node {$\varepsilon, \varepsilon; \varepsilon$} (q3)
        (q3) edge  [loop below] node {$1, 1; \varepsilon$} (q3)
        (q3) edge  [bend left=0] node {$\varepsilon, \sun; \varepsilon$} (q4)
    ;
\end{tikzpicture}
\\
& \\
& \\
\hline
& \\
& \\
& \\
$\{ 0^i 1^j 0^k \mid i,j,k \geq 0 \}$ & \\
& \\
& \\
\end{tabular}
\end{center}

 \vfill
 {\it Note: alternate notation is to replace $;$ with $\to$}

\begin{comment}
{\it Extra practice}: Consider the state diagram of a PDA with input alphabet 
$\Sigma$ and stack alphabet $\Gamma$.

\begin{center}
\begin{tabular}{|c|c|}
\hline
Label & means \\
\hline
$a, b ; c$ when $a \in \Sigma$, $b\in \Gamma$, $c \in \Gamma$ 
& \hspace{3in} \\
& \\
& \\
& \\
& \\
&\\
\hline
$a, \varepsilon ; c$ when $a \in \Sigma$, $c \in \Gamma$ 
& \hspace{3in} \\
& \\
& \\
& \\
& \\
&\\
\hline
$a, b ; \varepsilon$ when $a \in \Sigma$, $b\in \Gamma$
& \hspace{3in} \\
& \\
& \\
& \\
& \\
&\\
\hline
$a, \varepsilon ; \varepsilon$ when $a \in \Sigma$
& \hspace{3in} \\
& \\
& \\
& \\
& \\
&\\
\hline
\end{tabular}
\end{center}


How does the meaning change if $a$ is replaced by $\varepsilon$?
\end{comment}
