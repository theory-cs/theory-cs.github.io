%! app: Decidable Languages, Undecidable Languages
%! outcome: Classify language, Classify decision problem, Reduction
 

Recall definition:  $A$ is  {\bf  mapping  reducible to} $B$  means there is a computable function 
$f : \Sigma^* \to \Sigma^*$ such that {\it for all} strings  $x$ in $\Sigma^*$, 
\[
x  \in  A \qquad \qquad \text{if and  only  if} \qquad \qquad f(x) \in B.
\]
Notation:  when $A$  is mapping reducible to $B$, we write $A  \leq_m B$.

{\it Intuition:} $A \leq_m B$ means $A$ is no harder than $B$, i.e. that the level 
of difficulty of $A$ is less than or equal the level of difficulty of $B$.


{\it Example}: $A_{TM} \leq_m A_{TM}$  \hfill 
$A_{TM} = \{ \langle M, w \rangle  \mid M \text{ is a TM and } w \text{ is a string and } w \in L(M) \}$

{\it To prove, need a witnessing function $f: \Sigma^* \to \Sigma^*$ that is (1) computable and 
(2) for each $x \in \Sigma^*$, $x \in A_{TM}$ iff $f(x) \in A_{TM}$}


\vfill


{\bf Corollary}: For any language $L$, $L \leq_m L$, as witnessed by 

\newpage

{\it Example}: $A_{DFA} \leq_m \{ ww \mid  w \in \{0,1\}^* \}$ 


{\it To prove, need a witnessing function $f: \Sigma^* \to \Sigma^*$ that is (1) computable and 
(2) for each $x \in \Sigma^*$, $x \in A_{DFA}$ iff $f(x) \in \{ ww \mid  w \in \{0,1\}^* \}$}


\vfill

{\bf Corollary}: For any language decidable language $X$ and any 
set $Y$ with at least one string string in $Y$ and at least one string not in $Y$, $X \leq_m Y$, as witnessed by 

\vfill

%{\it Example}: $EQ_{DFA} \leq_{m} A_{DFA}$
%
%\vfill

%{\it Example}: $\{ 0^i  1^j \mid i  \geq 0, j \geq 0 \} \leq_m A_{TM}$ 
%
%\vfill

Next: consider mapping reductions between potentially undecidable languages.
\newpage

{\bf Halting problem}
    \[
    HALT_{TM} = \{ \langle M, w \rangle \mid \text{$M$ is a  Turing machine, $w$ is  a string, and $M$ halts on $w$} \}
    \]


We know $A_{TM}$ is undecidable. If we could prove that $A_{TM} \leq_m HALT_{TM}$ then we could conclude that $HALT_{TM}$ is undecidable too.

\vfill

Could we adapt our approach from before by tweaking the identity map?

\vfill

\newpage
Define $F: \Sigma^* \to \Sigma^*$ by
    \[
    F(x) =  \begin{cases}
    const_{out} \qquad &\text{if  $x \neq \langle M,w \rangle$ for any Turing machine  $M$ and string  $w$ over the alphabet of $M$} \\
    \langle M'_x, w \rangle \qquad &  \text{if $x = \langle M, w \rangle$ for some Turing machine  $M$ and string $w$ over the alphabet of $M$.}
    \end{cases}
    \]
    where $const_{out}  =  \langle  
        %\includegraphics[width=1.5in]{../../resources/machines/Lect22TM1.png} 
        \begin{tikzpicture}[->,>=stealth',shorten >=1pt, auto, node distance=2cm, semithick]
            \tikzstyle{every state}=[text=black, fill=none]
            
            \node[initial,state] (q0)          {$q0$};
            \node[state,accepting]         (qacc) [right of=q0, xshift=20pt] {$q_{acc}$};
            
            \path (q0) edge  [loop above] node {\parbox{1cm}{$0; \square, R$\newline$1; \square, R$\newline $\square; \square, R$}} (q0)
            ;
          \end{tikzpicture},  \varepsilon  \rangle$
    and  $M'_x$ is a Turing machine that computes like $M$ except, if the computation of $M$ ever were to go to a  reject state,
    $M'_x$ loops instead.   
%    \vfill
%
%    $F( \langle \includegraphics[width=1.5in]{../../resources/machines/Lect22TM1.png} ,  001  \rangle)$ =
%
%   \vfill

    %$F( \langle \includegraphics[width=2.5in]{../../resources/machines/Lect22TM2.png} ,  \varepsilon  \rangle)$ =

    $F( \langle 
        \begin{tikzpicture}[->,>=stealth',shorten >=1pt, auto, node distance=2cm, semithick]
            \tikzstyle{every state}=[text=black, fill=none]
            
            \node[initial,state] (q0)          {$q0$};
            \node[state]         (q1) [right of=q0, xshift=80pt] {$q1$};
            \node[state,accepting]   (qacc) [above of=q0] {$q_{acc}$};
            
            \path (q0) edge [bend left=20] node {\parbox{1cm}{$0; 0,R$\newline $1; 1, R$}} (q1)
                (q1) edge [bend left=20] node {\parbox{1cm}{$0; 0,R$\newline $1; 1, R$}} (q0)
                (q0) edge  [bend left=0] node {$\square; \square, R$} (qacc)
            ;
        \end{tikzpicture}, \varepsilon \rangle)$ = 


    To use this function  to prove that $A_{TM} \leq_m HALT_{TM}$, we need  two claims:

    
    Claim (1): $F$ is computable \phantom{\hspace{2in}}
    
    \vfill

    Claim (2): for every  $x$,  $x \in  A_{TM}$ iff $F(x) \in HALT_{TM}$.  
    
    \vfill
    \vfill
    \vfill