%! app: Regular Languages
%! outcome: Regular expressions, Formal definition of automata, Informal definition of automata

{\bf Review}: Formal definition of DFA: $M = (Q, \Sigma, \delta, q_0, F)$ 

\begin{center}
\begin{multicols}{2}
\begin{itemize}
\setlength{\itemsep}{2pt}
\item Finite set of states $Q$
\item Alphabet $\Sigma$
\item Transition function $\delta$
\item Start state $q_0$
\item Accept (final) states $F$
\end{itemize}
\end{multicols}
\end{center}
Quick check: In the state diagram of $M$, how many outgoing arrows are there from each state?



{\bf Note}: We'll see a new kind of finite automaton. It will be helpful to distinguish it from the
machines we've been talking about so we'll use {\bf Deterministic Finite Automaton} (DFA) to refer to the machines 
from Section 1.1.

$M = ( \{ q0, q1, q2\}, \{a,b\}, \delta, q0, \{q0\} )$ 
where $\delta$ is  (rows labelled by states
and columns labelled by symbols):
\begin{center}
\begin{tabular}{c|cc}
$\delta$ & $a$ & $b$ \\
\hline
$q0$ & $q1$ & $q1$ \\
$q1$ & $q2$ & $q2$ \\
$q2$ & $q0$ & $q0$ \\
\end{tabular}
\end{center}

The state diagram for $M$ is 

\vfill


Give two examples of strings that are accepted by $M$ and two examples of strings that are rejected by $M$:

\vfill


A regular expression describing $L(M)$ is


\vfill 

A state diagram for a finite automaton recognizing
$$\{w \mid w~\text{is a string over $\{a,b\}$ whose length is not a multiple of $3$} \}$$

\vfill

Extra example: Let $n$ be an arbitrary positive integer. What is a formal definition for a finite automaton recognizing
\[
\{w \mid w~\text{is a string over $\{0,1\}$ whose length is not a multiple of $n$} \}?
\]

\newpage

Consider the alphabet $\Sigma_1 = \{0,1\}$.

A state diagram for a finite automaton that recognizes $\{w \mid w~\text{contains at most two $1$'s} \}$ is

\vspace{70pt}

A state diagram for a finite automaton that recognizes $\{w \mid w~\text{contains more than two $1$'s} \}$ is

\vspace{70pt}


\textbf{Strategy}: Add ``labels" for states in the state diagram, 
e.g. ``have not seen any of desired pattern yet'' or
``sink state''. Then, we can use the analysis of the roles of the states in the 
state diagram to work towards a description of the language recognized
by the finite automaton.


Or: decompose the language to a simpler one 
that we already know how to recognize with a DFA or NFA.


Textbook Exercise 1.14: 
Suppose $A$ is a language over an alphabet $\Sigma$. 
If there is a DFA $M$ such that $L(M) = A$ then there is another DFA, let's call it $M'$, such that 
$L(M') = \overline{A}$, the complement of $A$, defined as $\{ w \in \Sigma^* \mid w \notin A \}$.


{\bf Proof idea}:


\vfill
A useful bit of terminology: the {\bf iterated transition function} of a finite automaton
$M = (Q, \Sigma, \delta, q_0, F)$ is defined recursively by
\[
\delta^* (~(q,w)~) 
=\begin{cases}
q  \qquad &\text{if $q \in Q, w = \varepsilon$} \\
\delta( ~(q,a)~) \qquad &\text{if $q \in Q$, $w = a \in \Sigma$ } \\
\delta(~(\delta^*(~(q,u)~), a) ~) \qquad &\text{if $q \in Q$, $w = ua$ where $u \in  \Sigma^*$ and $a \in \Sigma$}
\end{cases}
\]

Using  this terminology, $M$ accepts a string $w$ over $\Sigma$ if and only if $\delta^*( ~(q_0,w)~) \in F$.


{\bf Proof}: 
\vfill


