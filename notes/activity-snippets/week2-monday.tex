%! app: Regular Languages
%! outcome: Regular expressions, Formal definition of automata, Informal definition of automata

{\bf Review}: Formal definition of DFA: $M = (Q, \Sigma, \delta, q_0, F)$ 

\begin{center}
\begin{multicols}{2}
\begin{itemize}
\setlength{\itemsep}{2pt}
\item Finite set of states $Q$
\item Alphabet $\Sigma$
\item Transition function $\delta$
\item Start state $q_0$
\item Accept (final) states $F$
\end{itemize}
\end{multicols}
\end{center}
In the state diagram of $M$, how many outgoing arrows are there from each state?

$M = ( \{ q, r, s\}, \{a,b\}, \delta, q, \{s\} )$ 
where $\delta$ is  (rows labelled by states
and columns labelled by symbols):
\begin{center}
\begin{tabular}{c|cc}
$\delta$ & $a$ & $b$ \\
\hline
$q$ & $r$ & $q$ \\
$r$ & $r$ & $s$ \\
$s$ & $s$ & $s$ \\
\end{tabular}
\end{center}

The state diagram for $M$ is 

\vfill



Give two examples of strings that are accepted by $M$ and two examples of strings that are rejected by $M$:

\vfill

Add ``labels" for states in the state diagram, e.g. ``have not seen any of desired pattern yet'' or
``sink state''.
\newpage

We can use the analysis of the roles of the states in the state diagram to describe the language
recognized by the DFA. 


$L(M) = $

A regular expression describing $L(M)$ is




\vspace{300pt}

Let the alphabet be $\Sigma_1 = \{0,1\}$.

A state diagram for a DFA that recognizes $\{w \mid w~\text{contains at most two $1$'s} \}$ is

\vspace{70pt}

A state diagram for a DFA that recognizes $\{w \mid w~\text{contains more than two $1$'s} \}$ is

\vspace{70pt}


\newpage
{\it Extra example:} A state diagram for DFA recognizing
$$\{w \mid w~\text{is a string over $\{0,1\}$ whose length is not a multiple of $3$} \}$$

\vspace{70pt}


Let $n$ be an arbitrary positive integer. What is a formal definition for a DFA recognizing
\[
\{w \mid w~\text{is a string over $\{0,1\}$ whose length is not a multiple of $n$} \}?
\]

\vfill

{\bf Note}: On Wednesday, we'll see a new kind of finite automaton. It will be helpful to distinguish it from the
machines we've been talking about so we'll use {\bf Deterministic Finite Automaton} (DFA) to refer to the machines 
from Section 1.1.