%! app: TODOapp
%! outcome: TODOoutcome

Recall the definition of linked lists from class.

Consider this (incomplete) definition:

{\bf Definition} The function $\textit{increment} : \underline{\hspace{6em}}$ 
that adds 1 to the data in each node of a linked list is defined by:
\[
\begin{array}{llll}
& & \textit{increment} : \underline{\hspace{3em}} & \to \underline{\hspace{3em}} \\
\textrm{Basis Step:} & & \textit{increment}([]) & = [] \\
\textrm{Recursive Step:} & \textrm{If } l \in L, n \in \mathbb{N} & \textit{increment}((n, l)) & = (1 + n, \textit{increment}(l))
\end{array}
\]

Consider this (incomplete) definition:

{\bf Definition} The function $\textit{sum} : L \to \mathbb{N}$ that adds 
together all the data in nodes of the list is defined by:
\[
\begin{array}{llll}
& & \textit{sum} : L & \to \mathbb{N} \\
\textrm{Basis Step:} & & \textit{sum}([]) & = 0 \\
\textrm{Recursive Step:} & \textrm{If } l \in L, n \in \mathbb{N} & \textit{sum}((n, l)) & = \underline{\hspace{8em}}
\end{array}
\]

You will compute a sample function application and then fill in the 
blanks for the domain and codomain of each of these functions.

\begin{enumerate}
    \item Based on the definition, what is the result of $\textit{increment}((4, (2, (7, []))))$? Write your answer directly with no spaces.
    
    \item Which of the following describes the domain and codomain of \textit{increment}?
    
    \begin{multicols}{2}
    \begin{enumerate}
        \item The domain is $L$ and the codomain is $\mathbb{N}$
        \item The domain is $L$ and the codomain is $\mathbb{N} \times L$
        \item The domain is $L \times \mathbb{N}$ and the codomain is $L$
        \item The domain is $L \times \mathbb{N}$ and the codomain is $\mathbb{N}$
        \item The domain is $L$ and the codomain is $L$
        \item None of the above
    \end{enumerate}
    \end{multicols}
    
    \item Assuming we would like $sum((5, (6, [])))$ to evaluate to $11$ and $sum((3, (1, (8, []))))$ to evaluate to $12$, which of the following could be used to fill in the definition of the recursive case of \textit{sum}?
    
     \begin{multicols}{2}
    \begin{enumerate}
        \item $\begin{cases}
            1 + \textit{sum}(l) & \textrm{when } n \neq 0 \\
            \textit{sum}(l) & \textrm{when } n = 0 \\
        \end{cases}$
        \item $1 + \textit{sum}(l)$
        \item $n + \textit{increment}(l)$
        \item $n + \textit{sum}(l)$
        \item None of the above
    \end{enumerate}
    \end{multicols}
    
    \newpage
    \item Choose only and all of the following statements that are \textbf{well-defined}; that is, they correctly reflect the domains and codomains of the functions and quantifiers, and respect the notational conventions we use in this class. Note that a well-defined statement may be true or false.

    \begin{multicols}{2}    
    \begin{enumerate}
        \item $\forall l \in L \, (\textit{sum}(l))$
        \item $\exists l \in L \, (\textit{sum}(l) \land \textit{length}(l))$
        \item $\forall l \in L \, (\textit{sum}(\textit{increment}(l)) = 10)$
        \item $\exists l \in L \, (\textit{sum}(\textit{increment}(l)) = 10)$
        \item $\forall l \in L \, \forall n \in \mathbb{N} \, ((n \times l) \subseteq L)$
        \item $\forall l_1 \in L \, \exists l_2 \in L \, (\textit{increment}(\textit{sum}(l_1)) = l_2)$
        \item $\forall l \in L \, (\textit{length}(\textit{increment}(l)) = \textit{length}(l))$
    \end{enumerate}
    \end{multicols}
    
    \item Choose only and all of the statements in the previous part that are both well-defined and true.
\end{enumerate}