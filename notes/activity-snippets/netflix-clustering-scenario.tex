%! app: TODOapp
%! outcome: TODOoutcome

{\bf Scenario}: Good morning! You're a user experience engineer at Netflix. A
product goal is to design customized home pages for groups of users who have
similar interests. Your manager tasks you with designing an algorithm for
producing a clustering of users based on their movie interests,
so that customized homepages can be engineered for each group.

{\bf Conventions for today}: 
We will use $U = \{r_1, r_2, \cdots, r_t\}$ to 
refer to an arbitrary set of user ratings (we'll pick some 
specific examples to explore) that are a subset of $Rt_5$. 
We will be interested in creating partitions $C_1, \cdots, C_m$ of 
$U$. We'll assume that each user represented by an element of $U$ 
has a unique ratings tuple.


Your idea: equivalence relations! You offer your manager three great options: 

\[
    E_{id} = \{ ( ~(x_1, x_2, x_3, x_4, x_5), (x_1, x_2, x_3, x_4, x_5)~) \mid 
    (x_1, x_2, x_3, x_4, x_5) \in Rt_5  \}
\]

{\it Describe how each homepage should be designed \ldots }

\vspace{100pt}



\[
    E_{proj} =  \{ ( ~(x_1, x_2, x_3, x_4, x_5), (y_1, y_2, y_3, y_4, y_5)~) \in
         Rt_5 \times Rt_5 ~\mid~(x_1 = y_1) \land  (x_2 = y_2) \land (x_3 = y_3) \}
\]


{\it Describe how each homepage should be designed \ldots }

\vspace{100pt}

\[
E_{circ} =  \{ (u,v) \in Rt_5 \times Rt_5 ~\mid~ d(~ ( ~(0,0,0,0,0)~, u)~ ) =  d( ~(~(0,0,0,0,0),v~)~) \}
\]

{\it Describe how each homepage should be designed \ldots }


\vspace{100pt}


{\bf Scenario}: Good morning! You're a user experience engineer at Netflix. A
product goal is to design customized home pages for groups of users who have
similar interests. You task your team with designing an algorithm for
producing a clustering of users based on their movie interests. Your team
implements two algorithms that produce different clusterings. How do you
decide which one to use? What feedback do you give the team in order to help
them improve? Clearly, you will need to use math.


Your idea: find a way to {\bf score} clusterings (partitions) 


{\bf Definition}: For a cluster of ratings $C = \{r_1, r_2, \cdots, r_n \} 
\subseteq U$, the {\bf diameter} of the cluster is defined by:

$$\textit{diameter}(C) = \max_{1 \leq i, j \leq n} (d(~(r_i, r_j)~))$$ 

Consider $x = (1, 0, 1, 0, 1)$, $y = (1, 1, 1, 0, 1)$, $z = (-1, -1, 0, 0, 0)$, $w = (0, 0, 0, 1, 0)$.

What is $\textit{diameter}(\{x, y, z\})$? $\textit{diameter}(\{x, y\})$? $\textit{diameter}(\{x, z, w\})$?

\vspace{100pt}

\textit{diameter} works on single clusters. One way to aggregate across a
clustering $C_1, \cdots, C_m$ is $\sum_{k=1}^m diameter(C_k)$


Is this a good score?

\vspace{100pt}

How can we express the idea of {\bf many elements within a small area}? Key idea: ``give credit'' to small diameter clusters with many elements.

{\bf Definition}: For a cluster of ratings $C = \{r_1, r_2, \cdots, r_n \} 
\subseteq U$, the {\bf density} of the cluster is defined by:
\[
    \frac{n}{1+ diameter(C)}
\]



\newpage

Can you use density to decide whether the partition given by 
the equivalence classes of $E_{proj}$ or $E_{circ}$ for this task?