%! app: Decidable Languages, Undecidable Languages
%! outcome: Formal definition of automata, Informal definition of automata, Classify language, Find example languages
    
Two models of computation are called {\bf equally expressive} when 
every language recognizable with the first model is recognizable with the second, and vice versa.

True / False: NFAs and PDAs are equally expressive.

True / False: Regular expressions and CFGs are equally expressive.


\begin{center}
{\large \it  Some examples of models that are {\bf equally expressive} with deterministic Turing machines: }
\end{center}

\fbox{ {\bf May-stay}  machines }
The May-stay machine model is the same as the usual Turing machine model,  except that
on each transition, the tape head may move L, move R, or Stay. 

Formally: $(Q, \Sigma, \Gamma, \delta, q_0, q_{accept}, q_{reject})$ where 
\[
  \delta: Q \times \Gamma \to Q \times \Gamma \times \{L, R, S\}
\]

{\bf Claim}: Turing machines and May-stay machines are equally expressive. {\it To prove \ldots}

To translate a standard TM to a may-stay machine: 

\vspace{100pt}




To translate one  of the  may-stay machines to standard TM:
any time TM would Stay, move right  then  left.


Formally: suppose $M_S =  (Q, \Sigma, \Gamma, \delta, q_0, q_{acc}, q_{rej})$
has $\delta: Q \times \Gamma \to Q \times \Gamma \times \{L, R, S\}$. Define
the Turing-machine
\[
  M_{new} =  (\phantom{\hspace{2.5in}})
\]

\vfill

\newpage

\phantom{$M_{new}$ construction here \vspace{400pt}}
\vfill


\fbox{ {\bf Multitape Turing machine}} A multitape Turing macihne with $k$ tapes
can be formally representated as 
$(Q, \Sigma,  \Gamma, \delta, q_0, q_{acc}, q_{rej})$ 
where $Q$ is the finite set of  states,
$\Sigma$ is the  input alphabet with  $\textvisiblespace \notin \Sigma$,
$\Gamma$  is the  tape alphabet with $\Sigma \subsetneq \Gamma$ ,
$\delta: Q\times \Gamma^k\to Q \times \Gamma^k \times \{L,R\}^k$ 
(where $k$ is  the number of  states)


If $M$ is a standard  TM, it is a $1$-tape machine.


To translate a $k$-tape machine  to  a standard TM:
Use a  new symbol to separate the contents of each tape
and keep track of location of  head with  special version of each
tape symbol. {\tiny Sipser Theorem 3.13} 

\includegraphics[width=2.5in]{../../resources/images/Figure314.png}


{\it Extra practice:} \fbox{ {\bf  Wikipedia Turing machine} }
Define a machine $(Q, \Gamma, b, \Sigma,  q_0, F, \delta)$
where $Q$ is the finite set  of  states
$\Gamma$  is the tape alphabet,
$b \in \Gamma$ is the blank symbol, 
$\Sigma \subsetneq \Gamma$ is the  input alphabet, 
$q_0 \in  Q$ is the start state, 
$F \subseteq Q$ is the set of accept states, 
$\delta: (Q \setminus F)  \times  \Gamma \not\to Q \times  \Gamma  \times \{L, R\}$
 is a partial transition function
If computation enters a state  in $F$, it  accepts 
If computation enters a configuration where
 $\delta$ is not defined, it  rejects . {\tiny Hopcroft and  Ullman, cited by  Wikipedia} 

\newpage
\fbox{ {\bf Enumerators} } Enumerators give a different
model of computation where a language is {\bf produced, one string at a time},
rather than recognized by accepting (or not) individual strings.

Each enumerator machine has finite state control, unlimited work tape, and a printer. The computation proceeds
according to transition function; at any point machine may ``send'' a string to the printer.
\[
E  = (Q, \Sigma, \Gamma, \delta, q_0, q_{print})  
\]
$Q$ is the finite set of states, $\Sigma$ is  the output alphabet, $\Gamma$ is the 
tape alphabet ($\Sigma  \subsetneq\Gamma, 
\textvisiblespace \in \Gamma \setminus \Sigma$), 
\[
\delta:  Q  \times  \Gamma \times \Gamma \to  Q \times  \Gamma \times  \Gamma \times \{L, R\} \times  \{L, R\}
\]
where in state $q$, when the working tape is scanning character $x$ and the printer tape is scanning character $y$,
$\delta( (q,x,y) ) = (q', x', y', d_w, d_p)$ means transition to control state $q'$, write $x'$ on 
the working tape, write $y'$ on the printer tape, move in direction $d_w$ on the working tape, and move in direction 
$d_p$ on the printer tape. The computation starts in $q_0$ and each time the computation enters $q_{print}$
the string from the leftmost edge of the printer tape to the first blank cell is considered to be printed.

The language  {\bf  enumerated} by  $E$, $L(E)$, is $\{ w \in \Sigma^* \mid \text{$E$ eventually, at finite  time, 
prints $w$} \}$.


\begin{center}
\begin{tabular}{cc}
\includegraphics[width=3.5in]{../../resources/machines/Lec15enumerator.png}  & 
\begin{tabular}{|c|c|c|c|c|c|c|}
\hline
\multicolumn{1}{|c}{$q0$} &  \multicolumn{6}{c|}{\phantom{A}}\\
\hline
$\textvisiblespace ~*$& $\textvisiblespace$  & $\textvisiblespace$ & $\textvisiblespace$& $\textvisiblespace$& $\textvisiblespace$&  $\textvisiblespace$\\
\hline
$\textvisiblespace  ~*$& $\textvisiblespace$  & $\textvisiblespace$ & $\textvisiblespace$& $\textvisiblespace$& $\textvisiblespace$&  $\textvisiblespace$\\
\hline\hline
\multicolumn{7}{|c|}{\phantom{A}}\\
\hline
\phantom{AA} & \phantom{AA}& \phantom{AA}& \phantom{AA}& \phantom{AA}& \phantom{AA}& \phantom{AA} \\
\hline
\phantom{AA} & \phantom{AA}& \phantom{AA}& \phantom{AA}& \phantom{AA}& \phantom{AA}& \phantom{AA} \\
\hline
\hline
\multicolumn{7}{|c|}{\phantom{A}}\\
\hline
\phantom{AA} & \phantom{AA}& \phantom{AA}& \phantom{AA}& \phantom{AA}& \phantom{AA}& \phantom{AA} \\
\hline
\phantom{AA} & \phantom{AA}& \phantom{AA}& \phantom{AA}& \phantom{AA}& \phantom{AA}& \phantom{AA} \\
\hline
\hline
\multicolumn{7}{|c|}{\phantom{A}}\\
\hline
\phantom{AA} & \phantom{AA}& \phantom{AA}& \phantom{AA}& \phantom{AA}& \phantom{AA}& \phantom{AA} \\
\hline
\phantom{AA} & \phantom{AA}& \phantom{AA}& \phantom{AA}& \phantom{AA}& \phantom{AA}& \phantom{AA} \\
\hline
\hline
\multicolumn{7}{|c|}{\phantom{A}}\\
\hline
\phantom{AA} & \phantom{AA}& \phantom{AA}& \phantom{AA}& \phantom{AA}& \phantom{AA}& \phantom{AA} \\
\hline
\phantom{AA} & \phantom{AA}& \phantom{AA}& \phantom{AA}& \phantom{AA}& \phantom{AA}& \phantom{AA} \\
\hline
\end{tabular}
\end{tabular}
\end{center}


\vfill 

{\bf Theorem 3.21} A language is Turing-recognizable iff some enumerator enumerates it.
{\it Proof next time \ldots }