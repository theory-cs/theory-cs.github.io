%! app: Decidable Languages, Undecidable Languages
%! outcome: Classify language, Classify decision problem, Reduction
 



Recall:  $A$ is  {\bf  mapping  reducible to} $B$, written $A \leq_m B$,  means there is a computable function 
$f : \Sigma^* \to \Sigma^*$ such that {\it for all} strings  $x$ in $\Sigma^*$, 
\[
x  \in  A \qquad \qquad \text{if and  only  if} \qquad \qquad f(x) \in B.
\]

So far: 
\begin{itemize}
\item $A_{TM}$ is recognizable, undecidable, and not-co-recognizable.
\item $\overline{A_{TM}}$ is unrecognizable, undecidable, and co-recognizable.
\item $HALT_{TM}$ is recognizable, undecidable, and not-co-recognizable.
\item $\overline{HALT_{TM}}$ is unrecognizable, undecidable, and co-recognizable.
\item $E_{TM}$ is unrecognizable, undecidable, and co-recognizable.
\item $\overline{E_{TM}}$ is recognizable, undecidable, and not-co-recognizable.
\end{itemize}


\[
EQ_{TM} = \{ \langle M_1, M_2 \rangle \mid \text{$M_1$ and $M_2$ are both Turing machines and $L(M_1) =L(M_2)$} \}
\]


Can we find algorithms to recognize

$EQ_{TM}$  ? 

$\overline{EQ_{TM}}$ ? 

\vfill

{\it Goal}: Show that $EQ_{TM}$ is not recognizable and that $\overline{EQ_{TM}}$ is not recognizable.

Using Corollary to {\bf Theorem 5.28}: If  $A \leq_m B$ and $A$ is unrecognizable, then $B$ is unrecognizable,
it's enough to prove that 
\begin{itemize}
    \item[] $\overline{HALT_{TM}} \leq_m EQ_{TM}$ \hfill aka $HALT_{TM} \leq_m \overline{EQ_{TM}}$
    \item[] $\overline{HALT_{TM}}  \leq_m \overline{EQ_{TM}}$ \hfill aka $HALT_{TM} \leq_m EQ_{TM}$
\end{itemize}

\vfill

\newpage 
Need computable function  $F_1: \Sigma^* \to \Sigma^*$  such that  $x \in HALT_{TM}$ iff 
$F_1(x)  \notin  EQ_{TM}$.



{\it Strategy}:

\vspace{-15pt}

Map strings $\langle M, w \rangle$ to strings $\langle M'_{x},
\scalebox{0.5}{\begin{tikzpicture}[->,>=stealth',shorten >=1pt, auto, node distance=2cm, semithick]
      \tikzstyle{every state}=[text=black, fill=none]
      \node[initial,state] (q0)                    {$q_0$};
      \node[state,accepting] (qacc) [right of = q0, xshift = 20]{$q_{acc}$};
      \path (q0) edge  [loop above] node {$0, 1, \scalebox{1.5}{\textvisiblespace} \to R$} (q0)
     ;
    \end{tikzpicture}}
    \rangle$ 
    . This image string is not in $EQ_{TM}$ when $L(M'_x) \neq \emptyset$.
    
We will build $M'_x$ so that 
    $L(M'_{x}) = \Sigma^*$ when $M$ halts on $w$ and $L(M'_x) = \emptyset$ when $M$ loops on $w$.


Thus: when $\langle M,w \rangle \in HALT_{TM}$ it gets mapped to a string not in $EQ_{TM}$ and 
when $\langle M,w \rangle \notin HALT_{TM}$ it gets mapped to a string that is in $EQ_{TM}$.

\vfill

Define

\vspace{-15pt}

\begin{quote}
$F_1 =  ``$ On input $x$,
\begin{itemize}
\item[1.] Type-check whether  $x = \langle M, w \rangle$ for some TM $M$ and string $w$. 
If so, move to step 2; if  not, output  $\langle \hspace{2in} \rangle$
\item[2.] Construct the following machine $M'_x$:
\vspace{50pt}
\item[3.] Output $\langle M'_{x},
\scalebox{0.5}{\begin{tikzpicture}[->,>=stealth',shorten >=1pt, auto, node distance=2cm, semithick]
      \tikzstyle{every state}=[text=black, fill=none]
      \node[initial,state] (q0)                    {$q_0$};
      \node[state,accepting] (qacc) [right of = q0, xshift = 20]{$q_{acc}$};
      \path (q0) edge  [loop above] node {$0, 1, \scalebox{1.5}{\textvisiblespace} \to R$} (q0)
     ;
    \end{tikzpicture}}
    \rangle$ "
\end{itemize}
\end{quote}

\vfill

Verifying correctness: (1) Is function well-defined and computable? (2) Does it have the 
translation property $x \in HALT_{TM}$ iff its image is {\bf not} in $EQ_{TM}$ ? 
\begin{center}
\begin{tabular}{|c|c|}
\hline
Input string &  Output string \\
\hline
$\langle M, w \rangle$ where  $M$ halts on $w$ & \phantom{\hspace{4in}} \\
& \\
& \\
& \\
$\langle M, w \rangle$ where $M$ loops on $w$ & \\
& \\
&\\ & \\
$x$ not encoding any pair of  TM and string   &  \\
& \\
& \\
\hline
\end{tabular}
\end{center}


\vfill

Conclude: $HALT_{TM} \leq_m \overline{EQ_{TM}}$
\newpage

\newpage 
Need computable function  $F_2: \Sigma^* \to \Sigma^*$  such that  $x \in HALT_{TM}$ iff 
$F_2(x)  \in  EQ_{TM}$.



{\it Strategy}:

\vspace{-15pt}

Map strings $\langle M, w \rangle$ to strings $\langle M'_{x},
\scalebox{0.5}{\begin{tikzpicture}[->,>=stealth',shorten >=1pt, auto, node distance=2cm, semithick]
      \tikzstyle{every state}=[text=black, fill=none]
      \node[initial,state,accepting] (q0)                    {$q_0$};
     ;
    \end{tikzpicture}}
    \rangle$ 
    . This image string is in $EQ_{TM}$ when $L(M'_x) = \Sigma^*$.
    
We will build $M'_x$ so that 
    $L(M'_{x}) = \Sigma^*$ when $M$ halts on $w$ and $L(M'_x) = \emptyset$ when $M$ loops on $w$.


Thus: when $\langle M,w \rangle \in HALT_{TM}$ it gets mapped to a string  in $EQ_{TM}$ and 
when $\langle M,w \rangle \notin HALT_{TM}$ it gets mapped to a string that is not in $EQ_{TM}$.

\vfill

Define

\vspace{-15pt}

\begin{quote}
$F_2 =  ``$ On input $x$,
\begin{itemize}
\item[1.] Type-check whether  $x = \langle M, w \rangle$ for some TM $M$ and string $w$. 
If so, move to step 2; if  not, output  $\langle \hspace{2in} \rangle$
\item[2.] Construct the following machine $M'_x$:
\vspace{50pt}
\item[3.] Output $\langle M'_{x},
\scalebox{0.5}{\begin{tikzpicture}[->,>=stealth',shorten >=1pt, auto, node distance=2cm, semithick]
      \tikzstyle{every state}=[text=black, fill=none]
      \node[initial,state,accepting] (q0)                    {$q_0$};
     ;
    \end{tikzpicture}}
    \rangle$ "
\end{itemize}
\end{quote}

\vfill

Verifying correctness: (1) Is function well-defined and computable? (2) Does it have the 
translation property $x \in HALT_{TM}$ iff its image is in $EQ_{TM}$ ? 
\begin{center}
\begin{tabular}{|c|c|}
\hline
Input string &  Output string \\
\hline
$\langle M, w \rangle$ where  $M$ halts on $w$ & \phantom{\hspace{4in}} \\
& \\
& \\
& \\
$\langle M, w \rangle$ where $M$ loops on $w$ & \\
& \\
&\\ & \\
$x$ not encoding any pair of  TM and string   &  \\
& \\
& \\
\hline
\end{tabular}
\end{center}


\vfill

Conclude: $HALT_{TM} \leq_m EQ_{TM}$

\newpage

Two models of computation are called {\bf equally expressive} when 
every language recognizable with the first model is recognizable with the second, and vice versa.

{\bf  Church-Turing Thesis} (Sipser p. 183): The informal notion of algorithm is formalized completely  and correctly by the 
formal definition of a  Turing machine. In other words: all reasonably expressive models of 
computation are equally expressive with the standard Turing machine.


\begin{center}
{\large \it  Some examples of models that are {\bf equally expressive} with deterministic Turing machines: }
\end{center}

\vfill

\fbox{ {\bf May-stay}  machines }
The May-stay machine model is the same as the usual Turing machine model,  except that
on each transition, the tape head may move L, move R, or Stay. 

Formally: $(Q, \Sigma, \Gamma, \delta, q_0, q_{accept}, q_{reject})$ where 
\[
  \delta: Q \times \Gamma \to Q \times \Gamma \times \{L, R, S\}
\]

{\bf Claim}: Turing machines and May-stay machines are equally expressive. {\it To prove \ldots}

To translate a standard TM to a may-stay machine: never use the direction $S$!


To translate one  of the  may-stay machines to standard TM:
any time TM would Stay, move right  then  left.

\begin{comment}
Formally: suppose $M_S =  (Q, \Sigma, \Gamma, \delta, q_0, q_{acc}, q_{rej})$
has $\delta: Q \times \Gamma \to Q \times \Gamma \times \{L, R, S\}$. Define
the Turing-machine
\[
  M_{new} =  (\phantom{\hspace{2.5in}})
\]

\vfill


\phantom{$M_{new}$ construction here \vspace{400pt}}
\vfill
\end{comment}

\vfill 

\fbox{ {\bf Multitape Turing machine}} A multitape Turing macihne with $k$ tapes
can be formally representated as 
$(Q, \Sigma,  \Gamma, \delta, q_0, q_{acc}, q_{rej})$ 
where $Q$ is the finite set of  states,
$\Sigma$ is the  input alphabet with  $\textvisiblespace \notin \Sigma$,
$\Gamma$  is the  tape alphabet with $\Sigma \subsetneq \Gamma$ ,
$\delta: Q\times \Gamma^k\to Q \times \Gamma^k \times \{L,R\}^k$ 
(where $k$ is  the number of  states)


If $M$ is a standard  TM, it is a $1$-tape machine.


To translate a $k$-tape machine  to  a standard TM:
Use a  new symbol to separate the contents of each tape
and keep track of location of  head with  special version of each
tape symbol. {\tiny Sipser Theorem 3.13} 

\includegraphics[width=2.5in]{../../resources/images/Figure314.png}

\newpage
\fbox{ {\bf Enumerators} } Enumerators give a different
model of computation where a language is {\bf produced, one string at a time},
rather than recognized by accepting (or not) individual strings.

Each enumerator machine has finite state control, unlimited work tape, and a printer. The computation proceeds
according to transition function; at any point machine may ``send'' a string to the printer.
\[
E  = (Q, \Sigma, \Gamma, \delta, q_0, q_{print})  
\]
$Q$ is the finite set of states, $\Sigma$ is  the output alphabet, $\Gamma$ is the 
tape alphabet ($\Sigma  \subsetneq\Gamma, 
\textvisiblespace \in \Gamma \setminus \Sigma$), 
\[
\delta:  Q  \times  \Gamma \times \Gamma \to  Q \times  \Gamma \times  \Gamma \times \{L, R\} \times  \{L, R\}
\]
where in state $q$, when the working tape is scanning character $x$ and the printer tape is scanning character $y$,
$\delta( (q,x,y) ) = (q', x', y', d_w, d_p)$ means transition to control state $q'$, write $x'$ on 
the working tape, write $y'$ on the printer tape, move in direction $d_w$ on the working tape, and move in direction 
$d_p$ on the printer tape. The computation starts in $q_0$ and each time the computation enters $q_{print}$
the string from the leftmost edge of the printer tape to the first blank cell is considered to be printed.

The language  {\bf  enumerated} by  $E$, $L(E)$, is $\{ w \in \Sigma^* \mid \text{$E$ eventually, at finite  time, 
prints $w$} \}$.

\begin{comment}
\begin{center}
\begin{tabular}{cc}
\includegraphics[width=3.5in]{../../resources/machines/Lec15enumerator.png}  & 
\begin{tabular}{|c|c|c|c|c|c|c|}
\hline
\multicolumn{1}{|c}{$q0$} &  \multicolumn{6}{c|}{\phantom{A}}\\
\hline
$\textvisiblespace ~*$& $\textvisiblespace$  & $\textvisiblespace$ & $\textvisiblespace$& $\textvisiblespace$& $\textvisiblespace$&  $\textvisiblespace$\\
\hline
$\textvisiblespace  ~*$& $\textvisiblespace$  & $\textvisiblespace$ & $\textvisiblespace$& $\textvisiblespace$& $\textvisiblespace$&  $\textvisiblespace$\\
\hline\hline
\multicolumn{7}{|c|}{\phantom{A}}\\
\hline
\phantom{AA} & \phantom{AA}& \phantom{AA}& \phantom{AA}& \phantom{AA}& \phantom{AA}& \phantom{AA} \\
\hline
\phantom{AA} & \phantom{AA}& \phantom{AA}& \phantom{AA}& \phantom{AA}& \phantom{AA}& \phantom{AA} \\
\hline
\hline
\multicolumn{7}{|c|}{\phantom{A}}\\
\hline
\phantom{AA} & \phantom{AA}& \phantom{AA}& \phantom{AA}& \phantom{AA}& \phantom{AA}& \phantom{AA} \\
\hline
\phantom{AA} & \phantom{AA}& \phantom{AA}& \phantom{AA}& \phantom{AA}& \phantom{AA}& \phantom{AA} \\
\hline
\hline
\multicolumn{7}{|c|}{\phantom{A}}\\
\hline
\phantom{AA} & \phantom{AA}& \phantom{AA}& \phantom{AA}& \phantom{AA}& \phantom{AA}& \phantom{AA} \\
\hline
\phantom{AA} & \phantom{AA}& \phantom{AA}& \phantom{AA}& \phantom{AA}& \phantom{AA}& \phantom{AA} \\
\hline
\hline
\multicolumn{7}{|c|}{\phantom{A}}\\
\hline
\phantom{AA} & \phantom{AA}& \phantom{AA}& \phantom{AA}& \phantom{AA}& \phantom{AA}& \phantom{AA} \\
\hline
\phantom{AA} & \phantom{AA}& \phantom{AA}& \phantom{AA}& \phantom{AA}& \phantom{AA}& \phantom{AA} \\
\hline
\end{tabular}
\end{tabular}
\end{center}


\newpage
\end{comment}

{\bf Theorem 3.21} A language is Turing-recognizable iff some enumerator enumerates it.

{\bf Proof, part 1}: Assume $L$ is enumerated by some enumerator, $E$, so $L = L(E)$. We'll use $E$ in a subroutine
within a high-level description of a new Turing machine that we will build to recognize $L$.

{\bf Goal}: build Turing machine $M_E$ with $L(M_E) = L(E)$.

Define $M_E$ as follows: $M_E = $ ``On input $w$,
\begin{enumerate}
\item Run $E$. For each string $x$ printed by $E$.
\item \qquad Check if $x = w$. If so, accept (and halt); otherwise, continue."
\end{enumerate}

{\bf Proof, part 2}: Assume $L$ is Turing-recognizable and there 
is a Turing  machine  $M$ with  $L = L(M)$. We'll use $M$ in a subroutine
within a high-level description of an enumerator that we will build to enumerate $L$.

{\bf Goal}: build enumerator $E_M$ with $L(E_M) = L(M)$.

{\bf Idea}: check each string in turn to see if it is in $L$.

{\it How?} Run computation of $M$ on each string.  {\it But}: need to be careful 
about computations that don't halt.

{\it Recall} String order for $\Sigma = \{0,1\}$: $s_1 = \varepsilon$, $s_2 = 0$, $s_3 = 1$, $s_4 = 00$, $s_5 = 01$, $s_6  = 10$, 
$s_7  =  11$, $s_8 = 000$, \ldots

Define $E_M$ as follows: $E_{M} = $ `` {\it ignore any input.} Repeat the following for $i=1, 2, 3, \ldots$
\begin{enumerate}
  \item Run the computations of $M$ on $s_1$, $s_2$, \ldots, $s_i$ for (at most) $i$ steps each
  \item For each of these $i$ computations that accept during the (at most) $i$ steps, print
  out the accepted string."
\end{enumerate}

\vfill

\fbox{ {\bf Nondeterministic Turing machine}}

At any point in the computation, the nondeterministic machine may proceed according to 
several possibilities: $(Q, \Sigma, \Gamma, \delta, q_0, q_{acc}, q_{rej})$ where 
\[
\delta: Q \times \Gamma \to \mathcal{P}(Q \times \Gamma \times \{L, R\})  
\]
The computation of a nondeterministic Turing machine is a tree with branching
when the next step of the computation has multiple possibilities. A nondeterministic
Turing machine accepts a string exactly when some branch of the computation tree 
enters the accept state.

Given a nondeterministic machine, we can use a $3$-tape Turing machine to 
simulate it by doing a breadth-first search of computation tree: one tape 
is ``read-only'' input tape, one tape simulates the tape of the nondeterministic
computation, and one tape tracks nondeterministic branching. {\tiny Sipser page 178} 

\vfill

{\bf Summary}

Two models of computation are called {\bf equally expressive} when 
every language recognizable with the first model is recognizable with the second, and vice versa.

To prove the existence of a Turing machine that decides / recognizes some language, 
it's enough to construct an example using any of the equally expressive models.

But: some of the {\bf performance} properties of these models are not equivalent.

\vfill