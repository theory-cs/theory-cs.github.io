%! app: Decidable Languages, Undecidable Languages
%! outcome: Classify language, Find example languages, Define decision problem, Classify decision problem
    
{\bf The Church-Turing thesis posits that each algorithm can be implemented by some Turing machine.}

{\bf Describing  algorithms} (Sipser p. 185) To define a Turing machine, we could give a 
\begin{itemize}
\item {\bf Formal definition}: the $7$-tuple of parameters including set of states, 
input alphabet, tape alphabet, transition function, start state, accept state, and reject state.
This is the low-level programming view that models the logic computation flow in a processor.
\item {\bf Implementation-level definition}: English prose that describes the Turing machine head 
movements relative to contents of tape, and conditions for accepting / rejecting based on those contents.
This level describes memory management and implementing data access with data structures.
  \begin{itemize}
    \item Mention the tape or its contents (e.g.\ ``Scan the tape from left to right until a blank is seen.'')
    \item Mention the tape head (e.g.\ ``Return the tape head to the left end of the tape.'')
  \end{itemize}
\item {\bf High-level description} of algorithm executed by Turing machine: 
description of algorithm (precise sequence of instructions), 
without implementation details of machine. 
High-level descriptions of  Turing machine algorithms are written as indented text within quotation marks.   
Stages of the algorithm are typically numbered consecutively.
The first line specifies the input to the machine, which must be a string.
  \begin{itemize}
    \item Use other Turing machines as subroutines (e.g.\ ``Run $M$ on $w$'')
    \item Build new machines from existing machines using previously shown results (e.g.\ 
    ``Given NFA $A$ construct an NFA $B$ such that $L(B) = \overline{L(A)}$'')
    \item Use previously shown conversions and constructions (e.g.\ ``Convert regular expression $R$ 
    to an NFA $N$'')
  \end{itemize}
\end{itemize}

{\bf Formatted inputs to Turing machine algorithms}

The input to a Turing machine is always a string. 
The format of the input to a Turing machine can be checked to interpret 
this string as representing structured data (like a csv file, the formal definition of a DFA, another Turing machine, etc.)


This string may be the encoding of some object or  list of  objects.  

{\bf Notation:} $\langle O \rangle$ is the string that encodes the object $O$.
$\langle O_1, \ldots, O_n \rangle$ is the string that encodes the list of objects $O_1, \ldots, O_n$.

{\bf Assumption}: There are algorithms (Turing  machines) that can be called as subroutines
to decode the string representations of common objects and  interact with these objects as intended
(data structures). These algorithms are able to ``type-check'' and string representations for different
data structures are unique.
  
\newpage
For example, since there are algorithms to answer each of the following questions,
by Church-Turing thesis, there is a Turing machine that accepts exactly those strings for which the 
answer to the question is ``yes''
\begin{itemize}
    \item Does a string over $\{0,1\}$ have even length?

    \item Does a string over $\{0,1\}$ encode a string of ASCII characters?\footnote{An introduction to ASCII 
    is available on the w3 tutorial \href{https://www.w3schools.com/charsets/ref_html_ascii.asp}{here}.}

    \item Does a DFA have a specific number of states?

    \item Do two NFAs have any state names in common?

    \item Do two CFGs have the same start variable?

  \end{itemize}



A {\bf computational problem} is decidable iff language encoding its positive problem instances
is decidable.

The computational problem ``Does a specific DFA accept a given string?'' is encoded by the language
\begin{align*}
  &\{ \textrm{representations of DFAs $M$ and strings $w$ such that $w \in L(M)$}\}  \\
  =& \{ \langle M, w \rangle \mid M \textrm{ is a DFA}, w \textrm{ is a string}, w \in L(M) \}
\end{align*}

The computational problem ``Is the language generated by a CFG empty?'' is encoded by the language
\begin{align*}
  &\{ \textrm{representations of CFGs $G$  such that $L(G) = \emptyset$}\}  \\
  =& \{ \langle G \rangle \mid G \textrm{ is a CFG},  L(G) = \emptyset \}
\end{align*}



The computational problem ``Is the given Turing machine a decider?'' is encoded by the language
\begin{align*}
  &\{ \textrm{representations of TMs $M$  such that $M$ halts on every input}\}  \\
  =& \{ \langle M \rangle \mid M \textrm{ is a TM and for each string } w, \textrm{$M$ halts on $w$} \}
\end{align*}


{\it Note: writing down the language encoding a computational problem is only the first step in 
determining if it's recognizable, decidable, or \ldots }

Deciding a computational problem means building / defining a Turing 
machine that recognizes the language encoding the computational problem, and that 
is a decider.


Some classes of computational problems will 
help us understand the differences between the machine models we've been studying. (Sipser Section 4.1)

    \begin{center}
    \begin{tabular}{|lcl|}
    \hline
    \multicolumn{3}{|l|}{{\bf  Acceptance problem} } \\
    & & \\
    \ldots for DFA & $A_{DFA}$ & $\{ \langle B,w \rangle \mid  \text{$B$ is a  DFA that accepts input 
    string $w$}\}$ \\
    \ldots for NFA & $A_{NFA}$ & $\{ \langle B,w \rangle \mid  \text{$B$ is a  NFA that accepts input 
    string $w$}\}$ \\
    \ldots for regular expressions & $A_{REX}$ & $\{ \langle R,w \rangle \mid  \text{$R$ is a  regular
    expression that generates input string $w$}\}$ \\
    \ldots for CFG & $A_{CFG}$ & $\{ \langle G,w \rangle \mid  \text{$G$ is a context-free grammar 
    that generates input string $w$}\}$ \\
    \ldots for PDA & $A_{PDA}$ & $\{ \langle B,w \rangle \mid  \text{$B$ is a PDA that accepts input string $w$}\}$ \\
    & & \\
    \hline
    \multicolumn{3}{|l|}{{\bf Language emptiness  testing} } \\
    & & \\
    \ldots for DFA & $E_{DFA}$ & $\{ \langle A \rangle \mid  \text{$A$ is a  DFA and  $L(A) = \emptyset$\}}$ \\
    \ldots for NFA & $E_{NFA}$ & $\{ \langle A\rangle \mid  \text{$A$ is a NFA and  $L(A) = \emptyset$\}}$ \\
    \ldots for regular expressions & $E_{REX}$ & $\{ \langle R \rangle \mid  \text{$R$ is a  regular
    expression and  $L(R) = \emptyset$\}}$ \\
    \ldots for CFG & $E_{CFG}$ & $\{ \langle G \rangle \mid  \text{$G$ is a context-free grammar 
    and  $L(G) = \emptyset$\}}$ \\
    \ldots for PDA & $E_{PDA}$ & $\{ \langle A \rangle \mid  \text{$A$ is a PDA and  $L(A) = \emptyset$\}}$ \\
    & & \\
    \hline
    \multicolumn{3}{|l|}{{\bf Language equality testing} } \\
    & & \\
    \ldots for DFA & $EQ_{DFA}$ & $\{ \langle A, B \rangle \mid  \text{$A$ and $B$ are DFAs and  $L(A) =L(B)$\}}$\\
    \ldots for NFA & $EQ_{NFA}$ & $\{ \langle A, B \rangle \mid  \text{$A$ and $B$ are NFAs and  $L(A) =L(B)$\}}$\\
    \ldots for regular expressions & $EQ_{REX}$ & $\{ \langle R, R' \rangle \mid  \text{$R$ and $R'$ are regular
    expressions and  $L(R) =L(R')$\}}$\\
    \ldots for CFG & $EQ_{CFG}$ & $\{ \langle G, G' \rangle \mid  \text{$G$ and $G'$ are CFGs and  $L(G) =L(G')$\}}$ \\
    \ldots for PDA & $EQ_{PDA}$ & $\{ \langle A, B \rangle \mid  \text{$A$ and $B$ are PDAs and  $L(A) =L(B)$\}}$ \\
    \hline
    \end{tabular}
    \end{center}
    

    Example strings in $A_{DFA}$
    
    \vfill
    
    Example strings in  $E_{DFA}$
    
    \vfill
    
    Example strings in  $EQ_{DFA}$
    
    \vfill
