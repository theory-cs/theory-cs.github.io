%! app: Regular Languages 
%! outcome: Regular expressions, Formal definition of automata, Informal definition of automata


**This definition was in the pre-class reading**
A finite automaton (FA) is specified by  $M = (Q, \Sigma, \delta, q_0, F)$.
This $5$-tuple is called the {\bf formal definition} of the FA. The FA can also 
be represented by its state diagram: with nodes for the state, labelled edges specifying the 
transition function, and decorations on nodes denoting the start and accept states.

\begin{quote}
Finite set of states $Q$ can be labelled by any collection of distinct names. Often
we use default state labels $q0, q1, \ldots$ 
\end{quote}

\begin{quote}  
The alphabet $\Sigma$ determines the possible inputs to the automaton. 
Each input to the automaton is a string over  $\Sigma$, and the automaton ``processes'' the input
one symbol (or character) at a time.
\end{quote}

\begin{quote}
The transition function $\delta$ gives the next state of the automaton based on the current state of 
the machine and on the next input symbol.
\end{quote}

\begin{quote}
The start state $q_0$ is an element of $Q$.  Each computation of the machine starts at the  start  state.
\end{quote}

\begin{quote}
The accept (final) states $F$ form a subset of the states of the automaton, $F \subseteq  Q$. 
These states are used to flag if the machine accepts or rejects an input string.
\end{quote}


\begin{quote}
The computation of a machine on an input string is a sequence of states
in the machine,  starting with the start state, determined by transitions 
of the machine as it reads successive input symbols.
\end{quote}

\begin{quote}
The finite automaton $M$ accepts the given input string exactly when the computation of $M$ on the input string
ends in an accept state. $M$ rejects the given input string exactly when the computation of 
$M$ on the input string ends in a nonaccept state, that is, a state that is not in $F$.
\end{quote}

\begin{quote} 
The language of $M$, $L(M)$, is defined as the set of  all strings that are each accepted 
by the machine $M$. Each string that is rejected by $M$ is not in $L(M)$.
The language of $M$ is also called the language recognized by $M$.
\end{quote}   
   
What is {\bf finite} about all finite automata? (Select all that apply)
\begin{itemize}
   \item[$\square$] The size of the machine (number of states, number of arrows)
   \item[$\square$] The length of each computation of the machine
   \item[$\square$] The number of strings that are accepted by the machine
\end{itemize}
\newpage
  
\begin{center}
\begin{tikzpicture}[->,>=stealth',shorten >=1pt, auto, node distance=2cm, semithick]
   \tikzstyle{every state}=[text=black, fill=none]
   
   \node[initial,state,accepting] (q0)          {$q0$};
   \node[state,accepting]         (q1) [above right of=q0, xshift=20pt] {$q1$};
   \node[state,accepting]         (q2) [below right of=q0, xshift=20pt] {$q2$};
   \node[state]                   (q3) [below right of=q1, xshift=40pt] {$q3$};
   
   \path (q0) edge  [bend left=0] node {$a$} (q1)
       (q1) edge [loop above] node {$a$} (q1)
       (q1) edge [bend left=0] node {$b$} (q3)
       (q0) edge [bend left=0] node {$b$} (q2)
       (q2) edge [loop above] node {$b$} (q2)
       (q2) edge [bend left=0] node {$a$} (q3)
       (q3) edge [loop above] node {$a,b$} (q3)
   ;
\end{tikzpicture}
\end{center}
The formal definition of this FA is
   
\vfill
\vfill
   

Classify each string $a, aa, ab, ba, bb, \varepsilon$ as accepted by the FA or rejected by the FA.  

{\it Why are these the only two options?}

\vspace{200pt}


The language recognized by this automaton is
  

\vfill

\newpage

\begin{center}
   \begin{tikzpicture}[->,>=stealth',shorten >=1pt, auto, node distance=2cm, semithick]
      \tikzstyle{every state}=[text=black, fill=none]
      
      \node[initial,state,accepting] (q0)          {$q0$};
      \node[state,accepting]         (q1) [right of=q0, xshift=20pt] {$q1$};
      \node[state]         (q2) [right of=q1, xshift=20pt] {$q2$};
      
      \path (q0) edge  [bend left=0] node {$b$} (q1)
          (q0) edge [loop above] node {$a$} (q0)
          (q1) edge [bend left=0] node {$a$} (q2)
          (q1) edge [loop above] node {$b$} (q1)
          (q2) edge [loop above] node {$a,b$} (q2)
      ;
   \end{tikzpicture}
\end{center}

The language recognized by this automaton is
  


\vfill

\hrule

\begin{center}
   \begin{tikzpicture}[->,>=stealth',shorten >=1pt, auto, node distance=2cm, semithick]
      \tikzstyle{every state}=[text=black, fill=none]
      
      \node[initial,state] (q0)          {$q0$};
      \node[state,accepting]         (q1) [above right of=q0, xshift=10pt, yshift=20pt] {$q1$};
      \node[state,accepting]         (q2) [right of=q1, xshift=20pt] {$q2$};
      \node[state]          (q3) [below right of=q2, xshift=10pt, yshift=-20pt] {$q3$};
      
      \path (q0) edge  [bend left=0] node {$a$} (q1)
          (q0) edge [bend left=0] node {$b$} (q3)
          (q1) edge [bend left=0] node {$b$} (q2)
          (q1) edge [bend left=0] node {$a$} (q3)
          (q2) edge [bend left=0] node {$a,b$} (q3)
          (q3) edge [loop right] node {$a,b$} (q3)
      ;
   \end{tikzpicture}
\end{center}

The language recognized by this automaton is
  

\vfill