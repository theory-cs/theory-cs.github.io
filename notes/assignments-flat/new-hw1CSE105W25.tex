\documentclass[12pt, oneside]{article}

\usepackage[letterpaper, scale=0.8, centering]{geometry}
\usepackage{fancyhdr}
\setlength{\parindent}{0em}
\setlength{\parskip}{1em}

\pagestyle{fancy}
\fancyhf{}
\renewcommand{\headrulewidth}{0pt}
\rfoot{{\footnotesize Copyright Mia Minnes, 2025, Version \today~(\thepage)}}

\usepackage{titlesec}

\author{CSE105W25}

\newcommand{\instructions}{{\bf For all HW assignments:} Weekly homework 
may be done individually or in groups of up to 3 students. 
You may switch HW partners for different HW assignments. 
Please ensure your name(s) and PID(s) are clearly visible on the first page of your homework submission 
and then upload the PDF to Gradescope. If working in a group, submit only one submission per group: 
one partner uploads the submission through their Gradescope account and then adds the other group member(s) 
to the Gradescope submission by selecting their name(s) in the ``Add Group Members" dialog box. 
You will need to re-add your group member(s) every time you resubmit a new version of your assignment.
 Each homework question will be graded either for correctness (including clear and precise explanations and 
 justifications of all answers) or fair effort completeness. 
 On the ``graded for correctness"
 questions, you may only collaborate with CSE 105 students in your group; if your
 group has questions about a problem, you may ask in drop-in help hours or post a private
post (visible only to the Instructors) on Piazza. On the "graded for completeness" questions, you 
may collaborate with all other CSE 105 students this quarter, and you may make public posts about these questions 
on Piazza.

All submitted homework for this class must be typed. 
You can use a word processing editor if you like (Microsoft Word, Open Office, Notepad, Vim, Google Docs, etc.) 
but you might find it useful to take this opportunity to learn LaTeX. 
LaTeX is a markup language used widely in computer science and mathematics. 
The homework assignments are typed using LaTeX and you can use the source files 
as templates for typesetting your solutions.
To generate state diagrams of machines, you can (1) use the LaTex tikzpicture
environment (see templates in the class notes), or (2) use the software tools Flap.js or
JFLAP described in the class syllabus (and include a screenshot in your PDF), or (3) you can carefully
and clearly hand-draw
the diagram and take a picture and include it in your PDF.
We recommend that you
submit early drafts to Gradescope so that in case of any technical difficulties, at least some of your
work is present. You may update your submission as many times as you'd like up to the deadline.


{\bf Integrity reminders}
\begin{itemize}
\item Problems should be solved together, not divided up between the partners. The homework is
designed to give you practice with the main concepts and techniques of the course, 
while getting to know and learn from your classmates.
\item On the ``graded for correctness"
questions, you may only collaborate with CSE 105 students in your group.
You may ask questions about the homework in office hours (of the instructor, TAs, and/or tutors) and 
on Piazza (as private notes viewable only to the Instructors).  
You \emph{cannot} use any online resources about the course content other than the class material 
from this quarter -- this is primarily to ensure that we all use consistent notation and
definitions (aligned with the textbook) and also to protect the learning experience you will have when
the `aha' moments of solving the problem authentically happen.
\item Do not share written solutions or partial solutions for homework with 
other students in the class who are not in your group. Doing so would dilute their learning 
experience and detract from their success in the class.
\end{itemize}

}

\newcommand{\gradeCorrect}{({\it Graded for correctness}) }
\newcommand{\gradeCorrectFirst}{\gradeCorrect\footnote{This means your solution 
will be evaluated not only on the correctness of your answers, but on your ability
to present your ideas clearly and logically. You should explain how you 
arrived at your conclusions, using
mathematically sound reasoning. Whether you use formal proof techniques or 
write a more informal argument
for why something is true, your answers should always be well-supported. 
Your goal should be to convince the
reader that your results and methods are sound.} }
\newcommand{\gradeComplete}{({\it Graded for completeness}) }
\newcommand{\gradeCompleteFirst}{\gradeComplete\footnote{This means you will 
get full credit so long as your submission demonstrates honest effort to 
answer the question. You will not be penalized for incorrect answers. 
To demonstrate your honest effort in answering the question, we 
expect you to include your attempt to answer *each* part of the question. 
If you get stuck with your attempt, you can still demonstrate 
your effort by explaining where you got stuck and what 
you did to try to get unstuck.} }

\usepackage{tikz}
\usetikzlibrary{automata,positioning,arrows}

\usepackage{amssymb,amsmath,pifont,amsfonts,comment,enumerate,enumitem}
\usepackage{currfile,xstring,hyperref,tabularx,graphicx,wasysym}
\usepackage[labelformat=empty]{caption}
\usepackage{xcolor}
\usepackage{multicol,multirow,array,listings,tabularx,lastpage,textcomp,booktabs}

\lstnewenvironment{algorithm}[1][] {   
    \lstset{ mathescape=true,
        frame=tB,
        numbers=left, 
        numberstyle=\tiny,
        basicstyle=\rmfamily\scriptsize, 
        keywordstyle=\color{black}\bfseries,
        keywords={,procedure, div, for, to, input, output, return, datatype, function, in, if, else, foreach, while, begin, end, }
        numbers=left,
        xleftmargin=.04\textwidth,
        #1
    }
}
{}

\newcommand\abs[1]{\lvert~#1~\rvert}
\newcommand{\st}{\mid}

\newcommand{\cmark}{\ding{51}}
\newcommand{\xmark}{\ding{55}}
 
\newcommand{\SUBSTRING}{\textsc{Substring}}
\newcommand{\REP}{\textsc{Rep}}
\newcommand{\blank}{\scalebox{1.5}{\textvisiblespace}}
 
\title{HW1CSE105W25: Homework assignment 1}
\date{Due: January 16th at 5pm, via Gradescope}


\begin{document}
\maketitle
\thispagestyle{fancy}

{\bf In this assignment,}

You will practice reading and
applying the definitions of alphabets, strings, languages, Kleene star, and regular expressions.
You will use regular expressions and relate them to languages.


{\bf Resources}: To review the topics 
for this assignment, see the class material from Weeks 0 and 1 and Review Quiz 1.
We will post frequently asked questions and our answers to them in a 
pinned Piazza post.

{\bf Reading and extra practice problems}: Sipser Section 0, 1.3.
Chapter 0 exercises 0.1, 0.2, 0.3, 0.5, 0.6, 0.9. Chapter 1 exercises 1.19, 1.23.

\instructions

You will submit this assignment via Gradescope
(\href{https://www.gradescope.com}{https://www.gradescope.com}) 
in the assignment called ``hw1CSE105W25''.

{\bf Assigned questions}


\begin{enumerate}[wide, labelwidth=!, labelindent=0pt]
\item\textbf{Strings and languages: finding examples and edge cases} (12 points):
    \begin{enumerate}
    \item\gradeCompleteFirst  Give five (different) example alphabets that are meaningful or useful to you in some way. Specify them formally, either with 
    roster notation (which means listing all and only distict elements between $\{$ and $\}$ and separated by commas) or with another approach to precisely define all and only the elements of the alphabet.

    \item\gradeComplete Give an example of a finite set over an alphabet  and an infinite set over an alphabet. You get to choose the alphabet, and you get to choose the sets.  The goal is to practice communicating your choices and definitions with clear and precise notation. One habit that will be useful (for this course, and beyond), is to think of your response for each question as a well-formed paragraph: include all the information that is relevant so that your solution is self-contained, and so that each sentence is grammatically constructed.
    
    \item\gradeCorrectFirst Define an alphabet $\Sigma_1$ and an alphabet $\Sigma_2$ and a language $L_1$ over $\Sigma_1$ that is also a language over $\Sigma_2$ and a language $L_2$ over $\Sigma_2$ that is {\bf not} a language over $\Sigma_1$. 
    A complete and correct answer will use clear and precise notation
    (consistent with the textbook and class notes) and will include a description of why the given example $L_1$
    is a language over both $\Sigma_1$ and $\Sigma_2$ and a description 
    of why the given example $L_2$ is a language over $\Sigma_2$ and not over $\Sigma_1$.

    \end{enumerate}

\item\textbf{Regular expressions} (20 points):

    \begin{enumerate}
    \item\gradeComplete  Give three regular expressions that all describe the set of all strings over $\{a,b\}$ that have 
    odd length. Ungraded bonus challenge: Make the expressions as different as possible!

    \item\gradeComplete  A friend tells you that each regular expression that has a Kleene star ($~^*$) describes an
    infinite language. Are they right? Either help them justify their claim or give a counterexample to disprove it
    and explain your counterexample.

    \item\gradeCorrect For this question, the alphabet is $\{a,b,c\}$. A friend is trying to design a regular expression that describes the set of all strings over this alphabet that end in $c$. Classify each of the following attempts as 
    \begin{itemize}
    \item Correct. Explain why.
    \item Error Type 1: Incorrect, because (even though each string that is in the language described by the regular expression ends in $c$) there is a string that ends in $c$ that is not in the language described by the regular expression. Give this example string and explain why it proves we're in this case. 
    \item Error Type 2: Incorrect, because (even though each string that ends in $c$ is in the language described by the regular expression), there is a string in the language described by the regular expression that does not end in $c$. Give this example string and explain why it proves we're in this case. 
    \item Error Type 3: Incorrect, because there are two counterexample strings, one which is a string that ends in $c$ that is not in the language described by the regular expression and one which is in the language described by the regular expression but does not end in $c$.Give both example strings and describe why each has the given property.
    \end{itemize}

    \vfill
    \newpage

    \hrule
    {\it Worked example for reference:} Consider the regular expression $(a\cup b\cup c)^*$. This regular expression has {\bf Error Type 2} because it describes the set of all strings over $\{a,b,c\}$, so even though each string that ends in $c$ is in this language, there is an example, say $ab$ that is a string in the language described by the regular expression (because we consider the string formed as a result of the Kleene star operation which has 2 slots and where the first slot matches the $a$ in $a \cup b \cup c$ and the second slot matches $b$ in $a \cup b \cup c$) but does not end in $c$ (it ends in $b$).
    \hrule
    \begin{enumerate}
        \item The regular expression is   
        \[
        (a\cup b)^* \circ c
        \]
        \item The regular expression is  
        \[
        (a \circ b \circ c)^*
        \]
        \item The regular expression is  
        \[
        a^*c ~\cup~ b^*c ~\cup~ c^*c
        \]
    \end{enumerate}
    \end{enumerate}

\item\textbf{Functions over languages} (18 points):

For each language $L$ over an alphabet $\Sigma$, we have the 
associated sets of strings (also over $\Sigma$)
\[
    L^* = \{ w_1 \cdots w_k \mid k \geq 0 \textrm{ and each } w_i \in L\}
\]
and
\[
    SUBSTRING(L) = \{ w \in \Sigma^* ~|~ \text{there exist } x,y \in \Sigma^* \text{ such that } xwy \in L\}
\]
and 
\[
    EXTEND(L) = \{ w \in \Sigma^* ~|~ w = uv \text{ for some strings } u \in L \text{ and } v \in \Sigma^* \}
\]
Also, recall the set operations union and intersection: for any sets $X$ and $Y$
\[
X \cup Y = \{ w \mid w \in X \text{ or } w \in Y \}
\]
\[
X \cap Y = \{ w \mid w \in X \text{ and } w \in Y \}
\]

    \begin{enumerate}
    \item\gradeComplete Specify an example language $A$ over $\{0,1\}$ such that 
    $$SUBSTRING(A) = EXTEND(A)$$
    or explain why there is no such example. 
    A complete solution will include either (1) a precise and
    clear description of your example language $A$ 
    and a precise and clear description of
    the result of computing $SUBSTRING(A)$, $EXTEND(A)$ (using the given definitions)
    to justify this description and to justify the set equality,
    or (2) a sufficiently general and correct argument
    why there is no such example, referring back to the relevant definitions.

    \item\gradeCorrect Specify an example language $B$ over $\{0,1\}$ such that 
    $$SUBSTRING(B) \cap EXTEND(B) = \{\varepsilon\}$$ and $$SUBSTRING(B) \cup EXTEND(B) = \{0,1\}^*$$
    or explain why there is no such example. 
    A complete solution will include either (1) a precise and
    clear description of your example language $B$ 
    and a precise and clear description of
    the result of computing $SUBSTRING(B)$, $EXTEND(B)$ (using the given definitions)
    to justify this description and to justify the set equality with 
    $\{\varepsilon\}$ and $\{0,1\}^*$ (respectively), or (2) a sufficiently general and correct argument
    why there is no such example, referring back to the relevant definitions.

    \item\gradeCorrect Specify an example {\bf infinite} language $C$ over $\{0,1\}$ such that 
    $$SUBSTRING(C) \neq \{0,1\}^*$$ and $$SUBSTRING(C) = C^*$$or 
    explain why there is no such example.
    A complete solution will include either (1) a precise and
    clear description of your example language $C$ 
    and a precise and clear description of
    the result of computing $SUBSTRING(C)$, $C^*$ (using the given definitions)
    to justify this description and to justify the set nonequality claims, 
    or (2) a sufficiently general and correct argument
    why there is no such example, referring back to the relevant definitions.


    \item\gradeCorrect Specify an example {\bf finite} language $D$ over $\{0,1\}$ such that 
    $$EXTEND(D) \neq \{0,1\}^*$$ and $$EXTEND(D) = D^*$$or 
    explain why there is no such example.
    A complete solution will include either (1) a precise and
    clear description of your example language $D$ 
    and a precise and clear description of
    the result of computing $EXTEND(D)$, $D^*$ (using the given definitions)
    to justify this description and to justify the set nonequality claims, 
    or (2) a sufficiently general and correct argument
    why there is no such example, referring back to the relevant definitions.
    \end{enumerate}



    
    \end{enumerate}
\end{document}