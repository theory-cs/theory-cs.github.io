\documentclass[12pt, oneside]{article}

\usepackage[letterpaper, scale=0.8, centering]{geometry}
\usepackage{fancyhdr}
\setlength{\parindent}{0em}
\setlength{\parskip}{1em}

\pagestyle{fancy}
\fancyhf{}
\renewcommand{\headrulewidth}{0pt}
\rfoot{{\footnotesize Copyright Mia Minnes, 2022, Version \today~(\thepage)}}

\author{CSE105Sp22}

\newcommand{\instructions}{{\bf For all HW assignments:}

Weekly homework may be done individually or in groups of up to 3 students. 
You may switch HW partners for different HW assignments. 
The lowest HW score will not be included in your overall HW average. 
Please ensure your name(s) and PID(s) are clearly visible on the first page of your homework submission 
and then upload the PDF to Gradescope. If working in a group, submit only one submission per group: 
one partner uploads the submission through their Gradescope account and then adds the other group member(s) 
to the Gradescope submission by selecting their name(s) in the ``Add Group Members" dialog box. 
You will need to re-add your group member(s) every time you resubmit a new version of your assignment.
 Each homework question will be graded either for correctness (including clear and precise explanations and 
 justifications of all answers) or fair effort completeness. You may only collaborate on HW with CSE 105 students 
 in your group; if your group has questions about a HW problem, you may ask in drop-in help hours or post a private 
 post (visible only to the Instructors) on Piazza.

All submitted homework for this class must be typed. 
You can use a word processing editor if you like (Microsoft Word, Open Office, Notepad, Vim, Google Docs, etc.) 
but you might find it useful to take this opportunity to learn LaTeX. 
LaTeX is a markup language used widely in computer science and mathematics. 
The homework assignments are typed using LaTeX and you can use the source files 
as templates for typesetting your solutions.
To generate state diagrams of machines, we recommend using Flap.js
or JFLAP. Photographs of clearly hand-drawn diagrams may also be used. We recommend that you
submit early drafts to Gradescope so that in case of any technical difficulties, at least some of your
work is present. You may update your submission as many times as you'd like up to the deadline.


{\bf Integrity reminders}
\begin{itemize}
\item Problems should be solved together, not divided up between the partners. The homework is
designed to give you practice with the main concepts and techniques of the course, 
while getting to know and learn from your classmates.
\item You may not collaborate on homework with anyone other than your group members.
You may ask questions about the homework in office hours (of the instructor, TAs, and/or tutors) and 
on Piazza (as private notes viewable only to the Instructors).  
You \emph{cannot} use any online resources about the course content other than the class material 
from this quarter -- this is primarily to ensure that we all use consistent notation and
definitions we will use this quarter and also to protect the learning experience you will have when
the `aha' moments of solving the problem authentically happen.
\item Do not share written solutions or partial solutions for homework with 
other students in the class who are not in your group. Doing so would dilute their learning 
experience and detract from their success in the class.
\end{itemize}

}
\usepackage{amssymb,amsmath,pifont,amsfonts,comment,enumerate,enumitem}
\usepackage{currfile,xstring,hyperref,tabularx,graphicx,wasysym}
\usepackage[labelformat=empty]{caption}
\usepackage[dvipsnames,table]{xcolor}
\usepackage{multicol,multirow,array,listings,tabularx,lastpage,textcomp,booktabs}

\lstnewenvironment{algorithm}[1][] {   
    \lstset{ mathescape=true,
        frame=tB,
        numbers=left, 
        numberstyle=\tiny,
        basicstyle=\rmfamily\scriptsize, 
        keywordstyle=\color{black}\bfseries,
        keywords={,procedure, div, for, to, input, output, return, datatype, function, in, if, else, foreach, while, begin, end, }
        numbers=left,
        xleftmargin=.04\textwidth,
        #1
    }
}
{}
\lstnewenvironment{java}[1][]
{   
    \lstset{
        language=java,
        mathescape=true,
        frame=tB,
        numbers=left, 
        numberstyle=\tiny,
        basicstyle=\ttfamily\scriptsize, 
        keywordstyle=\color{black}\bfseries,
        keywords={, int, double, for, return, if, else, while, }
        numbers=left,
        xleftmargin=.04\textwidth,
        #1
    }
}
{}

\newcommand\abs[1]{\lvert~#1~\rvert}
\newcommand{\st}{\mid}

\newcommand{\A}[0]{\texttt{A}}
\newcommand{\C}[0]{\texttt{C}}
\newcommand{\G}[0]{\texttt{G}}
\newcommand{\U}[0]{\texttt{U}}

\newcommand{\cmark}{\ding{51}}
\newcommand{\xmark}{\ding{55}}
 
 
\title{Project}
\date{Part 1 due TBA; Part 2 due TBA; Part 3 due TBA}

\begin{document}
\maketitle
\thispagestyle{fancy}
The project component of this class will be an opportunity for you to extend your 
work on assignments and explore applications of your choosing. 

{\it Why?}
TBA

{\it How?} During emergency remote instruction last academic year, we discovered
that video assessement and some open-ended personalized projects help ensure fairness
and can be less stressful for students than in-person midterm exams. Asynchronous project
submission also gives flexibility and allows more physical distancing.

Your videos: We will delete all the videos we receive from you after assigning final grades for the course, 
and they will be stored in a university-controlled Google Drive directory 
only accessible to the course staff during the quarter. 
Please send an email to the instructor (minnes@eng.ucsd.edu) if you have 
concerns about 
the video / screencast components of this project or cannot complete projects in this style for some reason.

You may produce screencasts with any software you choose. 
One option is to record yourself with Zoom; a tutorial on how to use Zoom to record a 
screencast (courtesy of Prof. Joe Politz)  is here: 

\url{https://drive.google.com/open?id=1KROMAQuTCk40zwrEFotlYSJJQdcG_GUU}.

The video that was produced from that recording session in Zoom is here:

\url{https://drive.google.com/open?id=1MxJN6CQcXqIbOekDYMxjh7mTt1TyRVMl}

\subsection*{What resources can you use?}
This project must be completed individually, without any help from other people, 
including the course staff (other than logistics support if you get stuck with screencast). 

You can use any of this quarter's CSE 20 offering (notes, readings, class videos, homework feedback). 
These resources should be more than enough. If you are struggling to get started and want to 
look elsewhere online, you must acknowledge this by listing and citing any resources you consult 
(even if you do not explicitly quote them). Link directly to them and include the name of the 
author / video creator and the reason you consulted this reference. The work you submit for 
the project needs to be your own. Again, you shouldn't need to look anywhere other 
than this quarter's material and doing so may result in definitions or notations 
that conflict with our norms in this class so think carefully before you go down this path.

The project has three parts. 
\begin{itemize}
    \item Part 1 of Project: due TBA
    \item Part 2 of Project: due TBA
    \item Part 3 of Project: due TBA
\end{itemize}

\newpage
\subsection*{Part 1: due TBA}
\subsubsection*{Written component}


\subsubsection*{Video component}
Presenting your reasoning and demonstrating it via screenshare are important 
skills that also show us a lot of your learning. Getting practice with this style of 
presentation is a good thing for you to learn in general and a rich way for us to assess your skills. 

Prepare a 3-5 minute screencast video that starts with 
your face and your student ID for a few seconds at the beginning, and introduce yourself audibly while on screen. 
You don't have to be on camera for the rest of the video, though it's fine if you are. 
We are looking for a brief confirmation that it's you creating the video and doing the work 
submitted for the project.

Then, explain your work in question 1 of the written component.
Discuss at least one potential mistake that someone solving 
a similar question should avoid (this could be a mistake you made while thinking about this 
problem or something you anticipate a classmate might struggle with); explain why the 
mistake is wrong and how to fix it.
 
TBA

Gradescope online submission

\subsubsection*{Checklist (this is how we will grade Part 1 of the project)}
\begin{itemize}
\item Question 1: TBA
\end{itemize}

\newpage
\subsection*{Part 2: due TBA}
\subsubsection*{Written component}
\begin{enumerate}
\item In this part of the project, you will select one question from one of the review quizzes 
TBA to revisit. 
Include the problem statement, why you picked this question (e.g. what is interesting about it, 
what is hard about it, or why you wanted to take a second look at it), and your solution. 
    \begin{itemize}
        \item Question selection: you can pick any {\bf one question} listed in the Review 
        sections of the relevant notes documents, and you must address all of its parts.
        \item For each part of your chosen question: prepare a complete solution 
        (you can use the homework solutions we post for guidance about the style). 
        Your submission will be evaluated not only on the correctness of your answers, 
        but on your ability to present your ideas clearly and logically. 
        You should explain how you arrived at your conclusions, using mathematically 
        sound reasoning. Your goal should be to convince the reader that your results 
        and methods are sound. Imagine you are preparing these solutions for someone else 
        taking CSE 20 who missed that week and is ``catching up".
    \end{itemize}
\item In this part of the project, you'll TBA

\end{enumerate}

\subsubsection*{Video component}

Presenting your reasoning and demonstrating it via screenshare are important skills that 
also show us a lot of your learning. Getting practice with this style of presentation 
is a good thing for you to learn in general and a rich way for us to assess your skills. 

Prepare a 3-5 minute screencast video explaining your work in question 1 of the written component.
During your solution presentation, point out at least one potential mistake that someone 
solving a similar question should avoid (this could be a mistake you made while thinking 
about this problem or something you anticipate a classmate might struggle with); 
explain why the mistake is wrong and how to fix it. 

You do not need to include complete details of every part of your solution. 
It is up to you to choose what is most important so that you can stick to the 
timing guidelines and still have time to include discussing potential mistakes.

Include your face and your student ID (we'd like a photo ID that includes your name 
and picture if possible) for a few seconds at the beginning, and introduce yourself 
audibly while on screen. You don't have to be on camera the whole time, though it's fine 
if you are. We are looking for a brief confirmation that it's you creating the 
video/doing the work attached to the video.


Then, explain your work in question 1 of the written component.
Discuss at least one potential mistake that someone solving 
a similar question should avoid (this could be a mistake you made while thinking about this 
problem or something you anticipate a classmate might struggle with); explain why the 
mistake is wrong and how to fix it.
 
TBA

\subsubsection*{Checklist (this is how we will grade Part 2 of the project)}
\begin{itemize}
\item Question 1
    \begin{itemize}
        \item Selected review quiz question is labelled clearly, including the day 
        it belongs to and the statement of the question.
        \item Solution is complete: it addresses each part of the review quiz question selected.
        \item Solution is correct: it clearly and correctly justifies the correct answer 
        for each part of the question. You are welcome to check your answers with the 
        Gradescope autograder (we will be reopening the review quizzes for this purpose). 
        We will evaluate your submissions for the quality of your justification.
    \end{itemize}
\item Question 2
    \begin{itemize}
        \item A key lesson from each of the three references is stated clearly and 
        is relevant to the message of the articles. Supporting explanations are included.
        \item A specific example of an instance where using computers/ CS *caused* an error is described.
        \item A specific example of an instance where using computers/ CS helped *avoid* an error is described.
        \item Lesson(s) are drawn from the previous experiences.
        \item Specific strategies for increasing confidence in computation are described and justified.
    \end{itemize}
    \item Video
    \begin{itemize}
        \item Video loads correctly and is between 3 and 5 minutes. It includes your face and your student ID, 
        and you introduce yourself audibly while on screen.
        \item Video presents your solution for Question 1.
        \item A potential mistake is presented and discussed.
    \end{itemize}
\end{itemize}

\newpage
\subsection*{Part 3: due TBA}
\subsubsection*{Written component}
\begin{enumerate}
    \item In this part of the project, you will TBA
\end{enumerate}

\subsubsection*{Video component}
Presenting your reasoning and demonstrating it via screenshare are important skills that 
also show us a lot of your learning. Getting practice with this style of presentation 
is a good thing for you to learn in general and a rich way for us to assess your skills. 

Prepare a 3-5 minute screencast video explaining your work in question 1 parts (c) and (d)
of the written component (i.e. the negation and proof).
During your solution presentation, point out at least one potential mistake that someone 
solving a similar question should avoid (this could be a mistake you made while thinking 
about this problem or something you anticipate a classmate might struggle with); 
explain why the mistake is wrong and how to fix it. 

You do not need to include complete details of every part of your solution to these parts. 
It is up to you to choose what is most important so that you can stick to the 
timing guidelines and still have time to include discussing potential mistakes.

Include your face and your student ID (we'd like a photo ID that includes your name 
and picture if possible) for a few seconds at the beginning, and introduce yourself 
audibly while on screen. You don't have to be on camera the whole time, though it's fine 
if you are. We are looking for a brief confirmation that it's you creating the 
video/doing the work attached to the video.

Then, explain your work in question 1 of the written component.
Discuss at least one potential mistake that someone solving 
a similar question should avoid (this could be a mistake you made while thinking about this 
problem or something you anticipate a classmate might struggle with); explain why the 
mistake is wrong and how to fix it.
 
TBA


\subsubsection*{Checklist (this is how we will grade Part 3 of the project)}
\begin{itemize}
    \item Question 1 TBA
\item Video
    \begin{itemize}
        \item Video loads correctly and is between 3 and 5 minutes. It includes your face and your student ID, 
        and you introduce yourself audibly while on screen.
        \item Video presents your solution for Question 1 parts (c) and (d).
        \item A potential mistake is presented and discussed.
    \end{itemize}
\end{itemize}
\end{document}