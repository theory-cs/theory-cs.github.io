\documentclass[12pt, oneside]{article}

\usepackage[letterpaper, scale=0.89, centering]{geometry}
\usepackage{fancyhdr}
\setlength{\parindent}{0em}
\setlength{\parskip}{1em}

\usepackage{tikz}
\usetikzlibrary{automata,positioning,arrows}

\pagestyle{fancy}
\fancyhf{}
\renewcommand{\headrulewidth}{0pt}
\rfoot{\href{https://creativecommons.org/licenses/by-nc-sa/2.0/}{CC BY-NC-SA 2.0} Version \today~(\thepage)}

\usepackage{amssymb,amsmath,pifont,amsfonts,comment,enumerate,enumitem}
\usepackage{currfile,xstring,hyperref,tabularx,graphicx,wasysym}
\usepackage[labelformat=empty]{caption}
\usepackage{xcolor}
\usepackage{multicol,multirow,array,listings,tabularx,lastpage,textcomp,booktabs}

\lstnewenvironment{algorithm}[1][] {   
    \lstset{ mathescape=true,
        frame=tB,
        numbers=left, 
        numberstyle=\tiny,
        basicstyle=\rmfamily\scriptsize, 
        keywordstyle=\color{black}\bfseries,
        keywords={,procedure, div, for, to, input, output, return, datatype, function, in, if, else, foreach, while, begin, end, }
        numbers=left,
        xleftmargin=.04\textwidth,
        #1
    }
}
{}

\newcommand\abs[1]{\lvert~#1~\rvert}
\newcommand{\st}{\mid}

\newcommand{\cmark}{\ding{51}}
\newcommand{\xmark}{\ding{55}}
 
\begin{document}
\begin{flushright}
    \StrBefore{\currfilename}{.}
\end{flushright} \section*{Week8 monday}




\begin{center}
    \begin{tabular}{|lcl|}
    \hline
    \multicolumn{3}{|l|}{{\bf  Acceptance problem} } \\
    for Turing  machines  & $A_{TM}$ & $\{ \langle M,w \rangle \mid  \text{$M$ is a Turing machine that accepts input 
    string $w$}\}$ \\
    \hline
    \multicolumn{3}{|l|}{{\bf Language emptiness  testing} } \\
     for Turing machines & $E_{TM}$ & $\{ \langle M \rangle \mid  \text{$M$ is a Turing machine and  $L(M) = \emptyset$\}}$ \\
    \hline
    \multicolumn{3}{|l|}{{\bf Language equality testing} } \\
     for Turing machines& $EQ_{TM}$ & $\{ \langle  M_1, M_2 \rangle \mid  \text{$M_1$ and $M_2$ are Turing machines and  
     $L(M_1) =L(M_2)$\}}$\\
    \hline
    \end{tabular}
    \end{center}
    
    \begin{multicols}{3}
    $M_1$ \includegraphics[width=2in]{Week8WarmupTM1.png} 
    
    \columnbreak
    
    $M_2$ \includegraphics[width=2in]{Week8WarmupTM2.png}
    
    \columnbreak
    
    $M_3$ \includegraphics[width=2in]{Week8WarmupTM3.png}
    \end{multicols}
    
    Example strings in $A_{TM}$
    
    \vfill
    
    Example strings in  $E_{TM}$
    
    \vfill
    
    Example strings in  $EQ_{TM}$
    
    \vfill
    
    \newpage
    
    {\bf  Theorem}: $A_{TM}$  is  Turing-recognizable.
    
    
    {\bf  Strategy}:  To prove this theorem, we need  to  define  a Turing  machine  $R_{ATM}$ such that 
    $L(R_{ATM}) = A_{TM}$.
    
    
    Define $R_{ATM} =  $ ``
    
    \vspace{150pt}
    
    
    Proof of correctness: 
    
    
    \vfill
    \vfill
    
    We will show that $A_{TM}$ is undecidable.   {\it First, let's explore what that means.}
    
    








    


    


    
    To prove that a computational problem is {\bf decidable}, we find/ build a Turing 
    machine that recognizes the language encoding the computational problem, and that 
    is a decider.
    
    
    How do we prove a specific problem is {\bf not decidable}?
    
    How would we even find such a computational problem?
    
    
    {\it Counting arguments for the existence of an undecidable language:}
    \begin{itemize}
        \item The set of all Turing machines is countably infinite.
        \item Each recognizable language has at least one Turing machine that recognizes it (by definition), 
        so there can be no more Turing-recognizable
        languages than there are Turing machines. 
        \item Since there are infinitely many Turing-recognizable languages
        (think of the singleton sets), there are countably infinitely 
        many Turing-recognizable languages.
        \item Such the set of Turing-decidable languages is an infinite subset 
        of the set of Turing-recognizable languages, the set of 
        Turing-decidable languages is also countably infinite.
    \end{itemize}
    
    Since there are uncountably many languages (because $\mathcal{P}(\Sigma^*)$
    is uncountable), there are uncountably many unrecognizable languages
    and there are uncountably many undecidable languages.
    
    
    Thus, there's at least one undecidable language!
    
    \vfill
    
    {\bf What's a specific example of a language that is unrecognizable or undecidable?}
    
    To prove that a language is undecidable, we need to prove that there is no Turing machine that decides it.
    
    {\bf Key idea}: proof by contradiction relying on self-referential disagreement.
    
    

{\bf  Theorem}: $A_{TM}$  is  not  Turing-decidable.

{\bf  Proof}: Suppose {\bf towards a  contradiction}  that there  is a Turing machine  that decides $A_{TM}$.  
We call this presumed machine  $M_{ATM}$.

By  assumption, for every  Turing machine  $M$ and every  string $w$

\begin{itemize}
\item If $w \in L(M)$, then  the computation of $M_{ATM}$  on  $\langle M,w \rangle ~~ \underline{\phantom{\hspace{2.5in}}}$
\item If $w \notin L(M)$, then  the computation of $M_{ATM}$  on  $\langle M,w \rangle ~~ \underline{\phantom{\hspace{2.5in}}}$
\end{itemize}


Define  a {\bf new} Turing machine using  the high-level description:
\begin{quote}
$D =  $`` On  input $\langle M \rangle$, where  $M$  is  a Turing machine:
\begin{itemize}
\item[1.] Run  $M_{ATM}$ on  $\langle M, \langle M \rangle  \rangle$.
\item[2.] If $M_{ATM}$ accepts, reject; if  $M_{ATM}$ rejects, accept."
\end{itemize}
\end{quote}


Is $D$ a  Turing machine?

\vspace{50pt}

Is  $D$ a  decider? 

\vspace{50pt}

What is the result of the computation  of $D$  on  $\langle D \rangle$?

\vfill


\newpage
Definition: A language $L$ over an  alphabet $\Sigma$ is called {\bf co-recognizable} if its complement,  defined
as $\Sigma^* \setminus L  = \{ x  \in  \Sigma^* \mid x \notin  L \}$, is Turing-recognizable.


\vfill 
{\bf  Theorem} (Sipser Theorem 4.22): A  language is Turing-decidable if and only if both  it and its complement
are Turing-recognizable.

{\bf Proof, first direction:}  Suppose  language  $L$ is  Turing-decidable.   WTS  that both it and its complement 
are Turing-recognizable.

\vfill

{\bf Proof, second direction:}  Suppose  language  $L$ is  Turing-recognizable, and  so is  its complement.   WTS  that $L$
is Turing-decidable.
\vfill


Notation: The complement  of a set $X$ is denoted with  a superscript $c$, $X^c$, or an overline,  $\overline{X}$. \vfill
\end{document}