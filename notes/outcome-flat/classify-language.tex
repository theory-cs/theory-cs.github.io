\documentclass[12pt, oneside]{article}

\usepackage[letterpaper, scale=0.89, centering]{geometry}
\usepackage{fancyhdr}
\setlength{\parindent}{0em}
\setlength{\parskip}{1em}

\usepackage{tikz}
\usetikzlibrary{automata,positioning,arrows}

\pagestyle{fancy}
\fancyhf{}
\renewcommand{\headrulewidth}{0pt}
\rfoot{\href{https://creativecommons.org/licenses/by-nc-sa/2.0/}{CC BY-NC-SA 2.0} Version \today~(\thepage)}

\usepackage{amssymb,amsmath,pifont,amsfonts,comment,enumerate,enumitem}
\usepackage{currfile,xstring,hyperref,tabularx,graphicx,wasysym}
\usepackage[labelformat=empty]{caption}
\usepackage{xcolor}
\usepackage{multicol,multirow,array,listings,tabularx,lastpage,textcomp,booktabs}

\lstnewenvironment{algorithm}[1][] {   
    \lstset{ mathescape=true,
        frame=tB,
        numbers=left, 
        numberstyle=\tiny,
        basicstyle=\rmfamily\scriptsize, 
        keywordstyle=\color{black}\bfseries,
        keywords={,procedure, div, for, to, input, output, return, datatype, function, in, if, else, foreach, while, begin, end, }
        numbers=left,
        xleftmargin=.04\textwidth,
        #1
    }
}
{}

\newcommand\abs[1]{\lvert~#1~\rvert}
\newcommand{\st}{\mid}

\newcommand{\cmark}{\ding{51}}
\newcommand{\xmark}{\ding{55}}
 
\begin{document}
\begin{flushright}
    \StrBefore{\currfilename}{.}
\end{flushright} \section*{Week9 wednesday}


Recall:  $A$ is  {\bf  mapping  reducible to} $B$, written $A \leq_m B$,  means there is a computable function 
$f : \Sigma^* \to \Sigma^*$ such that {\it for all} strings  $x$ in $\Sigma^*$, 
\[
x  \in  A \qquad \qquad \text{if and  only  if} \qquad \qquad f(x) \in B.
\]

True or False: $\overline{A_{TM}} \leq_m \overline{HALT_{TM}}$

\vfill

True or False: $HALT_{TM} \leq_m A_{TM}$.

\vfill


{\bf Theorem} (Sipser 5.28): If $A \leq_m B$ and $B$ is recognizable, then $A$ is recognizable.

{\bf Proof}: 

\vfill

{\bf Corollary}: If  $A \leq_m B$ and $A$ is unrecognizable, then $B$ is unrecognizable.

\vfill

\newpage
{\it Strategy}:  

(i) To prove that a  recognizable language $R$ is undecidable, prove that $A_{TM} \leq_m R$.


(ii) To prove that a co-recognizable language $U$ is undecidable, prove that $\overline{A_{TM}} \leq_m U$,
 i.e. that $A_{TM} \leq_m \overline{U}$.

 

\[
E_{TM} = \{ \langle M \rangle \mid \text{$M$ is a Turing machine and $L(M) = \emptyset$} \}
\]

Example  string in  $E_{TM}$ is \underline{\phantom{\hspace{1.6in}}} .
Example  string not  in  $E_{TM}$ is \underline{\phantom{\hspace{1.6in}}} .

$E_{TM}$ is ~~decidable /  undecidable~~ and ~~recognizable /  unrecognizable~~.

$\overline{E_{TM}}$ is ~~decidable /  undecidable~~ and ~~recognizable /  unrecognizable~~.


{\bf Claim}: $\underline{\phantom{\hspace{1.6in}}}  \leq_m \overline{E_{TM}}$.

{\bf Proof}: Need computable function  $F: \Sigma^* \to \Sigma^*$  such that  $x \in A_{TM}$ iff $F(x)  \notin  E_{TM}$.
Define
\begin{quote}
$F =  ``$ On input $x$,
\begin{itemize}
\item[1.] Type-check whether  $x = \langle M, w \rangle$ for some TM $M$ and string $w$.   If so, move to step 2; if  not, output 
\item[2.] Construct the following machine $M'_x$:
\vspace{50pt}
\item[3.] Output $\langle M'_x \rangle$."
\end{itemize}
\end{quote}

Verifying correctness:
\begin{center}
\begin{tabular}{|c|c|}
\hline
Input string &  Output string \\
\hline
$\langle M, w \rangle$ where  $w \in L(M)$ & \phantom{\hspace{4in}} \\
& \\
$\langle M, w \rangle$ where $w \notin L(M)$ & \\
&\\
$x$ not encoding any pair of  TM and string   &  \\
\hline
\end{tabular}
\end{center}

\newpage \vfill
\section*{Week9 friday}


Recall:  $A$ is  {\bf  mapping  reducible to} $B$, written $A \leq_m B$,  means there is a computable function 
$f : \Sigma^* \to \Sigma^*$ such that {\it for all} strings  $x$ in $\Sigma^*$, 
\[
x  \in  A \qquad \qquad \text{if and  only  if} \qquad \qquad f(x) \in B.
\]
\[
EQ_{TM} = \{ \langle M, M' \rangle \mid \text{$M$ and $M'$ are both Turing machines and $L(M) =L(M')$} \}
\]

Example  string in  $EQ_{TM}$ is \underline{\phantom{\hspace{1.5in}}} .
Example  string not  in  $EQ_{TM}$ is \underline{\phantom{\hspace{1.5in}}} .

$EQ_{TM}$ is ~~decidable /  undecidable~~ and ~~recognizable /  unrecognizable~~.

$\overline{EQ_{TM}}$ is ~~decidable /  undecidable~~ and ~~recognizable /  unrecognizable~~.

To  prove, show that  $\underline{\phantom{\hspace{1.6in}}}  \leq_m EQ_{TM}$
and that $\underline{\phantom{\hspace{1.6in}}}  \leq_m \overline{EQ_{TM}}$.


\vfill

Verifying correctness:
\begin{center}
\begin{tabular}{|c|c|}
\hline
Input string &  Output string \\
\hline
$\langle M, w \rangle$ where  $M$ halts on $w$ & \phantom{\hspace{4in}} \\
& \\
$\langle M, w \rangle$ where $M$ loops on $w$ & \\
&\\
$x$ not encoding any pair of  TM and string   &  \\
\hline
\end{tabular}
\end{center}

\vfill

\newpage

In practice, computers (and Turing machines) don't have infinite tape, 
and we can't afford to wait unboundedly long for an answer.
``Decidable" isn't good enough - we want ``Efficiently decidable".

For a given algorithm working on a given input, how long do we need to wait for an answer? 
How does the running time depend on the input in the worst-case? average-case? 
We expect to have to spend more time on computations with larger inputs.


A language is {\bf recognizable} if \underline{\phantom{\hspace{4.5in}}}

A language is {\bf decidable} if \underline{\phantom{\hspace{4.7in}}}

A language is {\bf efficiently  decidable} if \underline{\phantom{\hspace{4in}}}

A function is {\bf computable} if \underline{\phantom{\hspace{4.7in}}}

A function is {\bf efficiently computable} if \underline{\phantom{\hspace{4in}}}\\

\vfill

Definition  (Sipser 7.1): For  $M$ a deterministic decider, its {\bf running time} is the function  $f: \mathbb{N} \to \mathbb{N}$
given  by
\[
f(n) =  \text{max number of  steps $M$ takes before halting, over all inputs  of length $n$}
\]

Definition (Sipser 7.7): For each function $t(n)$, the {\bf time complexity class}  $TIME(t(n))$, is defined  by
\[
TIME( t(n)) = \{ L \mid \text{$L$ is decidable by  a Turing machine with running time in  $O(t(n))$} \}
\]

An example of an element of  $TIME(  1  )$ is 

An example of an element of  $TIME(  n  )$ is 


Note: $TIME( 1) \subseteq TIME (n)  \subseteq TIME(n^2)$

\vfill

Definition (Sipser 7.12) : $P$ is the class of languages that  are decidable in polynomial time on 
a deterministic 1-tape  Turing  machine
\[
P  =  \bigcup_k TIME(n^k)
\]

{\it Compare to exponential time: brute-force search.}


Theorem (Sipser 7.8): Let $t(n)$ be a  function with  $t(n)  \geq n$.  Then every $t(n)$ time deterministic 
multitape Turing machine has an equivalent $O(t^2(n))$ time deterministic 1-tape Turing machine.

\newpage \vfill
\section*{Week8 monday}



{\bf  Theorem}: $A_{TM}$  is  not  Turing-decidable.

{\bf  Proof}: Suppose {\bf towards a  contradiction}  that there  is a Turing machine  that decides $A_{TM}$.  
We call this presumed machine  $M_{ATM}$.

By  assumption, for every  Turing machine  $M$ and every  string $w$

\begin{itemize}
\item If $w \in L(M)$, then  the computation of $M_{ATM}$  on  $\langle M,w \rangle ~~ \underline{\phantom{\hspace{2.5in}}}$
\item If $w \notin L(M)$, then  the computation of $M_{ATM}$  on  $\langle M,w \rangle ~~ \underline{\phantom{\hspace{2.5in}}}$
\end{itemize}


Define  a {\bf new} Turing machine using  the high-level description:
\begin{quote}
$D =  $`` On  input $\langle M \rangle$, where  $M$  is  a Turing machine:
\begin{itemize}
\item[1.] Run  $M_{ATM}$ on  $\langle M, \langle M \rangle  \rangle$.
\item[2.] If $M_{ATM}$ accepts, reject; if  $M_{ATM}$ rejects, accept."
\end{itemize}
\end{quote}


Is $D$ a  Turing machine?

\vspace{50pt}

Is  $D$ a  decider? 

\vspace{50pt}

What is the result of the computation  of $D$  on  $\langle D \rangle$?

\vfill

\newpage

{\bf  Theorem} (Sipser Theorem 4.22): A  language is Turing-decidable if and only if both  it and its complement
are Turing-recognizable.

{\bf Proof, first direction:}  Suppose  language  $L$ is  Turing-decidable.   WTS  that both it and its complement 
are Turing-recognizable.

\vfill

{\bf Proof, second direction:}  Suppose  language  $L$ is  Turing-recognizable, and  so is  its complement.   WTS  that $L$
is Turing-decidable.
\vfill


Give an example of a {\bf decidable} set: 

\vspace{20pt}

Give an example of a {\bf recognizable undecidable} set: 

\vspace{20pt}

Give an example of an {\bf unrecognizable} set: 

\vspace{20pt}


\newpage

{\bf True} or {\bf False}: The class of Turing-decidable languages is closed under complementation?

\vfill
\vfill
\vfill
Definition: A language $L$ over an  alphabet $\Sigma$ is called {\bf co-recognizable} if its complement,  defined
as $\Sigma^* \setminus L  = \{ x  \in  \Sigma^* \mid x \notin  L \}$, is Turing-recognizable.


Notation: The complement  of a set $X$ is denoted with  a superscript $c$, $X^c$, or an overline,  $\overline{X}$.
 \vfill
\section*{Week8 wednesday}


{\bf Mapping reduction}

Motivation: Proving that $A_{TM}$ is undecidable was hard. How can we leverage that work? 
Can we relate the decidability / undecidability of one problem to another?

\begin{quote}
If problem $X$ is {\bf no harder than} problem $Y$

\ldots and if $Y$ is easy,

\ldots then $X$ must be easy too.
\end{quote}


\begin{quote}
    If problem $X$ is {\bf no harder than} problem $Y$
    
    \ldots and if $X$ is hard,
    
    \ldots then $Y$ must be hard too.
\end{quote}

``Problem $X$ is no harder than problem $Y$'' means 
``Can answer questions about membership in $X$ by converting them to questions about membership in $Y$''.



Definition:  $A$ is  {\bf  mapping  reducible to} $B$  means there is a computable function 
$f : \Sigma^* \to \Sigma^*$ such that {\it for all} strings  $x$ in $\Sigma^*$, 
\[
x  \in  A \qquad \qquad \text{if and  only  if} \qquad \qquad f(x) \in B.
\]
Notation:  when $A$  is mapping reducible to $B$, we write $A  \leq_m B$.

{\it Intuition:} $A \leq_m B$ means $A$ is no harder than $B$, i.e. that the level 
of difficulty of $A$ is less than or equal the level of difficulty of $B$.

\vfill

\newpage
{\bf Computable functions}

Definition: A function $f: \Sigma^* \to \Sigma^*$ is a {\bf computable function} means there is some Turing machine such that, 
for each $x$, on input $x$ the Turing machine halts with exactly $f(x)$ followed by all blanks on the tape

{\it Examples of computable functions}:

The function that maps a string to a string which is one character longer and 
whose value, when interpreted as a fixed-width binary representation of a
nonnegative integer is twice the value of the input string (when interpreted as 
a fixed-width binary representation of a non-negative integer)
\[
f_1: \Sigma^* \to \Sigma^* \qquad  f_1(x)  = x0
\]

To prove $f_1$ is computable function, we define a Turing machine computing it.

{\it High-level description}
\begin{quote}
    ``On input $w$
    
    1. Append $0$ to $w$.
    
    2. Halt.''
\end{quote}

{\it Implementation-level description}
\begin{quote}
    ``On input $w$
    
    1. Sweep read-write head to the right until find first blank cell.
    
    2. Write 0.
    
    3. Halt.''
\end{quote}

{\it Formal definition} $(\{q0, qacc, qrej\}, \{0,1\}, \{0,1,\textvisiblespace\},\delta, q0, qacc, qrej)$
where $\delta$ is specified by the state diagram: 


\newpage


The function that maps a string to the result of repeating the string twice.
\[
f_2: \Sigma^* \to \Sigma^* \qquad f_2( x )  =  xx
\]

\vfill

The function that maps strings that are not the codes of Turing machines to the empty 
string and that maps strings that code Turing machines to the code of the 
related Turing machine that acts like the Turing machine coded by the input, except
that if this Turing machine coded by the input tries to reject, the 
new machine will go into a loop.
\[
f_3: \Sigma^* \to \Sigma^*  \qquad f_3( x )  =   \begin{cases}  \varepsilon \qquad&\text{if $x$ is not the code of  a TM} \\
\langle (Q \cup \{q_{trap} \}, \Sigma, \Gamma, \delta', q_0, q_{acc}, q_{rej} ) \rangle \qquad&\text{if $x = \langle (Q, \Sigma, \Gamma, \delta, q_0, q_{acc}, q_{rej} )\rangle$}\end{cases}
\]
where $q_{trap} \notin Q$ and 
\[\delta'( (q,x) ) = \begin{cases}
(r,y,d) &\text{if $q \in Q$, $x \in \Gamma$, $\delta ((q,x)) = (r,y,d)$, and  $r \neq  q_{rej}$} \\
(q_{trap}, \textvisiblespace, R) & \text{otherwise}
\end{cases}
\]
\vfill

\newpage


The function that maps strings that are not the codes of CFGs to the empty 
string and that maps strings that code CFGs to the code of a PDA that recognizes
the language generated by the CFG.


\vfill

{\it Other examples?}
 \vfill
\section*{Week8 friday}



Recall definition:  $A$ is  {\bf  mapping  reducible to} $B$  means there is a computable function 
$f : \Sigma^* \to \Sigma^*$ such that {\it for all} strings  $x$ in $\Sigma^*$, 
\[
x  \in  A \qquad \qquad \text{if and  only  if} \qquad \qquad f(x) \in B.
\]
Notation:  when $A$  is mapping reducible to $B$, we write $A  \leq_m B$.

{\it Intuition:} $A \leq_m B$ means $A$ is no harder than $B$, i.e. that the level 
of difficulty of $A$ is less than or equal the level of difficulty of $B$.


{\it Example}: $A_{TM} \leq_m A_{TM}$ 

\vfill

{\it Example}: $A_{DFA} \leq_m \{ ww \mid  w \in \{0,1\}^* \}$ 

\vfill



{\it Example}: $\{ 0^i  1^j \mid i  \geq 0, j \geq 0 \} \leq_m A_{TM}$ 

\vfill

{\bf Theorem} (Sipser 5.22): If $A \leq_m B$ and $B$ is decidable, then $A$ is decidable.
    

{\bf Theorem} (Sipser 5.23): If $A \leq_m B$ and $A$ is undecidable, then $B$ is undecidable.
    

\newpage

    {\bf Halting problem}
    \[
    HALT_{TM} = \{ \langle M, w \rangle \mid \text{$M$ is a  Turing machine, $w$ is  a string, and $M$ halts on $w$} \}
    \]
    
    Define $F: \Sigma^* \to \Sigma^*$ by
    \[
    F(x) =  \begin{cases}
    const_{out} \qquad &\text{if  $x \neq \langle M,w \rangle$ for any Turing machine  $M$ and string  $w$ over the alphabet of $M$} \\
    \langle M', w \rangle \qquad &  \text{if $x = \langle M, w \rangle$ for some Turing machine  $M$ and string $w$ over the alphabet of $M$.}
    \end{cases}
    \]
    where $const_{out}  =  \langle  \includegraphics[width=1.5in]{../../resources/machines/Lect22TM1.png} ,  \varepsilon  \rangle$
    and  $M'$ is a Turing machine that computes like $M$ except, if the computation ever were to go to a  reject state,
    $M'$ loops instead.
    
    \vfill

    $F( \langle \includegraphics[width=1.5in]{../../resources/machines/Lect22TM1.png} ,  001  \rangle)$ =

    \vfill

    $F( \langle \includegraphics[width=2.5in]{../../resources/machines/Lect22TM2.png} ,  1  \rangle)$ =

    \vfill
    
    \newpage
    To use this function  to prove that $A_{TM} \leq_m HALT_{TM}$, we need  two claims:

    
    Claim (1): $F$ is computable \phantom{\hspace{2in}}
    
    \vfill

    Claim (2): for every  $x$,  $x \in  A_{TM}$ iff $F(x) \in HALT_{TM}$.  
    
    \vfill
     \vfill
\section*{Week5 monday}


These definitions are on pages 101-102.

\vspace{-20pt}

\begin{center}
    \begin{tabular}{|p{2.4in}cp{3.6in}|}
    \hline 
    {\bf Term} & {\bf Typical symbol} & {\bf Meaning} \\
     & or {\bf Notation} & \\
    \hline
    \hline
    {\bf Context-free grammar} (CFG) & $G$ & $G = (V, \Sigma, R, S)$ \\
    The set of {\bf variables}& $V$ & Finite  set of symbols that represent phases in production pattern\\
    The set of {\bf terminals} & $\Sigma$ & Alphabet of symbols of strings generated  by CFG \\
    & & $V \cap \Sigma = \emptyset$ \\
    The set of {\bf rules}& $R$ & Each rule is  $A \to u$ with $A \in V$ and $u  \in (V  \cup \Sigma)^*$\\
    The {\bf start} variable&  $S$  & Usually  on left-hand-side of first/ topmost rule \\
    & &\\
    {\bf Derivation} & $S \Rightarrow \cdots \Rightarrow w$& 
    Sequence  of substitutions in a  CFG (also written $S \Rightarrow^* w$). At each step, we can apply one rule 
    to one occurrence of a variable in the current string by substituting that occurrence of the variable with the 
    right-hand-side of the rule. The derivation must end when the current string has only terminals (no variables)
    because then there are no instances of variables to apply a rule to.\\
    Language {\bf generated} by the context-free grammar $G$ & $L(G)$ &The set of strings for which there is a derivation in $G$. 
    Symbolically: $\{  w \in \Sigma^* \mid S \Rightarrow^* w \}$ i.e. $$\{  w \in \Sigma^* \mid \text{there is  derivation in $G$ that ends
    in $w$} \}$$\\
    {\bf Context-free language} & & A language that is the language generated by some context-free grammar\\
    \hline
    \end{tabular}
\end{center}

\vfill

  
{\bf Examples of context-free grammars, derivations in those grammars, and the languages generated by those grammars}
  
$G_1 =  (\{S\}, \{0\}, R, S)$ with rules
  \begin{align*}
    &S \to 0S\\
    &S \to 0\\
  \end{align*}
  In  $L(G_1)$ \ldots 
  
\vfill

  Not in $L(G_1)$ \ldots 


  \vfill

\newpage
  $G_2 =  (\{S\}, \{0,1\}, R, S)$
  \[
  S \to 0S \mid 1S \mid \varepsilon
  \]
  In  $L(G_2)$ \ldots 
  
  \vspace{110pt}
  
  Not in $L(G_2)$ \ldots 

  \vspace{110pt}

  $(\{S, T\}, \{0, 1\}, R, S)$ with  rules
  \begin{align*}
  &S \to T1T1T1T \\
  &T \to  0T \mid 1T \mid \varepsilon
  \end{align*}

  In  $L(G_3)$ \ldots 
  
  \vspace{110pt}
  
  Not in $L(G_3)$ \ldots 

  \vspace{110pt}

\newpage
  $G_4 =  (\{A, B\}, \{0, 1\}, R, A)$ with rules
  \[
    A \to 0A0 \mid  0A1 \mid 1A0  \mid 1A1 \mid  1
  \]
  In  $L(G_4)$ \ldots 
  
  \vspace{110pt}
  
  Not in $L(G_4)$ \ldots 

  \vspace{110pt}


  \begin{comment}
    I moved the following to the review quiz.

    {\it Extra practice}: Is there a CFG $G$ with $L(G) = \emptyset$?


  Three different CFGs that each generate the  language $\{abba\}$
  
  \begin{align*}
  & ( \{ S, T, V, W\}, \{a,b\}, \{ S \to aT, T \to bV, V \to bW, W \to a\}, S)\\
  & \\ 
  & \\ 
  & \\ 
  & ( \{ Q \}, \{a,b\}, \{Q \to abba\}, Q) \\
  & \\ 
  & \\ 
  & \\
  & ( \{ X,Y \}, \{a,b\}, \{X \to aYa, Y \to bb\}, X) 
  & \\ 
  & \\ 
  \end{align*} 
\end{comment}
  \newpage
  Design a CFG to generate the  language $\{a^n b^n \mid  n  \geq  0\}$
  
  \vspace{100pt}
  
  {\it Sample derivation:} 
  
  \vspace{100pt}
  
  
  \vfill 


\newpage
 \vfill
\section*{Week5 wednesday}


Warmup:   Design a CFG to generate the  language $\{a^i b^j \mid j \geq i  \geq 0\}$
  
\vfill
{\it Sample derivation:} 

\vspace{100pt}



Design a PDA to recognize the  language $\{a^i b^j \mid j \geq i  \geq 0\}$
  
\vspace{100pt}


\vfill
\newpage

{\bf Theorem  2.20}: A language is  generated by some context-free  grammar
if  and only if it is recognized by some push-down automaton.

Definition: a language is called {\bf context-free} if it is the language generated by a context-free grammar.
The class of all context-free language over a given alphabet $\Sigma$ is called {\bf CFL}.

Consequences:
\begin{itemize}
    \item Quick proof that every regular language is context free 
    \item To prove closure of the class of context-free languages under a given operation, we can choose 
    either of two modes 
    of proof (via CFGs or PDAs) depending on which is easier
    \item To fully specify a PDA we could give its $6$-tuple formal definition or we could give its input 
alphabet, stack alphabet, and state diagram.
An informal description of a PDA is a step-by-step description of how its computations 
would process input strings; the reader should be able to reconstruct the state diagram or formal 
definition precisely from such a descripton. The informal description of a PDA can refer to some 
common modules or subroutines that are computable by PDAs:
\begin{itemize}
  \item PDAs can ``test for emptiness of stack'' without providing details. 
  {\it How?} We can always push a special end-of-stack symbol, $\$$, at the start, before processing
  any input, and then use this symbol as a flag.
  \item PDAs can ``test for end of input'' without providing details.
  {\it How?} We can transform a PDA to one where accepting states are only those reachable 
  when there are no more input symbols.
\end{itemize}

\end{itemize}




\vfill

\newpage
Suppose $L_1$ and $L_2$ are context-free languages over $\Sigma$.  {\bf Goal}:  $L_1 \cup L_2$  is  also context-free.

{\it Approach 1: with  PDAs}

Let $M_1 = ( Q_1, \Sigma, \Gamma_1, \delta_1, q_1, F_1)$ and
$M_2 = ( Q_2, \Sigma, \Gamma_2, \delta_2, q_2, F_2)$ be PDAs with 
$L(M_1) =  L_1$  and  $L(M_2) = L_2$.

Define $M = $

\vfill

{\it Approach  2: with CFGs}

Let $G_1 = (V_1, \Sigma, R_1, S_1)$  and   $G_2 = (V_2, \Sigma, R_2, S_2)$  be CFGs  with
$L(G_1) =  L_1$  and  $L(G_2) = L_2$.

Define $G = $

\vfill

\newpage
Suppose $L_1$ and $L_2$ are context-free languages over $\Sigma$.  {\bf Goal}:  $L_1 \circ L_2$  is  also context-free.


{\it Approach 1: with  PDAs}

Let $M_1 = ( Q_1, \Sigma, \Gamma_1, \delta_1, q_1, F_1)$ and
$M_2 = ( Q_2, \Sigma, \Gamma_2, \delta_2, q_2, F_2)$ be PDAs with 
$L(M_1) =  L_1$  and  $L(M_2) = L_2$.

Define $M = $

\vfill

{\it Approach  2: with CFGs}

Let $G_1 = (V_1, \Sigma, R_1, S_1)$  and   $G_2 = (V_2, \Sigma, R_2, S_2)$  be CFGs  with
$L(G_1) =  L_1$  and  $L(G_2) = L_2$.

Define $G = $

\vfill
\newpage
{\it Summary}

Over a fixed alphabet $\Sigma$, a language $L$ is {\bf regular}

\vspace{-20pt}
\begin{center}
    iff it is described by some regular expression \\
    iff it is recognized by some DFA\\
    iff it is recognized by some NFA
\end{center}

Over a fixed alphabet $\Sigma$, a language $L$ is {\bf context-free}

\vspace{-20pt}
\begin{center}
    iff it is generated by some CFG\\
    iff it is recognized by some PDA
\end{center}

{\bf Fact}: Every regular language is a context-free language.

{\bf Fact}: There are context-free languages that are not nonregular.

{\bf Fact}: There are countably many regular languages.

{\bf Fact}: There are countably inifnitely many context-free languages.

{\it Consequence}: Most languages are {\bf not} context-free!

{\bf Examples  of non-context-free languages}

\begin{align*}
    &\{ a^n b^n c^n \mid 0 \leq n , n \in \mathbb{Z}\}\\
    &\{ a^i b^j c^k \mid 0 \leq i \leq j \leq k , i \in \mathbb{Z}, j \in \mathbb{Z}, k \in \mathbb{Z}\}\\
    &\{ ww \mid w \in \{0,1\}^* \}
\end{align*}
(Sipser Ex 2.36, Ex 2.37, 2.38)

There is a Pumping Lemma for CFL that can be used to prove a specific language is non-context-free: 
If $A$ is a context-free language, there there
is a number $p$ where, if $s$ is any string in $A$ of length at least $p$, then $s$ may be divided 
into five pieces $s = uvxyz$ where (1) for each $i \geq 0$, $uv^ixy^iz \in A$, (2) $|uv|>0$, (3) $|vxy| \leq p$.
{\it We will not go into the details of the proof or application of Pumping Lemma for CFLs this quarter.}

\begin{comment}
##Moved to review quiz
A set $X$ is said to be {\bf closed} under an operation $OP$ if, for any elements in $X$, applying 
$OP$ to them gives an element in $X$.  


\begin{center}
\begin{tabular}{|c|l|}
\hline
True/False & Closure claim \\
\hline
True &  The set of integers is closed under multiplication. \\
& $\forall x \forall y \left( ~(x \in \mathbb{Z} \wedge y \in \mathbb{Z})\to xy \in \mathbb{Z}~\right)$ \\
\hline
True & For each set $A$, the power set of $A$ is closed under intersection. \\
& $\forall A_1 \forall A_2 \left( ~(A_1 \in \mathcal{P}(A) \wedge A_2 \in \mathcal{P}(A) \in \mathbb{Z}) \to A_1 \cap A_2 \in \mathcal{P}(A)~\right)$ \\
\hline
  & The class of regular languages over $\Sigma$ is closed under complementation. \\
  & \\
 \hline
  & The class of regular languages over $\Sigma$ is closed under union. \\
  & \\
 \hline
  & The class of regular languages over $\Sigma$ is closed under intersection. \\
  & \\
  \hline
  & The class of regular languages over $\Sigma$ is closed under concatenation. \\
  & \\
 \hline
  & The class of regular languages over $\Sigma$ is closed under Kleene star. \\
  & \\
\hline
    & The class of context-free languages over $\Sigma$ is closed under complementation. \\
  & \\
\hline
    & The class of context-free languages over $\Sigma$ is closed under union. \\
  & \\
\hline
    & The class of context-free languages over $\Sigma$ is closed under intersection. \\
  & \\
\hline
    & The class of context-free languages over $\Sigma$ is closed under concatenation. \\
  & \\
\hline
    & The class of context-free languages over $\Sigma$ is closed under Kleene star. \\
  & \\
\hline
\end{tabular}
\end{center}
\end{comment}

 \vfill
\section*{Week5 friday}


 \vfill
\section*{Week4 monday}


Recap so far: In DFA, the only memory available is in the states. Automata can only
``remember'' finitely far in the past and finitely much information, because
they can have only finitely many states. If a computation path of a DFA visits 
the same state more than once, the machine can't tell the difference between 
the first time and future times it visits this state. Thus, if 
a DFA accepts one long string, then it must accept (infinitely) many 
similar strings.

{\bf Definition}  A positive integer $p$ is a {\bf pumping length} of a language $L$ over $\Sigma$ means
that, for each string $s  \in  \Sigma^*$, if  $|s| \geq p$ and $s \in L$, then there are strings $x,y,z$
such that 
\[
s = xyz
\]
and  
\[
|y| > 0,  \qquad \qquad 
\text{ for each $i \geq 0$, $xy^i z \in L$}, \qquad \text{and}
\qquad  \qquad
|xy| \leq p.
\]

{\bf Negation}: A positive integer  $p$  is {\bf not a pumping length} of a language  $L$ over  $\Sigma$  iff
\[
\exists s \left(~  |s| \geq  p \wedge s \in L \wedge \forall x \forall y \forall z  \left( ~\left( s = xyz \wedge 
|y| > 0 \wedge |xy| \leq p~ \right) \to \exists i  (  i \geq 0  \wedge xy^iz  \notin L ) \right) ~\right) 
\]
{\it Informally: }


Restating {\bf Pumping Lemma}: If $L$ is a regular language, then it  has
a pumping length.


{\bf Contrapositive}: If $L$ has no pumping length, then  it is nonregular.

\vfill

{\Large The Pumping Lemma {\it cannot} be used to prove that a language {\it is} regular.} 

{\Large The Pumping Lemma {\bf can} be used to prove that a language {\it is not} regular.}

{\it Extra practice}: Exercise 1.49 in the book.


\vfill

{\bf Proof strategy}: To prove that a language $L$ is {\bf not} regular, 
\begin{itemize}
    \item Consider an arbitrary positive integer $p$
    \item Prove that $p$ is not a pumping length for $L$
    \item Conclude that $L$ does not have {\it any} pumping length, and therefore it is not regular.
\end{itemize}

\newpage
{\bf Example}: $\Sigma  =  \{0,1\}$, $L = \{ 0^n 1^n \mid n  \geq 0\}$.

Fix $p$ an arbitrary positive integer. List strings that are in $L$ and have length  greater than or equal  to $p$:

\vspace{20pt}

Pick $s = $


Suppose $s = xyz$ with  $|xy|  \leq  p$ and $|y| > 0$.
\begin{center}
\begin{tabular}{|c|}
\hline
 \\
\hspace{4in} \\
\hline
\end{tabular}
\end{center}

Then when $i = \hspace{1in}$, $xy^i z  = \hspace{1in}$

\newpage

{\bf Example}: $\Sigma  =  \{0,1\}$, $L = \{w w^{\mathcal{R}} \mid w \in \{0,1\}^*\}$.
Remember that the reverse of a string $w$ is denoted $w^\mathcal{R}$  
and means to write $w$  in  the opposite order, if $w = w_1 \cdots  w_n$ then $w^\mathcal{R} = w_n \cdots  w_1$. Note: $\varepsilon^\mathcal{R} = \varepsilon$.


Fix $p$ an arbitrary positive integer. List strings that are in $L$ and have length  greater than or equal  to $p$:

\vspace{10pt}

Pick $s = $

Suppose $s = xyz$ with  $|xy|  \leq  p$ and $|y| > 0$.
\begin{center}
\begin{tabular}{|c|}
\hline
 \\
\hspace{4in} \\
\hline
\end{tabular}
\end{center}
Then when $i = \hspace{1in}$, $xy^i z  = \hspace{1in}$


\vspace{30pt} 

{\bf Example}: $\Sigma  =  \{0,1\}$, $L = \{0^j1^k  \mid j \geq k  \geq 0\}$.

Fix $p$ an arbitrary positive integer. List strings that are in $L$ and have length  greater than or equal  to $p$:

\vspace{10pt}

Pick $s = $


Suppose $s = xyz$ with  $|xy|  \leq  p$ and $|y| > 0$.
\begin{center}
\begin{tabular}{|c|}
\hline
 \\
\hspace{4in} \\
\hline
\end{tabular}
\end{center}
Then when $i = \hspace{1in}$, $xy^i z  = \hspace{1in}$



\vspace{30pt} 

{\bf Example}: $\Sigma  =  \{0,1\}$, $L = \{0^n1^m0^n  \mid m,n  \geq 0\}$.

Fix $p$ an arbitrary positive integer. List strings that are in $L$ and have length  greater than or equal  to $p$:

\vspace{10pt}

Pick $s = $


Suppose $s = xyz$ with  $|xy|  \leq  p$ and $|y| > 0$.
\begin{center}
\begin{tabular}{|c|}
\hline
 \\
\hspace{4in} \\
\hline
\end{tabular}
\end{center}
Then when $i = \hspace{1in}$, $xy^i z  = \hspace{1in}$

\newpage
{\it Extra practice}:


\begin{center}
    \begin{tabular}{c|c| c| c}
    Language & $s \in L$ & $s \notin L$ & Is the language regular or nonregular?  \\
    \hline
     & \hspace{1in} & \hspace{1in}  &  \\
    $\{a^nb^n \mid 0  \leq n  \leq 5 \}$ & & & \\
     & & & \\
    $\{b^n a^n \mid  n  \geq 2\}$  & & & \\
     & & & \\
    $\{a^m b^n \mid  0 \leq m\leq n\}$  & & & \\
     & & & \\
    $\{a^m b^n \mid  m \geq n+3,  n \geq 0\}$  & & & \\
     & & & \\
    $\{b^m a^n \mid  m \geq 1, n \geq  3\}$  & & & \\
     & & & \\
    $\{ w  \in \{a,b\}^* \mid w = w^\mathcal{R} \}$ & & & \\
     & & & \\ 
    $\{ ww^\mathcal{R} \mid w\in \{a,b\}^* \}$ & & & \\
     & & & \\ 
    \end{tabular}
\end{center}
     \vfill
\section*{Week4 wednesday}


Regular sets are not the end of the story
\begin{itemize}
    \item Many nice / simple / important sets are not regular
    \item Limitation of the finite-state automaton model: Can't ``count", Can only remember finitely far into the past,
    Can't backtrack, Must make decisions in ``real-time"
    \item We know actual computers are more powerful than this model...
\end{itemize}

The {\bf next} model of computation. Idea: allow some memory of unbounded size. How? 
\begin{itemize}
    \item To generalize regular expressions: {\bf context-free grammars}\\
    \item To generalize NFA: {\bf Pushdown automata}, which is like an NFA with access to a stack: 
    Number of states is fixed, number of entries in stack is unbounded. At each step
    (1) Transition to new state based on current state, letter read, and top letter of stack, then
    (2) (Possibly) push or pop a letter to (or from) top of stack. Accept a string iff
    there is some sequence of states and some sequence of stack contents 
    which helps the PDA processes the entire input string and ends in an accepting state.
\end{itemize}

\vfill

\vfill

Is there a PDA that recognizes the nonregular language $\{0^n1^n \mid n \geq 0 \}$?

\vfill

\newpage


\includegraphics[width=4in]{../../resources/machines/Lect9PDA.png}

The PDA with state diagram above can be informally described as:
\begin{quote}
    Read symbols from the input. As each 0 is read, push it onto the stack. 
    As soon as 1s are seen, pop a 0 off the stack for each 1 read. 
    If the stack becomes empty and we are at the end of the input string, accept the input. 
    If the stack becomes empty and there are 1s left to read, 
    or if 1s are finished while the stack still contains 0s, or if any 0s
    appear in the string following 1s, 
    reject the input.
\end{quote}
    

Trace the computation of this PDA on the input string $01$.

\vfill
    
Trace the computation of this PDA on the input string $011$.

\vfill

\newpage
A PDA recognizing the set $\{ \hspace{1.5 in} \}$ can be informally described as:
\begin{quote}
    Read symbols from the input. As each 0 is read, push it onto the stack. 
    As soon as 1s are seen, pop a 0 off the stack for each 1 read. 
    If the stack becomes empty and there is exactly one 1 left to read, read that 1 and accept the input. 
    If the stack becomes empty and there are either zero or more than one 1s left to read, 
    or if the 1s are finished while the stack still contains 0s, or if any 0s appear in the input following 1s, 
    reject the input.
\end{quote}
Modify the state diagram below to get a PDA that implements this description:

\includegraphics[width=4in]{../../resources/machines/Lect9PDA.png}


\vfill
{\bf Definition} A {\bf pushdown automaton} (PDA) is  specified by a  $6$-tuple $(Q, \Sigma, \Gamma, \delta, q_0, F)$
where $Q$ is the finite set of states, $\Sigma$ is the input alphabet,  $\Gamma$ is the stack alphabet,
\[
    \delta: Q \times \Sigma_\varepsilon  \times  \Gamma_\varepsilon \to \mathcal{P}( Q \times \Gamma_\varepsilon)
\]
is the transition function,  $q_0 \in Q$ is the start state, $F \subseteq  Q$ is the set of accept states.
    
 \vfill
\section*{Week4 friday}




Draw the state diagram and give the formal definition of a PDA with $\Sigma = \Gamma$.

\vfill

Draw the state diagram and give the formal definition of a PDA with $\Sigma \cap \Gamma = \emptyset$.
    
\vfill

\newpage
For the PDA state diagrams below, $\Sigma = \{0,1\}$.


\begin{center}
\begin{tabular}{c c}
Mathematical description of language & State diagram of PDA recognizing language\\
\hline
& $\Gamma = \{ \$, \#\}$ \hspace{2.3in} \\
& \\
& \includegraphics[width=3.5in]{../../resources/machines/Lect10PDA1.png}\\
& \\
& \\
\hline
& $\Gamma = \{ {@}, 1\}$ \hspace{2.3in} \\
& \\
& \includegraphics[width=3.5in]{../../resources/machines/Lect10PDA2.png}\\
& \\
& \\
\hline
& \\
& \\
& \\
$\{ 0^i 1^j 0^k \mid i,j,k \geq 0 \}$ & \\
& \\
& \\
\end{tabular}
\end{center}

 \vfill
 {\it Note: alternate notation is to replace $;$ with $\to$}

\begin{comment}
{\it Extra practice}: Consider the state diagram of a PDA with input alphabet 
$\Sigma$ and stack alphabet $\Gamma$.

\begin{center}
\begin{tabular}{|c|c|}
\hline
Label & means \\
\hline
$a, b ; c$ when $a \in \Sigma$, $b\in \Gamma$, $c \in \Gamma$ 
& \hspace{3in} \\
& \\
& \\
& \\
& \\
&\\
\hline
$a, \varepsilon ; c$ when $a \in \Sigma$, $c \in \Gamma$ 
& \hspace{3in} \\
& \\
& \\
& \\
& \\
&\\
\hline
$a, b ; \varepsilon$ when $a \in \Sigma$, $b\in \Gamma$
& \hspace{3in} \\
& \\
& \\
& \\
& \\
&\\
\hline
$a, \varepsilon ; \varepsilon$ when $a \in \Sigma$
& \hspace{3in} \\
& \\
& \\
& \\
& \\
&\\
\hline
\end{tabular}
\end{center}


How does the meaning change if $a$ is replaced by $\varepsilon$?
\end{comment}


{\it Big picture}: PDAs were motivated by wanting to add some memory of unbounded size to NFA. How 
do we accomplish a similar enhancement of regular expressions to get a syntactic model that is 
more expressive?

DFA, NFA, PDA: Machines process one input string at a time; the computation of a machine on its input string 
reads the input from left to right.

Regular expressions: Syntactic descriptions of all strings that match a particular pattern; the language 
described by a regular expression is built up recursively according to the expression's syntax

{\bf Context-free grammars}: Rules to produce one string at a time, adding characters from the middle, beginning, 
or end of the final string as the derivation proceeds.

 \vfill
\section*{Week6 monday}


We are ready to introduce a formal model that will capture a notion of general purpose computation.
\begin{itemize}
\item {\it Similar to DFA, NFA, PDA}: input will be an arbitrary string over a fixed alphabet.
\item {\it Different from NFA, PDA}: machine is deterministic.
\item {\it Different from DFA, NFA, PDA}: read-write head can move both to the left and to the right,
and can extend to the right past the original input.
\item {\it Similar to DFA, NFA, PDA}: transition function drives computation one step at a time 
by moving within a finite set of states, always starting at designated start state.
\item {\it Different from DFA, NFA, PDA}: the special states for rejecting and accepting take effect immediately.
\end{itemize}

\vspace{-10pt}

(See more details: Sipser p. 166)

\vfill

Formally: a  Turing machine is $M= (Q, \Sigma, \Gamma, \delta, q_0, q_{accept}, q_{reject})$ 
where $\delta$ is the {\bf transition function} 
\[
  \delta: Q\times \Gamma \to Q \times \Gamma \times \{L, R\}
\]
The {\bf computation} of $M$ on a string $w$ over $\Sigma$  is:

\vspace{-10pt}

\begin{itemize}
\setlength{\itemsep}{0pt}
\item Read/write head starts at leftmost position on tape. 
\item Input string is written on $|w|$-many leftmost cells of tape, 
rest of  the tape cells have  the blank symbol. {\bf Tape alphabet} 
is $\Gamma$ with $\textvisiblespace\in \Gamma$ and $\Sigma \subseteq \Gamma$.
The blank symbol $\textvisiblespace \notin \Sigma$.
\item Given current state of machine and current symbol being read at the tape head, 
the machine transitions to next state, writes a symbol to the current position  of the 
tape  head (overwriting existing symbol), and moves the tape head L or R (if possible). 
\item Computation ends {\bf if and when} machine enters either the accept or the reject state.
This is called {\bf halting}.
Note: $q_{accept} \neq q_{reject}$.
\end{itemize}

The {\bf language recognized by the  Turing machine} $M$,  is  $L(M) = \{ w \in \Sigma^* \mid w \textrm{ is accepted by } M\}$,
which is defined as
\[
  \{ w \in \Sigma^* \mid \textrm{computation of $M$ on $w$ halts after entering the accept state}\}
\]




\newpage
\begin{multicols}{2}
\includegraphics[width=2.5in]{../../resources/machines/Lect13TM1.png}

\columnbreak
Formal definition:

\vspace{10pt}

Sample computation: 

\begin{tabular}{|c|c|c|c|c|c|c|}
\hline
\multicolumn{1}{|c}{$q0\downarrow$} &  \multicolumn{6}{c|}{\phantom{A}}\\
\hline
$0$ & $0$  & $0$ & $\textvisiblespace $& $\textvisiblespace $& $\textvisiblespace $&  $\textvisiblespace $\\
\hline
\multicolumn{7}{|c|}{\phantom{A}}\\
\hline
\phantom{AA} & \phantom{AA}& \phantom{AA}& \phantom{AA}& \phantom{AA}& \phantom{AA}& \phantom{AA} \\
\hline
\multicolumn{7}{|c|}{\phantom{A}}\\
\hline
\phantom{AA} & \phantom{AA}& \phantom{AA}& \phantom{AA}& \phantom{AA}& \phantom{AA}& \phantom{AA} \\
\hline
\multicolumn{7}{|c|}{\phantom{A}}\\
\hline
\phantom{AA} & \phantom{AA}& \phantom{AA}& \phantom{AA}& \phantom{AA}& \phantom{AA}& \phantom{AA} \\
\hline
\multicolumn{7}{|c|}{\phantom{A}}\\
\hline
\phantom{AA} & \phantom{AA}& \phantom{AA}& \phantom{AA}& \phantom{AA}& \phantom{AA}& \phantom{AA} \\
\hline
\end{tabular}
\end{multicols}
\vfill

The language recognized by this machine is \ldots

\vfill
 

{\bf Describing  Turing machines} (Sipser p. 185) To define a Turing machine, we could give a 
\begin{itemize}
\item {\bf Formal definition}: the $7$-tuple of parameters including set of states, 
input alphabet, tape alphabet, transition function, start state, accept state, and reject state; or,
\item {\bf Implementation-level definition}: English prose that describes the Turing machine head 
movements relative to contents of tape, and conditions for accepting / rejecting based on those contents.
\item {\bf High-level description}: description of algorithm (precise sequence of instructions), 
without implementation details of machine. As part of this description, can ``call" and run 
another TM as a subroutine.
\end{itemize}
  
\newpage
Fix $\Sigma = \{0,1\}$, $\Gamma = \{ 0, 1, \textvisiblespace\}$ for the Turing machines with  the following state diagrams:
  
\begin{center}
  \includegraphics[width=2in]{../../resources/machines/Lect14TM1.png}
\end{center}

Example of string accepted: \\
Example of string rejected: \\


Implementation-level description

\vfill

High-level description

\vfill

\begin{center}
  \includegraphics[width=2in]{../../resources/machines/Lect14TM2.png}
\end{center}

Example of string accepted: \\
Example of string rejected: \\


Implementation-level description

\vfill

High-level description

\vfill

\newpage
\begin{center}
  \includegraphics[width=2in]{../../resources/machines/Lect14TM3.png}
\end{center}

Example of string accepted: \\
Example of string rejected: \\


Implementation-level description

\vfill

High-level description

\vfill

\begin{center}
  \includegraphics[width=2in]{../../resources/machines/Lect14TM4.png}
\end{center}

Example of string accepted: \\
Example of string rejected: \\


Implementation-level description

\vfill

High-level description

\vfill

\newpage
 \vfill
\section*{Week6 wednesday}



{\it Sipser Figure  3.10}

{\bf Conventions in state diagram of TM}: $b \to R$ label means $b \to b, R$ and
all arrows missing from diagram represent transitions with output $(q_{reject}, \textvisiblespace , R)$

\begin{multicols}{2}
\vspace{-20pt}
\begin{center}
\includegraphics[width=4in]{../../resources/machines/Lect13TM3.png}
\end{center}

Implementation level description of this machine:
\begin{quote}
Zig-zag across tape to corresponding positions on either side of $\#$ to check whether the 
characters in these positions agree. If they do not, or if there is no $\#$, reject. If they 
do, cross them off.

Once all symbols to the left of the $\#$ are crossed off, check for any un-crossed-off symbols 
to the right of $\#$; if there are any, reject; if there aren't, accept.
\end{quote}

The language recognized by this machine is
\[
  \{ w \# w \mid w \in \{0,1\}^* \}
\]

\columnbreak

Computation on  input  string  $01\#01$

\begin{tabular}{|c|c|c|c|c|c|c|}
\hline
\multicolumn{1}{|c}{$q_1 \downarrow$} &  \multicolumn{6}{c|}{\phantom{A}}\\
\hline
$0$ & $1$  & $\#$  & $0$ & $1$ & $\textvisiblespace $& $\textvisiblespace $\\
\hline
  \multicolumn{7}{|c|}{\phantom{A}}\\
  \hline
  \phantom{AA} & \phantom{AA}& \phantom{AA}& \phantom{AA}& \phantom{AA}& \phantom{AA}& \phantom{AA} \\
  \hline
  \multicolumn{7}{|c|}{\phantom{A}}\\
  \hline
  \phantom{AA} & \phantom{AA}& \phantom{AA}& \phantom{AA}& \phantom{AA}& \phantom{AA}& \phantom{AA} \\
  \hline
  \multicolumn{7}{|c|}{\phantom{A}}\\
  \hline
  \phantom{AA} & \phantom{AA}& \phantom{AA}& \phantom{AA}& \phantom{AA}& \phantom{AA}& \phantom{AA} \\
  \hline
  \multicolumn{7}{|c|}{\phantom{A}}\\
  \hline
  \phantom{AA} & \phantom{AA}& \phantom{AA}& \phantom{AA}& \phantom{AA}& \phantom{AA}& \phantom{AA} \\
  \hline
  \multicolumn{7}{|c|}{\phantom{A}}\\
  \hline
  \phantom{AA} & \phantom{AA}& \phantom{AA}& \phantom{AA}& \phantom{AA}& \phantom{AA}& \phantom{AA} \\
  \hline
  \multicolumn{7}{|c|}{\phantom{A}}\\
  \hline
  \phantom{AA} & \phantom{AA}& \phantom{AA}& \phantom{AA}& \phantom{AA}& \phantom{AA}& \phantom{AA} \\
  \hline
  \multicolumn{7}{|c|}{\phantom{A}}\\
  \hline
  \phantom{AA} & \phantom{AA}& \phantom{AA}& \phantom{AA}& \phantom{AA}& \phantom{AA}& \phantom{AA} \\
  \hline
  \multicolumn{7}{|c|}{\phantom{A}}\\
  \hline
  \phantom{AA} & \phantom{AA}& \phantom{AA}& \phantom{AA}& \phantom{AA}& \phantom{AA}& \phantom{AA} \\
  \hline
  \multicolumn{7}{|c|}{\phantom{A}}\\
  \hline
  \phantom{AA} & \phantom{AA}& \phantom{AA}& \phantom{AA}& \phantom{AA}& \phantom{AA}& \phantom{AA} \\
  \hline
  \multicolumn{7}{|c|}{\phantom{A}}\\
  \hline
  \phantom{AA} & \phantom{AA}& \phantom{AA}& \phantom{AA}& \phantom{AA}& \phantom{AA}& \phantom{AA} \\
  \hline
  \multicolumn{7}{|c|}{\phantom{A}}\\
  \hline
  \phantom{AA} & \phantom{AA}& \phantom{AA}& \phantom{AA}& \phantom{AA}& \phantom{AA}& \phantom{AA} \\
  \hline
  \multicolumn{7}{|c|}{\phantom{A}}\\
  \hline
  \phantom{AA} & \phantom{AA}& \phantom{AA}& \phantom{AA}& \phantom{AA}& \phantom{AA}& \phantom{AA} \\
  \hline
  \multicolumn{7}{|c|}{\phantom{A}}\\
  \hline
  \phantom{AA} & \phantom{AA}& \phantom{AA}& \phantom{AA}& \phantom{AA}& \phantom{AA}& \phantom{AA} \\
  \hline
  \multicolumn{7}{|c|}{\phantom{A}}\\
  \hline
  \phantom{AA} & \phantom{AA}& \phantom{AA}& \phantom{AA}& \phantom{AA}& \phantom{AA}& \phantom{AA} \\
  \hline
  \multicolumn{7}{|c|}{\phantom{A}}\\
  \hline
  \phantom{AA} & \phantom{AA}& \phantom{AA}& \phantom{AA}& \phantom{AA}& \phantom{AA}& \phantom{AA} \\
  \hline
  \multicolumn{7}{|c|}{\phantom{A}}\\
  \hline
  \phantom{AA} & \phantom{AA}& \phantom{AA}& \phantom{AA}& \phantom{AA}& \phantom{AA}& \phantom{AA} \\
  \hline
  \multicolumn{7}{|c|}{\phantom{A}}\\
  \hline
  \phantom{AA} & \phantom{AA}& \phantom{AA}& \phantom{AA}& \phantom{AA}& \phantom{AA}& \phantom{AA} \\
  \hline
  \multicolumn{7}{|c|}{\phantom{A}}\\
  \hline
  \phantom{AA} & \phantom{AA}& \phantom{AA}& \phantom{AA}& \phantom{AA}& \phantom{AA}& \phantom{AA} \\
  \hline
  \end{tabular}
\end{multicols}

\begin{multicols}{2}
High-level description of this machine is

\vfill


{\it Recall:} 
High-level descriptions of  Turing machine algorithms are written as indented text within quotation marks.   
Stages of the algorithm are typically numbered consecutively.
The first line specifies the input to the machine, which must be a string.

\columnbreak

{\it Extra practice}

Computation on  input  string  $01\#1$

\begin{tabular}{|c|c|c|c|c|c|c|}
\hline
\multicolumn{1}{|c}{$q_1\downarrow$} &  \multicolumn{6}{c|}{\phantom{A}}\\
\hline
$0$ & $1$  & $\#$  & $1$ & $\textvisiblespace $& $\textvisiblespace $&  $\textvisiblespace $\\
\hline
\multicolumn{7}{|c|}{\phantom{A}}\\
\hline
\phantom{AA} & \phantom{AA}& \phantom{AA}& \phantom{AA}& \phantom{AA}& \phantom{AA}& \phantom{AA} \\
\hline
\multicolumn{7}{|c|}{\phantom{A}}\\
\hline
\phantom{AA} & \phantom{AA}& \phantom{AA}& \phantom{AA}& \phantom{AA}& \phantom{AA}& \phantom{AA} \\
\hline
\multicolumn{7}{|c|}{\phantom{A}}\\
\hline
\phantom{AA} & \phantom{AA}& \phantom{AA}& \phantom{AA}& \phantom{AA}& \phantom{AA}& \phantom{AA} \\
\hline
\multicolumn{7}{|c|}{\phantom{A}}\\
\hline
\phantom{AA} & \phantom{AA}& \phantom{AA}& \phantom{AA}& \phantom{AA}& \phantom{AA}& \phantom{AA} \\
\hline
\multicolumn{7}{|c|}{\phantom{A}}\\
\hline
\phantom{AA} & \phantom{AA}& \phantom{AA}& \phantom{AA}& \phantom{AA}& \phantom{AA}& \phantom{AA} \\
\hline
\multicolumn{7}{|c|}{\phantom{A}}\\
\hline
\phantom{AA} & \phantom{AA}& \phantom{AA}& \phantom{AA}& \phantom{AA}& \phantom{AA}& \phantom{AA} \\
\hline
\multicolumn{7}{|c|}{\phantom{A}}\\
\hline
\phantom{AA} & \phantom{AA}& \phantom{AA}& \phantom{AA}& \phantom{AA}& \phantom{AA}& \phantom{AA} \\
\hline
\multicolumn{7}{|c|}{\phantom{A}}\\
\hline
\phantom{AA} & \phantom{AA}& \phantom{AA}& \phantom{AA}& \phantom{AA}& \phantom{AA}& \phantom{AA} \\
\hline
\multicolumn{7}{|c|}{\phantom{A}}\\
\hline
\phantom{AA} & \phantom{AA}& \phantom{AA}& \phantom{AA}& \phantom{AA}& \phantom{AA}& \phantom{AA} \\
\hline
\multicolumn{7}{|c|}{\phantom{A}}\\
\hline
\phantom{AA} & \phantom{AA}& \phantom{AA}& \phantom{AA}& \phantom{AA}& \phantom{AA}& \phantom{AA} \\
\hline
\multicolumn{7}{|c|}{\phantom{A}}\\
\hline
\phantom{AA} & \phantom{AA}& \phantom{AA}& \phantom{AA}& \phantom{AA}& \phantom{AA}& \phantom{AA} \\
\hline
\multicolumn{7}{|c|}{\phantom{A}}\\
\hline
\phantom{AA} & \phantom{AA}& \phantom{AA}& \phantom{AA}& \phantom{AA}& \phantom{AA}& \phantom{AA} \\
\hline
\multicolumn{7}{|c|}{\phantom{A}}\\
\hline
\phantom{AA} & \phantom{AA}& \phantom{AA}& \phantom{AA}& \phantom{AA}& \phantom{AA}& \phantom{AA} \\
\hline
\multicolumn{7}{|c|}{\phantom{A}}\\
\hline
\phantom{AA} & \phantom{AA}& \phantom{AA}& \phantom{AA}& \phantom{AA}& \phantom{AA}& \phantom{AA} \\
\hline
\multicolumn{7}{|c|}{\phantom{A}}\\
\hline
\phantom{AA} & \phantom{AA}& \phantom{AA}& \phantom{AA}& \phantom{AA}& \phantom{AA}& \phantom{AA} \\
\hline
\multicolumn{7}{|c|}{\phantom{A}}\\
\hline
\phantom{AA} & \phantom{AA}& \phantom{AA}& \phantom{AA}& \phantom{AA}& \phantom{AA}& \phantom{AA} \\
\hline
\multicolumn{7}{|c|}{\phantom{A}}\\
\hline
\phantom{AA} & \phantom{AA}& \phantom{AA}& \phantom{AA}& \phantom{AA}& \phantom{AA}& \phantom{AA} \\
\hline
\multicolumn{7}{|c|}{\phantom{A}}\\
\hline
\phantom{AA} & \phantom{AA}& \phantom{AA}& \phantom{AA}& \phantom{AA}& \phantom{AA}& \phantom{AA} \\
\hline
\multicolumn{7}{|c|}{\phantom{A}}\\
\hline
\phantom{AA} & \phantom{AA}& \phantom{AA}& \phantom{AA}& \phantom{AA}& \phantom{AA}& \phantom{AA} \\
\hline
\multicolumn{7}{|c|}{\phantom{A}}\\
\hline
\phantom{AA} & \phantom{AA}& \phantom{AA}& \phantom{AA}& \phantom{AA}& \phantom{AA}& \phantom{AA} \\
\hline
\end{tabular}

\end{multicols}
\newpage



A language $L$ is {\bf recognized by} a Turing machine $M$ means

\vfill

A Turing  machine  $M$ {\bf  recognizes} a language $L$ means

\vfill

A Turing machine $M$ is a {\bf decider}  means

\vfill

A language  $L$ is {\bf decided by} a Turing  machine  $M$  means

\vfill

A  Turing machine $M$ {\bf decides} a language $L$ means

\vfill

Fix $\Sigma = \{0,1\}$, $\Gamma = \{ 0, 1, \textvisiblespace\}$ for the Turing machines with  the following state diagrams:
  
  \begin{center}
  \begin{tabular}{|c|c|}
  \hline
  \hspace{0.8in}\includegraphics[width=2in]{../../resources/machines/Lect14TM1.png} \phantom{\hspace{0.8in}}&\hspace{0.8in} \includegraphics[width=2in]{../../resources/machines/Lect14TM2.png} \phantom{\hspace{0.8in}}\\
  Decider? Yes~~~/ ~~~No
  &Decider? Yes~~~/ ~~~No\\
  & \\
  \hline
  \includegraphics[width=2in]{../../resources/machines/Lect14TM3.png} & \includegraphics[width=2in]{../../resources/machines/Lect14TM4.png} \\
  Decider? Yes~~~/ ~~~No
  &Decider? Yes~~~/ ~~~No\\
  & \\
  
  \hline
  \end{tabular}
  \end{center}
  \newpage

 \vfill
\section*{Week6 friday}




A {\bf Turing-recognizable} language is a set of strings that 
is the language recognized by some Turing machine. We also 
say that such languages are recognizable.

A {\bf Turing-decidable} language is a set of strings that 
is the language recognized by some decider. We also 
say that such languages are decidable.


An {\bf unrecognizable} language is a language that is not Turing-recognizable.

An {\bf undecidable} language is a language that is not Turing-decidable.

\vfill

{\bf  True} or {\bf False}: Any  decidable language  is  also  recognizable.

\vfill

{\bf  True} or {\bf False}: Any  recognizable language  is  also  decidable.

\vfill

{\bf  True} or {\bf False}: Any  undecidable language  is  also  unrecognizable.

\vfill

{\bf  True} or {\bf False}: Any  unrecognizable language  is  also  undecidable.

\vfill

\newpage


{\bf True} or {\bf False}: The class of Turing-decidable languages is closed under complementation.

\vfill

Using formal definition:
\vfill

Using high-level description:
\vfill


{\bf  Church-Turing Thesis} (Sipser p. 183): The informal notion of algorithm is formalized completely  and correctly by the 
formal definition of a  Turing machine. In other words: all reasonably expressive models of 
computation are equally expressive with the standard Turing machine.

\newpage
Definition: A language $L$ over an  alphabet $\Sigma$ is called {\bf co-recognizable} if its complement,  defined
as $\Sigma^* \setminus L  = \{ x  \in  \Sigma^* \mid x \notin  L \}$, is Turing-recognizable.


\vfill 
{\bf  Theorem} (Sipser Theorem 4.22): A  language is Turing-decidable if and only if both  it and its complement
are Turing-recognizable.

{\bf Proof, first direction:}  Suppose  language  $L$ is  Turing-decidable.   WTS  that both it and its complement 
are Turing-recognizable.

\vfill

{\bf Proof, second direction:}  Suppose  language  $L$ is  Turing-recognizable, and  so is  its complement.   WTS  that $L$
is Turing-decidable.
\vfill


Notation: The complement  of a set $X$ is denoted with  a superscript $c$, $X^c$, or an overline,  $\overline{X}$.

\newpage

{\bf Claim}: If two languages  (over a fixed alphabet  $\Sigma$) are Turing-decidable, then  their union  is  as well.

{\bf Proof}:


\vfill
\newpage

{\bf Claim}: If two languages  (over a fixed alphabet  $\Sigma$) are Turing-recognizable, then  their union  is  as well.

{\bf Proof}:

 \vfill
\section*{Week7 wednesday}


{\bf The Church-Turing thesis posits that each algorithm can be implemented by some Turing machine.}

{\bf Describing  algorithms} (Sipser p. 185) To define a Turing machine, we could give a 
\begin{itemize}
\item {\bf Formal definition}: the $7$-tuple of parameters including set of states, 
input alphabet, tape alphabet, transition function, start state, accept state, and reject state.
This is the low-level programming view that models the logic computation flow in a processor.
\item {\bf Implementation-level definition}: English prose that describes the Turing machine head 
movements relative to contents of tape, and conditions for accepting / rejecting based on those contents.
This level describes memory management and implementing data access with data structures.
  \begin{itemize}
    \item Mention the tape or its contents (e.g.\ ``Scan the tape from left to right until a blank is seen.'')
    \item Mention the tape head (e.g.\ ``Return the tape head to the left end of the tape.'')
  \end{itemize}
\item {\bf High-level description} of algorithm executed by Turing machine: 
description of algorithm (precise sequence of instructions), 
without implementation details of machine. 
High-level descriptions of  Turing machine algorithms are written as indented text within quotation marks.   
Stages of the algorithm are typically numbered consecutively.
The first line specifies the input to the machine, which must be a string.
  \begin{itemize}
    \item Use other Turing machines as subroutines (e.g.\ ``Run $M$ on $w$'')
    \item Build new machines from existing machines using previously shown results (e.g.\ 
    ``Given NFA $A$ construct an NFA $B$ such that $L(B) = \overline{L(A)}$'')
    \item Use previously shown conversions and constructions (e.g.\ ``Convert regular expression $R$ 
    to an NFA $N$'')
  \end{itemize}
\end{itemize}

{\bf Formatted inputs to Turing machine algorithms}

The input to a Turing machine is always a string. 
The format of the input to a Turing machine can be checked to interpret 
this string as representing structured data (like a csv file, the formal definition of a DFA, another Turing machine, etc.)


This string may be the encoding of some object or  list of  objects.  

{\bf Notation:} $\langle O \rangle$ is the string that encodes the object $O$.
$\langle O_1, \ldots, O_n \rangle$ is the string that encodes the list of objects $O_1, \ldots, O_n$.

{\bf Assumption}: There are algorithms (Turing  machines) that can be called as subroutines
to decode the string representations of common objects and  interact with these objects as intended
(data structures). These algorithms are able to ``type-check'' and string representations for different
data structures are unique.
  
\newpage
For example, since there are algorithms to answer each of the following questions,
by Church-Turing thesis, there is a Turing machine that accepts exactly those strings for which the 
answer to the question is ``yes''
\begin{itemize}
    \item Does a string over $\{0,1\}$ have even length?

    \item Does a string over $\{0,1\}$ encode a string of ASCII characters?\footnote{An introduction to ASCII 
    is available on the w3 tutorial \href{https://www.w3schools.com/charsets/ref_html_ascii.asp}{here}.}

    \item Does a DFA have a specific number of states?

    \item Do two NFAs have any state names in common?

    \item Do two CFGs have the same start variable?

  \end{itemize}



A {\bf computational problem} is decidable iff language encoding its positive problem instances
is decidable.

The computational problem ``Does a specific DFA accept a given string?'' is encoded by the language
\begin{align*}
  &\{ \textrm{representations of DFAs $M$ and strings $w$ such that $w \in L(M)$}\}  \\
  =& \{ \langle M, w \rangle \mid M \textrm{ is a DFA}, w \textrm{ is a string}, w \in L(M) \}
\end{align*}

The computational problem ``Is the language generated by a CFG empty?'' is encoded by the language
\begin{align*}
  &\{ \textrm{representations of CFGs $G$  such that $L(G) = \emptyset$}\}  \\
  =& \{ \langle G \rangle \mid G \textrm{ is a CFG},  L(G) = \emptyset \}
\end{align*}



The computational problem ``Is the given Turing machine a decider?'' is encoded by the language
\begin{align*}
  &\{ \textrm{representations of TMs $M$  such that $M$ halts on every input}\}  \\
  =& \{ \langle M \rangle \mid M \textrm{ is a TM and for each string } w, \textrm{$M$ halts on $w$} \}
\end{align*}


{\it Note: writing down the language encoding a computational problem is only the first step in 
determining if it's recognizable, decidable, or \ldots }

Deciding a computational problem means building / defining a Turing 
machine that recognizes the language encoding the computational problem, and that 
is a decider.

 \vfill
\section*{Week7 friday}



Some classes of computational problems will 
help us understand the differences between the machine models we've been studying. (Sipser Section 4.1)

    \begin{center}
    \begin{tabular}{|lcl|}
    \hline
    \multicolumn{3}{|l|}{{\bf  Acceptance problem} } \\
    & & \\
    \ldots for DFA & $A_{DFA}$ & $\{ \langle B,w \rangle \mid  \text{$B$ is a  DFA that accepts input 
    string $w$}\}$ \\
    \ldots for NFA & $A_{NFA}$ & $\{ \langle B,w \rangle \mid  \text{$B$ is a  NFA that accepts input 
    string $w$}\}$ \\
    \ldots for regular expressions & $A_{REX}$ & $\{ \langle R,w \rangle \mid  \text{$R$ is a  regular
    expression that generates input string $w$}\}$ \\
    \ldots for CFG & $A_{CFG}$ & $\{ \langle G,w \rangle \mid  \text{$G$ is a context-free grammar 
    that generates input string $w$}\}$ \\
    \ldots for PDA & $A_{PDA}$ & $\{ \langle B,w \rangle \mid  \text{$B$ is a PDA that accepts input string $w$}\}$ \\
    & & \\
    \hline
    \multicolumn{3}{|l|}{{\bf Language emptiness  testing} } \\
    & & \\
    \ldots for DFA & $E_{DFA}$ & $\{ \langle A \rangle \mid  \text{$A$ is a  DFA and  $L(A) = \emptyset$\}}$ \\
    \ldots for NFA & $E_{NFA}$ & $\{ \langle A\rangle \mid  \text{$A$ is a NFA and  $L(A) = \emptyset$\}}$ \\
    \ldots for regular expressions & $E_{REX}$ & $\{ \langle R \rangle \mid  \text{$R$ is a  regular
    expression and  $L(R) = \emptyset$\}}$ \\
    \ldots for CFG & $E_{CFG}$ & $\{ \langle G \rangle \mid  \text{$G$ is a context-free grammar 
    and  $L(G) = \emptyset$\}}$ \\
    \ldots for PDA & $E_{PDA}$ & $\{ \langle A \rangle \mid  \text{$A$ is a PDA and  $L(A) = \emptyset$\}}$ \\
    & & \\
    \hline
    \multicolumn{3}{|l|}{{\bf Language equality testing} } \\
    & & \\
    \ldots for DFA & $EQ_{DFA}$ & $\{ \langle A, B \rangle \mid  \text{$A$ and $B$ are DFAs and  $L(A) =L(B)$\}}$\\
    \ldots for NFA & $EQ_{NFA}$ & $\{ \langle A, B \rangle \mid  \text{$A$ and $B$ are NFAs and  $L(A) =L(B)$\}}$\\
    \ldots for regular expressions & $EQ_{REX}$ & $\{ \langle R, R' \rangle \mid  \text{$R$ and $R'$ are regular
    expressions and  $L(R) =L(R')$\}}$\\
    \ldots for CFG & $EQ_{CFG}$ & $\{ \langle G, G' \rangle \mid  \text{$G$ and $G'$ are CFGs and  $L(G) =L(G')$\}}$ \\
    \ldots for PDA & $EQ_{PDA}$ & $\{ \langle A, B \rangle \mid  \text{$A$ and $B$ are PDAs and  $L(A) =L(B)$\}}$ \\
    \hline
    \end{tabular}
    \end{center}
    

    Example strings in $A_{DFA}$
    
    \vfill
    
    Example strings in  $E_{DFA}$
    
    \vfill
    
    Example strings in  $EQ_{DFA}$
    
    \vfill

  
    \newpage

  \begin{quote}
  $M_1 = $ ``On input $\langle M,w\rangle$, where $M$ is a DFA and $w$ is a string:
  \begin{enumerate}
  \setcounter{enumi}{-1}
  \item Type check encoding to check input is correct type. If not, reject.
  \item Simulate $M$ on input $w$ (by keeping track of states in $M$, transition function of $M$, etc.) 
  \item If the simulations ends in an accept state of $M$, accept. If it ends in a non-accept state of $M$, reject. "
  \end{enumerate}
  \end{quote}
  

What is $L(M_1)$? 

\vfill

Is $M_1$ a decider?

\vfill


{\it Alternate description}:
Sometimes omit step 0 from listing and do implicit type check.

Synonyms: ``Simulate'', ``run'', ``call''.

\newpage

  True / False: $A_{REX} = A_{NFA} = A_{DFA}$

  True / False: $A_{REX} \cap A_{NFA} = \emptyset$, $A_{REX} \cap A_{DFA} = \emptyset$, $A_{DFA} \cap A_{NFA} = \emptyset$

  
  A Turing machine that  decides $A_{NFA}$ is: 
  
  \vfill
  
  A Turing machine that  decides $A_{REX}$ is: 
  
  \vfill

  $E_{DFA} = \{ \langle A \rangle \mid  \text{$A$ is a  DFA and  $L(A) = \emptyset$}\}$. 
  True/False: A Turing machine that  decides $E_{DFA}$ is

  \begin{quote}
  $M_2 =  $``On  input  $\langle M\rangle$ where $M$ is a  DFA,
  \begin{enumerate}
  \item For integer  $i = 1, 2, \ldots$
  \item \qquad Let  $s_i$ be the  $i$th string over  the alphabet of  $M$ (ordered in  string order).
  \item \qquad Run $M$ on  input  $s_i$.
  \item \qquad If $M$  accepts,  $\underline{\phantom{FILL  IN BLANK}}$.  If $M$  rejects, increment $i$ and keep going."
  \end{enumerate}
  \end{quote}
  

Choose the correct option to help fill in the blank so that $M_2$ recognizes $E_{DFA}$
\begin{itemize}
\item[A.] accepts
\item[B.] rejects
\item[C.] loop for ever
\item[D.] We can't fill in the blank in any way to make this work
\end{itemize}

\newpage
  

  
  \begin{quote}
  $M_3 =  $ `` On  input $\langle M \rangle$ where $M$ is  a  DFA,
  \begin{enumerate}
  \item Mark the start  state  of $M$.
  \item Repeat until no  new states get marked:
  \item \qquad Loop over the states of $M$. 
  \item \qquad Mark any unmarked  state  that  has an incoming  edge  from a marked state.
  \item If  no  accept state of $A$ is  marked, $\underline{\phantom{FILL  IN BLANK}}$;  otherwise, 
  $\underline{\phantom{FILL  IN BLANK}}$".
  \end{enumerate}
  \end{quote}
  
  
  
To build a Turing machine that decides $EQ_{DFA}$, notice that 
\[
L_1 = L_2 \qquad\textrm{iff}\qquad (~(L_1 \cap \overline{L_2}) \cup (L_2 \cap \overline L_1)~) = \emptyset  
\]
{\it There are no elements that are in one set and not the other}


$M_{EQDFA} = $ 


  \vfill
  

  {\bf Summary}:  We can use the decision procedures (Turing machines) of decidable problems
  as subroutines in other algorithms. For example, we have subroutines for deciding each of 
  $A_{DFA}$, $E_{DFA}$, $EQ_{DFA}$.  We can also use algorithms for known constructions
  as subroutines in other algorithms. For example, we have subroutines for: counting the number 
  of states in a state diagram, counting the number of characters in an alphabet, converting DFA
  to a DFA recognizing the complement of the original language or a DFA recognizing the 
  Kleene star of the original language, constructing a DFA or NFA from two DFA or NFA so that 
  we have a machine recognizing the language of the union (or intersection, concatenation)
  of the languages of the original machines; converting regular expressions to equivalent DFA; 
  converting DFA to equivalent regular expressions, etc.
 \vfill
\section*{Week3 friday}


{\bf Definition and Theorem}: For an alphabet $\Sigma$, a language $L$ over $\Sigma$ is called {\bf regular}
exactly when $L$ is recognized by some DFA, which happens exactly when $L$ is recognized by some NFA, 
and happens exactly when $L$ is described by some regular expression

{\bf We saw that}: The class of regular languages is closed under complementation, union, 
intersection, set-wise concatenation, and Kleene star.

{\bf Prove or Disprove}: There is some alphabet $\Sigma$ for which there is 
some language recognized by an NFA but not by any DFA.

\vfill

{\bf Prove or Disprove}: There is some alphabet $\Sigma$ for which there is 
some finite language not described by any regular expression over $\Sigma$.

\vfill

{\bf Prove or Disprove}: If a language is recognized by an NFA 
then the complement of this language is not recognized by any DFA.

\vfill

\newpage

{\bf Fix alphabet $\Sigma$. Is every language $L$ over $\Sigma$ regular?}

\begin{center}
\begin{tabular}{c|c}
Set & Cardinality \\
\hline
& \\
$\{0,1\}$ & \\
& \\
$\{0,1\}^*$ & \\
& \\
$\mathcal{P}( \{0,1\})$ & \\
& \\
The set of all languages over $\{0,1\}$ & \\
& \\
The set of all regular expressions over $\{0,1\}$ & \\
& \\
The set of all regular languages over $\{0,1\}$ & \\
& \\
\end{tabular}
\end{center}



\vfill
Strategy: Find an {\bf invariant} property that is true of all regular languages. When analyzing 
a given language, if the invariant is not true about it, then the language is not regular.
\newpage

{\bf Pumping Lemma} (Sipser Theorem 1.70): If $A$ is a regular language, then there
is a number $p$ (a {\it pumping length}) where, if $s$ is any string in $A$ of length at least $p$, 
then $s$ may be divided into three pieces, $s = xyz$ such that
\vspace{-10pt}
\begin{itemize}
\item $|y| > 0$
\item for each $i \geq 0$, $xy^i z \in A$
\item $|xy| \leq p$.
\end{itemize}


{\bf Proof illustration}


\vfill




{\bf True or False}: A pumping length for $A = \{ 0,1 \}^*$ is $p = 5$.

\vspace{50pt} \vfill
\section*{Week10 monday}


Recall Definition  (Sipser 7.1): For  $M$ a deterministic decider, its {\bf running time} is the function  $f: \mathbb{N} \to \mathbb{N}$
given  by
\[
f(n) =  \text{max number of  steps $M$ takes before halting, over all inputs  of length $n$}
\]

Recall Definition (Sipser 7.7): For each function $t(n)$, the {\bf time complexity class}  $TIME(t(n))$, is defined  by
\[
TIME( t(n)) = \{ L \mid \text{$L$ is decidable by  a Turing machine with running time in  $O(t(n))$} \}
\]
Recall Definition (Sipser 7.12) : $P$ is the class of languages that  are decidable in polynomial time on 
a deterministic 1-tape  Turing  machine
\[
P  =  \bigcup_k TIME(n^k)
\]

Definition (Sipser  7.9): For $N$ a nodeterministic decider.  
The {\bf running time} of $N$ is the function $f: \mathbb{N} \to \mathbb{N}$ given  by
\[
f(n) =  \text{max number of  steps $N$ takes on  any branch before halting, over all inputs  of length $n$}
\]

Definition (Sipser 7.21): For each function $t(n)$, the {\bf nondeterministic time complexity class}  
$NTIME(t(n))$, is defined  by
\[
NTIME( t(n)) = \{ L \mid \text{$L$ is decidable by a nondeterministic Turing machine with running time in $O(t(n))$} \}
\]
\[
NP = \bigcup_k NTIME(n^k)
\]


{\bf True} or {\bf False}: $TIME(n^2) \subseteq NTIME(n^2)$

\vfill

{\bf True} or {\bf False}: $NTIME(n^2) \subseteq DTIME(n^2)$

\vfill
\newpage

{\bf Examples in $P$ }

{\it Can't use nondeterminism; Can use multiple tapes; Often need to be “more clever” than naïve / brute force approach}
\[
    PATH = \{\langle G,s,t\rangle \mid \textrm{$G$ is digraph with $n$ nodes there is path from s to t}\}
\]
Use breadth first search to show in $P$
\[
    RELPRIME = \{ \langle x,y\rangle \mid \textrm{$x$ and $y$ are relatively prime integers}\}
\]
Use Euclidean Algorithm to show in $P$
\[
    L(G) = \{w \mid \textrm{$w$ is generated by $G$}\} 
\]
(where $G$ is a context-free grammar). Use dynamic programming to show in $P$.

\vfill
{\bf Examples in $NP$}

{\it ``Verifiable" i.e. NP, Can be decided by a nondeterministic TM in polynomial time,
best known deterministic solution may be brute-force, 
solution can be verified by a deterministic TM in polynomial time.}

\[
    HAMPATH = \{\langle G,s,t \rangle \mid \textrm{$G$ is digraph with $n$ nodes, there is path
from $s$ to $t$ that goes through every node exactly once}\}
\]
\[
    VERTEX-COVER = \{ \langle G,k\rangle \mid \textrm{$G$ is an undirected graph with $n$
nodes that has a $k$-node vertex cover}\}
\]
\[
    CLIQUE = \{ \langle G,k\rangle \mid \textrm{$G$ is an undirected graph with $n$ nodes that has a $k$-clique}\}
\]
\[
    SAT =\{ \langle X \rangle \mid \textrm{$X$ is a satisfiable Boolean formula with $n$ variables}\}
\]

\vfill
\newpage
{\bf Every problem in NP is decidable with an exponential-time algorithm}

Nondeterministic approach: guess a possible solution, verify that it works.

Brute-force (worst-case exponential time) approach: iterate over all possible solutions, for each 
one, check if it works.



\begin{center}
\begin{tabular}{c|c}
    {\bf Problems in $P$} & {\bf Problems in $NP$}\\
    \hline
    (Membership in any) regular language & Any problem in $P$ \\
    (Membership in any) context-free language &  \\
    $A_{DFA}$ & $SAT$\\
    $E_{DFA}$ & $CLIQUE$ \\
    $EQ_{DFA}$ & $VERTEX-COVER$ \\
    $PATH$ & $HAMPATH$ \\
    $RELPRIME$ &  $\ldots$ \\
$\ldots$ &\\
\end{tabular}
\end{center}

Million-dollar question: Is $P = NP$?


One approach to trying to answer it is to look for {\it hardest} problems in $NP$ and 
then (1) if we can show that there are efficient algorithms for them, then we can get 
efficient algorithms for all problems in $NP$ so $P = NP$, or (2) these problems might 
be good candidates for showing that there are problems in $NP$ for which there 
are no efficient algorithms.

\vfill
\newpage
 \vfill
\section*{Week10 wednesday}



Definition (Sipser 7.29) Language  $A$ is {\bf polynomial-time mapping reducible} to language $B$, written $A \leq_P B$,
means there is a polynomial-time computable function $f: \Sigma^* \to \Sigma^*$  such that for every $x \in \Sigma^*$
\[
x \in A \qquad \text{iff} \qquad f(x) \in B.
\]
The  function $f$ is  called the  polynomial time reduction of $A$ to $B$.

{\bf  Theorem}  (Sipser 7.31):  If $A \leq_P B$ and $B  \in P$ then $A \in P$.

Proof: 

\vfill

Definition (Sipser 7.34; based in Stephen Cook and Leonid Levin's work in the 1970s): 
A language $B$ is {\bf  NP-complete} means (1) $B$ is in NP {\bf and}  (2) every language
$A$ in $NP$ is polynomial time reducible to $B$.

{\bf  Theorem}  (Sipser 7.35):  If $B$ is NP-complete and $B \in P$ then $P = NP$.

Proof: 

\vfill

\newpage

{\bf 3SAT}: A literal is a Boolean variable (e.g.  $x$) or a negated Boolean variable (e.g.  $\bar{x}$).  
A Boolean formula is a {\bf  3cnf-formula} if it is a Boolean formula in conjunctive normal form (a conjunction  
of  disjunctive clauses of literals) and each clause  has  three literals.
\[
3SAT  = \{  \langle  \phi \rangle \mid \text{$\phi$ is  a  satisfiable 3cnf-formula} \}
\]


Example strings  in $3SAT$
\vfill



Example  strings not  in $3SAT$

\vfill




{\bf Cook-Levin Theorem}: $3SAT$ is $NP$-complete.


{\it Are there other $NP$-complete problems?} To prove that $X$ is $NP$-complete
\begin{itemize}
\item {\it From scratch}: prove $X$ is in $NP$ and that all $NP$ problems are polynomial-time
reducible to $X$.
\item {\it Using reduction}: prove $X$ is in $NP$ and that a known-to-be $NP$-complete problem 
is polynomial-time reducible to $X$.
\end{itemize}

\vfill
\vfill


\newpage

{\bf CLIQUE}: A {\bf $k$-clique} in an undirected graph is a maximally connected subgraph with $k$  nodes.
\[
CLIQUE  = \{  \langle G, k \rangle \mid \text{$G$ is an  undirected graph with  a $k$-clique} \}
\]


Example strings  in $CLIQUE$

\vfill

Example  strings not  in $CLIQUE$

\vfill

Theorem (Sipser 7.32):
\[
3SAT  \leq_P CLIQUE
\]

Given a Boolean formula in conjunctive normal form with $k$ clauses and three literals per clause, we will 
map it to a graph so that the graph has a clique if the original formula is satisfiable and the 
graph does not have a clique if the original formula is not satisfiable.

The graph has $3k$ vertices (one for each literal in each clause) and an edge between all vertices except
\begin{itemize}
    \item vertices for two literals in the same clause
    \item vertices for literals that are negations of one another
\end{itemize}

Example: $(x \vee \bar{y} \vee {\bar z}) \wedge (\bar{x}  \vee y  \vee  z) \wedge (x \vee y  \vee z)$

\vfill

\newpage
 \vfill
\section*{Week10 friday}


\begin{center}
    \begin{tabular}{|p{4in}|p{3.5in}|}
        \hline
        & \\
        {\bf Model of Computation} & {\bf Class of Languages}\\
        &\\
        \hline
        & \\
        {\bf Deterministic finite automata}:
        formal definition, how to design for a given language, 
        how to describe language of a machine?
        {\bf Nondeterministic finite automata}:
        formal definition, how to design for a given language, 
        how to describe language of a machine?
        {\bf Regular expressions}: formal definition, how to design for a given language, 
        how to describe language of expression?
        {\it Also}: converting between different models. &
        {\bf Class of regular languages}: what are the closure 
        properties of this class? which languages are not in the class?
        using {\bf pumping lemma} to prove nonregularity.\\
        & \\
        \hline
        & \\
        {\bf Push-down automata}:
        formal definition, how to design for a given language, 
        how to describe language of a machine?
        {\bf Context-free grammars}:
        formal definition, how to design for a given language, 
        how to describe language of a grammar? &
        {\bf Class of context-free languages}: what are the closure 
        properties of this class? which languages are not in the class?\\
        & \\
        \hline
        & \\
        Turing machines that always halt in polynomial time
        & $P$ \\
        & \\
        Nondeterministic Turing machines that always halt in polynomial time 
        & $NP$ \\
        & \\
        \hline
        & \\
        {\bf Deciders} (Turing machines that always halt): 
        formal definition, how to design for a given language, 
        how to describe language of a machine? &
        {\bf Class of decidable languages}: what are the closure properties 
        of this class? which languages are not in the class? using diagonalization
        and mapping reduction to show undecidability \\
        & \\
        \hline
        & \\
        {\bf Turing machines}
        formal definition, how to design for a given language, 
        how to describe language of a machine? &
        {\bf Class of recognizable languages}: what are the closure properties 
        of this class? which languages are not in the class? using closure
        and mapping reduction to show unrecognizability \\
        & \\
        \hline
    \end{tabular}
\end{center}

\newpage

{\bf Given a language, prove it is regular}

{\it Strategy 1}: construct DFA recognizing the language and prove it works.

{\it Strategy 2}: construct NFA recognizing the language and prove it works.

{\it Strategy 3}: construct regular expression recognizing the language and prove it works.

{\it ``Prove it works'' means \ldots}

\vspace{100pt}

{\bf Example}: $L  = \{ w \in \{0,1\}^* \mid \textrm{$w$ has odd number of $1$s or starts with $0$}\}$

Using NFA

\vfill

Using regular expressions

\vfill


\newpage

{\bf Example}: Select all and only the options that result in a true statement: ``To show 
a language $A$ is not regular, we can\ldots'' 

\begin{enumerate}
    \item[a.] Show $A$ is finite
    \item[b.] Show there is a CFG generating $A$
    \item[c.] Show $A$ has no pumping length
    \item[d.] Show $A$ is undecidable
\end{enumerate}

\newpage

{\bf Example}: What is the language generated by the CFG with rules
\begin{align*}
    S &\to aSb \mid bY \mid Ya \\
    Y &\to bY \mid Ya \mid \varepsilon 
\end{align*}

\newpage

{\bf Example}: Prove that the language 
$T = \{ \langle M \rangle \mid \textrm{$M$ is a Turing machine and $L(M)$ is infinite}\}$ 
is undecidable.

\newpage

{\bf Example}: Prove that the class of decidable languages is closed under concatenation.
 \vfill
\end{document}