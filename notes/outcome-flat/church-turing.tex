\documentclass[12pt, oneside]{article}

\usepackage[letterpaper, scale=0.89, centering]{geometry}
\usepackage{fancyhdr}
\setlength{\parindent}{0em}
\setlength{\parskip}{1em}

\pagestyle{fancy}
\fancyhf{}
\renewcommand{\headrulewidth}{0pt}
\rfoot{\href{https://creativecommons.org/licenses/by-nc-sa/2.0/}{CC BY-NC-SA 2.0} Version \today~(\thepage)}

\usepackage{amssymb,amsmath,pifont,amsfonts,comment,enumerate,enumitem}
\usepackage{currfile,xstring,hyperref,tabularx,graphicx,wasysym}
\usepackage[labelformat=empty]{caption}
\usepackage[dvipsnames,table]{xcolor}
\usepackage{multicol,multirow,array,listings,tabularx,lastpage,textcomp,booktabs}

\lstnewenvironment{algorithm}[1][] {   
    \lstset{ mathescape=true,
        frame=tB,
        numbers=left, 
        numberstyle=\tiny,
        basicstyle=\rmfamily\scriptsize, 
        keywordstyle=\color{black}\bfseries,
        keywords={,procedure, div, for, to, input, output, return, datatype, function, in, if, else, foreach, while, begin, end, }
        numbers=left,
        xleftmargin=.04\textwidth,
        #1
    }
}
{}
\lstnewenvironment{java}[1][]
{   
    \lstset{
        language=java,
        mathescape=true,
        frame=tB,
        numbers=left, 
        numberstyle=\tiny,
        basicstyle=\ttfamily\scriptsize, 
        keywordstyle=\color{black}\bfseries,
        keywords={, int, double, for, return, if, else, while, }
        numbers=left,
        xleftmargin=.04\textwidth,
        #1
    }
}
{}

\newcommand\abs[1]{\lvert~#1~\rvert}
\newcommand{\st}{\mid}

\newcommand{\A}[0]{\texttt{A}}
\newcommand{\C}[0]{\texttt{C}}
\newcommand{\G}[0]{\texttt{G}}
\newcommand{\U}[0]{\texttt{U}}

\newcommand{\cmark}{\ding{51}}
\newcommand{\xmark}{\ding{55}}
 
\begin{document}
\begin{flushright}
    \StrBefore{\currfilename}{.}
\end{flushright} \section*{Week6 friday}



To define a Turing machine, we could give a 
\begin{itemize}
\item {\bf Formal definition}: the $7$-tuple of parameters including set of states, 
input alphabet, tape alphabet, transition function, start state, accept state, and reject state; or,
\item {\bf Implementation-level definition}: English prose that describes the Turing machine head 
movements relative to contents of tape, and conditions for accepting / rejecting based on those contents.
\item {\bf High-level description}: description of algorithm (precise sequence of instructions), 
without implementation details of machine. As part of this description, can ``call" and run 
another TM as a subroutine.
\end{itemize}


{\bf Theorem 3.21} A language is Turing-recognizable iff some enumerator enumerates it.

{\bf Proof}:

Assume $L$ is enumerated by some enumerator, $E$, so $L = L(E)$.
We'll use $E$ in a subroutine
within a high-level description of a new Turing machine that we will build to recognize $L$.

{\bf Goal}: build Turing machine $M_E$ with $L(M_E) = L(E)$.

Define $M_E$ as follows: $M_E = $ ``On input $w$,
\begin{enumerate}
\item Run $E$. For each string $x$ printed by $E$.
\item \qquad Check if $x = w$. If so, accept (and halt); otherwise, continue."
\end{enumerate}


\vfill 



Assume $L$ is Turing-recognizable and there 
is a Turing  machine  $M$ with  $L = L(M)$. We'll use $M$ in a subroutine
within a high-level description of an enumerator that we will build to enumerate $L$.

{\bf Goal}: build enumerator $E_M$ with $L(E_M) = L(M)$.

{\bf Idea}: check each string in turn to see if it is in $L$.

{\it How?} Run computation of $M$ on each string.  {\it But}: need to be careful 
about computations that don't halt.

{\it Recall} String order for $\Sigma = \{0,1\}$: $s_1 = \varepsilon$, $s_2 = 0$, $s_3 = 1$, $s_4 = 00$, $s_5 = 01$, $s_6  = 10$, 
$s_7  =  11$, $s_8 = 000$, \ldots

Define $E_M$ as follows: $E_{M} = $ `` {\it ignore any input.} Repeat the following for $i=1, 2, 3, \ldots$
\begin{enumerate}
  \item Run the computations of $M$ on $s_1$, $s_2$, \ldots, $s_i$ for (at most) $i$ steps each
  \item For each of these $i$ computations that accept during the (at most) $i$ steps, print
  out the accepted string."
\end{enumerate}

\newpage

{\bf Nondeterministic Turing machine}

At any point in the computation, the nondeterministic machine may proceed according to 
several possibilities: $(Q, \Sigma, \Gamma, \delta, q_0, q_{acc}, q_{rej})$ where 
\[
\delta: Q \times \Gamma \to \mathcal{P}(Q \times \Gamma \times \{L, R\})  
\]
The computation of a nondeterministic Turing machine is a tree with branching
when the next step of the computation has multiple possibilities. A nondeterministic
Turing machine accepts a string exactly when some branch of the computation tree 
enters the accept state.

Given a nondeterministic machine, we can use a $3$-tape Turing machine to 
simulate it by doing a breadth-first search of computation tree: one tape 
is ``read-only'' input tape, one tape simulates the tape of the nondeterministic
computation, and one tape tracks nondeterministic branching. {\tiny Sipser page 178} 

\vfill
Two models of computation are called {\bf equally expressive} when 
every language recognizable with the first model is recognizable with the second, and vice versa.

{\bf  Church-Turing Thesis} (Sipser p. 183): The informal notion of algorithm is formalized completely  and correctly by the 
formal definition of a  Turing machine. In other words: all reasonably expressive models of 
computation are equally expressive with the standard Turing machine.

\vfill

\newpage

{\bf Claim}: If two languages  (over a fixed alphabet  $\Sigma$) are Turing-recognizable, then  their union  is  as well.

{\bf Proof using Turing machines}:

\vfill

{\bf Proof using nondeterministic Turing machines}: 

\vfill  

{\bf  Proof using enumerators}:

\vfill

 \vfill
\end{document}