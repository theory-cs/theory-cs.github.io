\documentclass[12pt, oneside]{article}

\usepackage[letterpaper, scale=0.89, centering]{geometry}
\usepackage{fancyhdr}
\setlength{\parindent}{0em}
\setlength{\parskip}{1em}

\usepackage{tikz}
\usetikzlibrary{automata,positioning,arrows}

\pagestyle{fancy}
\fancyhf{}
\renewcommand{\headrulewidth}{0pt}
\rfoot{\href{https://creativecommons.org/licenses/by-nc-sa/2.0/}{CC BY-NC-SA 2.0} Version \today~(\thepage)}

\usepackage{amssymb,amsmath,pifont,amsfonts,comment,enumerate,enumitem}
\usepackage{currfile,xstring,hyperref,tabularx,graphicx,wasysym}
\usepackage[labelformat=empty]{caption}
\usepackage{xcolor}
\usepackage{multicol,multirow,array,listings,tabularx,lastpage,textcomp,booktabs}

\lstnewenvironment{algorithm}[1][] {   
    \lstset{ mathescape=true,
        frame=tB,
        numbers=left, 
        numberstyle=\tiny,
        basicstyle=\rmfamily\scriptsize, 
        keywordstyle=\color{black}\bfseries,
        keywords={,procedure, div, for, to, input, output, return, datatype, function, in, if, else, foreach, while, begin, end, }
        numbers=left,
        xleftmargin=.04\textwidth,
        #1
    }
}
{}

\newcommand\abs[1]{\lvert~#1~\rvert}
\newcommand{\st}{\mid}

\newcommand{\cmark}{\ding{51}}
\newcommand{\xmark}{\ding{55}}
 
\begin{document}
\begin{flushright}
    \StrBefore{\currfilename}{.}
\end{flushright} \section*{Week8 monday}



{\bf  Theorem}: $A_{TM}$  is  not  Turing-decidable.

{\bf  Proof}: Suppose {\bf towards a  contradiction}  that there  is a Turing machine  that decides $A_{TM}$.  
We call this presumed machine  $M_{ATM}$.

By  assumption, for every  Turing machine  $M$ and every  string $w$

\begin{itemize}
\item If $w \in L(M)$, then  the computation of $M_{ATM}$  on  $\langle M,w \rangle ~~ \underline{\phantom{\hspace{2.5in}}}$
\item If $w \notin L(M)$, then  the computation of $M_{ATM}$  on  $\langle M,w \rangle ~~ \underline{\phantom{\hspace{2.5in}}}$
\end{itemize}


Define  a {\bf new} Turing machine using  the high-level description:
\begin{quote}
$D =  $`` On  input $\langle M \rangle$, where  $M$  is  a Turing machine:
\begin{itemize}
\item[1.] Run  $M_{ATM}$ on  $\langle M, \langle M \rangle  \rangle$.
\item[2.] If $M_{ATM}$ accepts, reject; if  $M_{ATM}$ rejects, accept."
\end{itemize}
\end{quote}


Is $D$ a  Turing machine?

\vspace{50pt}

Is  $D$ a  decider? 

\vspace{50pt}

What is the result of the computation  of $D$  on  $\langle D \rangle$?

\vfill

\newpage

{\bf  Theorem} (Sipser Theorem 4.22): A  language is Turing-decidable if and only if both  it and its complement
are Turing-recognizable.

{\bf Proof, first direction:}  Suppose  language  $L$ is  Turing-decidable.   WTS  that both it and its complement 
are Turing-recognizable.

\vfill

{\bf Proof, second direction:}  Suppose  language  $L$ is  Turing-recognizable, and  so is  its complement.   WTS  that $L$
is Turing-decidable.
\vfill


Give an example of a {\bf decidable} set: 

\vspace{20pt}

Give an example of a {\bf recognizable undecidable} set: 

\vspace{20pt}

Give an example of an {\bf unrecognizable} set: 

\vspace{20pt}


\newpage

{\bf True} or {\bf False}: The class of Turing-decidable languages is closed under complementation?

\vfill
\vfill
\vfill
Definition: A language $L$ over an  alphabet $\Sigma$ is called {\bf co-recognizable} if its complement,  defined
as $\Sigma^* \setminus L  = \{ x  \in  \Sigma^* \mid x \notin  L \}$, is Turing-recognizable.


Notation: The complement  of a set $X$ is denoted with  a superscript $c$, $X^c$, or an overline,  $\overline{X}$.
 \vfill
\section*{Week5 monday}


These definitions are on pages 101-102.

\vspace{-20pt}

\begin{center}
    \begin{tabular}{|p{2.4in}cp{3.6in}|}
    \hline 
    {\bf Term} & {\bf Typical symbol} & {\bf Meaning} \\
     & or {\bf Notation} & \\
    \hline
    \hline
    {\bf Context-free grammar} (CFG) & $G$ & $G = (V, \Sigma, R, S)$ \\
    The set of {\bf variables}& $V$ & Finite  set of symbols that represent phases in production pattern\\
    The set of {\bf terminals} & $\Sigma$ & Alphabet of symbols of strings generated  by CFG \\
    & & $V \cap \Sigma = \emptyset$ \\
    The set of {\bf rules}& $R$ & Each rule is  $A \to u$ with $A \in V$ and $u  \in (V  \cup \Sigma)^*$\\
    The {\bf start} variable&  $S$  & Usually  on left-hand-side of first/ topmost rule \\
    & &\\
    {\bf Derivation} & $S \Rightarrow \cdots \Rightarrow w$& 
    Sequence  of substitutions in a  CFG (also written $S \Rightarrow^* w$). At each step, we can apply one rule 
    to one occurrence of a variable in the current string by substituting that occurrence of the variable with the 
    right-hand-side of the rule. The derivation must end when the current string has only terminals (no variables)
    because then there are no instances of variables to apply a rule to.\\
    Language {\bf generated} by the context-free grammar $G$ & $L(G)$ &The set of strings for which there is a derivation in $G$. 
    Symbolically: $\{  w \in \Sigma^* \mid S \Rightarrow^* w \}$ i.e. $$\{  w \in \Sigma^* \mid \text{there is  derivation in $G$ that ends
    in $w$} \}$$\\
    {\bf Context-free language} & & A language that is the language generated by some context-free grammar\\
    \hline
    \end{tabular}
\end{center}

\vfill

  
{\bf Examples of context-free grammars, derivations in those grammars, and the languages generated by those grammars}
  
$G_1 =  (\{S\}, \{0\}, R, S)$ with rules
  \begin{align*}
    &S \to 0S\\
    &S \to 0\\
  \end{align*}
  In  $L(G_1)$ \ldots 
  
\vfill

  Not in $L(G_1)$ \ldots 


  \vfill

\newpage
  $G_2 =  (\{S\}, \{0,1\}, R, S)$
  \[
  S \to 0S \mid 1S \mid \varepsilon
  \]
  In  $L(G_2)$ \ldots 
  
  \vspace{110pt}
  
  Not in $L(G_2)$ \ldots 

  \vspace{110pt}

  $(\{S, T\}, \{0, 1\}, R, S)$ with  rules
  \begin{align*}
  &S \to T1T1T1T \\
  &T \to  0T \mid 1T \mid \varepsilon
  \end{align*}

  In  $L(G_3)$ \ldots 
  
  \vspace{110pt}
  
  Not in $L(G_3)$ \ldots 

  \vspace{110pt}

\newpage
  $G_4 =  (\{A, B\}, \{0, 1\}, R, A)$ with rules
  \[
    A \to 0A0 \mid  0A1 \mid 1A0  \mid 1A1 \mid  1
  \]
  In  $L(G_4)$ \ldots 
  
  \vspace{110pt}
  
  Not in $L(G_4)$ \ldots 

  \vspace{110pt}


  \begin{comment}
    I moved the following to the review quiz.

    {\it Extra practice}: Is there a CFG $G$ with $L(G) = \emptyset$?


  Three different CFGs that each generate the  language $\{abba\}$
  
  \begin{align*}
  & ( \{ S, T, V, W\}, \{a,b\}, \{ S \to aT, T \to bV, V \to bW, W \to a\}, S)\\
  & \\ 
  & \\ 
  & \\ 
  & ( \{ Q \}, \{a,b\}, \{Q \to abba\}, Q) \\
  & \\ 
  & \\ 
  & \\
  & ( \{ X,Y \}, \{a,b\}, \{X \to aYa, Y \to bb\}, X) 
  & \\ 
  & \\ 
  \end{align*} 
\end{comment}
  \newpage
  Design a CFG to generate the  language $\{a^n b^n \mid  n  \geq  0\}$
  
  \vspace{100pt}
  
  {\it Sample derivation:} 
  
  \vspace{100pt}
  
  
  \vfill 


\newpage
 \vfill
\section*{Week5 wednesday}


Warmup:   Design a CFG to generate the  language $\{a^i b^j \mid j \geq i  \geq 0\}$
  
\vfill
{\it Sample derivation:} 

\vspace{100pt}



Design a PDA to recognize the  language $\{a^i b^j \mid j \geq i  \geq 0\}$
  
\vspace{100pt}


\vfill
\newpage

{\bf Theorem  2.20}: A language is  generated by some context-free  grammar
if  and only if it is recognized by some push-down automaton.

Definition: a language is called {\bf context-free} if it is the language generated by a context-free grammar.
The class of all context-free language over a given alphabet $\Sigma$ is called {\bf CFL}.

Consequences:
\begin{itemize}
    \item Quick proof that every regular language is context free 
    \item To prove closure of the class of context-free languages under a given operation, we can choose 
    either of two modes 
    of proof (via CFGs or PDAs) depending on which is easier
    \item To fully specify a PDA we could give its $6$-tuple formal definition or we could give its input 
alphabet, stack alphabet, and state diagram.
An informal description of a PDA is a step-by-step description of how its computations 
would process input strings; the reader should be able to reconstruct the state diagram or formal 
definition precisely from such a descripton. The informal description of a PDA can refer to some 
common modules or subroutines that are computable by PDAs:
\begin{itemize}
  \item PDAs can ``test for emptiness of stack'' without providing details. 
  {\it How?} We can always push a special end-of-stack symbol, $\$$, at the start, before processing
  any input, and then use this symbol as a flag.
  \item PDAs can ``test for end of input'' without providing details.
  {\it How?} We can transform a PDA to one where accepting states are only those reachable 
  when there are no more input symbols.
\end{itemize}

\end{itemize}




\vfill

\newpage
Suppose $L_1$ and $L_2$ are context-free languages over $\Sigma$.  {\bf Goal}:  $L_1 \cup L_2$  is  also context-free.

{\it Approach 1: with  PDAs}

Let $M_1 = ( Q_1, \Sigma, \Gamma_1, \delta_1, q_1, F_1)$ and
$M_2 = ( Q_2, \Sigma, \Gamma_2, \delta_2, q_2, F_2)$ be PDAs with 
$L(M_1) =  L_1$  and  $L(M_2) = L_2$.

Define $M = $

\vfill

{\it Approach  2: with CFGs}

Let $G_1 = (V_1, \Sigma, R_1, S_1)$  and   $G_2 = (V_2, \Sigma, R_2, S_2)$  be CFGs  with
$L(G_1) =  L_1$  and  $L(G_2) = L_2$.

Define $G = $

\vfill

\newpage
Suppose $L_1$ and $L_2$ are context-free languages over $\Sigma$.  {\bf Goal}:  $L_1 \circ L_2$  is  also context-free.


{\it Approach 1: with  PDAs}

Let $M_1 = ( Q_1, \Sigma, \Gamma_1, \delta_1, q_1, F_1)$ and
$M_2 = ( Q_2, \Sigma, \Gamma_2, \delta_2, q_2, F_2)$ be PDAs with 
$L(M_1) =  L_1$  and  $L(M_2) = L_2$.

Define $M = $

\vfill

{\it Approach  2: with CFGs}

Let $G_1 = (V_1, \Sigma, R_1, S_1)$  and   $G_2 = (V_2, \Sigma, R_2, S_2)$  be CFGs  with
$L(G_1) =  L_1$  and  $L(G_2) = L_2$.

Define $G = $

\vfill
\newpage
{\it Summary}

Over a fixed alphabet $\Sigma$, a language $L$ is {\bf regular}

\vspace{-20pt}
\begin{center}
    iff it is described by some regular expression \\
    iff it is recognized by some DFA\\
    iff it is recognized by some NFA
\end{center}

Over a fixed alphabet $\Sigma$, a language $L$ is {\bf context-free}

\vspace{-20pt}
\begin{center}
    iff it is generated by some CFG\\
    iff it is recognized by some PDA
\end{center}

{\bf Fact}: Every regular language is a context-free language.

{\bf Fact}: There are context-free languages that are not nonregular.

{\bf Fact}: There are countably many regular languages.

{\bf Fact}: There are countably inifnitely many context-free languages.

{\it Consequence}: Most languages are {\bf not} context-free!

{\bf Examples  of non-context-free languages}

\begin{align*}
    &\{ a^n b^n c^n \mid 0 \leq n , n \in \mathbb{Z}\}\\
    &\{ a^i b^j c^k \mid 0 \leq i \leq j \leq k , i \in \mathbb{Z}, j \in \mathbb{Z}, k \in \mathbb{Z}\}\\
    &\{ ww \mid w \in \{0,1\}^* \}
\end{align*}
(Sipser Ex 2.36, Ex 2.37, 2.38)

There is a Pumping Lemma for CFL that can be used to prove a specific language is non-context-free: 
If $A$ is a context-free language, there there
is a number $p$ where, if $s$ is any string in $A$ of length at least $p$, then $s$ may be divided 
into five pieces $s = uvxyz$ where (1) for each $i \geq 0$, $uv^ixy^iz \in A$, (2) $|uv|>0$, (3) $|vxy| \leq p$.
{\it We will not go into the details of the proof or application of Pumping Lemma for CFLs this quarter.}

\begin{comment}
##Moved to review quiz
A set $X$ is said to be {\bf closed} under an operation $OP$ if, for any elements in $X$, applying 
$OP$ to them gives an element in $X$.  


\begin{center}
\begin{tabular}{|c|l|}
\hline
True/False & Closure claim \\
\hline
True &  The set of integers is closed under multiplication. \\
& $\forall x \forall y \left( ~(x \in \mathbb{Z} \wedge y \in \mathbb{Z})\to xy \in \mathbb{Z}~\right)$ \\
\hline
True & For each set $A$, the power set of $A$ is closed under intersection. \\
& $\forall A_1 \forall A_2 \left( ~(A_1 \in \mathcal{P}(A) \wedge A_2 \in \mathcal{P}(A) \in \mathbb{Z}) \to A_1 \cap A_2 \in \mathcal{P}(A)~\right)$ \\
\hline
  & The class of regular languages over $\Sigma$ is closed under complementation. \\
  & \\
 \hline
  & The class of regular languages over $\Sigma$ is closed under union. \\
  & \\
 \hline
  & The class of regular languages over $\Sigma$ is closed under intersection. \\
  & \\
  \hline
  & The class of regular languages over $\Sigma$ is closed under concatenation. \\
  & \\
 \hline
  & The class of regular languages over $\Sigma$ is closed under Kleene star. \\
  & \\
\hline
    & The class of context-free languages over $\Sigma$ is closed under complementation. \\
  & \\
\hline
    & The class of context-free languages over $\Sigma$ is closed under union. \\
  & \\
\hline
    & The class of context-free languages over $\Sigma$ is closed under intersection. \\
  & \\
\hline
    & The class of context-free languages over $\Sigma$ is closed under concatenation. \\
  & \\
\hline
    & The class of context-free languages over $\Sigma$ is closed under Kleene star. \\
  & \\
\hline
\end{tabular}
\end{center}
\end{comment}

 \vfill
\section*{Week5 friday}


 \vfill
\section*{Week4 monday}


Recap so far: In DFA, the only memory available is in the states. Automata can only
``remember'' finitely far in the past and finitely much information, because
they can have only finitely many states. If a computation path of a DFA visits 
the same state more than once, the machine can't tell the difference between 
the first time and future times it visits this state. Thus, if 
a DFA accepts one long string, then it must accept (infinitely) many 
similar strings.

{\bf Definition}  A positive integer $p$ is a {\bf pumping length} of a language $L$ over $\Sigma$ means
that, for each string $s  \in  \Sigma^*$, if  $|s| \geq p$ and $s \in L$, then there are strings $x,y,z$
such that 
\[
s = xyz
\]
and  
\[
|y| > 0,  \qquad \qquad 
\text{ for each $i \geq 0$, $xy^i z \in L$}, \qquad \text{and}
\qquad  \qquad
|xy| \leq p.
\]

{\bf Negation}: A positive integer  $p$  is {\bf not a pumping length} of a language  $L$ over  $\Sigma$  iff
\[
\exists s \left(~  |s| \geq  p \wedge s \in L \wedge \forall x \forall y \forall z  \left( ~\left( s = xyz \wedge 
|y| > 0 \wedge |xy| \leq p~ \right) \to \exists i  (  i \geq 0  \wedge xy^iz  \notin L ) \right) ~\right) 
\]
{\it Informally: }


Restating {\bf Pumping Lemma}: If $L$ is a regular language, then it  has
a pumping length.


{\bf Contrapositive}: If $L$ has no pumping length, then  it is nonregular.

\vfill

{\Large The Pumping Lemma {\it cannot} be used to prove that a language {\it is} regular.} 

{\Large The Pumping Lemma {\bf can} be used to prove that a language {\it is not} regular.}

{\it Extra practice}: Exercise 1.49 in the book.


\vfill

{\bf Proof strategy}: To prove that a language $L$ is {\bf not} regular, 
\begin{itemize}
    \item Consider an arbitrary positive integer $p$
    \item Prove that $p$ is not a pumping length for $L$
    \item Conclude that $L$ does not have {\it any} pumping length, and therefore it is not regular.
\end{itemize}

\newpage
{\bf Example}: $\Sigma  =  \{0,1\}$, $L = \{ 0^n 1^n \mid n  \geq 0\}$.

Fix $p$ an arbitrary positive integer. List strings that are in $L$ and have length  greater than or equal  to $p$:

\vspace{20pt}

Pick $s = $


Suppose $s = xyz$ with  $|xy|  \leq  p$ and $|y| > 0$.
\begin{center}
\begin{tabular}{|c|}
\hline
 \\
\hspace{4in} \\
\hline
\end{tabular}
\end{center}

Then when $i = \hspace{1in}$, $xy^i z  = \hspace{1in}$

\newpage

{\bf Example}: $\Sigma  =  \{0,1\}$, $L = \{w w^{\mathcal{R}} \mid w \in \{0,1\}^*\}$.
Remember that the reverse of a string $w$ is denoted $w^\mathcal{R}$  
and means to write $w$  in  the opposite order, if $w = w_1 \cdots  w_n$ then $w^\mathcal{R} = w_n \cdots  w_1$. Note: $\varepsilon^\mathcal{R} = \varepsilon$.


Fix $p$ an arbitrary positive integer. List strings that are in $L$ and have length  greater than or equal  to $p$:

\vspace{10pt}

Pick $s = $

Suppose $s = xyz$ with  $|xy|  \leq  p$ and $|y| > 0$.
\begin{center}
\begin{tabular}{|c|}
\hline
 \\
\hspace{4in} \\
\hline
\end{tabular}
\end{center}
Then when $i = \hspace{1in}$, $xy^i z  = \hspace{1in}$


\vspace{30pt} 

{\bf Example}: $\Sigma  =  \{0,1\}$, $L = \{0^j1^k  \mid j \geq k  \geq 0\}$.

Fix $p$ an arbitrary positive integer. List strings that are in $L$ and have length  greater than or equal  to $p$:

\vspace{10pt}

Pick $s = $


Suppose $s = xyz$ with  $|xy|  \leq  p$ and $|y| > 0$.
\begin{center}
\begin{tabular}{|c|}
\hline
 \\
\hspace{4in} \\
\hline
\end{tabular}
\end{center}
Then when $i = \hspace{1in}$, $xy^i z  = \hspace{1in}$



\vspace{30pt} 

{\bf Example}: $\Sigma  =  \{0,1\}$, $L = \{0^n1^m0^n  \mid m,n  \geq 0\}$.

Fix $p$ an arbitrary positive integer. List strings that are in $L$ and have length  greater than or equal  to $p$:

\vspace{10pt}

Pick $s = $


Suppose $s = xyz$ with  $|xy|  \leq  p$ and $|y| > 0$.
\begin{center}
\begin{tabular}{|c|}
\hline
 \\
\hspace{4in} \\
\hline
\end{tabular}
\end{center}
Then when $i = \hspace{1in}$, $xy^i z  = \hspace{1in}$

\newpage
{\it Extra practice}:


\begin{center}
    \begin{tabular}{c|c| c| c}
    Language & $s \in L$ & $s \notin L$ & Is the language regular or nonregular?  \\
    \hline
     & \hspace{1in} & \hspace{1in}  &  \\
    $\{a^nb^n \mid 0  \leq n  \leq 5 \}$ & & & \\
     & & & \\
    $\{b^n a^n \mid  n  \geq 2\}$  & & & \\
     & & & \\
    $\{a^m b^n \mid  0 \leq m\leq n\}$  & & & \\
     & & & \\
    $\{a^m b^n \mid  m \geq n+3,  n \geq 0\}$  & & & \\
     & & & \\
    $\{b^m a^n \mid  m \geq 1, n \geq  3\}$  & & & \\
     & & & \\
    $\{ w  \in \{a,b\}^* \mid w = w^\mathcal{R} \}$ & & & \\
     & & & \\ 
    $\{ ww^\mathcal{R} \mid w\in \{a,b\}^* \}$ & & & \\
     & & & \\ 
    \end{tabular}
\end{center}
     \vfill
\section*{Week4 friday}




Draw the state diagram and give the formal definition of a PDA with $\Sigma = \Gamma$.

\vfill

Draw the state diagram and give the formal definition of a PDA with $\Sigma \cap \Gamma = \emptyset$.
    
\vfill

\newpage
For the PDA state diagrams below, $\Sigma = \{0,1\}$.


\begin{center}
\begin{tabular}{c c}
Mathematical description of language & State diagram of PDA recognizing language\\
\hline
& $\Gamma = \{ \$, \#\}$ \hspace{2.3in} \\
& \\
& \includegraphics[width=3.5in]{Lect10PDA1.png}\\
& \\
& \\
\hline
& $\Gamma = \{ {@}, 1\}$ \hspace{2.3in} \\
& \\
& \includegraphics[width=3.5in]{Lect10PDA2.png}\\
& \\
& \\
\hline
& \\
& \\
& \\
$\{ 0^i 1^j 0^k \mid i,j,k \geq 0 \}$ & \\
& \\
& \\
\end{tabular}
\end{center}

 \vfill
 {\it Note: alternate notation is to replace $;$ with $\to$}

\begin{comment}
{\it Extra practice}: Consider the state diagram of a PDA with input alphabet 
$\Sigma$ and stack alphabet $\Gamma$.

\begin{center}
\begin{tabular}{|c|c|}
\hline
Label & means \\
\hline
$a, b ; c$ when $a \in \Sigma$, $b\in \Gamma$, $c \in \Gamma$ 
& \hspace{3in} \\
& \\
& \\
& \\
& \\
&\\
\hline
$a, \varepsilon ; c$ when $a \in \Sigma$, $c \in \Gamma$ 
& \hspace{3in} \\
& \\
& \\
& \\
& \\
&\\
\hline
$a, b ; \varepsilon$ when $a \in \Sigma$, $b\in \Gamma$
& \hspace{3in} \\
& \\
& \\
& \\
& \\
&\\
\hline
$a, \varepsilon ; \varepsilon$ when $a \in \Sigma$
& \hspace{3in} \\
& \\
& \\
& \\
& \\
&\\
\hline
\end{tabular}
\end{center}


How does the meaning change if $a$ is replaced by $\varepsilon$?
\end{comment}


{\it Big picture}: PDAs were motivated by wanting to add some memory of unbounded size to NFA. How 
do we accomplish a similar enhancement of regular expressions to get a syntactic model that is 
more expressive?

DFA, NFA, PDA: Machines process one input string at a time; the computation of a machine on its input string 
reads the input from left to right.

Regular expressions: Syntactic descriptions of all strings that match a particular pattern; the language 
described by a regular expression is built up recursively according to the expression's syntax

{\bf Context-free grammars}: Rules to produce one string at a time, adding characters from the middle, beginning, 
or end of the final string as the derivation proceeds.

 \vfill
\section*{Week6 monday}


For Turing machine $M= (Q, \Sigma, \Gamma, \delta, q_0, q_{accept}, q_{reject})$ 
where $\delta$ is the {\bf transition function} 
\[
  \delta: Q\times \Gamma \to Q \times \Gamma \times \{L, R\}
\]
the {\bf computation} of $M$ on a string $w$ over $\Sigma$  is:

\begin{itemize}
\setlength{\itemsep}{0pt}
\item Read/write head starts at leftmost position on tape. 
\item Input string is written on $|w|$-many leftmost cells of tape, 
rest of  the tape cells have  the blank symbol. {\bf Tape alphabet} 
is $\Gamma$ with $\textvisiblespace\in \Gamma$ and $\Sigma \subseteq \Gamma$.
The blank symbol $\textvisiblespace \notin \Sigma$.
\item Given current state of machine and current symbol being read at the tape head, 
the machine transitions to next state, writes a symbol to the current position  of the 
tape  head (overwriting existing symbol), and moves the tape head L or R (if possible). 
\item Computation ends {\bf if and when} machine enters either the accept or the reject state.
This is called {\bf halting}.
Note: $q_{accept} \neq q_{reject}$.
\end{itemize}

The {\bf language recognized by the  Turing machine} $M$,  is  $L(M) = \{ w \in \Sigma^* \mid w \textrm{ is accepted by } M\}$,
which is defined as
\[
  \{ w \in \Sigma^* \mid \textrm{computation of $M$ on $w$ halts after entering the accept state}\}
\]
  
\newpage
\begin{multicols}{2}
\includegraphics[width=2.5in]{Lect13TM1.png}

\columnbreak
Formal definition:

\vspace{10pt}

Sample computation: 

\begin{tabular}{|c|c|c|c|c|c|c|}
\hline
\multicolumn{1}{|c}{$q0\downarrow$} &  \multicolumn{6}{c|}{\phantom{A}}\\
\hline
$0$ & $0$  & $0$ & $\textvisiblespace $& $\textvisiblespace $& $\textvisiblespace $&  $\textvisiblespace $\\
\hline
\multicolumn{7}{|c|}{\phantom{A}}\\
\hline
\phantom{AA} & \phantom{AA}& \phantom{AA}& \phantom{AA}& \phantom{AA}& \phantom{AA}& \phantom{AA} \\
\hline
\multicolumn{7}{|c|}{\phantom{A}}\\
\hline
\phantom{AA} & \phantom{AA}& \phantom{AA}& \phantom{AA}& \phantom{AA}& \phantom{AA}& \phantom{AA} \\
\hline
\multicolumn{7}{|c|}{\phantom{A}}\\
\hline
\phantom{AA} & \phantom{AA}& \phantom{AA}& \phantom{AA}& \phantom{AA}& \phantom{AA}& \phantom{AA} \\
\hline
\multicolumn{7}{|c|}{\phantom{A}}\\
\hline
\phantom{AA} & \phantom{AA}& \phantom{AA}& \phantom{AA}& \phantom{AA}& \phantom{AA}& \phantom{AA} \\
\hline
\end{tabular}
\end{multicols}
\vfill

The language recognized by this machine is \ldots

\vfill


To define a Turing machine, we could give a 
\begin{itemize}
\item {\bf Formal definition}, namely the $7$-tuple of parameters including set of states, 
input alphabet, tape alphabet, transition function, start state, accept state, and reject state; or,
\item {\bf Implementation-level definition}: English prose that describes the Turing machine head 
movements relative to contents of tape, and conditions for accepting / rejecting based on those contents.
\end{itemize}

Conventions for drawing state diagrams of Turing machines: (1) omit the reject state from the diagram (unless 
it's the  start state), (2) any missing transitions in the state diagram have value $(q_{reject}, ~\textvisiblespace~ , R)$.
\newpage

{\it Sipser Figure  3.10}
\begin{multicols}{2}
\vspace{-20pt}
\begin{center}
\includegraphics[width=4in]{Lect13TM3.png}
\end{center}

Implementation level description of this machine:
\begin{quote}
Zig-zag across tape to corresponding positions on either side of $\#$ to check whether the 
characters in these positions agree. If they do not, or if there is no $\#$, reject. If they 
do, cross them off.

Once all symbols to the left of the $\#$ are crossed off, check for any un-crossed-off symbols 
to the right of $\#$; if there are any, reject; if there aren't, accept.
\end{quote}

\columnbreak

Computation on  input  string  $01\#01$

\begin{tabular}{|c|c|c|c|c|c|c|}
\hline
\multicolumn{1}{|c}{$q_1 \downarrow$} &  \multicolumn{6}{c|}{\phantom{A}}\\
\hline
$0$ & $1$  & $\#$  & $0$ & $1$ & $\textvisiblespace $& $\textvisiblespace $\\
\hline
  \multicolumn{7}{|c|}{\phantom{A}}\\
  \hline
  \phantom{AA} & \phantom{AA}& \phantom{AA}& \phantom{AA}& \phantom{AA}& \phantom{AA}& \phantom{AA} \\
  \hline
  \multicolumn{7}{|c|}{\phantom{A}}\\
  \hline
  \phantom{AA} & \phantom{AA}& \phantom{AA}& \phantom{AA}& \phantom{AA}& \phantom{AA}& \phantom{AA} \\
  \hline
  \multicolumn{7}{|c|}{\phantom{A}}\\
  \hline
  \phantom{AA} & \phantom{AA}& \phantom{AA}& \phantom{AA}& \phantom{AA}& \phantom{AA}& \phantom{AA} \\
  \hline
  \multicolumn{7}{|c|}{\phantom{A}}\\
  \hline
  \phantom{AA} & \phantom{AA}& \phantom{AA}& \phantom{AA}& \phantom{AA}& \phantom{AA}& \phantom{AA} \\
  \hline
  \multicolumn{7}{|c|}{\phantom{A}}\\
  \hline
  \phantom{AA} & \phantom{AA}& \phantom{AA}& \phantom{AA}& \phantom{AA}& \phantom{AA}& \phantom{AA} \\
  \hline
  \multicolumn{7}{|c|}{\phantom{A}}\\
  \hline
  \phantom{AA} & \phantom{AA}& \phantom{AA}& \phantom{AA}& \phantom{AA}& \phantom{AA}& \phantom{AA} \\
  \hline
  \multicolumn{7}{|c|}{\phantom{A}}\\
  \hline
  \phantom{AA} & \phantom{AA}& \phantom{AA}& \phantom{AA}& \phantom{AA}& \phantom{AA}& \phantom{AA} \\
  \hline
  \multicolumn{7}{|c|}{\phantom{A}}\\
  \hline
  \phantom{AA} & \phantom{AA}& \phantom{AA}& \phantom{AA}& \phantom{AA}& \phantom{AA}& \phantom{AA} \\
  \hline
  \multicolumn{7}{|c|}{\phantom{A}}\\
  \hline
  \phantom{AA} & \phantom{AA}& \phantom{AA}& \phantom{AA}& \phantom{AA}& \phantom{AA}& \phantom{AA} \\
  \hline
  \multicolumn{7}{|c|}{\phantom{A}}\\
  \hline
  \phantom{AA} & \phantom{AA}& \phantom{AA}& \phantom{AA}& \phantom{AA}& \phantom{AA}& \phantom{AA} \\
  \hline
  \multicolumn{7}{|c|}{\phantom{A}}\\
  \hline
  \phantom{AA} & \phantom{AA}& \phantom{AA}& \phantom{AA}& \phantom{AA}& \phantom{AA}& \phantom{AA} \\
  \hline
  \multicolumn{7}{|c|}{\phantom{A}}\\
  \hline
  \phantom{AA} & \phantom{AA}& \phantom{AA}& \phantom{AA}& \phantom{AA}& \phantom{AA}& \phantom{AA} \\
  \hline
  \multicolumn{7}{|c|}{\phantom{A}}\\
  \hline
  \phantom{AA} & \phantom{AA}& \phantom{AA}& \phantom{AA}& \phantom{AA}& \phantom{AA}& \phantom{AA} \\
  \hline
  \multicolumn{7}{|c|}{\phantom{A}}\\
  \hline
  \phantom{AA} & \phantom{AA}& \phantom{AA}& \phantom{AA}& \phantom{AA}& \phantom{AA}& \phantom{AA} \\
  \hline
  \multicolumn{7}{|c|}{\phantom{A}}\\
  \hline
  \phantom{AA} & \phantom{AA}& \phantom{AA}& \phantom{AA}& \phantom{AA}& \phantom{AA}& \phantom{AA} \\
  \hline
  \multicolumn{7}{|c|}{\phantom{A}}\\
  \hline
  \phantom{AA} & \phantom{AA}& \phantom{AA}& \phantom{AA}& \phantom{AA}& \phantom{AA}& \phantom{AA} \\
  \hline
  \multicolumn{7}{|c|}{\phantom{A}}\\
  \hline
  \phantom{AA} & \phantom{AA}& \phantom{AA}& \phantom{AA}& \phantom{AA}& \phantom{AA}& \phantom{AA} \\
  \hline
  \multicolumn{7}{|c|}{\phantom{A}}\\
  \hline
  \phantom{AA} & \phantom{AA}& \phantom{AA}& \phantom{AA}& \phantom{AA}& \phantom{AA}& \phantom{AA} \\
  \hline
  \end{tabular}
\end{multicols}


The language recognized by this machine is
\[
  \{ w \# w \mid w \in \{0,1\}^* \}
\]

\newpage
{\it Extra practice}

Computation on  input  string  $01\#1$

\begin{tabular}{|c|c|c|c|c|c|c|}
\hline
\multicolumn{1}{|c}{$q_1\downarrow$} &  \multicolumn{6}{c|}{\phantom{A}}\\
\hline
$0$ & $1$  & $\#$  & $1$ & $\textvisiblespace $& $\textvisiblespace $&  $\textvisiblespace $\\
\hline
\multicolumn{7}{|c|}{\phantom{A}}\\
\hline
\phantom{AA} & \phantom{AA}& \phantom{AA}& \phantom{AA}& \phantom{AA}& \phantom{AA}& \phantom{AA} \\
\hline
\multicolumn{7}{|c|}{\phantom{A}}\\
\hline
\phantom{AA} & \phantom{AA}& \phantom{AA}& \phantom{AA}& \phantom{AA}& \phantom{AA}& \phantom{AA} \\
\hline
\multicolumn{7}{|c|}{\phantom{A}}\\
\hline
\phantom{AA} & \phantom{AA}& \phantom{AA}& \phantom{AA}& \phantom{AA}& \phantom{AA}& \phantom{AA} \\
\hline
\multicolumn{7}{|c|}{\phantom{A}}\\
\hline
\phantom{AA} & \phantom{AA}& \phantom{AA}& \phantom{AA}& \phantom{AA}& \phantom{AA}& \phantom{AA} \\
\hline
\multicolumn{7}{|c|}{\phantom{A}}\\
\hline
\phantom{AA} & \phantom{AA}& \phantom{AA}& \phantom{AA}& \phantom{AA}& \phantom{AA}& \phantom{AA} \\
\hline
\multicolumn{7}{|c|}{\phantom{A}}\\
\hline
\phantom{AA} & \phantom{AA}& \phantom{AA}& \phantom{AA}& \phantom{AA}& \phantom{AA}& \phantom{AA} \\
\hline
\multicolumn{7}{|c|}{\phantom{A}}\\
\hline
\phantom{AA} & \phantom{AA}& \phantom{AA}& \phantom{AA}& \phantom{AA}& \phantom{AA}& \phantom{AA} \\
\hline
\multicolumn{7}{|c|}{\phantom{A}}\\
\hline
\phantom{AA} & \phantom{AA}& \phantom{AA}& \phantom{AA}& \phantom{AA}& \phantom{AA}& \phantom{AA} \\
\hline
\multicolumn{7}{|c|}{\phantom{A}}\\
\hline
\phantom{AA} & \phantom{AA}& \phantom{AA}& \phantom{AA}& \phantom{AA}& \phantom{AA}& \phantom{AA} \\
\hline
\multicolumn{7}{|c|}{\phantom{A}}\\
\hline
\phantom{AA} & \phantom{AA}& \phantom{AA}& \phantom{AA}& \phantom{AA}& \phantom{AA}& \phantom{AA} \\
\hline
\multicolumn{7}{|c|}{\phantom{A}}\\
\hline
\phantom{AA} & \phantom{AA}& \phantom{AA}& \phantom{AA}& \phantom{AA}& \phantom{AA}& \phantom{AA} \\
\hline
\multicolumn{7}{|c|}{\phantom{A}}\\
\hline
\phantom{AA} & \phantom{AA}& \phantom{AA}& \phantom{AA}& \phantom{AA}& \phantom{AA}& \phantom{AA} \\
\hline
\multicolumn{7}{|c|}{\phantom{A}}\\
\hline
\phantom{AA} & \phantom{AA}& \phantom{AA}& \phantom{AA}& \phantom{AA}& \phantom{AA}& \phantom{AA} \\
\hline
\multicolumn{7}{|c|}{\phantom{A}}\\
\hline
\phantom{AA} & \phantom{AA}& \phantom{AA}& \phantom{AA}& \phantom{AA}& \phantom{AA}& \phantom{AA} \\
\hline
\multicolumn{7}{|c|}{\phantom{A}}\\
\hline
\phantom{AA} & \phantom{AA}& \phantom{AA}& \phantom{AA}& \phantom{AA}& \phantom{AA}& \phantom{AA} \\
\hline
\multicolumn{7}{|c|}{\phantom{A}}\\
\hline
\phantom{AA} & \phantom{AA}& \phantom{AA}& \phantom{AA}& \phantom{AA}& \phantom{AA}& \phantom{AA} \\
\hline
\multicolumn{7}{|c|}{\phantom{A}}\\
\hline
\phantom{AA} & \phantom{AA}& \phantom{AA}& \phantom{AA}& \phantom{AA}& \phantom{AA}& \phantom{AA} \\
\hline
\multicolumn{7}{|c|}{\phantom{A}}\\
\hline
\phantom{AA} & \phantom{AA}& \phantom{AA}& \phantom{AA}& \phantom{AA}& \phantom{AA}& \phantom{AA} \\
\hline
\multicolumn{7}{|c|}{\phantom{A}}\\
\hline
\phantom{AA} & \phantom{AA}& \phantom{AA}& \phantom{AA}& \phantom{AA}& \phantom{AA}& \phantom{AA} \\
\hline
\multicolumn{7}{|c|}{\phantom{A}}\\
\hline
\phantom{AA} & \phantom{AA}& \phantom{AA}& \phantom{AA}& \phantom{AA}& \phantom{AA}& \phantom{AA} \\
\hline
\end{tabular}

 \vfill
\section*{Week6 wednesday}



  
  Fix $\Sigma = \{0,1\}$, $\Gamma = \{ 0, 1, \textvisiblespace\}$ for the Turing machines with  the following state diagrams:
  
  \begin{center}
  \begin{tabular}{|c|c|}
  \hline
  \hspace{0.8in}\includegraphics[width=2in]{Lect14TM1.png} \phantom{\hspace{0.8in}}&\hspace{0.8in} \includegraphics[width=2in]{Lect14TM2.png} \phantom{\hspace{0.8in}}\\
  Implementation  level description:  \phantom{\hspace{1in}} &Implementation  level description:  \phantom{\hspace{1in}} \\
  &\\
  &\\
  &\\
  Example of string accepted: \phantom{\hspace{1.5in}}& Example of string accepted: \phantom{\hspace{1.5in}}\\
  Example of string rejected: \phantom{\hspace{1.5in}}& Example of string  rejected: \phantom{\hspace{1.5in}}\\
  & \\
  \hline
  \includegraphics[width=2in]{Lect14TM3.png} & \includegraphics[width=2in]{Lect14TM4.png} \\
  Implementation  level description:  \phantom{\hspace{1in}} &Implementation  level description:  \phantom{\hspace{1in}} \\
  &\\
  &\\
  &\\
  Example of string accepted: \phantom{\hspace{1.5in}}& Example of string accepted: \phantom{\hspace{1.5in}}\\
  Example of string rejected: \phantom{\hspace{1.5in}}& Example of string  rejected: \phantom{\hspace{1.5in}}\\
  & \\
  
  \hline
  \end{tabular}
  \end{center}

Two models of computation are called {\bf equally expressive} when 
every language recognizable with the first model is recognizable with the second, and vice versa.

True / False: NFAs and PDAs are equally expressive.

True / False: Regular expressions and CFGs are equally expressive.


To say a language is {\bf Turing-recognizable} means that there is some Turing machine that recognizes it.

\begin{center}
{\large \it  Some examples of models that are {\bf equally expressive} with deterministic Turing machines: }
\end{center}

\newpage
\fbox{ {\bf May-stay}  machines }
The May-stay machine model is the same as the usual Turing machine model,  except that
on each transition, the tape head may move L, move R, or Stay. 

Formally: $(Q, \Sigma, \Gamma, \delta, q_0, q_{accept}, q_{reject})$ where 
\[
  \delta: Q \times \Gamma \to Q \times \Gamma \times \{L, R, S\}
\]

{\bf Claim}: Turing machines and May-stay machines are equally expressive. {\it To prove \ldots}

To translate a standard TM to a may-stay machine: 

\vspace{100pt}




To translate one  of the  may-stay machines to standard TM:
any time TM would Stay, move right  then  left.


Formally: suppose $M_S =  (Q, \Sigma, \Gamma, \delta, q_0, q_{acc}, q_{rej})$
has $\delta: Q \times \Gamma \to Q \times \Gamma \times \{L, R, S\}$. Define
the Turing-machine
\[
  M_{new} =  (\phantom{\hspace{2.5in}})
\]

\vfill


\phantom{$M_{new}$ construction here \vspace{400pt}}
\vfill

\newpage

\fbox{ {\bf Multitape Turing machine}} A multitape Turing macihne with $k$ tapes
can be formally representated as 
$(Q, \Sigma,  \Gamma, \delta, q_0, q_{acc}, q_{rej})$ 
where $Q$ is the finite set of  states,
$\Sigma$ is the  input alphabet with  $\textvisiblespace \notin \Sigma$,
$\Gamma$  is the  tape alphabet with $\Sigma \subsetneq \Gamma$ ,
$\delta: Q\times \Gamma^k\to Q \times \Gamma^k \times \{L,R\}^k$ 
(where $k$ is  the number of  states)


If $M$ is a standard  TM, it is a $1$-tape machine.


To translate a $k$-tape machine  to  a standard TM:
Use a  new symbol to separate the contents of each tape
and keep track of location of  head with  special version of each
tape symbol. {\tiny Sipser Theorem 3.13} 

\includegraphics[width=2.5in]{resources/images/Figure314.png}

\vfill

{\it Extra practice:} \fbox{ {\bf  Wikipedia Turing machine} }
Define a machine $(Q, \Gamma, b, \Sigma,  q_0, F, \delta)$
where $Q$ is the finite set  of  states
$\Gamma$  is the tape alphabet,
$b \in \Gamma$ is the blank symbol, 
$\Sigma \subsetneq \Gamma$ is the  input alphabet, 
$q_0 \in  Q$ is the start state, 
$F \subseteq Q$ is the set of accept states, 
$\delta: (Q \setminus F)  \times  \Gamma \not\to Q \times  \Gamma  \times \{L, R\}$
 is a partial transition function
If computation enters a state  in $F$, it  accepts 
If computation enters a configuration where
 $\delta$ is not defined, it  rejects . {\tiny Hopcroft and  Ullman, cited by  Wikipedia} 

\vfill

\newpage
\fbox{ {\bf Enumerators} } Enumerators give a different
model of computation where a language is {\bf produced, one string at a time},
rather than recognized by accepting (or not) individual strings.

Each enumerator machine has finite state control, unlimited work tape, and a printer. The computation proceeds
according to transition function; at any point machine may ``send'' a string to the printer.
\[
E  = (Q, \Sigma, \Gamma, \delta, q_0, q_{print})  
\]
$Q$ is the finite set of states, $\Sigma$ is  the output alphabet, $\Gamma$ is the 
tape alphabet ($\Sigma  \subsetneq\Gamma, 
\textvisiblespace \in \Gamma \setminus \Sigma$), 
\[
\delta:  Q  \times  \Gamma \times \Gamma \to  Q \times  \Gamma \times  \Gamma \times \{L, R\} \times  \{L, R\}
\]
where in state $q$, when the working tape is scanning character $x$ and the printer tape is scanning character $y$,
$\delta( (q,x,y) ) = (q', x', y', d_w, d_p)$ means transition to control state $q'$, write $x'$ on 
the working tape, write $y'$ on the printer tape, move in direction $d_w$ on the working tape, and move in direction 
$d_p$ on the printer tape. The computation starts in $q_0$ and each time the computation enters $q_{print}$
the string from the leftmost edge of the printer tape to the first blank cell is considered to be printed.

The language  {\bf  enumerated} by  $E$, $L(E)$, is $\{ w \in \Sigma^* \mid \text{$E$ eventually, at finite  time, 
prints $w$} \}$.


\begin{center}
\begin{tabular}{cc}
\includegraphics[width=3.5in]{Lec15enumerator.png}  & 
\begin{tabular}{|c|c|c|c|c|c|c|}
\hline
\multicolumn{1}{|c}{$q0$} &  \multicolumn{6}{c|}{\phantom{A}}\\
\hline
$\textvisiblespace ~*$& $\textvisiblespace$  & $\textvisiblespace$ & $\textvisiblespace$& $\textvisiblespace$& $\textvisiblespace$&  $\textvisiblespace$\\
\hline
$\textvisiblespace  ~*$& $\textvisiblespace$  & $\textvisiblespace$ & $\textvisiblespace$& $\textvisiblespace$& $\textvisiblespace$&  $\textvisiblespace$\\
\hline\hline
\multicolumn{7}{|c|}{\phantom{A}}\\
\hline
\phantom{AA} & \phantom{AA}& \phantom{AA}& \phantom{AA}& \phantom{AA}& \phantom{AA}& \phantom{AA} \\
\hline
\phantom{AA} & \phantom{AA}& \phantom{AA}& \phantom{AA}& \phantom{AA}& \phantom{AA}& \phantom{AA} \\
\hline
\hline
\multicolumn{7}{|c|}{\phantom{A}}\\
\hline
\phantom{AA} & \phantom{AA}& \phantom{AA}& \phantom{AA}& \phantom{AA}& \phantom{AA}& \phantom{AA} \\
\hline
\phantom{AA} & \phantom{AA}& \phantom{AA}& \phantom{AA}& \phantom{AA}& \phantom{AA}& \phantom{AA} \\
\hline
\hline
\multicolumn{7}{|c|}{\phantom{A}}\\
\hline
\phantom{AA} & \phantom{AA}& \phantom{AA}& \phantom{AA}& \phantom{AA}& \phantom{AA}& \phantom{AA} \\
\hline
\phantom{AA} & \phantom{AA}& \phantom{AA}& \phantom{AA}& \phantom{AA}& \phantom{AA}& \phantom{AA} \\
\hline
\hline
\multicolumn{7}{|c|}{\phantom{A}}\\
\hline
\phantom{AA} & \phantom{AA}& \phantom{AA}& \phantom{AA}& \phantom{AA}& \phantom{AA}& \phantom{AA} \\
\hline
\phantom{AA} & \phantom{AA}& \phantom{AA}& \phantom{AA}& \phantom{AA}& \phantom{AA}& \phantom{AA} \\
\hline
\end{tabular}
\end{tabular}
\end{center}


\newpage

{\bf Theorem 3.21} A language is Turing-recognizable iff some enumerator enumerates it.

{\bf Proof}:

Assume $L$ is enumerated by some enumerator, $E$, so $L = L(E)$.
We'll use $E$ in a subroutine
within a high-level description of a new Turing machine that we will build to recognize $L$.

{\bf Goal}: build Turing machine $M_E$ with $L(M_E) = L(E)$.

Define $M_E$ as follows: $M_E = $ ``On input $w$,
\begin{enumerate}
\item Run $E$. For each string $x$ printed by $E$.
\item \qquad Check if $x = w$. If so, accept (and halt); otherwise, continue."
\end{enumerate}


\vfill 



Assume $L$ is Turing-recognizable and there 
is a Turing  machine  $M$ with  $L = L(M)$. We'll use $M$ in a subroutine
within a high-level description of an enumerator that we will build to enumerate $L$.

{\bf Goal}: build enumerator $E_M$ with $L(E_M) = L(M)$.

{\bf Idea}: check each string in turn to see if it is in $L$.

{\it How?} Run computation of $M$ on each string.  {\it But}: need to be careful 
about computations that don't halt.

{\it Recall} String order for $\Sigma = \{0,1\}$: $s_1 = \varepsilon$, $s_2 = 0$, $s_3 = 1$, $s_4 = 00$, $s_5 = 01$, $s_6  = 10$, 
$s_7  =  11$, $s_8 = 000$, \ldots

Define $E_M$ as follows: $E_{M} = $ `` {\it ignore any input.} Repeat the following for $i=1, 2, 3, \ldots$
\begin{enumerate}
  \item Run the computations of $M$ on $s_1$, $s_2$, \ldots, $s_i$ for (at most) $i$ steps each
  \item For each of these $i$ computations that accept during the (at most) $i$ steps, print
  out the accepted string."
\end{enumerate}
 \vfill
\section*{Week6 friday}


{\bf Nondeterministic Turing machine}

At any point in the computation, the nondeterministic machine may proceed according to 
several possibilities: $(Q, \Sigma, \Gamma, \delta, q_0, q_{acc}, q_{rej})$ where 
\[
\delta: Q \times \Gamma \to \mathcal{P}(Q \times \Gamma \times \{L, R\})  
\]
The computation of a nondeterministic Turing machine is a tree with branching
when the next step of the computation has multiple possibilities. A nondeterministic
Turing machine accepts a string exactly when some branch of the computation tree 
enters the accept state.

Given a nondeterministic machine, we can use a $3$-tape Turing machine to 
simulate it by doing a breadth-first search of computation tree: one tape 
is ``read-only'' input tape, one tape simulates the tape of the nondeterministic
computation, and one tape tracks nondeterministic branching. {\tiny Sipser page 178} 

\vfill
Two models of computation are called {\bf equally expressive} when 
every language recognizable with the first model is recognizable with the second, and vice versa.

{\bf  Church-Turing Thesis} (Sipser p. 183): The informal notion of algorithm is formalized completely  and correctly by the 
formal definition of a  Turing machine. In other words: all reasonably expressive models of 
computation are equally expressive with the standard Turing machine.

\vfill

\newpage


A language $L$ is {\bf recognized by} a Turing machine $M$ means

\vspace{15pt}

A Turing  machine  $M$ {\bf  recognizes} a language $L$ if means

\vspace{15pt}

A Turing machine $M$ is a {\bf decider}  means

\vspace{15pt}

A language  $L$ is {\bf decided by} a Turing  machine  $M$  means

\vspace{15pt}

A  Turing machine $M$ {\bf decides} a language $L$ means

\vspace{15pt}

Fix $\Sigma = \{0,1\}$, $\Gamma = \{ 0, 1, \textvisiblespace\}$ for the Turing machines with  the following state diagrams:
  
  \begin{center}
  \begin{tabular}{|c|c|}
  \hline
  \hspace{0.8in}\includegraphics[width=2in]{Lect14TM1.png} \phantom{\hspace{0.8in}}&\hspace{0.8in} \includegraphics[width=2in]{Lect14TM2.png} \phantom{\hspace{0.8in}}\\
  Decider? Yes~~~/ ~~~No
  &Decider? Yes~~~/ ~~~No\\
  & \\
  \hline
  \includegraphics[width=2in]{Lect14TM3.png} & \includegraphics[width=2in]{Lect14TM4.png} \\
  Decider? Yes~~~/ ~~~No
  &Decider? Yes~~~/ ~~~No\\
  & \\
  
  \hline
  \end{tabular}
  \end{center}
  \newpage



{\bf Claim}: If two languages  (over a fixed alphabet  $\Sigma$) are Turing-recognizable, then  their union  is  as well.

{\bf Proof using Turing machines}:

\vfill

{\bf Proof using nondeterministic Turing machines}: 

\vfill  

{\bf  Proof using enumerators}:

\vfill

\newpage
    
{\bf Describing  Turing machines} (Sipser p. 185)


To define a Turing machine, we could give a 
\begin{itemize}
\item {\bf Formal definition}: the $7$-tuple of parameters including set of states, 
input alphabet, tape alphabet, transition function, start state, accept state, and reject state; or,
\item {\bf Implementation-level definition}: English prose that describes the Turing machine head 
movements relative to contents of tape, and conditions for accepting / rejecting based on those contents.
\item {\bf High-level description}: description of algorithm (precise sequence of instructions), 
without implementation details of machine. As part of this description, can ``call" and run 
another TM as a subroutine.
\end{itemize}


The Church-Turing thesis posits that each algorithm can be implemented by some Turing machine

High-level descriptions of  Turing machine algorithms are written as indented text within quotation marks.   

Stages of the algorithm are typically numbered consecutively.

The first line specifies the input to the machine, which must be a string.
This string may be the encoding of some object or  list of  objects.  

{\bf Notation:} $\langle O \rangle$ is the string that encodes the object $O$.
$\langle O_1, \ldots, O_n \rangle$ is the string that encodes the list of objects $O_1, \ldots, O_n$.

{\bf Assumption}: There are Turing  machines that can be called as subroutines
to decode the string representations of common objects and  interact with these objects as intended
(data structures).
  
For example, since there are algorithms to answer each of the following questions,
by Church-Turing thesis, there is a Turing machine that accepts exactly those strings for which the 
answer to the question is ``yes''
\begin{itemize}
    \item Does a string over $\{0,1\}$ have even length?

    \vfill

    \item Does a string over $\{0,1\}$ encode a string of ASCII characters?\footnote{An introduction to ASCII 
    is available on the w3 tutorial \href{https://www.w3schools.com/charsets/ref_html_ascii.asp}{here}.}

    \vfill

    \item Does a DFA have a specific number of states?

    \vfill

    \item Do two NFAs have any state names in common?

    \vfill

    \item Do two CFGs have the same start variable?

    \vfill

  \end{itemize}
 \vfill
\section*{Week7 monday}



\begin{center}
    \begin{tabular}{|l|l|l|l|}
    \hline
    & Suppose $M$ is  a TM & Suppose $D$ is  a TM & Suppose $E$ is  an
    enumerator  \\
    &that  recognizes $L$  &that  decides $L$  &that enumerates $L$ \\
    \hline
    If string $w$ is in  $L$ then  \ldots  &&& \\
    &&&\\
    &&&\\
    If string $w$ is not in  $L$ then  \ldots  && &\\
    &&&\\
    &&&\\
    \hline
    \end{tabular}
\end{center}


A language $L$ is {\bf recognized by} a Turing machine $M$ means

\vspace{15pt}

A Turing  machine  $M$ {\bf  recognizes} a language $L$ if means

\vspace{15pt}

A Turing machine $M$ is a {\bf decider}  means

\vspace{15pt}

A language  $L$ is {\bf decided by} a Turing  machine  $M$  means

\vspace{15pt}

A  Turing machine $M$ {\bf decides} a language $L$ means

\vspace{15pt}


{\it From Friday's review quiz: }
Which of the following sentences make sense? Which of those are true?

A language is a decider if it always halts.

\vfill

The union of two deciders is a decider.

\vfill

A language is decidable if and only if it is recognizable.

\vfill

There is a Turing machine that isn't decidable.

\vfill

There is a recognizable language that isn't decided by any Turing machine.

\vfill

\newpage


{\bf Claim}: If two languages  (over a fixed alphabet  $\Sigma$) are Turing-recognizable, then  their union  is  as well.

{\bf Proof using Turing machines}:

\vfill

{\bf Proof using nondeterministic Turing machines}: 

\vfill  

{\bf  Proof using enumerators}:

\vfill

\newpage


The first line of a {\bf high-level description} of a Turing machine specifies the input to the machine, which must be a string.
This string may be the encoding of some object or  list of  objects.  

{\bf Notation:} $\langle O \rangle$ is the string that encodes the object $O$.
$\langle O_1, \ldots, O_n \rangle$ is the string that encodes the list of objects $O_1, \ldots, O_n$.

{\bf Assumption}: There are Turing  machines that can be called as subroutines
to decode the string representations of common objects and  interact with these objects as intended
(data structures).
  
For example, since there are algorithms to answer each of the following questions,
by Church-Turing thesis, there is a Turing machine that accepts exactly those strings for which the 
answer to the question is ``yes''
\begin{itemize}
    \item Does a string over $\{0,1\}$ have even length?

    \vfill

    \item Does a string over $\{0,1\}$ encode a string of ASCII characters?\footnote{An introduction to ASCII 
    is available on the w3 tutorial \href{https://www.w3schools.com/charsets/ref_html_ascii.asp}{here}.}

    \vfill

    \item Does a DFA have a specific number of states?

    \vfill

    \item Do two NFAs have any state names in common?

    \vfill

    \item Do two CFGs have the same start variable?

    \vfill

  \end{itemize}

A {\bf computational problem} is decidable iff language encoding its positive problem instances
is decidable.

The computational problem ``Does a specific DFA accept a given string?'' is encoded by the language
\begin{align*}
  &\{ \textrm{representations of DFAs $M$ and strings $w$ such that $w \in L(M)$}\}  \\
  =& \{ \langle M, w \rangle \mid M \textrm{ is a DFA}, w \textrm{ is a string}, w \in L(M) \}
\end{align*}

The computational problem ``Is the language generated by a CFG empty?'' is encoded by the language
\begin{align*}
  &\{ \textrm{representations of CFGs $G$  such that $L(G) = \emptyset$}\}  \\
  =& \{ \langle G \rangle \mid G \textrm{ is a CFG},  L(G) = \emptyset \}
\end{align*}



The computational problem ``Is the given Turing machine a decider?'' is encoded by the language
\begin{align*}
  &\{ \textrm{representations of TMs $M$  such that $M$ halts on every input}\}  \\
  =& \{ \langle M \rangle \mid M \textrm{ is a TM and for each string } w, \textrm{$M$ halts on $w$} \}
\end{align*}


{\it Note: writing down the language encoding a computational problem is only the first step in 
determining if it's recognizable, decidable, or \ldots }
 \vfill
\section*{Week7 wednesday}


Deciding a computational problem means building / defining a Turing 
machine that recognizes the language encoding the computational problem, and that 
is a decider.


{\bf Some classes of computational problems help us understand the differences between the machine models we've been studying:}


    \begin{center}
    \begin{tabular}{|lcl|}
    \hline
    \multicolumn{3}{|l|}{{\bf  Acceptance problem} } \\
    & & \\
    \ldots for DFA & $A_{DFA}$ & $\{ \langle B,w \rangle \mid  \text{$B$ is a  DFA that accepts input 
    string $w$}\}$ \\
    \ldots for NFA & $A_{NFA}$ & $\{ \langle B,w \rangle \mid  \text{$B$ is a  NFA that accepts input 
    string $w$}\}$ \\
    \ldots for regular expressions & $A_{REX}$ & $\{ \langle R,w \rangle \mid  \text{$R$ is a  regular
    expression that generates input string $w$}\}$ \\
    \ldots for CFG & $A_{CFG}$ & $\{ \langle G,w \rangle \mid  \text{$G$ is a context-free grammar 
    that generates input string $w$}\}$ \\
    \ldots for PDA & $A_{PDA}$ & $\{ \langle B,w \rangle \mid  \text{$B$ is a PDA that accepts input string $w$}\}$ \\
    & & \\
    & & \\
    \hline
    \multicolumn{3}{|l|}{{\bf Language emptiness  testing} } \\
    & & \\
    \ldots for DFA & $E_{DFA}$ & $\{ \langle A \rangle \mid  \text{$A$ is a  DFA and  $L(A) = \emptyset$\}}$ \\
    \ldots for NFA & $E_{NFA}$ & $\{ \langle A\rangle \mid  \text{$A$ is a NFA and  $L(A) = \emptyset$\}}$ \\
    \ldots for regular expressions & $E_{REX}$ & $\{ \langle R \rangle \mid  \text{$R$ is a  regular
    expression and  $L(R) = \emptyset$\}}$ \\
    \ldots for CFG & $E_{CFG}$ & $\{ \langle G \rangle \mid  \text{$G$ is a context-free grammar 
    and  $L(G) = \emptyset$\}}$ \\
    \ldots for PDA & $E_{PDA}$ & $\{ \langle A \rangle \mid  \text{$A$ is a PDA and  $L(A) = \emptyset$\}}$ \\
    & & \\
    & & \\
    \hline
    \multicolumn{3}{|l|}{{\bf Language equality testing} } \\
    & & \\
    \ldots for DFA & $EQ_{DFA}$ & $\{ \langle A, B \rangle \mid  \text{$A$ and $B$ are DFAs and  $L(A) =L(B)$\}}$\\
    \ldots for NFA & $EQ_{NFA}$ & $\{ \langle A, B \rangle \mid  \text{$A$ and $B$ are NFAs and  $L(A) =L(B)$\}}$\\
    \ldots for regular expressions & $EQ_{REX}$ & $\{ \langle R, R' \rangle \mid  \text{$R$ and $R'$ are regular
    expressions and  $L(R) =L(R')$\}}$\\
    \ldots for CFG & $EQ_{CFG}$ & $\{ \langle G, G' \rangle \mid  \text{$G$ and $G'$ are CFGs and  $L(G) =L(G')$\}}$ \\
    \ldots for PDA & $EQ_{PDA}$ & $\{ \langle A, B \rangle \mid  \text{$A$ and $B$ are PDAs and  $L(A) =L(B)$\}}$ \\
    \hline
    Sipser Section 4.1 &&\\
    \hline
    \end{tabular}
    \end{center}
    
    
    
    \newpage
    
    \begin{center}
    \begin{tabular}{|c|c|c|}
    \hline
    $M_1$  \includegraphics[width=2in]{Lect17DFA1.png} &  
    $M_2$ \includegraphics[width=2in]{Lect17DFA2.png} &  
    $M_3$ \includegraphics[width=2in]{Lect17DFA3.png} \\ 
    && \\
    && \\
    && \\
    && \\
    \hline
    \end{tabular}
    \end{center}
    
    Example strings in $A_{DFA}$
    
    \vfill
    
    Example strings in  $E_{DFA}$
    
    \vfill
    
    Example strings in  $EQ_{DFA}$
    
    \vfill

  
  \begin{quote}
  $M_1 = $ ``On input $\langle M,w\rangle$, where $M$ is a DFA and $w$ is a string:
  \begin{enumerate}
  \setcounter{enumi}{-1}
  \item Type check encoding to check input is correct type.
  \item Simulate $M$ on input $w$ (by keeping track of states in $M$, transition function of $M$, etc.) 
  \item If the simulations ends in an accept state of $M$, accept. If it ends in a non-accept state of $M$, reject. "
  \end{enumerate}
  \end{quote}
  

What is $L(M_1)$? 

\vfill

Is $M_1$ a decider?

\vfill

  
  \begin{quote}
  $M_2 =  $``On  input  $\langle M, w \rangle$ where $M$ is a  DFA and  $w$ is  a string, 
  \begin{enumerate}
  \item Run $M$ on  input  $w$.
  \item If $M$  accepts, accept; if $M$ rejects, reject."
  \end{enumerate}
  \end{quote}
  

  What is $L(M_2)$? 

  \vfill
  
  Is $M_2$ a decider?
  
  \vfill
  
    
\newpage
  $A_{REX} = $

  $A_{NFA} = $


  True / False: $A_{REX} = A_{NFA} = A_{DFA}$

  True / False: $A_{REX} \cap A_{NFA} = \emptyset$, $A_{REX} \cap A_{DFA} = \emptyset$, $A_{DFA} \cap A_{NFA} = \emptyset$

  
  A Turing machine that  decides $A_{NFA}$ is: 
  
  \vfill
  
  A Turing machine that  decides $A_{REX}$ is: 
  
  \vfill
  \newpage
  
  \begin{quote}
  $M_3 =  $``On  input  $\langle M\rangle$ where $M$ is a  DFA,
  \begin{enumerate}
  \item For integer  $i = 1, 2, \ldots$
  \item \qquad Let  $s_i$ be the  $i$th string over  the alphabet of  $M$ (ordered in  string order).
  \item \qquad Run $M$ on  input  $s_i$.
  \item \qquad If $M$  accepts,  $\underline{\phantom{FILL  IN BLANK}}$.  If $M$  rejects, increment $i$ and keep going."
  \end{enumerate}
  \end{quote}
  

Choose the correct option to help fill in the blank so that $M_3$ recognizes $E_{DFA}$
\begin{itemize}
\item[A.] accepts
\item[B.] rejects
\item[C.] loop for ever
\item[D.] We can't fill in the blank in any way to make this work
\item[E.] None of the above
\end{itemize}

  
  \begin{quote}
  $M_4 =  $ `` On  input $\langle M \rangle$ where $M$ is  a  DFA,
  \begin{enumerate}
  \item Mark the start  state  of $M$.
  \item Repeat until no  new states get marked:
  \item \qquad Loop over the states of $M$. 
  \item \qquad Mark any unmarked  state  that  has an incoming  edge  from a marked state.
  \item If  no  accept state of $A$ is  marked, $\underline{\phantom{FILL  IN BLANK}}$;  otherwise, 
  $\underline{\phantom{FILL  IN BLANK}}$".
  \end{enumerate}
  \end{quote}
  
  
  
To build a Turing machine that decides $EQ_{DFA}$, notice that 
\[
L_1 = L_2 \qquad\textrm{iff}\qquad (~(L_1 \cap \overline{L_2}) \cup (L_2 \cap \overline L_1)~) = \emptyset  
\]
{\it There are no elements that are in one set and not the other}


$M_{EQDFA} = $ 


  \vfill
  

  {\bf Summary}:  We can use the decision procedures (Turing machines) of decidable problems
  as subroutines in other algorithms. For example, we have subroutines for deciding each of 
  $A_{DFA}$, $E_{DFA}$, $EQ_{DFA}$.  We can also use algorithms for known constructions
  as subroutines in other algorithms. For example, we have subroutines for: counting the number 
  of states in a state diagram, counting the number of characters in an alphabet, converting DFA
  to a DFA recognizing the complement of the original language or a DFA recognizing the 
  Kleene star of the original language, constructing a DFA or NFA from two DFA or NFA so that 
  we have a machine recognizing the language of the union (or intersection, concatenation)
  of the languages of the original machines; converting regular expressions to equivalent DFA; 
  converting DFA to equivalent regular expressions, etc.

\newpage \vfill
\section*{Week7 friday}


\begin{center}
    \begin{tabular}{|lcl|}
    \hline
    \multicolumn{3}{|l|}{{\bf  Acceptance problem} } \\
    & & \\
    \ldots for DFA & $A_{DFA}$ & $\{ \langle B,w \rangle \mid  \text{$B$ is a  DFA that accepts input 
    string $w$}\}$ \\
    \ldots for NFA & $A_{NFA}$ & $\{ \langle B,w \rangle \mid  \text{$B$ is a  NFA that accepts input 
    string $w$}\}$ \\
    \ldots for regular expressions & $A_{REX}$ & $\{ \langle R,w \rangle \mid  \text{$R$ is a  regular
    expression that generates input string $w$}\}$ \\
    \ldots for CFG & $A_{CFG}$ & $\{ \langle G,w \rangle \mid  \text{$G$ is a context-free grammar 
    that generates input string $w$}\}$ \\
    \ldots for PDA & $A_{PDA}$ & $\{ \langle B,w \rangle \mid  \text{$B$ is a PDA that accepts input string $w$}\}$ \\
    & & \\
    & & \\
    \hline
    \end{tabular}
\end{center}

\begin{center}
\begin{tabular}{|lcl|}
\hline
\multicolumn{3}{|l|}{{\bf  Acceptance problem} } \\
for Turing  machines  & $A_{TM}$ & $\{ \langle M,w \rangle \mid  \text{$M$ is a Turing machine that accepts input 
string $w$}\}$ \\
\hline
\multicolumn{3}{|l|}{{\bf Language emptiness  testing} } \\
 for Turing machines & $E_{TM}$ & $\{ \langle M \rangle \mid  \text{$M$ is a Turing machine and  $L(M) = \emptyset$\}}$ \\
\hline
\multicolumn{3}{|l|}{{\bf Language equality testing} } \\
 for Turing machines& $EQ_{TM}$ & $\{ \langle  M_1, M_2 \rangle \mid  \text{$M_1$ and $M_2$ are Turing machines and  
 $L(M_1) =L(M_2)$\}}$\\
\hline
Sipser Section 4.1 &&\\
\hline
\end{tabular}
\end{center}

\begin{multicols}{3}
$M_1$ \includegraphics[width=2in]{Week8WarmupTM1.png} 

\columnbreak

$M_2$ \includegraphics[width=2in]{Week8WarmupTM2.png}

\columnbreak

$M_3$ \includegraphics[width=2in]{Week8WarmupTM3.png}
\end{multicols}

Example strings in $A_{TM}$

\vfill

Example strings in  $E_{TM}$

\vfill

Example strings in  $EQ_{TM}$

\vfill

\newpage

{\bf  Theorem}: $A_{TM}$  is  Turing-recognizable.


{\bf  Strategy}:  To prove this theorem, we need  to  define  a Turing  machine  $R_{ATM}$ such that 
$L(R_{ATM}) = A_{TM}$.


Define $R_{ATM} =  $ ``

\vspace{150pt}


Proof of correctness: 


\vfill
\vfill

We will show that $A_{TM}$ is undecidable.   {\it First, let's explore what that means.}

\newpage

A {\bf Turing-recognizable} language is a set of strings that 
is the language recognized by some Turing machine. We also 
say that such languages are recognizable.

A {\bf Turing-decidable} language is a set of strings that 
is the language recognized by some decider. We also 
say that such languages are decidable.

An {\bf unrecognizable} language is a language that is not Turing-recognizable.

An {\bf undecidable} language is a language that is not Turing-decidable.


{\bf  True} or {\bf False}: Any  undecidable language  is  also  unrecognizable.


{\bf  True} or {\bf False}: Any  unrecognizable language  is  also  undecidable.


To prove that a computational problem is {\bf decidable}, we find/ build a Turing 
machine that recognizes the language encoding the computational problem, and that 
is a decider.


How do we prove a specific problem is {\bf not decidable}?

How would we even find such a computational problem?


{\it Counting arguments for the existence of an undecidable language:}
\begin{itemize}
    \item The set of all Turing machines is countably infinite.
    \item Each Turing-recognizable language is associated with a Turing machine
    in a one-to-one relationship, so there can be no more Turing-recognizable
    languages than there are Turing machines. 
    \item Since there are infinitely many Turing-recognizable languages
    (think of the singleton sets), there are countably infinitely 
    many Turing-recognizable languages.
    \item Such the set of Turing-decidable languages is an infinite subset 
    of the set of Turing-recognizable languages, the set of 
    Turing-decidable languages is also countably infinite.
\end{itemize}

Since there are uncountably many languages (because $\mathcal{P}(\Sigma^*)$
is uncountable), there are uncountably many unrecognizable languages
and there are uncountably many undecidable languages.


Thus, there's at least one undecidable language!

\vfill

{\bf What's a specific example of a language that is unrecognizable or undecidable?}

To prove that a language is undecidable, we need to prove that there is no Turing machine that decides it.

{\bf Key idea}: proof by contradiction relying on self-referential disagreement.

 \vfill
\section*{Week3 friday}


{\bf Definition and Theorem}: For an alphabet $\Sigma$, a language $L$ over $\Sigma$ is called {\bf regular}
exactly when $L$ is recognized by some DFA, which happens exactly when $L$ is recognized by some NFA, 
and happens exactly when $L$ is described by some regular expression

{\bf We saw that}: The class of regular languages is closed under complementation, union, 
intersection, set-wise concatenation, and Kleene star.

{\bf Prove or Disprove}: There is some alphabet $\Sigma$ for which there is 
some language recognized by an NFA but not by any DFA.

\vfill

{\bf Prove or Disprove}: There is some alphabet $\Sigma$ for which there is 
some finite language not described by any regular expression over $\Sigma$.

\vfill

{\bf Prove or Disprove}: If a language is recognized by an NFA 
then the complement of this language is not recognized by any DFA.

\vfill

\newpage

{\bf Fix alphabet $\Sigma$. Is every language $L$ over $\Sigma$ regular?}

\begin{center}
\begin{tabular}{c|c}
Set & Cardinality \\
\hline
& \\
$\{0,1\}$ & \\
& \\
$\{0,1\}^*$ & \\
& \\
$\mathcal{P}( \{0,1\})$ & \\
& \\
The set of all languages over $\{0,1\}$ & \\
& \\
The set of all regular expressions over $\{0,1\}$ & \\
& \\
The set of all regular languages over $\{0,1\}$ & \\
& \\
\end{tabular}
\end{center}



\vfill
Strategy: Find an {\bf invariant} property that is true of all regular languages. When analyzing 
a given language, if the invariant is not true about it, then the language is not regular.
\newpage

{\bf Pumping Lemma} (Sipser Theorem 1.70): If $A$ is a regular language, then there
is a number $p$ (a {\it pumping length}) where, if $s$ is any string in $A$ of length at least $p$, 
then $s$ may be divided into three pieces, $s = xyz$ such that
\vspace{-10pt}
\begin{itemize}
\item $|y| > 0$
\item for each $i \geq 0$, $xy^i z \in A$
\item $|xy| \leq p$.
\end{itemize}


{\bf Proof illustration}


\vfill




{\bf True or False}: A pumping length for $A = \{ 0,1 \}^*$ is $p = 5$.

\vspace{50pt} \vfill
\end{document}