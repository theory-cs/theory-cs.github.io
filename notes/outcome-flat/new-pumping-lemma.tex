\documentclass[12pt, oneside]{article}

\usepackage[letterpaper, scale=0.89, centering]{geometry}
\usepackage{fancyhdr}
\setlength{\parindent}{0em}
\setlength{\parskip}{1em}

\usepackage{tikz}
\usetikzlibrary{automata,positioning,arrows}

\pagestyle{fancy}
\fancyhf{}
\renewcommand{\headrulewidth}{0pt}
\rfoot{\href{https://creativecommons.org/licenses/by-nc-sa/2.0/}{CC BY-NC-SA 2.0} Version \today~(\thepage)}

\usepackage{amssymb,amsmath,pifont,amsfonts,comment,enumerate,enumitem}
\usepackage{currfile,xstring,hyperref,tabularx,graphicx,wasysym}
\usepackage[labelformat=empty]{caption}
\usepackage{xcolor}
\usepackage{multicol,multirow,array,listings,tabularx,lastpage,textcomp,booktabs}

\lstnewenvironment{algorithm}[1][] {   
    \lstset{ mathescape=true,
        frame=tB,
        numbers=left, 
        numberstyle=\tiny,
        basicstyle=\rmfamily\scriptsize, 
        keywordstyle=\color{black}\bfseries,
        keywords={,procedure, div, for, to, input, output, return, datatype, function, in, if, else, foreach, while, begin, end, }
        numbers=left,
        xleftmargin=.04\textwidth,
        #1
    }
}
{}

\newcommand\abs[1]{\lvert~#1~\rvert}
\newcommand{\st}{\mid}

\newcommand{\cmark}{\ding{51}}
\newcommand{\xmark}{\ding{55}}
 
\begin{document}
\begin{flushright}
    \StrBefore{\currfilename}{.}
\end{flushright} \section*{Day9}


{\bf Definition and Theorem}: For an alphabet $\Sigma$, a language $L$ over $\Sigma$ is called {\bf regular}
exactly when $L$ is recognized by some DFA, which happens exactly when $L$ is recognized by some NFA, 
and happens exactly when $L$ is described by some regular expression

{\bf We saw that}: The class of regular languages is closed under complementation, union, 
intersection, set-wise concatenation, and Kleene star.

{\it Extra practice}: 

{\bf Disprove}: There is some alphabet $\Sigma$ for which there is 
some language recognized by an NFA but not by any DFA.

\vfill

{\bf Disprove}: There is some alphabet $\Sigma$ for which there is 
some finite language not described by any regular expression over $\Sigma$.

\vfill

{\bf Disprove}: If a language is recognized by an NFA 
then the complement of this language is not recognized by any DFA.

\vfill


{\bf Fix alphabet $\Sigma$. Is every language $L$ over $\Sigma$ regular?}

\begin{center}
\begin{tabular}{c|c}
Set & Cardinality \\
\hline
& \\
$\{0,1\}$ & \\
& \\
$\{0,1\}^*$ & \\
& \\
$\mathcal{P}( \{0,1\})$ & \\
& \\
The set of all languages over $\{0,1\}$ & \\
& \\
The set of all regular expressions over $\{0,1\}$ & \\
& \\
The set of all regular languages over $\{0,1\}$ & \\
& \\
\end{tabular}
\end{center}

\newpage

Strategy: Find an {\bf invariant} property that is true of all regular languages. When analyzing 
a given language, if the invariant is not true about it, then the language is not regular.

\vfill

{\bf Pumping Lemma} (Sipser Theorem 1.70): If $A$ is a regular language, then there
is a number $p$ (a {\it pumping length}) where, if $s$ is any string in $A$ of length at least $p$, 
then $s$ may be divided into three pieces, $s = xyz$ such that
\vspace{-10pt}
\begin{itemize}
\item $|y| > 0$
\item for each $i \geq 0$, $xy^i z \in A$
\item $|xy| \leq p$.
\end{itemize}
\vfill

{\bf Proof idea}: In DFA, the only memory available is in the states. 
Automata can only
``remember'' finitely far in the past and finitely much information, because
they can have only finitely many states. If a computation path of a DFA visits 
the same state more than once, the machine can't tell the difference between 
the first time and future times it visits this state. Thus, if 
a DFA accepts one long string, then it must accept (infinitely) many 
similar strings.
\vfill

{\bf Proof illustration}


\vfill
\vfill


\newpage

{\bf True or False}: A pumping length for $A = \{ 0,1 \}^*$ is $p = 5$.

\vfill


{\bf True or False}: A pumping length for $A = \{ 0,1 \}^*$ is $p = 2$.

\vfill



{\bf True or False}: A pumping length for $A = \{ 0,1 \}^*$ is $p = 105$.

\vfill



Restating {\bf Pumping Lemma}: If $L$ is a regular language, then it  has
a pumping length.


{\bf Contrapositive}: If $L$ has no pumping length, then  it is nonregular.

\vfill

{\Large The Pumping Lemma {\it cannot} be used to prove that a language {\it is} regular.} 

{\Large The Pumping Lemma {\bf can} be used to prove that a language {\it is not} regular.}

{\it Extra practice}: Exercise 1.49 in the book.


\vfill

{\bf Proof strategy}: To prove that a language $L$ is {\bf not} regular, 
\begin{itemize}
    \item Consider an arbitrary positive integer $p$
    \item Prove that $p$ is not a pumping length for $L$
    \item Conclude that $L$ does not have {\it any} pumping length, and therefore it is not regular.
\end{itemize}


{\bf Negation}: A positive integer  $p$  is {\bf not a pumping length} of a language  $L$ over  $\Sigma$  iff
\[
\exists s \left(~  |s| \geq  p \wedge s \in L \wedge \forall x \forall y \forall z  \left( ~\left( s = xyz \wedge 
|y| > 0 \wedge |xy| \leq p~ \right) \to \exists i  (  i \geq 0  \wedge xy^iz  \notin L ) \right) ~\right) 
\] \vfill
\section*{Day10}



{\bf Proof strategy}: To prove that a language $L$ is {\bf not} regular, 
\begin{itemize}
    \item Consider an arbitrary positive integer $p$
    \item Prove that $p$ is not a pumping length for $L$. A positive integer  $p$  is {\bf not a pumping length} of a language  $L$ over  $\Sigma$  iff
    \[
    \exists s \left(~  |s| \geq  p \wedge s \in L \wedge \forall x \forall y \forall z  \left( ~\left( s = xyz \wedge 
    |y| > 0 \wedge |xy| \leq p~ \right) \to \exists i  (  i \geq 0  \wedge xy^iz  \notin L ) \right) ~\right) 
    \]
    {\it Informally: }
    
\vfill

    \item Conclude that $L$ does not have {\it any} pumping length, and therefore it is not regular.
\end{itemize}

\vfill
{\bf Example}: $\Sigma  =  \{0,1\}$, $L = \{ 0^n 1^n \mid n  \geq 0\}$.

Fix $p$ an arbitrary positive integer. List strings that are in $L$ and have length  greater than or equal  to $p$:

\vspace{20pt}

Pick $s = $


Suppose $s = xyz$ with  $|xy|  \leq  p$ and $|y| > 0$.
\begin{center}
\begin{tabular}{|c|}
\hline
 \\
\hspace{4in} \\
\hline
\end{tabular}
\end{center}

Then when $i = \hspace{1in}$, $xy^i z  = \hspace{1in}$

\vfill

\newpage

{\bf Example}: $\Sigma  =  \{0,1\}$, $L = \{w w^{\mathcal{R}} \mid w \in \{0,1\}^*\}$.
Remember that the reverse of a string $w$ is denoted $w^\mathcal{R}$  
and means to write $w$  in  the opposite order, if $w = w_1 \cdots  w_n$ then $w^\mathcal{R} = w_n \cdots  w_1$. Note: $\varepsilon^\mathcal{R} = \varepsilon$.


Fix $p$ an arbitrary positive integer. List strings that are in $L$ and have length  greater than or equal  to $p$:

\vspace{10pt}

Pick $s = $

Suppose $s = xyz$ with  $|xy|  \leq  p$ and $|y| > 0$.
\begin{center}
\begin{tabular}{|c|}
\hline
 \\
\hspace{4in} \\
\hline
\end{tabular}
\end{center}
Then when $i = \hspace{1in}$, $xy^i z  = \hspace{1in}$


\vfill


{\bf Example}: $\Sigma  =  \{0,1\}$, $L = \{0^j1^k  \mid j \geq k  \geq 0\}$.

Fix $p$ an arbitrary positive integer. List strings that are in $L$ and have length  greater than or equal  to $p$:

\vspace{10pt}

Pick $s = $


Suppose $s = xyz$ with  $|xy|  \leq  p$ and $|y| > 0$.
\begin{center}
\begin{tabular}{|c|}
\hline
 \\
\hspace{4in} \\
\hline
\end{tabular}
\end{center}
Then when $i = \hspace{1in}$, $xy^i z  = \hspace{1in}$


\vfill


{\bf Example}: $\Sigma  =  \{0,1\}$, $L = \{0^n1^m0^n  \mid m,n  \geq 0\}$.

Fix $p$ an arbitrary positive integer. List strings that are in $L$ and have length  greater than or equal  to $p$:

\vspace{10pt}

Pick $s = $


Suppose $s = xyz$ with  $|xy|  \leq  p$ and $|y| > 0$.
\begin{center}
\begin{tabular}{|c|}
\hline
 \\
\hspace{4in} \\
\hline
\end{tabular}
\end{center}
Then when $i = \hspace{1in}$, $xy^i z  = \hspace{1in}$

\newpage
{\it Extra practice}:


\begin{center}
    \begin{tabular}{c|c| c| c}
    Language & $s \in L$ & $s \notin L$ & Is the language regular or nonregular?  \\
    \hline
     & \hspace{1in} & \hspace{1in}  &  \\
    $\{a^nb^n \mid 0  \leq n  \leq 5 \}$ & & & \\
     & & & \\
    $\{b^n a^n \mid  n  \geq 2\}$  & & & \\
     & & & \\
    $\{a^m b^n \mid  0 \leq m\leq n\}$  & & & \\
     & & & \\
    $\{a^m b^n \mid  m \geq n+3,  n \geq 0\}$  & & & \\
     & & & \\
    $\{b^m a^n \mid  m \geq 1, n \geq  3\}$  & & & \\
     & & & \\
    $\{ w  \in \{a,b\}^* \mid w = w^\mathcal{R} \}$ & & & \\
     & & & \\ 
    $\{ ww^\mathcal{R} \mid w\in \{a,b\}^* \}$ & & & \\
     & & & \\ 
    \end{tabular}
\end{center}
     \vfill
\end{document}