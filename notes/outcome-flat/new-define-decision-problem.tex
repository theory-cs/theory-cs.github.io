\documentclass[12pt, oneside]{article}

\usepackage[letterpaper, scale=0.89, centering]{geometry}
\usepackage{fancyhdr}
\setlength{\parindent}{0em}
\setlength{\parskip}{1em}

\usepackage{tikz}
\usetikzlibrary{automata,positioning,arrows}

\pagestyle{fancy}
\fancyhf{}
\renewcommand{\headrulewidth}{0pt}
\rfoot{\href{https://creativecommons.org/licenses/by-nc-sa/2.0/}{CC BY-NC-SA 2.0} Version \today~(\thepage)}

\usepackage{amssymb,amsmath,pifont,amsfonts,comment,enumerate,enumitem}
\usepackage{currfile,xstring,hyperref,tabularx,graphicx,wasysym}
\usepackage[labelformat=empty]{caption}
\usepackage{xcolor}
\usepackage{multicol,multirow,array,listings,tabularx,lastpage,textcomp,booktabs}

\lstnewenvironment{algorithm}[1][] {   
    \lstset{ mathescape=true,
        frame=tB,
        numbers=left, 
        numberstyle=\tiny,
        basicstyle=\rmfamily\scriptsize, 
        keywordstyle=\color{black}\bfseries,
        keywords={,procedure, div, for, to, input, output, return, datatype, function, in, if, else, foreach, while, begin, end, }
        numbers=left,
        xleftmargin=.04\textwidth,
        #1
    }
}
{}

\newcommand\abs[1]{\lvert~#1~\rvert}
\newcommand{\st}{\mid}

\newcommand{\cmark}{\ding{51}}
\newcommand{\xmark}{\ding{55}}
 
\begin{document}
\begin{flushright}
    \StrBefore{\currfilename}{.}
\end{flushright} \section*{Day1}


The CSE 105 vocabulary and notation build on discrete
math and introduction to proofs classes.  Some of the conventions may 
be a bit different from what you saw before so we'll draw your attention to them.

For consistency, we will use the notation from this class' textbook\footnote{Page references are to 
the 3rd edition of Sipser's Introduction to the Theory of Computation,
available through various sources for approximately \$30. You may be able to 
opt in to purchase a digital copy through Canvas. Copies of the book are also available 
for those who can't access the book
to borrow from the course instructor, while supplies last (minnes@ucsd.edu)}.

These definitions are on pages 3, 4, 6, 13, 14, 53.

\begin{center}
    \begin{tabular}{|p{2.6in}cp{3.5in}|}
    \hline 
    {\bf Term} & {\bf Typical symbol} & {\bf Meaning} \\
     & or {\bf Notation} & \\
    \hline
    && \\
    Alphabet & $\Sigma$, $\Gamma$ & A non-empty finite set	 \\
    Symbol over $\Sigma$  & $\sigma$, $b$, $x$ & An element of the alphabet $\Sigma$\\ 
    String over $\Sigma$  &	$u$, $v$, $w$ & A finite list of symbols from $\Sigma$\\
    (The) empty string &$\varepsilon$ & The (only) string of length $0$\\
    The set of all strings over $\Sigma$ & $\Sigma^*$ & The collection of all possible strings formed from symbols from $\Sigma$ \\ 
    (Some) language over $\Sigma$& $L$ & (Some) set of strings over $\Sigma$ \\ 
    (The) empty language &$\emptyset$ & The empty set, i.e. the set that has no strings (and no other elements either)\\
    && \\
    \hline
    && \\
    The power set of a set $X$ &$\mathcal{P}(X)$ & The set of all subsets of $X$ \\
    (The set of) natural numbers &$\mathcal{N}$ & The set of positive integers \\ 
    (Some) finite set & & The empty set or a set whose distinct elements can be counted by a natural number\\
    (Some) infinite set & & A set that is not finite.\\ 
    &&\\
    \hline
    && \\
    Reverse of a string $w$ & $w^\mathcal{R}$  & write $w$  in  the opposite order, if $w = w_1 \cdots  w_n$ then $w^\mathcal{R} = w_n \cdots  w_1$. Note: $\varepsilon^\mathcal{R} = \varepsilon$\\
    Concatenating strings $x$ and $y$ & $xy$ &  take $x = x_1 \cdots x_m$, $y=y_1 \cdots y_n$ and form $xy = x_1 \cdots x_m y_1 \cdots y_n$\\
    String $z$ is a substring of string $w$ & & there are strings $u,v$ such that $w = uzv$\\
    String $x$ is a prefix of string $y$ & & there is a string $z$ such that $y = xz$ \\
    String $x$ is a proper prefix of string $y$ & & $x$ is a prefix of $y$ and $x \neq y$\\
    && \\
    \hline
    &&\\
    Shortlex order, also known as string order over alphabet $\Sigma$ & & Order strings over  $\Sigma$ first 
    by length and then according to the dictionary order, assuming symbols in $\Sigma$  have an ordering\\ \hline
    \end{tabular}
\end{center}

\vfill
    
\newpage
Write out in words the meaning of the symbols below: 

\[
    \{ a, b, c\}
\]

\phantom{The set whose elements are $a$, $b$, and $c$}

\[
    | \{a, b, a \} | = 2
\]

\phantom{The number of elements in the set $\{a,b,a\}$ is $2$.}

\[
    | aba | = 3
\]

\phantom{The length of the string $aba$ is $3$.}

\begin{comment}

\[
    (a, 3, 2, b, b)
\]

\phantom{The $5$-tuple whose first components is $a$, second component 
is $3$, third component is $2$, fourth component is $b$, and fifth component is $b$.}
\end{comment}

{\it Circle the correct choice}:

A {\bf string} over an alphabet $\Sigma$ is \underline{~~an element of $\Sigma^*$ ~~ OR ~~ a subset of $\Sigma^*$}.
    
A {\bf language} over an alphabet $\Sigma$ is \underline{~~an element of $\Sigma^*$ ~~ OR ~~ a subset of $\Sigma^*$}.


With $\Sigma_1 = \{0,1\}$ and 
$\Sigma_2 = \{a,b,c,d,e,f,g,h,i,j,k,l,m,n,o,p,q,r,s,t,u,v,w,x,y,z\}$
and $\Gamma = \{0,1,x,y,z\}$

{\bf True} or {\bf False}: $\varepsilon \in \Sigma_1$

{\bf True} or {\bf False}: $\varepsilon$ is  a string over $\Sigma_1$

{\bf True} or {\bf False}: $\varepsilon$ is a language over $\Sigma_1$

{\bf True} or {\bf False}: $\varepsilon$ is a prefix of some string over  $\Sigma_1$

{\bf True} or {\bf False}: There is a string over $\Sigma_1$ that is a proper prefix of $\varepsilon$
    

The first five strings over $\Sigma_1$ in string order, using the ordering $0 <  1$: \vfill
    
The first five strings over $\Sigma_2$ in string order, using the usual alphabetical ordering for single letters: \vfill



 \vfill
\end{document}