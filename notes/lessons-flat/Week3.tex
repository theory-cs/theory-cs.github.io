\documentclass[12pt, oneside]{article}

\usepackage[letterpaper, scale=0.89, centering]{geometry}
\usepackage{fancyhdr}
\setlength{\parindent}{0em}
\setlength{\parskip}{1em}

\usepackage{tikz}
\usetikzlibrary{automata,positioning,arrows}

\pagestyle{fancy}
\fancyhf{}
\renewcommand{\headrulewidth}{0pt}
\rfoot{\href{https://creativecommons.org/licenses/by-nc-sa/2.0/}{CC BY-NC-SA 2.0} Version \today~(\thepage)}

\usepackage{amssymb,amsmath,pifont,amsfonts,comment,enumerate,enumitem}
\usepackage{currfile,xstring,hyperref,tabularx,graphicx,wasysym}
\usepackage[labelformat=empty]{caption}
\usepackage{xcolor}
\usepackage{multicol,multirow,array,listings,tabularx,lastpage,textcomp,booktabs}

\lstnewenvironment{algorithm}[1][] {   
    \lstset{ mathescape=true,
        frame=tB,
        numbers=left, 
        numberstyle=\tiny,
        basicstyle=\rmfamily\scriptsize, 
        keywordstyle=\color{black}\bfseries,
        keywords={,procedure, div, for, to, input, output, return, datatype, function, in, if, else, foreach, while, begin, end, }
        numbers=left,
        xleftmargin=.04\textwidth,
        #1
    }
}
{}

\newcommand\abs[1]{\lvert~#1~\rvert}
\newcommand{\st}{\mid}

\newcommand{\cmark}{\ding{51}}
\newcommand{\xmark}{\ding{55}}
 
\begin{document}
\begin{flushright}
    \StrBefore{\currfilename}{.}
\end{flushright} 
\subsection*{Week 3 at a glance}

\vspace{-15pt}

\subsubsection*{Textbook reading: Chapter 1}

\vspace{-15pt}

Before Monday, Theorem 1.47 + 1.48, Theorem 1.39 ``Proof Idea'', Example 1.41, Example 1.56.

Before Wednesday, read Introduction to Section 1.4 (page 77) which introduces nonregularity.

Before Friday, read Example 1.75, Example 1.77.

For Week 4 Monday: read Definition 2.13 (page 111-112) introducing Pushdown Automata.

\vspace{-20pt}

\subsubsection*{We will be learning and practicing to:}

\vspace{-15pt}

\begin{itemize}
\item Clearly and unambiguously communicate computational ideas using appropriate formalism. Translate across levels of abstraction.
\begin{itemize}
   \item Give examples of sets that are regular (and prove that they are).
   \begin{itemize}
      \item {\bf State the definition of the class of regular languages}
      \item {\bf Give examples of regular languages, using each of the three equivalent models of computation for proving regularity.}
      \item {\bf Choose between multiple models to prove that a language is regular}
      \item {\bf Explain the limits of the class of regular languages}
   \end{itemize}
   \item Describe and use models of computation that don't involve state machines.
   \begin{itemize}
      \item {\bf Given a DFA or NFA, find a regular expression that describes its language.}
      \item {\bf Given a regular expression, find a DFA or NFA that recognizes its language.}
   \end{itemize}
\end{itemize}

\item Know, select and apply appropriate computing knowledge and problem-solving techniques. 
\begin{itemize}
\item Apply classical techniques including pumping lemma, determinization, diagonalization, and reduction to analyze the complexity of languages and problems.
\begin{itemize}
    \item {\bf Justify why the Pumping Lemma is true.}
    \item {\bf Use the pumping lemma to prove that a given language is not regular.}
\end{itemize}
\end{itemize}

\item Understand, guide, shape impact of computing on society/the world. Connect the role of Theory CS classes to other applications (in undergraduate CS curriculum and beyond). Model problems using appropriate mathematical concepts.
 \begin{itemize}
     \item Explain nondeterminism and describe tools for simulating it with deterministic computation.
     \begin{itemize}
       \item {\bf Given a NFA, find a DFA that recognizes its language.}
       \item {\bf Convert between regular expressions and automata}
     \end{itemize}
 \end{itemize}

\end{itemize}

\vspace{-20pt}

\subsubsection*{TODO:}
\begin{list}{\itemsep-10pt}
   \item Schedule your Test 1 Attempt 1, Test 2 Attempt 1, Test 1 Attempt 2, and Test 2 Attempt 2 times 
   at PrairieTest (http://us.prairietest.com)
   \item Homework 2 submitted via Gradescope (https://www.gradescope.com/), due Tuesday 10/15/2024
   \item Review Quiz 3 on PrairieLearn (http://us.prairielearn.com), complete by Sunday 10/20/2024
\end{list}

\newpage

In Computer Science, we operationalize ``hardest'' as ``requires most resources'', where
resources might be memory, time, parallelism, randomness, power, etc.
To be able to compare ``hardness'' of problems, we use a consistent description of problems

{\bf Input}: String

{\bf Output}: Yes/ No, where Yes means that the input string matches the pattern or property described by the problem.

So far: we saw that regular expressions are convenient ways of describring patterns in strings.
{\bf Finite automata} give a model of computation for processing strings and and classifying them into Yes (accepted)
or No (rejected). We will see that each set of strings is described by a regular expression if and only 
if there is a FA that recognizes it.  Another way of thinking about it: properties described by regular
expressions require exactly the computational power of these finite automata.

\subsection*{Monday: Regular languages}



So far we have that: 
\begin{itemize}
\item If there is a DFA recognizing a language, there is a DFA recognizing its complement.
\item If there are NFA recognizing two languages, there is a NFA recognizing their union.
\item If there are DFA recognizing two languages, there is a DFA recognizing their union.
\item If there are DFA recognizing two languages, there is a DFA recognizing their intersection.
\end{itemize}

Our goals for today are (1) prove similar results about other set operations, (2) prove that 
NFA and DFA are equally expressive, and therefore (3) define an important class of languages.

\vfill

\newpage
Suppose $A_1, A_2$ are languages over an alphabet $\Sigma$.
{\bf Claim:} if there is a NFA $N_1$ such that $L(N_1) = A_1$ and 
NFA $N_2$ such that $L(N_2) = A_2$, then there is another NFA, let's call it $N$, such that 
$L(N) = A_1 \circ A_2$.

{\bf Proof idea}: Allow computation to move between $N_1$ and $N_2$ ``spontaneously" when reach an accepting state of 
$N_1$, guessing that we've reached the point where the two parts of the string in the set-wise concatenation 
are glued together.


{\bf Formal construction}: Let 
$N_1 = (Q_1, \Sigma, \delta_1, q_1, F_1)$ and $N_2 = (Q_2, \Sigma, \delta_2,q_2, F_2)$
and assume $Q_1 \cap Q_2 = \emptyset$.
Construct $N = (Q, \Sigma, \delta, q_0, F)$ where
\begin{itemize}
    \item $Q = $
    \item $q_0 = $
    \item $F = $
    \item $\delta: Q \times \Sigma_\varepsilon \to \mathcal{P}(Q)$ is defined by, for $q \in Q$ and $a \in \Sigma_{\varepsilon}$:
        \[
            \delta((q,a))=\begin{cases}  
                \delta_1 ((q,a)) &\qquad\text{if } q\in Q_1 \textrm{ and } q \notin F_1\\ 
                \delta_1 ((q,a)) &\qquad\text{if } q\in F_1 \textrm{ and } a \in \Sigma\\ 
                \delta_1 ((q,a)) \cup \{q_2\} &\qquad\text{if } q\in F_1 \textrm{ and } a = \varepsilon\\ 
                \delta_2 ((q,a)) &\qquad\text{if } q\in Q_2
            \end{cases}
        \]
\end{itemize}

\vfill

{\it Proof of correctness would prove that $L(N) = A_1 \circ A_2$ by considering
an arbitrary string accepted by $N$, tracing an accepting computation of $N$ on it, and using 
that trace to prove the string can be written as the result of concatenating two strings, 
the first in $A_1$ and the second in $A_2$; then, taking an arbitrary 
string in $A_1 \circ A_2$ and proving that it is accepted by $N$. Details left for extra practice.}

\newpage



Suppose $A$ is a language over an alphabet $\Sigma$.
{\bf Claim:} if there is a NFA $N$ such that $L(N) = A$, then there is another NFA, let's call it $N'$, such that 
$L(N') = A^*$.

{\bf Proof idea}: Add a fresh start state, which is an accept state. Add spontaneous 
moves from each (old) accept state to the old start state.

{\bf Formal construction}: Let 
$N = (Q, \Sigma, \delta, q_1, F)$ and assume $q_0 \notin Q$.
Construct $N' = (Q', \Sigma, \delta', q_0, F')$ where
\begin{itemize}
    \item $Q' = Q \cup \{q_0\}$
    \item $F' = F \cup \{q_0\}$
    \item $\delta': Q' \times \Sigma_\varepsilon \to \mathcal{P}(Q')$ is defined by, for $q \in Q'$ and $a \in \Sigma_{\varepsilon}$:
        \[
            \delta'((q,a))=\begin{cases}  
                \delta ((q,a)) &\qquad\text{if } q\in Q \textrm{ and } q \notin F\\ 
                \delta ((q,a)) &\qquad\text{if } q\in F \textrm{ and } a \in \Sigma\\ 
                \delta ((q,a)) \cup \{q_1\} &\qquad\text{if } q\in F \textrm{ and } a = \varepsilon\\ 
                \{q_1\} &\qquad\text{if } q = q_0 \textrm{ and } a = \varepsilon \\
                \emptyset &\qquad\text{if } q = q_0 \textrm { and } a \in \Sigma
            \end{cases}
        \]
\end{itemize}


{\it Proof of correctness would prove that $L(N') = A^*$ by considering
an arbitrary string accepted by $N'$, tracing an accepting computation of $N'$ on it, and using 
that trace to prove the string can be written as the result of concatenating some number of strings, 
each of which is in $A$; then, taking an arbitrary 
string in $A^*$ and proving that it is accepted by $N'$. Details left for extra practice.}


{\bf Application}: A state diagram for a NFA over $\Sigma = \{a,b\}$ 
that recognizes $L (( a^*b)^* )$:

\vfill
\newpage
Suppose $A$ is a language over an alphabet $\Sigma$.
{\bf Claim:} if there is a NFA $N$ such that $L(N) = A$ then 
there is a DFA $M$ such that $L(M) = A$.

{\bf Proof idea}: States in $M$ are ``macro-states" -- collections of states from $N$ -- 
that represent the set of possible states a computation of $N$ might be in.


{\bf Formal construction}: Let $N = (Q, \Sigma, \delta, q_0, F)$.  Define 
\[
M = (~ \mathcal{P}(Q), \Sigma, \delta', q',  \{ X \subseteq Q \mid X \cap F \neq \emptyset \}~ )
\]
where $q' = \{ q \in Q \mid \text{$q = q_0$ or is accessible from $q_0$ by spontaneous moves in $N$} \}$
and 
\[
    \delta' (~(X, x)~) = \{ q \in Q \mid q \in \delta( ~(r,x)~) ~\text{for some $r \in X$ or is accessible 
from such an $r$ by spontaneous moves in $N$} \}
\]


Consider the state diagram of an NFA over $\{a,b\}$. Use the ``macro-state'' construction 
to find an equivalent DFA.

\includegraphics[width=2.5in]{../../resources/machines/Lect6NFA1.png}




\vfill

Consider the state diagram of an NFA over $\{0,1\}$. Use the ``macro-state'' construction 
to find an equivalent DFA.


\includegraphics[width=1.8in]{../../resources/machines/Lect6NFA2.png}


\vfill

Note: We can often prune the DFAs that result from the ``macro-state'' constructions to get an 
equivalent DFA with fewer states (e.g.\ only the ``macro-states" reachable from the start state).

\newpage


{\bf The class of regular languages}

Fix an alphabet $\Sigma$. For each language $L$ over $\Sigma$:
\begin{center}
\begin{tabular}{cc}
    {\bf There is a DFA over $\Sigma$ that recognizes $L$}&$\exists M ~(M \textrm{ is a DFA and } L(M) = A)$\\
    {\it if and only if}&\\
    {\bf There is a NFA over $\Sigma$ that recognizes $L$}&$\exists N ~(N \textrm{ is a NFA and } L(N) = A)$\\
    {\it if and only if}&\\
    {\bf There is a regular expression over $\Sigma$ that describes $L$} &$\exists R ~(R \textrm{ is a regular expression and } L(R) = A)$\\
\end{tabular}
\end{center}

A language is called {\bf regular} when any (hence all) of the above three conditions are met.

We already proved that DFAs and NFAs are equally expressive. It remains to prove that regular expressions 
are too.

Part 1: Suppose $A$ is a language over an alphabet $\Sigma$.
If there is a regular expression $R$ such that $L(R) = A$, then there is a NFA, let's call it $N$, such that 
$L(N) = A$.

{\bf Structural induction}: Regular expression is built from basis regular expressions using inductive steps
(union, concatenation, Kleene star symbols). Use constructions to mirror these in NFAs.


{\bf Application}: A state diagram for a NFA over $\{a,b\}$ that recognizes $L(a^* (ab)^*)$:

\vfill

Part 2: Suppose $A$ is a language over an alphabet $\Sigma$.
If there is a DFA $M$ such that $L(M) = A$, then there is a regular expression, let's call it $R$, such that 
$L(R) = A$.

{\bf Proof idea}: Trace all possible paths from start state to accept state.  Express labels of these paths
as regular expressions, and union them all.

\begin{enumerate}
\item Add new start state with $\varepsilon$ arrow to old start state.
\item Add new accept state with $\varepsilon$ arrow from old accept states.  Make old accept states
non-accept.
\item Remove one (of the old) states at a time: modify regular expressions on arrows that went through removed
state to restore language recognized by machine.
\end{enumerate}

{\bf Application}: Find a regular expression describing the language recognized by the DFA with 
state diagram

\includegraphics[width=2.5in]{../../resources/machines/Lect6NFA3.png}

\vfill
 
\newpage
\subsection*{Wednesday: Nonregular languages}



{\bf Definition and Theorem}: For an alphabet $\Sigma$, a language $L$ over $\Sigma$ is called {\bf regular}
exactly when $L$ is recognized by some DFA, which happens exactly when $L$ is recognized by some NFA, 
and happens exactly when $L$ is described by some regular expression

{\bf We saw that}: The class of regular languages is closed under complementation, union, 
intersection, set-wise concatenation, and Kleene star.

{\bf Prove or Disprove}: There is some alphabet $\Sigma$ for which there is 
some language recognized by an NFA but not by any DFA.

\vfill

{\bf Prove or Disprove}: There is some alphabet $\Sigma$ for which there is 
some finite language not described by any regular expression over $\Sigma$.

\vfill

{\bf Prove or Disprove}: If a language is recognized by an NFA 
then the complement of this language is not recognized by any DFA.

\vfill

\newpage

{\bf Fix alphabet $\Sigma$. Is every language $L$ over $\Sigma$ regular?}

\begin{center}
\begin{tabular}{c|c}
Set & Cardinality \\
\hline
& \\
$\{0,1\}$ & \\
& \\
$\{0,1\}^*$ & \\
& \\
$\mathcal{P}( \{0,1\})$ & \\
& \\
The set of all languages over $\{0,1\}$ & \\
& \\
The set of all regular expressions over $\{0,1\}$ & \\
& \\
The set of all regular languages over $\{0,1\}$ & \\
& \\
\end{tabular}
\end{center}



\vfill
Strategy: Find an {\bf invariant} property that is true of all regular languages. When analyzing 
a given language, if the invariant is not true about it, then the language is not regular.
\newpage

{\bf Pumping Lemma} (Sipser Theorem 1.70): If $A$ is a regular language, then there
is a number $p$ (a {\it pumping length}) where, if $s$ is any string in $A$ of length at least $p$, 
then $s$ may be divided into three pieces, $s = xyz$ such that
\vspace{-10pt}
\begin{itemize}
\item $|y| > 0$
\item for each $i \geq 0$, $xy^i z \in A$
\item $|xy| \leq p$.
\end{itemize}


{\bf Proof illustration}


\vfill




{\bf True or False}: A pumping length for $A = \{ 0,1 \}^*$ is $p = 5$.

\vfill 
\newpage
\subsection*{Friday: Pumping Lemma}



Recap so far: In DFA, the only memory available is in the states. Automata can only
``remember'' finitely far in the past and finitely much information, because
they can have only finitely many states. If a computation path of a DFA visits 
the same state more than once, the machine can't tell the difference between 
the first time and future times it visits this state. Thus, if 
a DFA accepts one long string, then it must accept (infinitely) many 
similar strings.

{\bf Definition}  A positive integer $p$ is a {\bf pumping length} of a language $L$ over $\Sigma$ means
that, for each string $s  \in  \Sigma^*$, if  $|s| \geq p$ and $s \in L$, then there are strings $x,y,z$
such that 
\[
s = xyz
\]
and  
\[
|y| > 0,  \qquad \qquad 
\text{ for each $i \geq 0$, $xy^i z \in L$}, \qquad \text{and}
\qquad  \qquad
|xy| \leq p.
\]

{\bf Negation}: A positive integer  $p$  is {\bf not a pumping length} of a language  $L$ over  $\Sigma$  iff
\[
\exists s \left(~  |s| \geq  p \wedge s \in L \wedge \forall x \forall y \forall z  \left( ~\left( s = xyz \wedge 
|y| > 0 \wedge |xy| \leq p~ \right) \to \exists i  (  i \geq 0  \wedge xy^iz  \notin L ) \right) ~\right) 
\]
{\it Informally: }


Restating {\bf Pumping Lemma}: If $L$ is a regular language, then it  has
a pumping length.


{\bf Contrapositive}: If $L$ has no pumping length, then  it is nonregular.

\vfill

{\Large The Pumping Lemma {\it cannot} be used to prove that a language {\it is} regular.} 

{\Large The Pumping Lemma {\bf can} be used to prove that a language {\it is not} regular.}

{\it Extra practice}: Exercise 1.49 in the book.


\vfill

{\bf Proof strategy}: To prove that a language $L$ is {\bf not} regular, 
\begin{itemize}
    \item Consider an arbitrary positive integer $p$
    \item Prove that $p$ is not a pumping length for $L$
    \item Conclude that $L$ does not have {\it any} pumping length, and therefore it is not regular.
\end{itemize}

\newpage
{\bf Example}: $\Sigma  =  \{0,1\}$, $L = \{ 0^n 1^n \mid n  \geq 0\}$.

Fix $p$ an arbitrary positive integer. List strings that are in $L$ and have length  greater than or equal  to $p$:

\vspace{20pt}

Pick $s = $


Suppose $s = xyz$ with  $|xy|  \leq  p$ and $|y| > 0$.
\begin{center}
\begin{tabular}{|c|}
\hline
 \\
\hspace{4in} \\
\hline
\end{tabular}
\end{center}

Then when $i = \hspace{1in}$, $xy^i z  = \hspace{1in}$

\newpage

{\bf Example}: $\Sigma  =  \{0,1\}$, $L = \{w w^{\mathcal{R}} \mid w \in \{0,1\}^*\}$.
Remember that the reverse of a string $w$ is denoted $w^\mathcal{R}$  
and means to write $w$  in  the opposite order, if $w = w_1 \cdots  w_n$ then $w^\mathcal{R} = w_n \cdots  w_1$. Note: $\varepsilon^\mathcal{R} = \varepsilon$.


Fix $p$ an arbitrary positive integer. List strings that are in $L$ and have length  greater than or equal  to $p$:

\vspace{10pt}

Pick $s = $

Suppose $s = xyz$ with  $|xy|  \leq  p$ and $|y| > 0$.
\begin{center}
\begin{tabular}{|c|}
\hline
 \\
\hspace{4in} \\
\hline
\end{tabular}
\end{center}
Then when $i = \hspace{1in}$, $xy^i z  = \hspace{1in}$


\vspace{30pt} 

{\bf Example}: $\Sigma  =  \{0,1\}$, $L = \{0^j1^k  \mid j \geq k  \geq 0\}$.

Fix $p$ an arbitrary positive integer. List strings that are in $L$ and have length  greater than or equal  to $p$:

\vspace{10pt}

Pick $s = $


Suppose $s = xyz$ with  $|xy|  \leq  p$ and $|y| > 0$.
\begin{center}
\begin{tabular}{|c|}
\hline
 \\
\hspace{4in} \\
\hline
\end{tabular}
\end{center}
Then when $i = \hspace{1in}$, $xy^i z  = \hspace{1in}$



\vspace{30pt} 

{\bf Example}: $\Sigma  =  \{0,1\}$, $L = \{0^n1^m0^n  \mid m,n  \geq 0\}$.

Fix $p$ an arbitrary positive integer. List strings that are in $L$ and have length  greater than or equal  to $p$:

\vspace{10pt}

Pick $s = $


Suppose $s = xyz$ with  $|xy|  \leq  p$ and $|y| > 0$.
\begin{center}
\begin{tabular}{|c|}
\hline
 \\
\hspace{4in} \\
\hline
\end{tabular}
\end{center}
Then when $i = \hspace{1in}$, $xy^i z  = \hspace{1in}$

\newpage
{\it Extra practice}:


\begin{center}
    \begin{tabular}{c|c| c| c}
    Language & $s \in L$ & $s \notin L$ & Is the language regular or nonregular?  \\
    \hline
     & \hspace{1in} & \hspace{1in}  &  \\
    $\{a^nb^n \mid 0  \leq n  \leq 5 \}$ & & & \\
     & & & \\
    $\{b^n a^n \mid  n  \geq 2\}$  & & & \\
     & & & \\
    $\{a^m b^n \mid  0 \leq m\leq n\}$  & & & \\
     & & & \\
    $\{a^m b^n \mid  m \geq n+3,  n \geq 0\}$  & & & \\
     & & & \\
    $\{b^m a^n \mid  m \geq 1, n \geq  3\}$  & & & \\
     & & & \\
    $\{ w  \in \{a,b\}^* \mid w = w^\mathcal{R} \}$ & & & \\
     & & & \\ 
    $\{ ww^\mathcal{R} \mid w\in \{a,b\}^* \}$ & & & \\
     & & & \\ 
    \end{tabular}
\end{center}
     
\end{document}